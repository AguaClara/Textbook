%% Generated by Sphinx.
\def\sphinxdocclass{report}
\documentclass[letterpaper,10pt,english]{sphinxmanual}
\ifdefined\pdfpxdimen
   \let\sphinxpxdimen\pdfpxdimen\else\newdimen\sphinxpxdimen
\fi \sphinxpxdimen=.75bp\relax

\PassOptionsToPackage{warn}{textcomp}
\usepackage[utf8]{inputenc}
\ifdefined\DeclareUnicodeCharacter
 \ifdefined\DeclareUnicodeCharacterAsOptional
  \DeclareUnicodeCharacter{"00A0}{\nobreakspace}
  \DeclareUnicodeCharacter{"2500}{\sphinxunichar{2500}}
  \DeclareUnicodeCharacter{"2502}{\sphinxunichar{2502}}
  \DeclareUnicodeCharacter{"2514}{\sphinxunichar{2514}}
  \DeclareUnicodeCharacter{"251C}{\sphinxunichar{251C}}
  \DeclareUnicodeCharacter{"2572}{\textbackslash}
 \else
  \DeclareUnicodeCharacter{00A0}{\nobreakspace}
  \DeclareUnicodeCharacter{2500}{\sphinxunichar{2500}}
  \DeclareUnicodeCharacter{2502}{\sphinxunichar{2502}}
  \DeclareUnicodeCharacter{2514}{\sphinxunichar{2514}}
  \DeclareUnicodeCharacter{251C}{\sphinxunichar{251C}}
  \DeclareUnicodeCharacter{2572}{\textbackslash}
 \fi
\fi
\usepackage{cmap}
\usepackage[T1]{fontenc}
\usepackage{amsmath,amssymb,amstext}
\usepackage{babel}
\usepackage{times}
\usepackage[Bjarne]{fncychap}
\usepackage[,numfigreset=1,mathnumfig]{sphinx}

\usepackage{geometry}

% Include hyperref last.
\usepackage{hyperref}
% Fix anchor placement for figures with captions.
\usepackage{hypcap}% it must be loaded after hyperref.
% Set up styles of URL: it should be placed after hyperref.
\urlstyle{same}
\addto\captionsenglish{\renewcommand{\contentsname}{Acknowledgements}}

\addto\captionsenglish{\renewcommand{\figurename}{Fig.}}
\addto\captionsenglish{\renewcommand{\tablename}{Table}}
\addto\captionsenglish{\renewcommand{\literalblockname}{Listing}}

\addto\captionsenglish{\renewcommand{\literalblockcontinuedname}{continued from previous page}}
\addto\captionsenglish{\renewcommand{\literalblockcontinuesname}{continues on next page}}

\addto\extrasenglish{\def\pageautorefname{page}}

\setcounter{tocdepth}{0}


        \usepackage{cancel}
    

\title{AguaClara Textbook Documentation}
\date{Aug 06, 2018}
\release{}
\author{AguaClara Cornell}
\newcommand{\sphinxlogo}{\vbox{}}
\renewcommand{\releasename}{}
\makeindex

\begin{document}

\maketitle
\sphinxtableofcontents
\phantomsection\label{\detokenize{index::doc}}


This textbook is written and maintained in \sphinxhref{https://github.com/AguaClara/Textbook}{Github} via \sphinxhref{http://www.sphinx-doc.org/en/master/}{Sphinx}. It uses and refers to AguaClara code and functions in \sphinxhref{https://github.com/AguaClara/aide\_design}{aide\_design}. Listed below are the versions of the programs we use:


\begin{savenotes}\sphinxattablestart
\centering
\sphinxcapstartof{table}
\sphinxcaption{These are the software versions used to compile this textbook}\label{\detokenize{index:id2}}\label{\detokenize{index:software-versions}}
\sphinxaftercaption
\begin{tabular}[t]{|\X{10}{20}|\X{10}{20}|}
\hline
\sphinxstyletheadfamily 
Software
&\sphinxstyletheadfamily 
version
\\
\hline
Sphinx
&
1.7.5
\\
\hline
aide\_design
&
0.0.12
\\
\hline
Anaconda
&
4.5.4
\\
\hline
Python
&
3.6.5
\\
\hline
\end{tabular}
\par
\sphinxattableend\end{savenotes}


\chapter{Acknowledgements}
\label{\detokenize{Acknowledgements:acknowledgements}}\label{\detokenize{Acknowledgements:id1}}\label{\detokenize{Acknowledgements::doc}}
We gratefully acknowledge the funding provided by the Environmental Protection Agency and the National Science Foundation. Together they have provided over \$1 million in support of developing the next generation of sustainable drinking water treatment technologies.


\section{Environmental Protection Agency statement}
\label{\detokenize{Acknowledgements:environmental-protection-agency-statement}}
“This textbook was developed under numerous Assistance Agreements awarded by the U.S. Environmental Protection Agency to Cornell University. It has not been formally reviewed by EPA. The views expressed in this document are solely those of the authors and do not necessarily reflect those of the Agency. EPA does not endorse any products or commercial services mentioned in this publication.”

\begin{figure}[htbp]
\centering

\noindent\sphinxincludegraphics{{961a201f5bb2324746e0904245efc79fe2157900}.jpg}
\label{\detokenize{Acknowledgements:figure-nsf-logo}}\end{figure}


\section{National Science Foundation statement}
\label{\detokenize{Acknowledgements:national-science-foundation-statement}}
This material is based upon work supported by the National Science Foundation under Grant numbers CBET-1704472 and CBET-1437961. Any opinions, findings, and conclusions or recommendations expressed in this material are those of the authors and do not necessarily reflect the views of the National Science Foundation.


\begin{savenotes}\sphinxattablestart
\raggedright
\sphinxcapstartof{table}
\sphinxcaption{Table of funded research projects that contributed to the knowledge in this textbook.}\label{\detokenize{Acknowledgements:id2}}\label{\detokenize{Acknowledgements:table-funded-proposals}}
\sphinxaftercaption
\begin{tabular}[t]{|\X{30}{180}|\X{150}{180}|}
\hline
\sphinxstyletheadfamily 
Agency
&\sphinxstyletheadfamily 
Proposal Title
\\
\hline
NSF
&
Wrf: Experimental Observation and Modeling of Coagulant Mediated Contaminant Removal: Flocculation, Floc Blankets, and Sedimentation
\\
\hline
USEPA
&
AguaClara’s Ram Pump for Zero Electricity Drinking Water Treatment
\\
\hline
USEPA
&
Environment \& Community Friendly Wastewater Treatment
\\
\hline
USEPA
&
High Rate Sedimentation
\\
\hline
USEPA
&
Novel Reactor Design for Enhanced Removal of Fluoride Using A Modified Nalgonda Method
\\
\hline
USEPA
&
Novel Reactor Design for Enhanced Removal of Fluoride Using A Modified Nalgonda Method
\\
\hline
NSF
&
Experimental Evaluation And Modeling Of Hydraulic Flocculation Systems Under Conditions of Turbulent Flow
\\
\hline
USEPA
&
Application of Foam Filtration to Water Treatment for Rapid Emergency Response
\\
\hline
USEPA
&
Stacked Rapid Sand Filtration - A Robust Filtration Process for Sustainable Drinking Water
\\
\hline
USEPA
&
Sustainable Water Treatment Facility for Communities with Arsenic Contaminated Groundwater
\\
\hline
USEPA
&
Smart Turbidimeters for Remote Monitoring of Water Quality
\\
\hline
USEPA
&
Stacked Rapid Sand Filtration - A Robust Filtration Process for Sustainable Drinking Water Infrastructure
\\
\hline
USEPA
&
Developing A Point-of-Use Filter Utilizing Polyurethane Foam
\\
\hline
USEPA
&
Dose Controller for AguaClara Water Treatment Plants
\\
\hline
USEPA
&
Dose Controller for AguaClara Water Treatment Plants
\\
\hline
USEPA
&
AguaClara: Clean Water for Small Communities
\\
\hline
\end{tabular}
\par
\sphinxattableend\end{savenotes}
\begin{description}
\item[{More gratitude below!}] \leavevmode\begin{itemize}
\item {} 
Ken Brown and the Sanjuan Foundation

\item {} 
Duane Stiller

\item {} 
Countless other donors whose contributions have made it possible to develop new technologies and share those technologies with partner organizations and communities

\item {} 
Many hundreds of students who gave their time and creativity so that others could have safe water on tap

\item {} 
The Swiss Development Cooperation that has funded the construction of 5 AguaClara plants in Honduras and 2 in Nicaragua

\item {} 
The Cornell Engineering that provided generous funding in the startup years.

\end{itemize}

\end{description}


\chapter{Authors}
\label{\detokenize{Authors:authors}}\label{\detokenize{Authors:title-authors}}\label{\detokenize{Authors::doc}}
This text is a collaborative effort involving hundreds of people. Innovation requires collisions of ideas and the AguaClara program was designed to foster global and multidisciplinary interactions between students, faculty, field engineers, plant operators, implementation partner organizations, and community members. These interactions have provided a continuous and rich source of ideas that make it clear that in a social network it is impossible for anyone to claim ownership of an idea. Thus the inventions, equations, and reactor designs that are described in this text are the product of a large, collaborative, open-source community and none of us can claim that we are the sole authors. The list of authors below have contributed directly to this text.
\begin{itemize}
\item {} 
Juan Guzman

\item {} 
Monroe Weber-Shirk

\item {} 
Clare O’Connor

\item {} 
Ethan Keller

\end{itemize}


\chapter{Introduction to RST and Sphinx for Textbook Contributors}
\label{\detokenize{Textbook_Creation_Help/rst_intro:introduction-to-rst-and-sphinx-for-textbook-contributors}}\label{\detokenize{Textbook_Creation_Help/rst_intro:title-rst-intro}}\label{\detokenize{Textbook_Creation_Help/rst_intro::doc}}

\section{What is RST?}
\label{\detokenize{Textbook_Creation_Help/rst_intro:what-is-rst}}\label{\detokenize{Textbook_Creation_Help/rst_intro:heading-what-is-rst}}
RST stands for ReStructured Text. It is the standard markup language used for documenting python packages. \sphinxhref{http://www.sphinx-doc.org/en/master/}{Sphinx} is the Python package that generates an html website from RST files, and it is what we are using to generate this site. To read more about why we chose RST over markdown or Latex, read the following section, {\hyperref[\detokenize{Textbook_Creation_Help/rst_intro:heading-why-rst}]{\sphinxcrossref{\DUrole{std,std-ref}{Why RST?}}}}.


\subsection{Why RST?}
\label{\detokenize{Textbook_Creation_Help/rst_intro:why-rst}}\label{\detokenize{Textbook_Creation_Help/rst_intro:heading-why-rst}}
In the beginning, we used markdown. As we tried to add different features to markdown (\DUrole{red}{colored words}, image sizes, citations), we were forced to use raw html and various pre-processors. With these various band-aid solutions came added complexity. Adding sections became cumbersome and awkward as it required ill-defined html. Additionally, providing site-wide style updates was prohibitively time-consuming and complex. Essentially, we were trying to pack too much functionality into markdown. In the search for an alternative, restructured text provided several advantages. Out of the box, RST supports globally-defined styles, figure numbering and referencing, Latex function rendering, image display customization and more. Furthermore, restructured text was already the language of choice for the AIDE ecosystem’s documentation.


\section{Setting up RST for Development}
\label{\detokenize{Textbook_Creation_Help/rst_intro:setting-up-rst-for-development}}\label{\detokenize{Textbook_Creation_Help/rst_intro:heading-setting-up-rst}}
There are two ways to \sphinxstyleemphasis{quickly} view an RST file. The first is using an \sphinxhref{https://ide.atom.io/}{Atom} plugin that renders the view alongside the source code. This is a good initial test to make sure the RST is proper RST and looks \sphinxstyleemphasis{mostly} correct. However, some functionality, such as any extensions provided by \sphinxhref{http://www.sphinx-doc.org/en/master/}{Sphinx} won’t run in the preview. In order to see the final html that will display on the website, you’ll need to use the second method, running sphinx locally to fully generate the html code. Once you are satisfied with your work and want to push it to the textbook, you’ll need to incorporate it to the master branch. To do so, refer to {\hyperref[\detokenize{Textbook_Creation_Help/rst_intro:publishing-online}]{\sphinxcrossref{Publishing online}}}.


\subsection{Installing the Atom Plugins}
\label{\detokenize{Textbook_Creation_Help/rst_intro:installing-the-atom-plugins}}\label{\detokenize{Textbook_Creation_Help/rst_intro:heading-installing-atom}}
If you are using the Atom IDE to write RST, you can use the \sphinxhref{https://atom.io/packages/rst-preview-pandoc}{rst-preview-pandoc} plugin to auto-generate a live RST preview within atom (much like the markdown-preview-plus preview page.) To get rst-preview working, you’ll need to install \sphinxhref{https://atom.io/packages/language-restructuredtext}{language-restructuredtext} via atom and \sphinxhref{https://pandoc.org/installing.html}{Pandoc} via your command line (\sphinxcode{\sphinxupquote{pip install pandoc}}). If everything worked, you can use \sphinxcode{\sphinxupquote{ctrl + shift + e}} to toggle a display window for the live-updated RST preview.


\subsection{Building RST Locally with Sphinx}
\label{\detokenize{Textbook_Creation_Help/rst_intro:building-rst-locally-with-sphinx}}\label{\detokenize{Textbook_Creation_Help/rst_intro:heading-building-rst-locally}}
We use \sphinxhref{http://www.sphinx-doc.org/en/master/}{Sphinx} to build RST locally and remotely. Follow these steps to get \sphinxhref{http://www.sphinx-doc.org/en/master/}{Sphinx} and run it locally:
\begin{enumerate}
\item {} 
Install \sphinxhref{http://www.sphinx-doc.org/en/master/}{Sphinx}, disqus, and a sphinx visual theme using pip: \sphinxcode{\sphinxupquote{pip install sphinx -{-}user -U}} and \sphinxcode{\sphinxupquote{pip install git+https://github.com/rmk135/sphinxcontrib-disqus}}.

\item {} 
Generate all the html by navigating in the command line to the source directory /Textbook and creating the build in that directory with the command line \sphinxcode{\sphinxupquote{make html}}.

\item {} 
View the html generated in the /Textbook/\_build directory by copying the full file path of /Textbook/\_build/html/index.html and pasting it into your browser.

\end{enumerate}

\begin{sphinxadmonition}{note}{Note:}
Regarding \sphinxstylestrong{1.} the master branch for the package implementing disqus in sphinx \sphinxhref{https://github.com/Robpol86/sphinxcontrib-disqus/pull/7}{is broken}, which is why we use a non-standard pip/online installation. If you already have the incorrect sphinx-disqus version installed, uninstall it with \sphinxcode{\sphinxupquote{pip uninstall sphinxcontrib-disqus}} before installing the functioning version.
\end{sphinxadmonition}


\subsection{Publishing Online}
\label{\detokenize{Textbook_Creation_Help/rst_intro:publishing-online}}\label{\detokenize{Textbook_Creation_Help/rst_intro:heading-publishing-online}}
We use \sphinxhref{https://travis-ci.org/}{Travis} to ensure this site will always contain functional builds. To publish online, you need to:
\begin{enumerate}
\item {} 
Always test your build by first :ref:{}` building RST locally \textless{}heading\_building\_rst\_locally\textgreater{}{}`, and then following the {\hyperref[\detokenize{Textbook_Creation_Help/rst_intro:heading-testing-online}]{\sphinxcrossref{\DUrole{std,std-ref}{testing online}}}} instructions. Once you like how your build looks, follow the steps below to introduce it into the master branch.

\item {} 
Submit a \sphinxhref{https://github.com/AguaClara/Textbook/pulls}{pull request to master}. You’ll need to ask for someone else to review your work at this stage- “request reviewers”. Every pull request \sphinxstylestrong{must} be reviewed by at least one other person.

\item {} 
\sphinxhref{https://travis-ci.org/}{Travis} will build the site using \sphinxhref{http://www.sphinx-doc.org/en/master/}{Sphinx}, and if there aren’t any errors, Travis will report success to GitHub on the “checks” part of the pull request.

\item {} 
All your requested reviewers must now approve and comment on  your commit before the merge is allowed.

\item {} 
Once the PR passes Travis and is approved by another author, feel free to “merge to master.”

\item {} \begin{description}
\item[{To release the master branch, (build the html, pdf, and latex, and upload the pdf to Pages) you’ll need to publish a \sphinxhref{https://github.com/AguaClara/Textbook/releases/new}{GitHub release}. Include a \sphinxhref{https://semver.org/}{semver} version number as the tag (under “Tag: Choose or create”), and a brief description of the updates under “Release Title”. Finally, for the description, detail the changes as much as you see fit and when ready, hit “Publish release”. Example:}] \leavevmode\begin{itemize}
\item {} 
Tag name: 0.1.5

\item {} 
Release title: Filtration section maintenance

\item {} 
Description: Added filter code from aide\_design 0.2.6. Also updated all broken external links.

\end{itemize}

\end{description}

\item {} 
Travis will rebuild the site and push the html to Pages, and the PDF and LaTeX to GitHub Releases under the tag name.

\end{enumerate}

\begin{sphinxadmonition}{important}{Important:}
If your changes to the master branch aren’t pushing to gh-pages, then check the status of the \sphinxhref{https://travis-ci.org/AguaClara/Textbook}{Travis build here}.
\end{sphinxadmonition}


\subsection{Testing Online}
\label{\detokenize{Textbook_Creation_Help/rst_intro:testing-online}}\label{\detokenize{Textbook_Creation_Help/rst_intro:heading-testing-online}}
To test exactly what will be published, we have a test branch. The test branch is built by Travis and contains all the processed html that Travis produces in \_build/html. This branch is populated when ANY COMMIT IS PUSHED. Meaning, the last commit to be pushed, if it passes the Travis tests, will be built and the output will be placed in the test branch. Also, if the PDF=True environment variable is triggered for a Travis build, the PDF will also be generated and placed in the test branch. To do this, use the “Trigger Build” option in Travis and put:

\fvset{hllines={, ,}}%
\begin{sphinxVerbatim}[commandchars=\\\{\}]
\PYG{n}{script}\PYG{p}{:}
    \PYG{o}{\PYGZhy{}} \PYG{n}{PDF}\PYG{o}{=}\PYG{k+kc}{True}
\end{sphinxVerbatim}

\sphinxhref{https://rawgit.com/AguaClara/Textbook/test/html/index.html}{The website output is viewable here}.


\subsection{Sharing Test Output}
\label{\detokenize{Textbook_Creation_Help/rst_intro:sharing-test-output}}
if you want to share what your latest branch developments look like without having whoever is viewing it actually have to build it, you can push a commit, and find the \sphinxhref{https://rawgit.com/}{rawgit URL with this site} by entering the URL of the git file within the test branch that you’d like to share. Furthermore, if you want to point to the commit so that even if someone else pushes, the URL will still point to the code you intend it to, make sure to include the commit SHA within the rawgit URL like so: \sphinxurl{https://rawgit.com/AguaClara/Textbook/e5693e0485702b95e11d4d6bdf76d07c42fdbf99/html/index.html}. That link will never change where it is pointing. To share the PDF output, follow the {\hyperref[\detokenize{Textbook_Creation_Help/rst_intro:heading-testing-online}]{\sphinxcrossref{\DUrole{std,std-ref}{testing online}}}} instructions to build the PDF, and point to the commit with the PDF. Happy testing!


\section{Brief Best Practices}
\label{\detokenize{Textbook_Creation_Help/rst_intro:brief-best-practices}}\label{\detokenize{Textbook_Creation_Help/rst_intro:heading-brief-best-practices}}
When writing RST, there are often many ways to write the same thing. Almost always, the way with the fewest number of characters is the best way. Ideally, never copy and paste.


\subsection{How do I write RST?}
\label{\detokenize{Textbook_Creation_Help/rst_intro:how-do-i-write-rst}}\label{\detokenize{Textbook_Creation_Help/rst_intro:heading-how-do-i-write-rst}}
RST is friendly to learn and powerful. There are many useful cheatsheets, like \sphinxhref{https://thomas-cokelaer.info/tutorials/sphinx/rest\_syntax.html\#inserting-code-and-literal-blocks}{this one} and the next page on this site: \DUrole{xref,std,std-ref}{Functionality in RST and AguaClara Convention}, which is specifically for AguaClara and this textbook project. When you start writing RST, look at the cheat sheets all the time. Use \sphinxcode{\sphinxupquote{ctrl-f}} all the time to find whatever you need.

\sphinxstylestrong{Things not covered in most cheat sheets which are of critical importance:}
\begin{itemize}
\item {} 
A document is referred to by its title, as defined between the \sphinxcode{\sphinxupquote{*****}} signs at the top of the page, \sphinxstylestrong{NOT} the filename. So it is critical to have a title.

\item {} 
Anything else you’d like to add for the future…

\end{itemize}


\subsection{Example to Start From}
\label{\detokenize{Textbook_Creation_Help/rst_intro:example-to-start-from}}\label{\detokenize{Textbook_Creation_Help/rst_intro:heading-example-to-start-from}}
This file is written in RST. You can start there! Just click on “View page source” at the top of the page.

Also, the next page contains the convention, and is where we specify all AguaClara RST best practices: \DUrole{xref,std,std-ref}{Functionality in RST and AguaClara Convention}. I recommend looking at the raw RST and the rendered html side by side.


\section{Converting Markdown to RST}
\label{\detokenize{Textbook_Creation_Help/rst_intro:converting-markdown-to-rst}}\label{\detokenize{Textbook_Creation_Help/rst_intro:heading-converting-md-to-rst}}
Ideally, use pandoc to do the conversion in the command line: \sphinxcode{\sphinxupquote{pandoc -{-}from=markdown -{-}to=rst -{-}output=my\_file.rst my\_file.md}}.
Raw html will not be converted (because it is not actually markdown), and tables are converted poorly.
You’ll need to carefully review any page converted with pandoc.


\chapter{Parameter Convention List}
\label{\detokenize{Textbook_Creation_Help/parameter_convention_list:parameter-convention-list}}\label{\detokenize{Textbook_Creation_Help/parameter_convention_list:title-parameter-convention-list}}\label{\detokenize{Textbook_Creation_Help/parameter_convention_list::doc}}

\begin{savenotes}\sphinxattablestart
\centering
\sphinxcapstartof{table}
\sphinxcaption{Relevant Dimensions}\label{\detokenize{Textbook_Creation_Help/parameter_convention_list:id1}}\label{\detokenize{Textbook_Creation_Help/parameter_convention_list:table-dimension-table}}
\sphinxaftercaption
\begin{tabular}[t]{|\X{30}{90}|\X{30}{90}|\X{30}{90}|}
\hline
\sphinxstyletheadfamily 
Dimension
&\sphinxstyletheadfamily 
Abbreviation
&\sphinxstyletheadfamily 
Base Unit
\\
\hline
Length
&
\([L]\)
&
meter
\\
\hline
Mass
&
\([M]\)
&
kilogram
\\
\hline
Time
&
\([T]\)
&
second
\\
\hline
\end{tabular}
\par
\sphinxattableend\end{savenotes}

If you would like to be able to \sphinxcode{\sphinxupquote{ctrl+f}} some variables, click on ‘View page source’ on the top right of this window. If you want to know what a greek variable is but don’t know what it’s called, you can view the source text on the file where you found the variable. nu, mu, eta, who actually remembers what these all look like? The letter ‘v’ should sue ‘nu’ for copyright infringement. Or is it the other way around?


\begin{savenotes}\sphinxatlongtablestart\begin{longtable}{|\X{10}{70}|\X{15}{70}|\X{45}{70}|}
\caption{Parameter Guide\strut}\label{\detokenize{Textbook_Creation_Help/parameter_convention_list:id2}}\label{\detokenize{Textbook_Creation_Help/parameter_convention_list:table-parameter-table}}\\*[\sphinxlongtablecapskipadjust]
\hline
\sphinxstyletheadfamily 
Parameter
&\sphinxstyletheadfamily 
Description
&\sphinxstyletheadfamily 
Units
\\
\hline
\endfirsthead

\multicolumn{3}{c}%
{\makebox[0pt]{\sphinxtablecontinued{\tablename\ \thetable{} -- continued from previous page}}}\\
\hline
\sphinxstyletheadfamily 
Parameter
&\sphinxstyletheadfamily 
Description
&\sphinxstyletheadfamily 
Units
\\
\hline
\endhead

\hline
\multicolumn{3}{r}{\makebox[0pt][r]{\sphinxtablecontinued{Continued on next page}}}\\
\endfoot

\endlastfoot

\(m\)
&
Mass
&
\([M]\)
\\
\hline
\(z\)
&
Elevation
&
\([L]\)
\\
\hline
\(L\)
&
Length
&
\([L]\)
\\
\hline
\(W\)
&
Width
&
\([L]\)
\\
\hline
\(H\)
&
Height
&
\([L]\)
\\
\hline
\(D\)
&
Diameter
&
\([L]\)
\\
\hline
\(r\)
&
Radius
&
\([L]\)
\\
\hline
\(A\)
&
Area
&
\([L]^2\)
\\
\hline
\(\rlap{-} V\)
&
Volume
&
\([L]^3\)
\\
\hline
\(v\)
&
Velocity
&
\(\frac{[L]}{[T]}\)
\\
\hline
\(Q\)
&
Flow rate
&
\(\frac{[L]^3}{[T]}\)
\\
\hline
\(n\)
&
Number, Amount
&
Dimensionless
\\
\hline
\(C\)
&
Concentration
&
\(\frac{[M]}{[L]^3}\)
\\
\hline
\(p\)
&
Pressure
&
\(\frac{[M]}{[L][T]^2}\)
\\
\hline
\(g\)
&
Acceleration due to Gravity
&
\(\frac{[L]}{[T]^2}\)
\\
\hline
\(\rho\)
&
Density
&
\(\frac{[M]}{[L]^3}\)
\\
\hline
\(\mu\)
&
Dynamic viscosity
&
\(\frac{[M]}{[T][L]}\)
\\
\hline
\(\nu\)
&
Kinematic viscosity
&
\(\frac{[L]^2}{[T]}\)
\\
\hline
\(h\)
&
Head, Elevation
&
\([L]\)
\\
\hline
\(h_L\)
&
Headloss
&
\([L]\)
\\
\hline
\(h_{\rm f}\)
&
Major Loss (friction)
&
\([L]\)
\\
\hline
\(\epsilon\)
&
Surface roughness
&
\([L]\)
\\
\hline
\(\rm{f}\)
&
Darcy-Weisbach friction factor
&
Dimensionless
\\
\hline
\({\rm Re}\)
&
Reynolds Number
&
Dimensionless
\\
\hline
\(h_e\)
&
Minor Loss (expansion)
&
\([L]\)
\\
\hline
\(K\)
&
Minor Loss coefficient
&
Dimensionless
\\
\hline
\(\Pi\)
&
Dimensionless Proportionality Ratio
&
Dimensionless
\\
\hline
\(\Pi_{vc}\)
&
Vena Contracta Area Ratio
&
Dimensionless
\\
\hline
\(\Pi_{Error}\)
&
Linearity Error Ratio
&
Dimensionless
\\
\hline
\(M\)
&
Fluid Momentum
&
\(\frac{[M][L]}{[T]^2}\)
\\
\hline
\(F\)
&
Force
&
\(\frac{[M][L]}{[T]^2}\)
\\
\hline
\(t\)
&
Time
&
\([T]\)
\\
\hline
\(\theta\)
&
Residence Time
&
\([T]\)
\\
\hline
\(G\)
&
Velocity Gradient/Fluid Deformation
&
\(\frac{1}{[T]}\)
\\
\hline
\(\varepsilon\)
&
Energy Dissipation Rate
&
\(\frac{[L]^2}{[T]^3}\)
\\
\hline
\(\Pi_{\bar G}^{G_{Max}}\)
&
\(\frac{G_{Max}}{\bar G}\) Ratio in a Reactor
&
Dimensionless
\\
\hline
\(\Pi_{\bar \varepsilon}^{\varepsilon_{Max}}\)
&
\(\frac{\varepsilon_{Max}}{\bar \varepsilon}\) Ratio in a Reactor
&
Dimensionless
\\
\hline
\(\Pi_{HS}\)
&
Height to Baffle Spacing in a Flocculator
&
Dimensionless
\\
\hline
\(H_e\)
&
Height Between Flow Expansions in a Flocculator
&
\([L]\)
\\
\hline
\(S\)
&
Spacing Between Two Objects
&
\([L]\)
\\
\hline
\(B\)
&
Center-to-Center Spacing Between Two Objects
&
\([L]\)
\\
\hline
\(T\)
&
Object Thickness
&
\([L]\)
\\
\hline
\(P\)
&
Power
&
\(\frac{[M][L]^2}{[T]^3}\)
\\
\hline
\(\eta_K\)
&
Kolmogorov Length Scale
&
\([L]\)
\\
\hline
\(\lambda_\nu\)
&
Inner Viscous Length Scale
&
\([L]\)
\\
\hline
\(\Pi_{K\nu}\)
&
Ratio of Inner Viscous Length Scale to Kolmogorov Length Scale
&
Dimensionless
\\
\hline
\(\Lambda\)
&
Distance Between Particles
&
\([L]\)
\\
\hline
\end{longtable}\sphinxatlongtableend\end{savenotes}


\chapter{Introduction to AguaClara Water Treatment Design}
\label{\detokenize{Introduction/Introduction:introduction-to-aguaclara-water-treatment-design}}\label{\detokenize{Introduction/Introduction:title-introduction-to-aguaclara-water-treatment-design}}\label{\detokenize{Introduction/Introduction::doc}}

\section{A Different Kind of Textbook}
\label{\detokenize{Introduction/Introduction:a-different-kind-of-textbook}}\label{\detokenize{Introduction/Introduction:heading-a-different-kind-of-textbook}}
This textbook represents our cumulative insights from our journey that has been motivated by a quest to make the world a better place where everyone has access to safe water on tap, the engineering challenge of optimizing the design of drinking water treatment plants, and the curiosity to understand what controls their performance. We would like to understand what determines which contaminants make it the whole way through a water treatment plant. If we could understand what allows some contaminants to sneak the whole way through a water treatment plant, then we suspect that we could create better designs to more effectively remove contaminants.

Engineering textbooks provide a venue for authors to share what they’ve learned, to organize ideas, and to provide a guide for engineers as they design solutions for real world problems. Engineering textbooks are often intended to document the established core of knowledge. It seems reasonable to assume that what is in textbooks and in peer reviewed literature is mostly true.


\subsection{The edge of knowledge may be closer than we thought}
\label{\detokenize{Introduction/Introduction:the-edge-of-knowledge-may-be-closer-than-we-thought}}\label{\detokenize{Introduction/Introduction:heading-edge-of-knowledge}}
The assumption that what is written and passed down in oral history through the scientific community is true can lead to missed opportunities and lost insights. The hypotheses from one generation of scientists can too easily evolve into new theories in the next generation and then into established theories for the next. The history of drinking water treatment science is cloudy (think high turbidity!) with hypotheses that miss or misrepresent key concepts.

You might wonder why we care so much about getting the science right and being as clear as possible about what is known. After all, the core drinking water treatment technologies were invented before we were born and many of us have safe drinking water coming from our taps. Environmental Engineers have known how to design municipal drinking water treatment plants since they early 1900’s. We care about getting the science right because we hypothesize that there are many opportunities to significantly improve drinking water treatment technologies and that improved understandings of each unit process have the potential to lead to new breakthroughs.

Our contention is that no one has ever optimized the design of a drinking water treatment plant! We are reasonably certain of this because we don’t yet have models (with equations) that describe performance of most of the core unit processes (rapid mix, flocculation, floc blankets, sedimentation, sand filtration) used for surface water treatment. The only possible exception is lamellar sedimentation which can be characterized if we know the size and density distribution of the particles entering the sedimentation tank.

Traditional drinking water treatment textbooks can too easily miss the opportunity to advance the science of drinking water treatment technologies by presenting certainty where there should be skepticism. For example, rapid mix is described as process that occurs in a few seconds, flocculation is described as a process that should be fastest for high turbidity waters and slowest for low turbidity waters, and filtration performance is described by a model that predicts first order removal with respect to filter bed depth. We will demonstrate why each of these assumptions doesn’t match observations, we will discuss new insights into these processes, and we will identify high priority research questions that have the potential to lead to major improvements in drinking water treatment.

We want to encourage skepticism and we want to develop insights to guide thoughtful skepticism. A key skill for successful engineering is the ability to identify the location of the edge of knowledge. The ability to distinguish between what is reasonably certain and what is still in question is what powers the scientific method of slowly extending knowledge. New insights are difficult to obtain if the research  is based on a faulty premise.

\begin{figure}[htbp]
\centering
\capstart

\noindent\sphinxincludegraphics[width=500\sphinxpxdimen]{{Short_walk_to_the_edge}.jpg}
\caption{We’ve learned that we can find the edge of knowledge very soon after we begin researching a water treatment technology (artwork created by Yi Wen Ng in 2012).}\label{\detokenize{Introduction/Introduction:id12}}\label{\detokenize{Introduction/Introduction:figure-short-walk}}\end{figure}

There are significant knowledge gaps in every process that we cover in this textbook. We aren’t yet able to optimize surface water treatment processes because we don’t yet understand the fundamental physics of many of the processes. We are getting closer, join us on the journey.

We need the brightest and the best to create new and better solutions so we can meet the goal of providing everyone with safe drinking water. This challenge is apparently more difficult than building a space station, designing a fuel cell, or inventing the world wide web. So let’s role up our sleeves and begin.


\subsection{Tools to Find the Edge of Knowledge}
\label{\detokenize{Introduction/Introduction:tools-to-find-the-edge-of-knowledge}}\label{\detokenize{Introduction/Introduction:heading-tools-to-find-the-edge-of-knowledge}}\begin{quote}
\begin{itemize}
\item {} 
Don’t believe everything we say

\item {} 
Ask lots of questions

\end{itemize}
\begin{itemize}
\item {} 
How do you know that? The goal here is to identify the difference between what is known and what is hypothesized.

\item {} 
What is the equation that describes the physics of this process? If there isn’t an equation that describes the process and that can be used to design the reactor for the process, then it is likely that the physics of the process is not yet understood.

\item {} 
How could we improve this process? If the physics of a process are fully understood, then dimensionally correct equations can be used to obtain the optimal design for that process.

\item {} 
Is the process design based on “rules of thumb” or on physics? “Rules of thumb” or empirical design guidelines often can be identified by the use of physical parameters that have units. For example, if the design guideline specifies a length, time, or velocity then it is likely that the guideline is not based on physics. If the design guidelines are based on a dimensionless parameter then it is possible that it is based on physics.

\end{itemize}
\begin{itemize}
\item {} 
Evaluate the data to see if it matches predictions of the hypothesized model. Assess whether the authors acknowledge when their data doesn’t match hypothesized models.

\item {} 
Beware of the use of words that are poorly defined and that hide uncertainty. For example, creating a name for a supposed mechanism to describe all of the observations that don’t fit with your theory does NOT mean that you understand that mechanism. The ability to name something doesn’t mean it is understood.

\item {} 
Does this “theory” provide insights that have led to new discoveries or new applications?

\item {} 
Does the “theory” include equations that are based on the fundamental laws of nature?

\item {} 
Does the “theory” use dimensionless constants that are close to one?

\item {} 
Is it an elegant “theory” with no need for special cases?

\end{itemize}
\end{quote}


\subsection{Myth in Environmental Engineering}
\label{\detokenize{Introduction/Introduction:myth-in-environmental-engineering}}\label{\detokenize{Introduction/Introduction:heading-myth-in-environmental-engineering}}\begin{description}
\item[{The following list is designed to get you thinking. These are concepts that are present in the Environmental Engineering community and that may capture some elements of truth and that may also further misconceptions.}] \leavevmode\begin{itemize}
\item {} 
Dead bodies cause disease

\item {} 
Slow sand filters ripen (improve in ability to remove contaminants over time) because of biological growth in the filter bed

\item {} 
If a 20 cm deep sand filter removes 90\% of influent particles, then a 40 cm deep filter will remove 99\% of influent particles

\item {} 
If water is dirty, then you should filter it

\item {} 
Chlorine disinfects dirty water and makes it safe to drink

\item {} 
Chlorination and filtration eliminated typhoid fever from the US

\item {} 
Cessation of chlorination due to fear of disinfection by products caused the cholera outbreak in Peru in 1993

\item {} 
Sedimentation is simple

\item {} 
We already know how to solve the problem of the billions of people who do not having access to safe drinking water

\end{itemize}

\end{description}


\subsection{Uncertainty in Science and Engineering}
\label{\detokenize{Introduction/Introduction:uncertainty-in-science-and-engineering}}\label{\detokenize{Introduction/Introduction:heading-uncertainty-in-science-and-engineering}}
A challenge for authors is to recognize the difference between what is known with a reasonably high degree of certainty and what is assumed to be true without a solid basis. We struggle to tell the difference between fact and hypothesis. The time-honored approach in science is to rely on the peer review process. That process for vetting knowledge has been shown to be flawed.

Your question could be whether the distinction between fact and hypothesis really matters. If the hypothesis is widely accepted as fact and if it has been accepted for decades what benefit is there to calling it a hypothesis rather than a fact?

This question is at the core of our educational philosophy. Is this text the repository of knowledge that we are providing for you to drink or is this text a conversation where we invite you to join the effort to discover better ways to provide safe water on tap?


\subsection{Integrating Educational Philosophy with an Evolving Textbook}
\label{\detokenize{Introduction/Introduction:integrating-educational-philosophy-with-an-evolving-textbook}}\label{\detokenize{Introduction/Introduction:heading-textbook-philosophy}}
This is an evolving textbook. We don’t intend to ever print this book. This book has version numbers just like software with the idea that revisions are rapid and frequent. We commit to helping to accelerate the pace of knowledge generation and to revising this text as you help us identify places where we have presented hypotheses as theory and places where research provides a basis for better theoretical models of the water treatment processes.

Our students are co-creators of knowledge and not empty vessels to be filled with our wisdom. AguaClara technologies are inventions that are the result of idea collisions in the AguaClara labs and from observations and reflections with operators, technicians, and engineers in dozens of water treatment plants. Although we’ve learned a great deal about water treatment since 2005 when AguaClara was founded, there is still much more to be learned. And so it is with a spirit of curiosity that we write this textbook expecting to learn even more in the coming years.

Socrates said “\sphinxhref{https://www.goodreads.com/quotes/69267-education-is-the-kindling-of-a-flame-not-the-filling}{Education is the kindling of a flame, not the filling of a vessel}.” Our goal is to bring the spirit of play, discovery, and mystery into the challenge of improving the quality of life of everyone on the planet by sharing better methods to produce safe drinking water.

In We Make the Road by Walking: Conversations on Education and Social Change, Paulo Freire said,
“\sphinxhref{https://www.goodreads.com/author/quotes/41108.Paulo\_Freire?page=2}{The more we become able to become a child again, to keep ourselves childlike, the more we can understand…}”. We commit to playing together in a relationship where we are all learning and we are all teaching. “Education must begin with the solution of the teacher-student contradiction, by reconciling the poles of the contradiction so that both are simultaneously teachers and students.” - Paulo Freire


\subsection{Respect, Empathy, Love and Curiosity power the AguaClara Innovation System}
\label{\detokenize{Introduction/Introduction:respect-empathy-love-and-curiosity-power-the-aguaclara-innovation-system}}\label{\detokenize{Introduction/Introduction:heading-empathy}}
The AguaClara network of organizations has been methodically inventing improved water treatment technologies since 2005. Our success is based on respect, empathy and love. Innovation requires flocculation of ideas. The transport of ideas between organizations and individuals is mediated by respect. Respect as a cornerstone of organizational culture foster rapid and honest exchange of ideas. The rapid pace of innovation in the AguaClara network is sustained thru a shared culture of respect, empathy, and love.

Curiosity can flourish in a culture of love, respect, and empathy. Asking why and why not and pondering an ever growing number of questions has empowered student teams to take on the quest for new knowledge and new solutions.
\begin{description}
\item[{Any large organization will require a leadership hierarchy and any hierarchy will rely on respect based on fear or respect based on love. \sphinxhref{https://www.forbes.com/sites/lizryan/2015/11/25/the-five-characteristics-of-fear-based-leaders/\#a6179f38a968}{Fear-based hierarchies} impede the accurate sharing of information and can easily devolve into data-free and low-truth decision-making schemes. According to \sphinxhref{https://www.forbes.com/sites/lizryan/2015/11/25/the-five-characteristics-of-fear-based-leaders/\#a6179f38a968}{Liz Ryan}, the characteristics of fear-based leaders include:}] \leavevmode\begin{itemize}
\item {} 
They’ll Teach You, Whether You Like It or Not

\item {} 
Everyone is a Friend or a Foe

\item {} 
It’s All about the Trophies

\item {} 
They Don’t Step Outside Boxes

\item {} 
They’re Addicted to Yardsticks

\end{itemize}

\end{description}

Love-based hierarchies foster honesty and a free-flow of information. Reflection is encouraged across the organization and truth, honesty, and integrity are valued. Staff at the bottom of the hierarchy know that their opinions and reflections are valued and thus they will be willing to report problems to organization leaders and share their ideas.

Love-based leaders relate to others based on true respect for the other. They will take the time to converse with people at all levels of the organization and will value the opportunity to speak with people who are the interface between the organization and the rest of the world. A person’s value is based on being a person, not based on position in the hierarchy.

As water treatment plant designers it is critical that we spend time with a diverse set of stakeholders including community members and water treatment plant operators. Those relationships must begin with respect and valuing their insights. As we spend time together we can develop trust so that they communicate both the good and bad.

We’ve learned much from plant operators. They figured out how to reduce rising flocs at Agalteca, Honduras where we learned that conventional sedimentation tank inlet manifolds generate large circulation currents. Plant operators added curtains to the windows at Moroceli, Honduras (see figure\_Moroceli\_curtains) because they noticed that direct sunlight on the sedimentation tanks caused an increase in settled water turbidity.

\begin{figure}[htbp]
\centering
\capstart

\noindent\sphinxincludegraphics[width=500\sphinxpxdimen]{{Moroceli_curtains}.jpg}
\caption{Moroceli AguaClara water treatment plant operators installed curtains to reduce direct sunshine on sedimentation tanks. Solar heating produces density currents that carry flocs to the top of the sedimentation tank.}\label{\detokenize{Introduction/Introduction:id13}}\label{\detokenize{Introduction/Introduction:figure-moroceli-curtains}}\end{figure}

Empathy is fundamental in design. Empathy enables us to consider reality from another’s perspective. Empathy enables us to bring the people who will use or benefit from a technology into the design considerations. Empathy brings the insight that water treatment plants need to have roofs and provide a secure work environment both day and night. Empathy brings the insight that replacement parts must be readily available and that generic components are preferred over specialty proprietary components.


\section{The Global Context for Drinking Water Treatment}
\label{\detokenize{Introduction/Introduction:the-global-context-for-drinking-water-treatment}}\label{\detokenize{Introduction/Introduction:heading-the-global-context-for-drinking-water-treatment}}
The \sphinxhref{https://www.un.org/sustainabledevelopment/sustainable-development-goals/}{Sustainable Development Goals: SDGs} and specifically \sphinxhref{https://www.un.org/sustainabledevelopment/water-and-sanitation/}{SDG 6} provide the context and motivation for this text. The first SDG 6 target is: “By 2030, achieve universal and equitable access to safe and affordable drinking water for all.” That goal is daunting and won’t be met using the approaches of the past 5 decades. This text is about creating a new paradigm for the design of high performing water treatment technologies with the goal of making a real contribution toward SDG 6.1.

\begin{figure}[htbp]
\centering
\capstart

\noindent\sphinxincludegraphics[width=100\sphinxpxdimen]{{SDG6}.png}
\caption{Sustainable development goal 6 is all about clean water and sanitation.}\label{\detokenize{Introduction/Introduction:id14}}\label{\detokenize{Introduction/Introduction:figure-sdg6}}\end{figure}

The number of people who currently lack access to reliable safe water on tap is not known. Estimates range from “\sphinxhref{https://www.un.org/sustainabledevelopment/water-and-sanitation/}{1.8 billion who use a source of drinking water that is contaminated with feces}” to the Centers for Disease Control recommendations for where it is \sphinxhref{https://lifehacker.com/know-what-countries-guarantee-drinkable-tap-water-with-1635070463}{usually safe to drink tap water}.

\begin{figure}[htbp]
\centering
\capstart

\noindent\sphinxincludegraphics[width=600\sphinxpxdimen]{{CDC_Global_Safe_Tap_Water}.png}
\caption{There are relatively few countries where it is almost always safe to drink the tap water.}\label{\detokenize{Introduction/Introduction:id15}}\label{\detokenize{Introduction/Introduction:figure-cdc-global-safe-tap-water}}\end{figure}

The \sphinxhref{https://www.un.org/sustainabledevelopment/blog/2017/07/billions-around-the-world-lack-safe-water-proper-sanitation-facilities-reveals-un-report/}{UN estimate in 2017} was that 2.1 billion lack access to safe water. By 2030 there will be an additional \sphinxhref{https://news.un.org/en/story/2015/07/505352-un-projects-world-population-reach-85-billion-2030-driven-growth-developing}{1.2 billion from population growth}.

\begin{figure}[htbp]
\centering
\capstart

\noindent\sphinxincludegraphics[width=400\sphinxpxdimen]{{Population_Infographic_01}.jpg}
\caption{1.2 billion people will be added to the global population between 2015 and 2030.}\label{\detokenize{Introduction/Introduction:id16}}\label{\detokenize{Introduction/Introduction:figure-population-infographic-01}}\end{figure}

Thus by 2030 we need to provide safe water for at least 3.3 billion people AND maintain the water supply systems for the 5.2 billion who currently have access to safe water. That is a daunting number that requires some exploration!

\fvset{hllines={, ,}}%
\begin{sphinxVerbatim}[commandchars=\\\{\}]
\PYG{k+kn}{from} \PYG{n+nn}{aide\PYGZus{}design}\PYG{n+nn}{.}\PYG{n+nn}{play} \PYG{k}{import}\PYG{o}{*}
\PYG{k+kn}{import} \PYG{n+nn}{datetime}
\PYG{n}{People\PYGZus{}needing\PYGZus{}water\PYGZus{}2030} \PYG{o}{=} \PYG{l+m+mf}{3.3}\PYG{o}{*}\PYG{l+m+mi}{10}\PYG{o}{*}\PYG{o}{*}\PYG{l+m+mi}{9}
\PYG{n}{now} \PYG{o}{=} \PYG{n}{datetime}\PYG{o}{.}\PYG{n}{datetime}\PYG{o}{.}\PYG{n}{now}\PYG{p}{(}\PYG{p}{)}
\PYG{n}{Task\PYGZus{}time} \PYG{o}{=} \PYG{p}{(}\PYG{l+m+mi}{2030} \PYG{o}{\PYGZhy{}} \PYG{n}{now}\PYG{o}{.}\PYG{n}{year}\PYG{p}{)}\PYG{o}{*}\PYG{n}{u}\PYG{o}{.}\PYG{n}{year}
\PYG{c+c1}{\PYGZsh{}If we assume we will meet this demand by building the same amount of new capacity each year, then we have}
\PYG{n}{People\PYGZus{}per\PYGZus{}year} \PYG{o}{=} \PYG{n}{People\PYGZus{}needing\PYGZus{}water\PYGZus{}2030}\PYG{o}{/}\PYG{n}{Task\PYGZus{}time}
\PYG{n}{People\PYGZus{}per\PYGZus{}year}
\PYG{c+c1}{\PYGZsh{}The percapita demand for water}
\PYG{n}{Per\PYGZus{}capita\PYGZus{}demand} \PYG{o}{=} \PYG{l+m+mi}{3}\PYG{o}{*}\PYG{n}{u}\PYG{o}{.}\PYG{n}{mL}\PYG{o}{/}\PYG{n}{u}\PYG{o}{.}\PYG{n}{s}
\PYG{n}{Per\PYGZus{}capita\PYGZus{}demand}\PYG{o}{.}\PYG{n}{to}\PYG{p}{(}\PYG{n}{u}\PYG{o}{.}\PYG{n}{L}\PYG{o}{/}\PYG{n}{u}\PYG{o}{.}\PYG{n}{day}\PYG{p}{)}
\PYG{n}{Per\PYGZus{}capita\PYGZus{}demand}
\PYG{n}{Rate\PYGZus{}new\PYGZus{}water\PYGZus{}supply\PYGZus{}capacity} \PYG{o}{=} \PYG{p}{(}\PYG{n}{People\PYGZus{}per\PYGZus{}year} \PYG{o}{*} \PYG{n}{Per\PYGZus{}capita\PYGZus{}demand}\PYG{p}{)}\PYG{o}{.}\PYG{n}{to}\PYG{p}{(}\PYG{n}{u}\PYG{o}{.}\PYG{n}{L}\PYG{o}{/}\PYG{p}{(}\PYG{n}{u}\PYG{o}{.}\PYG{n}{s}\PYG{o}{*}\PYG{n}{u}\PYG{o}{.}\PYG{n}{year}\PYG{p}{)}\PYG{p}{)}
\PYG{n}{Rate\PYGZus{}new\PYGZus{}water\PYGZus{}supply\PYGZus{}capacity}
\PYG{n}{NYC\PYGZus{}water\PYGZus{}supply} \PYG{o}{=} \PYG{l+m+mi}{44000} \PYG{o}{*} \PYG{n}{u}\PYG{o}{.}\PYG{n}{L}\PYG{o}{/}\PYG{n}{u}\PYG{o}{.}\PYG{n}{s}
\PYG{n}{NYC\PYGZus{}per\PYGZus{}year} \PYG{o}{=} \PYG{n}{Rate\PYGZus{}new\PYGZus{}water\PYGZus{}supply\PYGZus{}capacity}\PYG{o}{/}\PYG{n}{NYC\PYGZus{}water\PYGZus{}supply}
\PYG{n}{NYC\PYGZus{}per\PYGZus{}year}
\end{sphinxVerbatim}

If we provide 260 L/day per person, then we need to provide the equivalent of 19 water supplies for New York City every year between now and 2030. The planet needs approximately 800,000 L/s of new capacity each year. AguaClara water treatment plants cost approximately \$10,000 per L/s of treatment capacity. Thus the budget for global water treatment needs to be 8 billion USD per year. Note that this doesn’t include any other aspects of supplying water. Managing water sources, transmission lines, storage, and distribution systems are even more expensive than water treatment.

The need for drinking water supplies isn’t limited to the global south. The California Urban Water Agencies \sphinxhref{https://static1.squarespace.com/static/5a565e93b07869c78112e2e5/t/5a5965934192024b3f610be1/1515808194305/CUWA2017\_AnnualReport.pdf}{estimate that 530,000 or more people in rural areas of California are unable to turn on their tap and access clean, safe water}.


\subsection{Why don’t 2 billion people have access to safe water?}
\label{\detokenize{Introduction/Introduction:why-don-t-2-billion-people-have-access-to-safe-water}}\label{\detokenize{Introduction/Introduction:heading-2-billion-without-access-to-safe-water}}
The simple answer is that they are too poor and are unable to afford safe water on tap. But it isn’t that simple! Families without access to safe water on tap often spend more for water than families with safe water on tap. There seem to be two key reasons why those with limited financial resources often have limited access to water, poor quality water, and yet pay a premium for that water.

The first reason for the lack of safe water has been the poor track record of water treatment infrastructure. The frequent failures and high operating costs of municipal scale water treatment systems have led many decision makers to conclude water treatment infrastructure isn’t a worthwhile investment. Politicians who invest political capital to bring water treatment to their community often find that after the initial ribbon cutting there is little political benefit because the system doesn’t deliver the benefits to the community that they had promised.

The second reason for the lack of safe water is the lack of access to capital for municipal scale infrastructure. Even though an AguaClara water treatment plant would pay for itself in a fraction of its useful life, there is not yet a financial mechanisms for communities to access a loan so that they can make the investment. A community would need to save enough money to be able to purchase a water treatment plant (as was the case for Las Vegas, Honduras), a bilateral donor can finance a plant through a donation, or the national government can use sovereign debt or taxes to finance plants. The challenge for a community is to obtain the financial or political power to access the needed funds.

As we work to solve a global challenge that has been plaguing humanity since the dawn of human civilization, then it will serve us well to understand a bit of the history that has led to our current reality. Water treatment history includes amazing successes, persistent failures, fortuitous discoveries, a heavy reliance on empiricism, and an occasional myth. Our goal is learn from and reflect on our history and then create even better solutions.


\section{Introduction to Surface Water Treatment}
\label{\detokenize{Introduction/Introduction:introduction-to-surface-water-treatment}}\label{\detokenize{Introduction/Introduction:heading-introduction-to-surface-water-treatment}}
We treat water because it doesn’t meet the requirements for its intended use. We need to understand the problem so that we can understand existing and novel water treatment technologies.


\subsection{Water Contaminants}
\label{\detokenize{Introduction/Introduction:water-contaminants}}\label{\detokenize{Introduction/Introduction:heading-water-contaminants}}
Many substances are able to dissolve in water and with it’s high density, water is able to carry suspended solids. The substances may be naturally occurring, anthropogenic, benign, or harmful. The types of contaminants are influenced by the water source. Contaminant concentrations are often highly variable over time.

A water treatment system must be able to handle the likely range of contaminant levels and produce treated water that meets the user requirements. In some cases the user may have the option of switching sources or reducing demand when a source becomes excessively contaminated for a limited period of time. For example, a municipal water supplier may be able to shut the plant down for a few hours to avoid having to treat a very dirty water after a rainstorm. This strategy can work well for water sources that have small watersheds and hence a rapid return to better water after the storm passes. In other cases the water treatment processes must be capable of treating the most contaminated water that the water source provides. In any case, selecting the best unit processes to treat a given water source for a particular use case can be challenging. It is common to find water treatment plants that are unable to adequately treat their water source.


\subsubsection{Particles}
\label{\detokenize{Introduction/Introduction:particles}}\label{\detokenize{Introduction/Introduction:heading-particles}}
Surface waters (rivers, streams, lakes) and some ground water (especially ground water under the influence of surface water) inevitable carry some suspended particles. “\sphinxhref{https://www.sciencedirect.com/science/article/pii/S0048969708010103}{Particles transported by rivers are composed of resistant primary minerals (e.g., quartz and zircon), secondary minerals (clays, metallic oxides and oxyhydroxides) and biogenic remains.”} Many of these particles may be harmless, but there is good reason to be hesitant to drink water with a high concentration of suspended particles.


\subsubsection{Pathogens}
\label{\detokenize{Introduction/Introduction:pathogens}}\label{\detokenize{Introduction/Introduction:heading-pathogens}}
Pathogens include viruses (100 nm), bacteria (1 \(\mu m\)), and protozoa (several \(\mu m\)). Pathogens are particles and are removed by processes that remove particles along with other microbes, organic and inorganic particles.


\subsubsection{Turbidity}
\label{\detokenize{Introduction/Introduction:turbidity}}\label{\detokenize{Introduction/Introduction:heading-turbidity}}
Turbidity or cloudiness is an indirect measure of particle concentration. Turbidity is an optical measurement of scattered light. Light scattering by refraction is primarily caused by particles that are smaller than but close to the wavelength of light. Particles that are close larger than the wavelength of light can reflect light. Turbidity measures both of these effects by shining a light into a water sample and then measuring the scattered light with a photodetector at 90°. The meter is then calibrated with standard suspensions.

For a given suspension the turbidity can be directly correlated with the suspended solids concentration. However, that relationship is complicated because the amount of scattered light is related to the particle size distribution because given the same mass concentration, smaller particles have more surface area and thus reflect more light.

Although turbidity would seem to be an odd parameter to use to measure water quality, it turns out to be the most widely used water quality measurement. The reasons are simple. First, turbidity is amazingly easy to measure over a very wide range of particle concentrations (perhaps 10 \(\mu g/L\) to 1 \(g/L\)). The test doesn’t require any reagents and it can be done in a flow through sample cell for real time measurements. Second, particle free water is pathogen free water. Third, disinfection processes (chlorination, ozonation, UV light) are all significantly less effective at inactivating pathogens if there are other particles present in the water.


\subsubsection{Dissolved Species}
\label{\detokenize{Introduction/Introduction:dissolved-species}}\label{\detokenize{Introduction/Introduction:heading-dissolved-species}}
The list of dissolved species that can be present in water in the environment is endless and ranges from natural organic matter (from decay of plants) to caffeine to atrazine. Usually the highest concentration class of molecules is dissolved natural organic matter (NOM). NOM has some similarity to inorganic particles in that it isn’t necessarily harmful and yet there are several reasons why removal of NOM is an important water treatment goal.

From an aesthetic perspective, NOM absorbs light at short wavelengths and this results in water that looks yellow or brown. While I enjoy drinking tea with a rich brown color, I’d prefer that my water be clear.
\begin{description}
\item[{NOM plays a supersized role in influencing performance of surface water treatment plants. NOM has three negative effects:}] \leavevmode
1) It requires higher dosages of coagulant for effective particle removal.
1) It reduces the disinfection effectiveness of chlorine, ozone, and UV. Chlorine partially oxidizes the NOM and thus more chlorine must be used to maintain a residual level of chlorine.
1) It can produce disinfection by-products that are toxic.

\end{description}

Thus removal of NOM is a water treatment goal. Fortunately the same coagulants that are used for particle removal also can remove a significant fraction of NOM. The interactions between NOM and coagulants will be discussed in the {\hyperref[\detokenize{Rapid_Mix/RM_Intro:title-rapid-mix-introduction}]{\sphinxcrossref{\DUrole{std,std-ref}{Introduction to Rapid Mix}}}}.

The removal of other dissolved species is beyond the scope of the first release of this textbook. The authors intend to add sections on the removal of some dissolved species in the near future.


\subsection{Chlorine (Might Have) Saved the World}
\label{\detokenize{Introduction/Introduction:chlorine-might-have-saved-the-world}}\label{\detokenize{Introduction/Introduction:heading-chlorine-saved-the-world}}
Chlorine is widely recognized for reducing mortality from water borne disease in the United States. A more careful review of the mortality data and of the ability of chlorine to inactive various pathogens makes it difficult to assess the role of chlorine. A classic graph (see \hyperref[\detokenize{Introduction/Introduction:figure-us-death-rate}]{Fig.\@ \ref{\detokenize{Introduction/Introduction:figure-us-death-rate}}}) has been used to suggest that chlorination of drinking water supplies resulted in a significant reduction in mortality

\begin{figure}[htbp]
\centering
\capstart

\noindent\sphinxincludegraphics[width=500\sphinxpxdimen]{{US_infectious_diseases_death_rate}.jpg}
\caption{\sphinxhref{https://www.cdc.gov/mmwr/preview/mmwrhtml/mm4829a1.htm}{Classic graph showing the reduction in the death rate for the United States from 1900 to 1996.}}\label{\detokenize{Introduction/Introduction:id17}}\label{\detokenize{Introduction/Introduction:figure-us-death-rate}}\end{figure}


\begin{savenotes}\sphinxattablestart
\centering
\sphinxcapstartof{table}
\sphinxcaption{Surface Water Treatment Technologies}\label{\detokenize{Introduction/Introduction:id18}}\label{\detokenize{Introduction/Introduction:table-surface-water-treatment-technologies}}
\sphinxaftercaption
\begin{tabular}[t]{|*{5}{\X{1}{5}|}}
\hline
\sphinxstyletheadfamily 
Technology
&\sphinxstyletheadfamily 
Description
&\sphinxstyletheadfamily 
Prerequisite
&\sphinxstyletheadfamily 
Owner
&\sphinxstyletheadfamily 
Year
\\
\hline
Simple sedimentation
&
particles settle
&
none
&
public
&
unknown
\\
\hline
Flocculation
&
aluminum and iron salts
&
none
&
public
&
\sphinxhref{https://www.iwapublishing.com/news/coagulation-and-flocculation-water-and-wastewater-treatment}{1757}
\\
\hline
Sedimentation
&
horizontal flow
&
flocculation
&
public
&
unknown
\\
\hline
Lamellar sedimentation
&
plate or tube settlers
&
flocculation or floc blanket
&
public
&
\sphinxhref{http://www.hydroflotech.com/inclined-plate-clarifier-basic-theory-of-operation}{1904}
\\
\hline
Roughing filter
&
simple sedimentation in a gravel bed
&
none
&
public
&
\sphinxhref{https://www.researchgate.net/publication/237827490\_Roughing\_filter\_for\_water\_pre-treatment\_technology\_in\_developing\_countries\_A\_review?enrichId=rgreq-bb1d04e6613378d626c78cadb6674ae8\&enrichSource=Y292ZXJQYWdlOzIzNzgyNzQ5MDtBUzoyMDAwMDczMDQxMjY0NjdAMTQyNDY5Njg2NTYxMQ\%3D\%3D\&el=1\_x\_2}{unknown}
\\
\hline
Slow sand filtration
&
Roughing filter or single step treatment for low NTU water
&
none
&
public
&
\sphinxhref{https://en.wikipedia.org/wiki/Slow\_sand\_filter}{1829}
\\
\hline
Rapid sand filtration
&
depth filtration
&
sedimentation
&
public
&
\sphinxhref{https://en.wikipedia.org/wiki/Rapid\_sand\_filter}{1920}
\\
\hline
Stacked rapid sand filter
&
gravity powered backwash
&
lamellar sedimentation
&
AguaClara Cornell open source
&
\sphinxhref{https://ascelibrary.org/doi/abs/10.1061/\%28ASCE\%29EE.1943-7870.0000562}{2012}
\\
\hline
Floc blanket
&
upflow fluidized suspension of flocs
&
flocculation
&
public
&
\sphinxhref{https://link.springer.com/chapter/10.1007\%2F978-3-642-61196-4\_2}{1930}
\\
\hline
Jet reverser floc blanket
&
first fully fluidized floc blanket
&
flocculation
&
AguaClara Cornell open source
&
\sphinxhref{http://cuaguaclara.blogspot.com/2012/08/the-floc-blanket-quest.html}{2012}
\\
\hline
Ballasted sedimentation
&
small sand carry particles downward
&\begin{itemize}
\item {} 
\end{itemize}
&
\sphinxhref{http://www.veoliawatertechnologies.com.au/medias/topics/focus\_actiflo.htm}{Actiflo Veolia}
&
\sphinxhref{https://patents.google.com/patent/US5840195}{1995}
\\
\hline
Superpulsator
&
pulsing flow through floc blanket
&
rapid mix
&
\sphinxhref{http://www.degremont-technologies.com/SUPERPULSATOR-R}{Degremont}
&
\sphinxhref{https://patents.google.com/patent/US3038608A}{1958}  \sphinxhref{https://patents.google.com/patent/US5143625}{1991}
\\
\hline
Dissolved air flotation
&
bubbles carry particles upward
&
flocculation
&
Public
&
\sphinxhref{https://iwaponline.com/wst/article-abstract/31/3-4/1/4138/Principles-and-applications-of-dissolved-air}{1905}
\\
\hline
\end{tabular}
\par
\sphinxattableend\end{savenotes}

See \sphinxhref{https://www.pnws-awwa.org/uploads/PDFs/conferences/2014/2.\%20PNWS\%20AWWA\%20WTC\%20Precon\%2005\%2007\%202014\%20Pretreatment\%20by\%20B\&V\%201\&2\%20-\%20R1.pdf}{Pretreatment Processes for Potable Water Treatment Plants by Jeff Lindgren for an excellent overview of available technologies, May 2014 (not including AguaClara innovations)}.


\subsection{Treatment Trains}
\label{\detokenize{Introduction/Introduction:treatment-trains}}\label{\detokenize{Introduction/Introduction:heading-treatment-trains}}
The prerequisites for the unit processes in \hyperref[\detokenize{Introduction/Introduction:table-surface-water-treatment-technologies}]{Table \ref{\detokenize{Introduction/Introduction:table-surface-water-treatment-technologies}}} reveal that surface water treatment almost always requires a series of treatment steps. A treatment train is a series of treatment steps (or unit processes) designed to convert a contaminated source water into a purified water meeting the use requirements.
\begin{description}
\item[{Example treatment trains include:}] \leavevmode\begin{itemize}
\item {} 
Conventional mechanized treatment: mechanical flocculation, lamellar sedimentation, rapid sand filtration, disinfection

\item {} 
Superpulsator: rapid mix, floc blanket, lamellar sedimentation, rapid sand filtration

\item {} 
AguaClara: hydraulic flocculation, floc blanket, lamellar sedimentation, stacked rapid sand filtration, disinfection

\item {} 
Membrane filtration: flocculation, sedimentation, rapid sand filtration, granular or powdered activated carbon, pre-oxidation (see \sphinxhref{https://pubs-acs-org.proxy.library.cornell.edu/doi/abs/10.1021\%2Fes802473r}{Review Article})

\end{itemize}

\end{description}


\section{The AguaClara Treatment Train}
\label{\detokenize{Introduction/Introduction:the-aguaclara-treatment-train}}\label{\detokenize{Introduction/Introduction:heading-the-aguaclara-treatment-train}}
Why does flocculation precedes sedimentation?
Which process removes the largest quantity of contaminants?

Sedimentation is the process of particles ‘falling’ because they have a higher density then the water, and its governing equation is:
\begin{equation}\label{equation:Introduction/Introduction:eq_laminar_terminal_velocity}
\begin{split} \bar v_t = \frac{D_{particle}^2 g}{18 \nu} \frac{\rho_p - \rho_w}{\rho_w}\end{split}
\end{equation}
\begin{DUlineblock}{0em}
\item[] Such that:
\item[] \(\bar v_t\) = terminal velocity of a particle, its downwards speed if it were in quiescent (still) water
\item[] \(D_{particle}\) = particle diameter
\item[] \(\rho\) = density. The \(p\) subscript stands for particle, while \(w\) stands for water
\end{DUlineblock}

\fvset{hllines={, ,}}%
\begin{sphinxVerbatim}[commandchars=\\\{\}]
\PYG{k+kn}{from} \PYG{n+nn}{aide\PYGZus{}design}\PYG{n+nn}{.}\PYG{n+nn}{play} \PYG{k}{import}\PYG{o}{*}
\PYG{k}{def} \PYG{n+nf}{v\PYGZus{}t}\PYG{p}{(}\PYG{n}{D\PYGZus{}particle}\PYG{p}{,}\PYG{n}{density\PYGZus{}particle}\PYG{p}{,}\PYG{n}{Temperature}\PYG{p}{)}\PYG{p}{:}
  \PYG{k}{return} \PYG{p}{(}\PYG{n}{D\PYGZus{}particle}\PYG{o}{*}\PYG{o}{*}\PYG{l+m+mi}{2}\PYG{o}{*}\PYG{n}{pc}\PYG{o}{.}\PYG{n}{gravity} \PYG{o}{*}\PYG{p}{(}\PYG{n}{density\PYGZus{}particle} \PYG{o}{\PYGZhy{}} \PYG{n}{pc}\PYG{o}{.}\PYG{n}{density\PYGZus{}water}\PYG{p}{(}\PYG{n}{Temperature}\PYG{p}{)}\PYG{p}{)}\PYG{o}{/}\PYG{p}{(}\PYG{l+m+mi}{18}\PYG{o}{*}\PYG{n}{pc}\PYG{o}{.}\PYG{n}{viscosity\PYGZus{}kinematic}\PYG{p}{(}\PYG{n}{Temperature}\PYG{p}{)}\PYG{o}{*}\PYG{n}{pc}\PYG{o}{.}\PYG{n}{density\PYGZus{}water}\PYG{p}{(}\PYG{n}{Temperature}\PYG{p}{)}\PYG{p}{)}\PYG{p}{)}\PYG{o}{.}\PYG{n}{to}\PYG{p}{(}\PYG{n}{u}\PYG{o}{.}\PYG{n}{m}\PYG{o}{/}\PYG{n}{u}\PYG{o}{.}\PYG{n}{s}\PYG{p}{)}
\PYG{n}{clay} \PYG{o}{=} \PYG{l+m+mi}{2650} \PYG{o}{*} \PYG{n}{u}\PYG{o}{.}\PYG{n}{kg}\PYG{o}{/}\PYG{n}{u}\PYG{o}{.}\PYG{n}{m}\PYG{o}{*}\PYG{o}{*}\PYG{l+m+mi}{3}
\PYG{n}{organic} \PYG{o}{=} \PYG{l+m+mi}{1040} \PYG{o}{*} \PYG{n}{u}\PYG{o}{.}\PYG{n}{kg}\PYG{o}{/}\PYG{n}{u}\PYG{o}{.}\PYG{n}{m}\PYG{o}{*}\PYG{o}{*}\PYG{l+m+mi}{3}
\PYG{n}{Temperature} \PYG{o}{=} \PYG{l+m+mi}{20} \PYG{o}{*} \PYG{n}{u}\PYG{o}{.}\PYG{n}{degC}
\PYG{n}{D\PYGZus{}particle} \PYG{o}{=} \PYG{n}{np}\PYG{o}{.}\PYG{n}{logspace}\PYG{p}{(}\PYG{o}{\PYGZhy{}}\PYG{l+m+mi}{6}\PYG{p}{,}\PYG{o}{\PYGZhy{}}\PYG{l+m+mi}{3}\PYG{p}{)}\PYG{o}{*}\PYG{n}{u}\PYG{o}{.}\PYG{n}{m}
\PYG{n}{fig}\PYG{p}{,} \PYG{n}{ax} \PYG{o}{=} \PYG{n}{plt}\PYG{o}{.}\PYG{n}{subplots}\PYG{p}{(}\PYG{p}{)}
\PYG{n}{ax}\PYG{o}{.}\PYG{n}{loglog}\PYG{p}{(}\PYG{n}{D\PYGZus{}particle}\PYG{o}{.}\PYG{n}{to}\PYG{p}{(}\PYG{n}{u}\PYG{o}{.}\PYG{n}{m}\PYG{p}{)}\PYG{p}{,}\PYG{n}{v\PYGZus{}t}\PYG{p}{(}\PYG{n}{D\PYGZus{}particle}\PYG{p}{,}\PYG{n}{clay}\PYG{p}{,}\PYG{n}{Temperature}\PYG{p}{)}\PYG{o}{.}\PYG{n}{to}\PYG{p}{(}\PYG{n}{u}\PYG{o}{.}\PYG{n}{m}\PYG{o}{/}\PYG{n}{u}\PYG{o}{.}\PYG{n}{s}\PYG{p}{)}\PYG{p}{)}
\PYG{n}{ax}\PYG{o}{.}\PYG{n}{loglog}\PYG{p}{(}\PYG{n}{D\PYGZus{}particle}\PYG{o}{.}\PYG{n}{to}\PYG{p}{(}\PYG{n}{u}\PYG{o}{.}\PYG{n}{m}\PYG{p}{)}\PYG{p}{,}\PYG{n}{v\PYGZus{}t}\PYG{p}{(}\PYG{n}{D\PYGZus{}particle}\PYG{p}{,}\PYG{n}{organic}\PYG{p}{,}\PYG{n}{Temperature}\PYG{p}{)}\PYG{o}{.}\PYG{n}{to}\PYG{p}{(}\PYG{n}{u}\PYG{o}{.}\PYG{n}{m}\PYG{o}{/}\PYG{n}{u}\PYG{o}{.}\PYG{n}{s}\PYG{p}{)}\PYG{p}{)}
\PYG{n}{ax}\PYG{o}{.}\PYG{n}{set}\PYG{p}{(}\PYG{n}{xlabel}\PYG{o}{=}\PYG{l+s+s1}{\PYGZsq{}}\PYG{l+s+s1}{Particle diameter (m)}\PYG{l+s+s1}{\PYGZsq{}}\PYG{p}{,} \PYG{n}{ylabel}\PYG{o}{=}\PYG{l+s+s1}{\PYGZsq{}}\PYG{l+s+s1}{Terminal velocity (m/s)}\PYG{l+s+s1}{\PYGZsq{}}\PYG{p}{)}
\PYG{n}{ax}\PYG{o}{.}\PYG{n}{legend}\PYG{p}{(}\PYG{p}{[}\PYG{l+s+s2}{\PYGZdq{}}\PYG{l+s+s2}{clay or sand}\PYG{l+s+s2}{\PYGZdq{}}\PYG{p}{,}\PYG{l+s+s2}{\PYGZdq{}}\PYG{l+s+s2}{organic particle}\PYG{l+s+s2}{\PYGZdq{}}\PYG{p}{]}\PYG{p}{)}
\PYG{n}{imagepath} \PYG{o}{=} \PYG{l+s+s1}{\PYGZsq{}}\PYG{l+s+s1}{Introduction/Images/}\PYG{l+s+s1}{\PYGZsq{}}
\PYG{n}{fig}\PYG{o}{.}\PYG{n}{savefig}\PYG{p}{(}\PYG{n}{imagepath}\PYG{o}{+}\PYG{l+s+s1}{\PYGZsq{}}\PYG{l+s+s1}{Terminal\PYGZus{}velocity}\PYG{l+s+s1}{\PYGZsq{}}\PYG{p}{)}
\end{sphinxVerbatim}

The terminal velocities of particles in surface waters range over many orders of magnitude especially if you consider that mountain streams can carry large rocks. But removing rocks from water is easily accomplished, gravity will take of it for us. Gravity is such a great force for separation of particles from water that we would like to use it to remove small particles too. Unfortunately, gravity becomes rather ineffective at separating pathogens and small inorganic particles such as clay. The terminal velocities (\eqref{equation:Introduction/Introduction:eq_laminar_terminal_velocity}) of these particles is given in \hyperref[\detokenize{Introduction/Introduction:figure-terminal-velocity}]{Fig.\@ \ref{\detokenize{Introduction/Introduction:figure-terminal-velocity}}}.

\begin{figure}[htbp]
\centering
\capstart

\noindent\sphinxincludegraphics[width=500\sphinxpxdimen]{{Terminal_velocity}.png}
\caption{The terminal velocity of a 1 \(\mu m\) bacteria cell is approximately 20 nanometers per second. The terminal velocity of a 5 \(\mu m\) clay particles is 30 \(\mu m/s\). The velocity estimates for the faster settling particles may be too slow because those particles are transitioning to turbulent flow.}\label{\detokenize{Introduction/Introduction:id19}}\label{\detokenize{Introduction/Introduction:figure-terminal-velocity}}\end{figure}

The low terminal velocities of particles that we need to remove from surface waters reveals that sedimentation alone will not work. The time required for a small particle to settle even a few mm would require excessively large sedimentation tanks. This is why flocculation, the process of sticking particles together so that they can attain higher sedimentation velocities, is perhaps the most important unit process in surface water treatment plants.
\begin{description}
\item[{The AguaClara treatment train consists of the following processes}] \leavevmode\begin{itemize}
\item {} 
flow measurement

\item {} 
metering of the coagulant (and chlorine) that will cause particles to stick together

\item {} 
mixing of the coagulant with the raw water

\item {} 
flocculation where the water is deformed to cause particle collisions

\item {} 
floc blanket where large flocs settle through water that is flowing upward causing collisions between small particles carried by the upward flowing water and the large flocs

\item {} 
lamellar sedimentation where gravity causes particles to settle to an inclined plate and then slide back down into the floc blanket

\item {} 
stacked rapid sand filtration where particles collide with previously deposited particles in a sand filter bed

\item {} 
disinfection with chlorine to inactivate any pathogens that escaped the previous unit processes

\end{itemize}

\end{description}


\subsection{Design Evolution}
\label{\detokenize{Introduction/Introduction:design-evolution}}
During the later half of the 20th century surface water treatment technologies evolved slowly. The slow evolution was likely a product of the regulatory environment, the high cost of water treatment infrastructure, and the low profit margin. The high cost of municipal scale water treatment infrastructure made experiments at scale infeasible and thus there was no mechanism to introduce disruptive innovations. With little opportunity for a significant return on investment there was little incentive to invest in the research and development that could have advanced the technologies. A final disincentive was the widely held belief that surface water treatment was a mature field with little opportunity for significant advancement. The advances of the latter half of the 20th century focused primarily on mechanization and automation (Supervisory Control and Data Acquisition - SCADA).

Design standards such as the {[}Great Lakes - Upper Mississippi River Board 10 States Standards{]}(\sphinxurl{http://10statesstandards.com/}) are evolving very slowly and retain an empirical approach to design. The empirical design methodology is a direct result of two confounding factors. The physics of particle interactions based on diffusion, fluid shear, and gravity are complex and given the challenges of characterizing surface water particle suspensions it was natural to assume that a mathematical description of the processes would be intractable.

Mechanized and automated water treatment plants performed reasonably well in communities with ready access to technical support services and supply chains that could reliably deliver replacement parts. In the global south municipal water treatment plants haven’t faired as well. In 2012, one of the main water treatment plants serving Kathmandu, Nepal had failed chlorine pumps and were using a red garden hose to siphon chlorine from the stock tank. They crimped the end of the hose to control the flow rate of the chlorine solution.

\begin{figure}[htbp]
\centering
\capstart

\noindent\sphinxincludegraphics[width=300\sphinxpxdimen]{{Kathmandu_chemical_feed_room}.png}
\caption{Failed chlorine doing system bypassed with a red tube that siphons the chlorine solution at a plant in Kathmandu, Nepal in 2012.}\label{\detokenize{Introduction/Introduction:id20}}\label{\detokenize{Introduction/Introduction:figure-kathmandu-chemical-feed-room}}\end{figure}

The ingenious and simple chemical dosing system that uses a siphon to completely bypass the failed pumps begs the question of whether design engineers could have invented a better option than the short lived pumps that they specified. We will investigate a gravity powered chemical dosing system that is far more reliable than chemical dosing pumps and that borrows from the simplicity of the garden hose solution used by the Nepali plant operators.

Chemical dosing systems are particularly vulnerable and their failures make plant operation very challenging. Providing the right coagulant dose is critical for efficient removal of particle and dissolved organics. Chemical dosing systems commonly rely on pumps and those pumps require regular maintenance and have relatively short mean times between failures.

\begin{figure}[htbp]
\centering
\capstart

\noindent\sphinxincludegraphics[width=300\sphinxpxdimen]{{Kathmandu_alum_dosing}.jpg}
\caption{Alum dosing system based on the rate that 25 kg blocks of alum are placed in the inlet channel of the plant.}\label{\detokenize{Introduction/Introduction:id21}}\label{\detokenize{Introduction/Introduction:figure-kathmandu-alum-dosing}}\end{figure}

The AguaClara Cornell program was founded in 2005 with the goal of creating a new generation of sustainable technologies that would perform well even in the rugged settings of rural communities. The goal wasn’t simply to create technologies that would work for communities with very limited resources. The goal was to create the next generation of technologies that would both perform well in communities with limited resources and would be the highest performing technologies on multiple metrics for all communities.


\subsection{Empirical Design}
\label{\detokenize{Introduction/Introduction:empirical-design}}\label{\detokenize{Introduction/Introduction:heading-empirical-design}}
For the past several decades surface water treatment technologies have been considered “mature” and when I (Monroe) took a design course on drinking water treatment in 1985 I had the impression that there was little room for further innovation. This perspective is remarkable given that with the exception of lamellar sedimentation there were no equations describing the core treatment processes.

Empirical design guidelines don’t provide insight into how designs could be optimized or even what the performance of a water treatment plant will be.


\section{Design for the Financers, Venders, Client, or Context?}
\label{\detokenize{Introduction/Introduction:design-for-the-financers-venders-client-or-context}}\label{\detokenize{Introduction/Introduction:heading-design-for-the-context}}
Tours of water treatment plants suggest that it is common for designs to be driven by the vender goal of a stable revenue stream for replacement parts rather than by a goal of meeting the client’s needs. Mandatory software upgrades, mechanical valves, chemical pumps, mixing units provide a steady demand for proprietary components. Financers often prefer projects that can be implemented quickly either because they have target expenditures for a fiscal year or because loan repayment begins when the facility is turned over to the client.

Design for the client would strive to reduce capital, operating, and maintenance expenses. Clients also place a high value on reliability, ease of maintenance, and the ability to handle repairs with their staff. Design for the context would account for the capabilities of local and national supply chains. A key design consideration is to ensure that the treatment capabilities of the treatment plant match the variable water quality of the proposed water source. There are numerous slow sand filtration plants installed in the global south that are attempting to treat water sources that can not be effectively treated by slow sand filtration. The cost of the failure to consider the client and the context is born by the communities who end up with water treatment systems that aren’t able to provide reliable safe water.

Design for the client requires empathy and a commitment to listen to and learn from plant operators. It also requires attention to detail and watching how plant operators interact with water treatment plants. Empathy leads to the goal of creating a work environment that makes it easy for the plant operators to do their routine tasks. This isn’t just to make the plant operators work easy. A plant that is designed with the plant operator in mind will also engender pride and that pride will lead to better plant performance.

An example of design for the operator is the elevation of the walkways in AguaClara plants. Conventional plants often have walkways that are above the tanks. That places the operator’s eyes several meters above the water surface and makes it difficult to see particles and flocs in the water. AguaClara plants have the walkways approximately 50 cm below the top of the tanks. This makes it easy for the plant operator to look into the tanks for quick visual inspections.

\begin{figure}[htbp]
\centering
\capstart

\noindent\sphinxincludegraphics[width=300\sphinxpxdimen]{{Improvised_ladder_access_to_sed_tank}.jpg}
\caption{A plant operator built a makeshift ladder to enable easier access to the flocculation and sedimentation tanks in a package plant. This ladder considerably shortened the distance between the coagulant dose controls and the flocculator. The ladder also makes it possible to look closely at the water to see the size of the flocs.}\label{\detokenize{Introduction/Introduction:id22}}\label{\detokenize{Introduction/Introduction:figure}}\end{figure}


\section{Design Bifurcations}
\label{\detokenize{Introduction/Introduction:design-bifurcations}}\label{\detokenize{Introduction/Introduction:heading-design-bifurcations}}
Seemingly small decisions can have a profound effect on the evolution of design. Often these decisions have a clear logic and a simple analysis would suggest that the decision must be the right one. It is common for design choices to have multiple consequences that can turn a seemingly great choice into a poor performer.


\subsection{Walls and a Roof}
\label{\detokenize{Introduction/Introduction:walls-and-a-roof}}\label{\detokenize{Introduction/Introduction:heading-walls-and-a-roof}}
Traditionally in tropical and temperate climates, flocculation and sedimentation units are built without an enclosing building because they aren’t in danger of freezing. Without protection from the sun the materials used for plant construction must be UV resistant and thus plastic can’t be used. This requires use of heavier and more expensive materials such stainless steel and aluminum. Metal plate settlers are heavy and thus they can’t be easily removed by the plant operator.

Without the ability to gain access to a sedimentation tank from above, conventional sedimentation tank cleaning must be done by providing operator access below the plate settlers. This in turn requires that the space below the plate settlers be tall enough to accommodate a plant operator. Thus sedimentation tanks that are built in the open have to be deeper than sedimentation tanks that are built under a roof and they are more difficult to maintain because the operator has to enter the tank through a waterproof access port. Operator access to the space below the stainless steel or aluminum plate settlers is through a port in the side of the tank (see the video \hyperref[\detokenize{Introduction/Introduction:figure-cleaning-a-sed-tank-with-fixed-plates}]{Fig.\@ \ref{\detokenize{Introduction/Introduction:figure-cleaning-a-sed-tank-with-fixed-plates}}}).

\begin{figure}[htbp]
\centering
\capstart
\sphinxhref{http://www.youtube.com/watch?v=TSh-ZNqaW8Y}{\sphinxincludegraphics[width=300\sphinxpxdimen]{{0}.jpg}}\caption{Plant operators opening hatch below plate settlers in a traditional sedimentation tank.}\label{\detokenize{Introduction/Introduction:id23}}\label{\detokenize{Introduction/Introduction:figure-cleaning-a-sed-tank-with-fixed-plates}}\end{figure}

AguaClara sedimentation tanks are designed to be taken off line one at a time so the water treatment plant can continue to operate during maintenance. Two plant operators can quickly open a sedimentation tank by removing the plastic plate settlers (see the video \hyperref[\detokenize{Introduction/Introduction:figure-removing-plate-settlers}]{Fig.\@ \ref{\detokenize{Introduction/Introduction:figure-removing-plate-settlers}}}). The zero settled sludge design of the AguaClara sedimentation tanks also reduces the need for cleaning.

\begin{figure}[htbp]
\centering
\capstart
\sphinxhref{http://www.youtube.com/watch?v=vZ2f6mduEls}{\sphinxincludegraphics[width=300\sphinxpxdimen]{{01}.jpg}}\caption{Plant operator removing plate settlers from an AguaClara sedimentation tank.}\label{\detokenize{Introduction/Introduction:id24}}\label{\detokenize{Introduction/Introduction:figure-removing-plate-settlers}}\end{figure}
\phantomsection\label{\detokenize{Introduction/Introduction:heading-mechanized-or-smart-hydraulics}}
There is another major consequence of building water treatment plants in a secure enclosed building. Many water treatment plants will operate around the clock and that requires plant operators to spend the night at the facility. Having a secure facility provides improved safety for the plant operator. That improved safety is very important for some potential operators and thus providing that safety will increase potential diversity.


\subsection{Mechanized or Smart Hydraulics}
\label{\detokenize{Introduction/Introduction:mechanized-or-smart-hydraulics}}
Dramatically different designs are also created when we choose gravity power and smart hydraulics rather than mechanical mixers, pumps, and mechanical controls for each of the unit processes. It appears that use of electricity in drinking water treatment plants became the popular choice about 100 years ago. Many gravity powered plants have been converted to use mechanical mixers for rapid mix and flocculation. That choice may not have been well founded from a water quality or performance perspective.

Automated plants often move the controls far away from the critical observation locations in the plant. This might be appropriate or necessary in some cases, but it has the disadvantage of making it more difficult for operators to directly observe what is happening in the plant. Direct observations are critical because even highly mechanized water treatment plants are not yet equipped with enough sensors to enable rapid troubleshooting from the control room.

AguaClara plants have a layout that places the coagulant dose controls within a few steps of the best places to observe floc formation in the flocculator. This provides plant operators with rapid feedback that is critical when the raw water changes rapidly at the beginning of a high runoff event. As operators spend time observing the processes in the plant they begin to associate cause and effect and can make operational changes to improve performance. For example, gas bubbles that carry flocs to the surface can indicate sludge accumulation in a sedimentation tank. Rising flocs without gas bubbles can indicate a poor inlet flow distribution for a sedimentation tank or density differences caused by temperature differences.


\chapter{Review: Fluid Mechanics}
\label{\detokenize{Review/Review_Fluid_Mechanics:review-fluid-mechanics}}\label{\detokenize{Review/Review_Fluid_Mechanics:title-review-fluid-mechanics}}\label{\detokenize{Review/Review_Fluid_Mechanics::doc}}
This document is meant to be a refresher on fluid mechanics. It will only cover the topics in fluids mechanics that will be used heavily in the course.

If you wish to review fluid mechanics in (much) more detail, please refer to \sphinxhref{https://github.com/AguaClara/CEE4540\_Master/wiki/Fluids-Review-Guide}{this guide} Note that to view this link, you will need a Github accounts. If you wish to review from a legitimate textbook, you can find a pdf of good book by Frank White \sphinxhref{https://hellcareers.files.wordpress.com/2016/01/fluid-mechanics-seventh-edition-by-frank-m-white.pdf}{here}.


\section{Important Terms and Equations}
\label{\detokenize{Review/Review_Fluid_Mechanics:important-terms-and-equations}}\label{\detokenize{Review/Review_Fluid_Mechanics:heading-fluids-terms-eqs}}
\sphinxstylestrong{Terms:}
\begin{enumerate}
\item {} 
{\hyperref[\detokenize{Review/Review_Fluid_Mechanics:heading-laminar-and-turbulent-flow}]{\sphinxcrossref{\DUrole{std,std-ref}{Laminar}}}}

\item {} 
{\hyperref[\detokenize{Review/Review_Fluid_Mechanics:heading-laminar-and-turbulent-flow}]{\sphinxcrossref{\DUrole{std,std-ref}{Turbulent}}}}

\item {} 
{\hyperref[\detokenize{Review/Review_Fluid_Mechanics:heading-laminar-and-turbulent-flow}]{\sphinxcrossref{\DUrole{std,std-ref}{Viscosity}}}}

\item {} 
{\hyperref[\detokenize{Review/Review_Fluid_Mechanics:heading-streamlines-and-control-volumes}]{\sphinxcrossref{\DUrole{std,std-ref}{Streamline}}}}

\item {} 
{\hyperref[\detokenize{Review/Review_Fluid_Mechanics:heading-streamlines-and-control-volumes}]{\sphinxcrossref{\DUrole{std,std-ref}{Control Volume}}}}

\item {} 
{\hyperref[\detokenize{Review/Review_Fluid_Mechanics:heading-bernoulli-equation}]{\sphinxcrossref{\DUrole{std,std-ref}{Head}}}}

\item {} 
{\hyperref[\detokenize{Review/Review_Fluid_Mechanics:heading-head-loss}]{\sphinxcrossref{\DUrole{std,std-ref}{Head loss}}}}

\item {} 
{\hyperref[\detokenize{Review/Review_Fluid_Mechanics:heading-head-loss-elevation-difference-trick}]{\sphinxcrossref{\DUrole{std,std-ref}{Driving head}}}}

\item {} 
{\hyperref[\detokenize{Review/Review_Fluid_Mechanics:heading-what-is-a-vena-contracta}]{\sphinxcrossref{\DUrole{std,std-ref}{Vena Contracta/Coefficient of Contraction}}}}

\end{enumerate}

\sphinxstylestrong{Equations:}
\begin{enumerate}
\item {} 
Continuity equation: \eqref{equation:Review/Review_Fluid_Mechanics:continuity_equation}

\item {} 
Reynolds number \eqref{equation:Review/Review_Fluid_Mechanics:reynolds_number_equation}

\item {} 
Bernoulli equation \eqref{equation:Review/Review_Fluid_Mechanics:bernoulli_equation}

\item {} 
Energy equation \eqref{equation:Review/Review_Fluid_Mechanics:energy_equation}

\item {} 
Darcy-Weisbach equation \eqref{equation:Review/Review_Fluid_Mechanics:darcy_weisbach}

\item {} 
Swamee-Jain equation \eqref{equation:Review/Review_Fluid_Mechanics:swamee_jain}

\item {} 
Hagen-Poiseuille equation \eqref{equation:Review/Review_Fluid_Mechanics:hagen_poiseuille}

\item {} 
Orifice equation \eqref{equation:Review/Review_Fluid_Mechanics:orifice_equation}

\end{enumerate}


\section{Introductory Concepts}
\label{\detokenize{Review/Review_Fluid_Mechanics:introductory-concepts}}\label{\detokenize{Review/Review_Fluid_Mechanics:heading-introductory-concepts}}
Before diving in to the rest of this document, there are a few important concepts to focus on which will be the foundation for building your understanding of fluid mechanics. One must walk before they can run, and similarly, the basics of fluid mechanics must be understood before moving on to the more fun (and exciting!) sections of this document.


\subsection{Continuity Equation}
\label{\detokenize{Review/Review_Fluid_Mechanics:continuity-equation}}\label{\detokenize{Review/Review_Fluid_Mechanics:heading-continuity-equation}}
Continuity is simply an application of mass balance to fluid mechanics. It states that the cross sectional area \(A\) that a fluid flows through multiplied by the fluid’s average flow velocity \(\bar v\) must equal the fluid’s flow rate \(Q\):
\begin{equation}\label{equation:Review/Review_Fluid_Mechanics:continuity_equation}
\begin{split}  Q = \bar v A\end{split}
\end{equation}
\begin{sphinxadmonition}{note}{Note:}
The line above the \(v\) is called a ‘bar,’ and represents an average. Any variable can have a bar. In this case, we are adding the bar to velocity \(v\), turning it into average velocity \(\bar v\). This variable is pronounced ‘v bar.’
\end{sphinxadmonition}

In this course, we deal primarily with flow through pipes. For a circular pipe, \(A = \pi r^2\). Substituting diameter in for radius, \(r = \frac{D}{2}\), we get \(A = \frac{\pi D^2}{4}\). You will often see this form of the continuity equation being used to relate the a pipe’s flow rate to its diameter and the velocity of the fluid flowing through it:
\begin{equation}\label{equation:Review/Review_Fluid_Mechanics:Review/Review_Fluid_Mechanics:0}
\begin{split}Q = \bar v \frac{\pi D^2}{4}\end{split}
\end{equation}
The continuity equation is also useful when flow is going from one geometry to another. In this case, the flow in one geometry must be the same as the flow in the other, \(Q_1 = Q_2\), which yields the following equations:
\begin{equation}\label{equation:Review/Review_Fluid_Mechanics:Review/Review_Fluid_Mechanics:1}
\begin{split}\bar v_1 A_1 = \bar v_2 A_2\end{split}
\end{equation}\begin{equation}\label{equation:Review/Review_Fluid_Mechanics:Review/Review_Fluid_Mechanics:2}
\begin{split}\bar v_1 \frac{\pi D_1^2}{4} = \bar v_2 \frac{\pi D_2^2}{4}\end{split}
\end{equation}
\begin{DUlineblock}{0em}
\item[] Such that:
\item[] \(Q =\) fluid flow rate
\item[] \(\bar v =\) fluid average velocity
\item[] \(A =\) pipe area
\item[] \(r =\) pipe radius
\item[] \(D =\) pipe diameter
\end{DUlineblock}

An example of changing flow geometries is when a change in pipe size occurs in a circular piping system, as is demonstrated below. The flow through \({\rm pipe} \, 1\) must be the same as the flow through \({\rm pipe} \, 2\).

\begin{figure}[htbp]
\centering
\capstart

\noindent\sphinxincludegraphics[width=700\sphinxpxdimen]{{continuity_pipes}.png}
\caption{Flow going from a small diameter pipe to a large one. The continuity principle states that the flow through each pipe must be the same.}\label{\detokenize{Review/Review_Fluid_Mechanics:id1}}\label{\detokenize{Review/Review_Fluid_Mechanics:figure-continuity-pipes}}\end{figure}


\subsection{Laminar and Turbulent Flow}
\label{\detokenize{Review/Review_Fluid_Mechanics:laminar-and-turbulent-flow}}\label{\detokenize{Review/Review_Fluid_Mechanics:heading-laminar-and-turbulent-flow}}
Considering that this class deals with the flow of water through a water treatment plant, understanding the characteristics of the flow is very important. Thus, it is necessary to understand the most common characteristic of fluid flow: whether it is \sphinxstylestrong{laminar} or \sphinxstylestrong{turbulent}. \sphinxhref{https://en.wikipedia.org/wiki/Laminar\_flow}{Laminar} flow is very smooth and highly ordered. \sphinxhref{https://en.wikipedia.org/wiki/Turbulence}{Turbulent} flow is chaotic, messy, and disordered. The best way to understand each flow and what it looks like is visually, like in the Wikipedia figure below \sphinxhref{https://youtu.be/qtvVN2qt968?t=131}{or in this video}. Please ignore the part of the video after the image of the tap.

\begin{figure}[htbp]
\centering
\capstart

\noindent\sphinxincludegraphics[width=400\sphinxpxdimen]{{Wikipedia_laminar_turbulent}.png}
\caption{This is a beautiful example of the difference between ordered and smooth laminar flow and chaotic turbulent flow.}\label{\detokenize{Review/Review_Fluid_Mechanics:id2}}\label{\detokenize{Review/Review_Fluid_Mechanics:figure-wikipedia-laminar-turbulent}}\end{figure}

A numeric way to determine whether flow is laminar or turbulent is by finding the \sphinxhref{https://en.wikipedia.org/wiki/Reynolds\_number}{Reynolds number}, \({\rm Re}\). The Reynolds number is a dimensionless parameter that compares inertia, represented by the average flow velocity \(\bar v\) times a length scale \(D\) to \sphinxhref{https://en.wikipedia.org/wiki/Viscosity}{viscosity}, represented by the kinematic viscosity \(\nu\). \sphinxhref{https://www.youtube.com/watch?v=DVQw0svRHZA}{Click here} for a brief video explanation of viscosity. If the Reynolds number is less than 2,100 the flow is considered laminar. If it is more than 2,100, it is considered turbulent.
\begin{equation}\label{equation:Review/Review_Fluid_Mechanics:Review/Review_Fluid_Mechanics:3}
\begin{split}{\rm Re = \frac{inertia}{viscosity}} = \frac{\bar vD}{\nu}\end{split}
\end{equation}
\sphinxhref{https://en.wikipedia.org/wiki/Laminar\%E2\%80\%93turbulent\_transition}{The transition between laminar and turbulent flow is not yet well understood}, which is why the concept of transitional flow is often simplified and neglected to make it possible to code for laminar or turbulent flow, which are better understood. We will assume that the transition occurs at \(\rm{Re} = 2100\). In aide\_design, this parameter shows us as \sphinxcode{\sphinxupquote{pc.RE\_TRANSITION\_PIPE}}.

Fluid can flow through very many different geometries, like a pipe, a rectangular channel, or any other shape. To account for this, the characteristic length scale for the Reynolds number, which was written in the equation above as \(D\), is quantified as the \sphinxhref{https://www.engineeringtoolbox.com/hydraulic-equivalent-diameter-d\_458.html}{hydraulic diameter}, \(D_h\) when considering a general cross-sectional area. For circular pipes, which are the most common geometry you’ll encounter in this class, the hydraulic diameter is simply the pipe’s diameter, \(D_h = D\).

Here are other commonly used forms of the Reynolds number equation \sphinxstyleemphasis{for circular pipes}. They are the same as the one above, just with the substitutions \(Q = \bar v \frac{\pi D^2}{4}\) and \(\nu = \frac{\mu}{\rho}\)
\begin{equation}\label{equation:Review/Review_Fluid_Mechanics:reynolds_number_equation}
\begin{split}  {\rm Re} = \frac{\bar vD}{\nu} = \frac{4Q}{\pi D\nu} = \frac{\rho \bar vD}{\mu}\end{split}
\end{equation}
\begin{DUlineblock}{0em}
\item[] Such that:
\item[] \(Q\) = fluid flow rate in pipe
\item[] \(D\) = pipe diameter
\item[] \(\bar v\) = fluid velocity
\item[] \(\nu\) = fluid kinematic viscosity
\item[] \(\mu\) = fluid dynamic viscosity
\end{DUlineblock}


\sphinxstrong{See also:}


\sphinxstylestrong{Function in aide\_design:} \sphinxcode{\sphinxupquote{pc.re\_pipe(FlowRate, Diam, Nu)}} Returns the Reynolds number \sphinxstyleemphasis{in a circular pipe}. Functions for finding the Reynolds number through other flow conduits and geometries can also be found in \sphinxhref{https://github.com/AguaClara/aide\_design/blob/master/aide\_design/physchem.py}{physchem.py} within aide\_design.



\begin{sphinxadmonition}{note}{Note:}
\sphinxstylestrong{Definition of Flow Regimes:} Laminar and turbulent flow are described as two different \sphinxstylestrong{flow regimes}. When there is a characteristic of flow and different categories of the characteristic, each category is referred to as a flow regime. For example, the Reynolds number describes a flow characteristic, and its categories, referred to as flow regimes, are laminar or turbulent.
\end{sphinxadmonition}


\subsection{Streamlines and Control Volumes}
\label{\detokenize{Review/Review_Fluid_Mechanics:streamlines-and-control-volumes}}\label{\detokenize{Review/Review_Fluid_Mechanics:heading-streamlines-and-control-volumes}}
Both \sphinxhref{https://en.wikipedia.org/wiki/Streamlines,\_streaklines,\_and\_pathlines}{streamlines} and \sphinxhref{https://www.engineersedge.com/fluid\_flow/control\_volume.htm}{control volumes} are tools to compare different parts of a system. For this class, this system will always be hydraulic.

Imagine water flowing through a pipe. A streamline is the path that a particle would take if it could be placed in the fluid without changing the original flow of the fluid. A more technical definition is “a line which is everywhere parallel to the local velocity vector.” Computational tools, \sphinxhref{https://www.nuclear-power.net/wp-content/uploads/2016/05/Flow-Regime.png?4b884b}{dyes (in water)}, or \sphinxhref{https://www.youtube.com/watch?v=E9ZSAX56m0E\&t=59s}{smoke (in air)} can be used to visualize streamlines.

A \sphinxstylestrong{control volume} is just an imaginary 3-dimensional shape in space. Its boundaries can be placed anywhere by the person applying the control volume, and once set the boundaries remain fixed in space over time. These boundaries are usually chosen to compare two relevant surfaces to each other. These surfaces are called \sphinxstyleemphasis{Control Surfaces}. The entirety of a control volume is usually not shown, as it is often unnecessary. This is demonstrated in the following image:

\begin{figure}[htbp]
\centering
\capstart

\noindent\sphinxincludegraphics[width=650\sphinxpxdimen]{{control_volume_simplification}.png}
\caption{While the image on the left indicates a complete control volume, control volumes are usually shortened to only include the relevant control surfaces, in which the control volume intersects the fluid. This is shown in the image on the right.}\label{\detokenize{Review/Review_Fluid_Mechanics:id3}}\label{\detokenize{Review/Review_Fluid_Mechanics:figure-control-volume-simplification}}\end{figure}

\begin{sphinxadmonition}{important}{Important:}
Many images will be used over the course of this class to show hydraulic systems. A standardized system of lines will be used throughout them all to distinguish reference elevations from control volumes from streamlines. This system is described in the image below.
\end{sphinxadmonition}

\begin{figure}[htbp]
\centering
\capstart

\noindent\sphinxincludegraphics[width=650\sphinxpxdimen]{{image_control_volumes}.png}
\caption{On the right, a control volume is applied to a hydraulic system. On the left, a streamline is applied to a hydraulic system. A figure-convention for control volumes and streamlines will be very helpful throughout this course as there will be very, very many figures.}\label{\detokenize{Review/Review_Fluid_Mechanics:id4}}\label{\detokenize{Review/Review_Fluid_Mechanics:figure-image-control-volumes}}\end{figure}


\section{The Bernoulli and Energy Equations}
\label{\detokenize{Review/Review_Fluid_Mechanics:the-bernoulli-and-energy-equations}}\label{\detokenize{Review/Review_Fluid_Mechanics:heading-bernoulli-and-energy-equations}}
As explained in almost every fluid mechanics class, the Bernoulli and energy equations are incredibly useful in understanding the transfer of the fluid’s energy throughout a streamline or through a control volume. The Bernoulli equation applies to two different points along one streamline, whereas the energy equation applies to fluid entering and exiting a control volume. The energy of a fluid has three forms: pressure, potential (deriving from elevation), and kinetic (deriving from velocity).


\subsection{The Bernoulli Equation}
\label{\detokenize{Review/Review_Fluid_Mechanics:the-bernoulli-equation}}\label{\detokenize{Review/Review_Fluid_Mechanics:heading-bernoulli-equation}}
These three forms of energy expressed above make up the Bernoulli equation:
\begin{equation}\label{equation:Review/Review_Fluid_Mechanics:bernoulli_equation}
\begin{split}  \frac{p_1}{\rho g} + {z_1} + \frac{v_1^2}{2g} = \frac{p_2}{\rho g} + {z_2} + \frac{v_2^2}{2g}\end{split}
\end{equation}
\begin{DUlineblock}{0em}
\item[] Such that:
\item[] \(p\) = pressure
\item[] \(\rho\) = fluid density
\item[] \(g\) = acceleration due to gravity, in aide\_design as \sphinxcode{\sphinxupquote{con.GRAVITY}}
\item[] \(z\) = elevation relative to a reference
\item[] \(v\) = fluid velocity
\end{DUlineblock}

Notice that each term in this form of the Bernoulli equation has units of \([L]\), even though the terms represent the energy of the fluid, which has units of \(\frac{[M] \cdot [L]^2}{[T]^2}\). When energy of the fluid is described in units of length, the term used is called \sphinxstylestrong{head} and referred to as \(h\).

There are two important distinctions to keep in mind when using head to talk about a fluid’s energy. First is that head is dependent on the density of the fluid under consideration. Take mercury, for example, which is around 13.6 times more dense than water. 1 meter of mercury head is therefore equivalent to around 13.6 meters of water head. Second is that head is independent of the amount of fluid being considered, \sphinxstyleemphasis{as long as all the fluid is the same density}. Thus, raising 1 liter of water up by one meter and raising 100 liters of water up by one meter are both equivalent to giving the water 1 meter of water head, even though it requires 100 times more energy to raise the hundred liters than to raise the single liter. Since we are concerned mainly with water in this class, we will refer to ‘water head’ simply as ‘head’.

Going back to the Bernoulli equation, the \(\frac{p}{\rho g}\) term is called the pressure head, \(z\) is called the elevation head, and \(\frac{v^2}{2g}\) is the velocity head. The following diagram shows these various forms of head via a 1 meter deep bucket (left) and a jet of water shooting out of the ground (right).

\begin{figure}[htbp]
\centering
\capstart

\noindent\sphinxincludegraphics[width=650\sphinxpxdimen]{{different_forms_of_head}.png}
\caption{The three forms of hydraulic head.}\label{\detokenize{Review/Review_Fluid_Mechanics:id5}}\label{\detokenize{Review/Review_Fluid_Mechanics:figure-different-forms-of-head}}\end{figure}


\subsubsection{Assumption in using the Bernoulli equation}
\label{\detokenize{Review/Review_Fluid_Mechanics:assumption-in-using-the-bernoulli-equation}}
Though there are \sphinxhref{https://en.wikipedia.org/wiki/Bernoulli\%27s\_principle\#Incompressible\_flow\_equation}{many assumptions needed to confirm that the Bernoulli equation can be used}, the main one for the purpose of this class is that energy is not gained or lost throughout the streamline being considered. If we consider more precise fluid mechanics terminology, then “friction by viscous forces must be negligible.” What this means is that the fluid along the streamline being considered is not losing energy to viscosity. As a result, using the Bernoulli equation implies that energy can’t be gained or lost. It can only be transferred between its three forms.


\subsubsection{Example problems}
\label{\detokenize{Review/Review_Fluid_Mechanics:example-problems}}
\sphinxhref{https://www.teachengineering.org/content/cub\_/lessons/cub\_bernoulli/cub\_bernoulli\_lesson01\_bepworksheetas\_draft4\_tedl\_dwc.pdf}{Here is a simple worksheet with very straightforward example problems using the Bernoulli equation.} Note that the solutions use the pressure-form of the Bernoulli equation. This just means that every term in the equation is multiplied by \(\rho g\), so the pressure term is just \(P\). The form of the equation does not affect the solution to the problem it helps solved.


\subsection{The Energy Equation}
\label{\detokenize{Review/Review_Fluid_Mechanics:the-energy-equation}}\label{\detokenize{Review/Review_Fluid_Mechanics:heading-energy-equation}}
The assumption necessary to use the Bernoulli equation, which is stated above, represents the key difference between the Bernoulli equation and the energy equation for the purpose of this class. The energy equation accounts for the potential addition or loss of fluid energy within the control volume. (L)oss of energy is usually due to viscous friction resisting fluid flow, \(h_L\), or the charging of a (T)urbine, \(h_T\). The most common input of fluid energy into a system is usually caused by a (P)ump within the control volume, \(h_P\).
\begin{equation}\label{equation:Review/Review_Fluid_Mechanics:Review/Review_Fluid_Mechanics:4}
\begin{split}\frac{p_{1}}{\rho g} + z_{1} + \alpha_{1} \frac{\bar v_{1}^2}{2g} + h_P = \frac{p_{2}}{\rho g} + z_{2} + {\alpha_{2}} \frac{\bar v_{2}^2}{2g} + h_T + h_L\end{split}
\end{equation}
You’ll also notice the \(\alpha\) term attached to the velocity head. This is a correction factor for kinetic energy, and will be neglected in this class; we assume that its value is 1. In the Bernoulli equation, the velocity of a streamline of the fluid is considered, \(v\). The energy equation, however compares control surfaces instead of streamlines, and the velocities across a control surface many not all be the same. Hence, \(\bar v\) is used to represent the average velocity. Since AguaClara does not use pumps nor turbines, \(h_P = h_T = 0\). With these simplifications, the energy equation can be written as follows:
\begin{equation}\label{equation:Review/Review_Fluid_Mechanics:energy_equation}
\begin{split}  \frac{p_{1}}{\rho g} + z_{1} + \frac{\bar v_{1}^2}{2g} = \frac{p_{2}}{\rho g} + z_{2} + \frac{\bar v_{2}^2}{2g} + h_L\end{split}
\end{equation}
\sphinxstylestrong{This is the form of the energy equation that you will see over and over again in this book.} To summarize, the main difference between the Bernoulli equation and the energy equation for the purposes of this class is energy loss. The energy equation accounts for the fluid’s loss of energy over time while the Bernoulli equation does not. So how can the fluid lose energy?


\section{Headloss}
\label{\detokenize{Review/Review_Fluid_Mechanics:headloss}}\label{\detokenize{Review/Review_Fluid_Mechanics:heading-head-loss}}
\sphinxstylestrong{Head(L)oss}, \(h_L\) is a term that is ubiquitous in both this class and fluid mechanics in general. Its definition is exactly as it sounds: it refers to the loss of energy of a fluid as it flows through space. There are two components to head loss: major losses caused by (f)riction between the fluid the surface it’s flowing over, \(h_{\rm{f}}\), and minor losses caused by fluid-fluid internal friction resulting from flow (e)xpansions, \(h_e\). These two components combine such that \(h_L = h_{\rm{f}} + h_e\).


\subsection{Major Losses}
\label{\detokenize{Review/Review_Fluid_Mechanics:major-losses}}\label{\detokenize{Review/Review_Fluid_Mechanics:heading-major-losses}}
These losses are the result of friction between the fluid and the surface over which the fluid is flowing. A force acting parallel to a surface is referred to as \sphinxhref{https://en.wikipedia.org/wiki/Shear\_force}{shear}. It can therefore be said that major losses are the result of shear between the fluid and the surface it’s flowing over. To help in understanding major losses, consider the following example: imagine, as you have so often in physics class, pushing a large box across the ground. Friction is what resists your efforts to push the box. The farther you push the box, the more energy you expend pushing against friction. The same is true for water moving through a pipe, where water is analogous to the box you want to move, the pipe is similar to the floor that provides the friction, and the major losses of the water through the pipe is analogous to the energy \sphinxstylestrong{you} expend by pushing the box.

In this class, we will be dealing primarily with major losses in circular pipes, as opposed to channels or pipes with other geometries. Fortunately for us, Henry Darcy and Julius Weisbach came up with a handy equation to determine the major losses in a circular pipe \sphinxstyleemphasis{under both laminar and turbulent flow conditions}. Their equation is logically and unoriginally named the \sphinxhref{https://en.wikipedia.org/wiki/Darcy\%E2\%80\%93Weisbach\_equation}{Darcy-Weisbach equation}. It is shown below:
\begin{equation}\label{equation:Review/Review_Fluid_Mechanics:darcy_weisbach}
\begin{split}  h_{\rm{f}} \, = \, {\rm{f}} \frac{L}{D} \frac{\bar v^2}{2g}\end{split}
\end{equation}
Substituting the continuity equation \(Q = \bar vA\) in the form of \(\bar v^2 = \frac{16Q^2}{\pi^2 D^4}\) gives another, equivalent form of Darcy-Weisbach which uses flow, \(Q\), instead of velocity, \(\bar v\):
\begin{equation}\label{equation:Review/Review_Fluid_Mechanics:Review/Review_Fluid_Mechanics:5}
\begin{split}h_{\rm{f}} \, = \,{\rm{f}} \frac{8}{g \pi^2} \frac{LQ^2}{D^5}\end{split}
\end{equation}
\begin{DUlineblock}{0em}
\item[] Such that:
\item[] \(h_{\rm{f}}\) = major loss
\item[] \(\rm{f}\) = Darcy friction factor
\item[] \(L\) = pipe length
\item[] \(Q\) = pipe flow rate
\item[] \(D\) = pipe diameter
\end{DUlineblock}


\sphinxstrong{See also:}


\sphinxstylestrong{Function in aide\_design:} \sphinxcode{\sphinxupquote{pc.headloss\_fric(FlowRate, Diam, Length, Nu, PipeRough)}} Returns only major losses. Works for both laminar and turbulent flow. PipeRough describes the pipe roughness \(\epsilon\) described shortly below.



Darcy-Weisbach is wonderful because it applies to both laminar and turbulent flow regimes and contains relatively easy to measure variables. The one exception is the Darcy friction factor, \(\rm{f}\). This parameter is an approximation for the magnitude of friction between the pipe walls and the fluid, and its value changes depending on the whether or not the flow is laminar or turbulent, and varies with the Reynolds number in both flow regimes.

For laminar flow, the friction factor can be determined from the following equation:
\begin{equation}\label{equation:Review/Review_Fluid_Mechanics:Review/Review_Fluid_Mechanics:6}
\begin{split}{\rm{f}} = \frac{64}{\rm{Re}}\end{split}
\end{equation}
For turbulent flow, the friction factor is more difficult to determine. In this class, we will use the \sphinxhref{https://en.wikipedia.org/wiki/Darcy\_friction\_factor\_formulae\#Swamee\%E2\%80\%93Jain\_equation}{Swamee-Jain equation}:
\begin{equation}\label{equation:Review/Review_Fluid_Mechanics:swamee_jain}
\begin{split}  {\rm{f}} = \frac{0.25} {\left[ \log \left( \frac{\epsilon }{3.7D} + \frac{5.74}{{\rm Re}^{0.9}} \right) \right]^2}\end{split}
\end{equation}
\begin{DUlineblock}{0em}
\item[] Such that:
\item[] \(\epsilon\) = pipe roughness, \([L]\)
\item[] \(D\) = pipe diameter, \([L]\)
\end{DUlineblock}


\sphinxstrong{See also:}


\sphinxstylestrong{Function in aide\_design:} \sphinxcode{\sphinxupquote{pc.fric(FlowRate, Diam, Nu, PipeRough)}} Returns \(\rm{f}\) for laminar \sphinxstyleemphasis{or} turbulent flow. For laminar flow, use zero for the \sphinxcode{\sphinxupquote{PipeRough}} input.



The simplicity of the equation for \(\rm{f}\) during laminar flow allows for substitutions to create a very useful, simplified equation for major losses during laminar flow. This simplification combines the Darcy-Weisbach equation, the equation for the Darcy friction factor during laminar flow, and the Reynold’s number formula:
\begin{equation}\label{equation:Review/Review_Fluid_Mechanics:Review/Review_Fluid_Mechanics:7}
\begin{split}h_{\rm{f}} \, = \,{\rm{f}} \frac{8}{g \pi^2} \frac{LQ^2}{D^5}\end{split}
\end{equation}\begin{equation}\label{equation:Review/Review_Fluid_Mechanics:Review/Review_Fluid_Mechanics:8}
\begin{split}{\rm{f}} = \frac{64}{\rm{Re}}\end{split}
\end{equation}\begin{equation}\label{equation:Review/Review_Fluid_Mechanics:Review/Review_Fluid_Mechanics:9}
\begin{split}{\rm{Re}}=\frac{4Q}{\pi D\nu}\end{split}
\end{equation}
To form the \sphinxhref{https://en.wikipedia.org/wiki/Hagen\%E2\%80\%93Poiseuille\_equation}{Hagen-Poiseuille equation} for major losses during laminar flow, and \sphinxstyleemphasis{only} during laminar flow:
\begin{equation}\label{equation:Review/Review_Fluid_Mechanics:hagen_poiseuille}
\begin{split}  h_{\rm{f}} = \frac{128\mu L Q}{\rho g\pi D^4}\end{split}
\end{equation}\begin{equation}\label{equation:Review/Review_Fluid_Mechanics:Review/Review_Fluid_Mechanics:10}
\begin{split}h_{\rm{f}} = \frac{32\nu L\bar v}{ g D^2}\end{split}
\end{equation}
The significance of this equation lies in its relationship between \(h_{\rm{f}}\) and \(Q\). Hagen-Poiseuille shows that the terms are directly proportional (\(h_{\rm{f}} \propto Q\)) during laminar flow, while Darcy-Weisbach shows that \(h_{\rm{f}}\) grows with the square of \(Q\) during turbulent flow (\(h_{\rm{f}} \propto Q^2\)). As you will soon see, minor losses, \(h_e\), will grow with the square of \(Q\) in both laminar and turbulent flow. This has implications that will be discussed in a future chapter: {\hyperref[\detokenize{Flow_Control_and_Measurement/FCM_Design:title-flow-control-design}]{\sphinxcrossref{\DUrole{std,std-ref}{Flow Control and Measurement Design}}}}.

In 1944, Lewis Ferry Moody plotted a ridiculous amount of experimental data, gathered by many people, on the Darcy-Weisbach friction factor to create what we now call the \sphinxhref{https://en.wikipedia.org/wiki/Moody\_chart}{Moody diagram}. This diagram has makes it easy to find the friction factor \(f\). \(\rm{f}\) is plotted on the left-hand y-axis, relative pipe roughness \(\frac{\epsilon}{D}\) is on the right-hand y-axis, and Reynolds number \(\rm{Re}\) is on the x-axis. The Moody diagram is an alternative to computational methods for finding \(\rm{f}\).

\begin{figure}[htbp]
\centering
\capstart

\noindent\sphinxincludegraphics[width=650\sphinxpxdimen]{{Moody}.jpg}
\caption{This is the famous and famously useful Moody diagram.}\label{\detokenize{Review/Review_Fluid_Mechanics:id6}}\label{\detokenize{Review/Review_Fluid_Mechanics:figure-moody}}\end{figure}


\subsection{Minor Losses}
\label{\detokenize{Review/Review_Fluid_Mechanics:minor-losses}}\label{\detokenize{Review/Review_Fluid_Mechanics:heading-minor-losses}}
Unfortunately, there is no simple ‘pushing a box across the ground’ example to explain minor losses. So instead, consider a \sphinxhref{https://www.youtube.com/watch?v=5spXXZX55C8}{hydraulic jump}. In the video, you can see lots of turbulence and eddies in the transition region between the fast, shallow flow and the slow, deep flow. The high amount of mixing of the water in the transition region of the hydraulic jump results in significant friction \sphinxstyleemphasis{between water and water}. This turbulent, eddy-induced, fluid-fluid friction results in  minor losses, much like fluid-pipe friction results in major losses.

As occurs in a hydraulic jump, a flow expansion (from shallow flow to deep flow) creates the turbulent eddies that result in minor losses. This will be a recurring theme in throughout the course: \sphinxstylestrong{minor losses are caused by flow expansions}. Imagine a pipe fitting that connects a small diameter pipe to a large diameter one, as shown in \hyperref[\detokenize{Review/Review_Fluid_Mechanics:figure-minor-loss-pipe-frd}]{Fig.\@ \ref{\detokenize{Review/Review_Fluid_Mechanics:figure-minor-loss-pipe-frd}}} below. The flow must expand to fill up the entire large diameter pipe. This expansion creates turbulent eddies near the union between the small and large pipes, and these eddies result in minor losses. You may already know the equation for minor losses, but understanding where it comes from is very important for effective AguaClara plant design. For this reason, you are strongly recommended to read through its full derivation: {\hyperref[\detokenize{Review/Review_Fluid_Mechanics_Derivations:title-review-fluid-mechanics-derivations}]{\sphinxcrossref{\DUrole{std,std-ref}{Review: Fluid Mechanics Derivations}}}}.

There are three forms of the minor loss equation that you will see in this class:
\begin{equation}\label{equation:Review/Review_Fluid_Mechanics:Review/Review_Fluid_Mechanics:11}
\begin{split}{\rm{ \mathbf{First \, form:} }} \quad h_e = \frac{\left( \bar v_{in}  - \bar v_{out} \right)^2}{2g}\end{split}
\end{equation}\begin{equation}\label{equation:Review/Review_Fluid_Mechanics:Review/Review_Fluid_Mechanics:12}
\begin{split}{\rm{ \mathbf{Second \, form:} }} \quad h_e = \left( 1 - \frac{A_{in}}{A_{out}} \right)^2 \, \frac{\bar v_{in}^2}{2g} \, \, = \, \, K_e^{'} \frac{\bar v_{in}^2}{2g}, \quad {\rm where} \quad K_e^{'} = \left( 1 - \frac{A_{in}}{A_{out}} \right)^2\end{split}
\end{equation}\begin{equation}\label{equation:Review/Review_Fluid_Mechanics:Review/Review_Fluid_Mechanics:13}
\begin{split}\color{purple}{
{\rm{ \mathbf{Third \, form:} }} \quad h_e = \left( \frac{A_{out}}{A_{in}} -1 \right)^2 \, \frac{\bar  v_{out}^2}{2g} \, \, = \, \, K_e \frac{\bar v_{out}^2}{2g}, \quad {\rm where} \quad K_e = \left( \frac{A_{out}}{A_{in}} - 1 \right)^2
}\end{split}
\end{equation}
\begin{DUlineblock}{0em}
\item[] Such that:
\item[] \(K_e^{'}, \,\, K_e\) = minor loss coefficients, dimensionless
\end{DUlineblock}

\begin{sphinxadmonition}{note}{Note:}
You will most often see \(K_e^{'}\) and \(K_e\) used without the \(e\) subscript,  as \(K^{'}\) and \(K\).
\end{sphinxadmonition}


\sphinxstrong{See also:}


\sphinxstylestrong{Function in aide\_design:} \sphinxcode{\sphinxupquote{pc.headloss\_exp\_general(Vel, KMinor)}} Returns \(h_e\). Can be either the second or third form due to user input of both velocity and minor loss coefficient. It is up to the user to use consistent \(\bar v\) and \(K_e\).




\sphinxstrong{See also:}


\sphinxstylestrong{Function in aide\_design:} \sphinxcode{\sphinxupquote{pc.headloss\_exp(FlowRate, Diam, KMinor)}} Returns \(h_e\). Uses third form, \(K_e\).



\begin{figure}[htbp]
\centering
\capstart

\noindent\sphinxincludegraphics[width=650\sphinxpxdimen]{{minor_loss_pipe}.png}
\caption{The \(in\) and \(out\) subscripts in each of the three forms of the minor loss equation refer to this diagram that was used for the derivation.}\label{\detokenize{Review/Review_Fluid_Mechanics:id7}}\label{\detokenize{Review/Review_Fluid_Mechanics:figure-minor-loss-pipe-frd}}\end{figure}

The second and third forms are the ones which you are probably most familiar with. The distinction between them, however, is critical. First, consider the magnitudes of \(A_{in}\) and \(A_{out}\). \(A_{in}\) can never be larger than \(A_{out}\), because the flow is expanding. When flow expands, the cross-sectional area it flows through must increase. As a result, both \(\frac{A_{out}}{A_{in}} > 1\) and \(\frac{A_{in}}{A_{out}} < 1\) must always be true. This means that \(K^{'}\) can never be greater than 1, while \(K\) technically has no upper limit.

If you have taken CEE 3310, you have seen tables of minor loss coefficients \sphinxhref{https://www.engineeringtoolbox.com/minor-loss-coefficients-pipes-d\_626.html}{like this
one}, and they almost all have coefficients greater than 1. This implies that these tables use the third form of the minor loss equation as we have defined it, where the velocity is \(\bar v_{out}\). There is a good reason for using the third form over the second one: \(\bar v_{out}\) is far easier to determine than \(\bar v_{in}\). Consider flow through a pipe elbow, as shown in the image below.

\begin{figure}[htbp]
\centering
\capstart

\noindent\sphinxincludegraphics[width=650\sphinxpxdimen]{{minor_loss_elbow}.png}
\caption{Flow around a pipe elbow results in a minor loss. ‘Control surface 1’ can be abbreviated as ‘CS 1’}\label{\detokenize{Review/Review_Fluid_Mechanics:id8}}\label{\detokenize{Review/Review_Fluid_Mechanics:figure-minor-loss-elbow}}\end{figure}

In order to find \(\bar v_{out}\), we first need to know what (or where) is \(out\) and what is \(in\). A simple way to distinguish the two surfaces is that \(in\) occurs when the flow is most contracted, and \(out\) occurs when the flow has fully expanded after that maximal contraction. Going on these guidelines, Control surface ‘2’ (CS 2) in the figure above above would be \(in\), since it represents the most contracted flow in the elbow-pipe system. Therefore, CS 3 would be \(out\), as it represents the flow having fully expanded after its compression at CS 2.

\(\bar v_{out}\) is easy to determine because it is the velocity of the fluid as it flows through the entire area of the pipe. Thus, \(\bar v_{out}\) can be found with the continuity equation, since the flow through the pipe and its diameter are easy to measure, \(\bar v_{out} = \frac{4 Q}{\pi D^2}\). On the other hand, \(\bar v_{in}\) is difficult to find, as the area of the contracted flow is dependent on the exact geometry of the elbow. This is why the third form of the minor loss equation, as we have defined it, is the most common:
\begin{equation}\label{equation:Review/Review_Fluid_Mechanics:Review/Review_Fluid_Mechanics:14}
\begin{split}h_e = K \frac{\bar v_{out}^2}{2g} = \,\,\,\, \left( \frac{A_{out}}{A_{in}} -1 \right)^2 \frac{\bar v_{out}^2}{2g}\end{split}
\end{equation}
\begin{sphinxadmonition}{note}{Note:}
When considering a hydraulic system within a control volume, there can be many sources of minor losses. Instead of saying \(h_e = K_1 \frac{\bar v_{out}^2}{2g} + K_2 \frac{\bar v_{out}^2}{2g} + ...\) we can simply lump all of the minor loss coefficients into one: \(\sum K = K_1 + K_2 + ...\). Thus, it is also common to see this form of the minor loss equation when finding the minor loss across control volumes: \(\sum K \frac{v_{out}^2}{2g}\).
\end{sphinxadmonition}


\subsection{Head Loss = Elevation Difference Trick}
\label{\detokenize{Review/Review_Fluid_Mechanics:head-loss-elevation-difference-trick}}\label{\detokenize{Review/Review_Fluid_Mechanics:heading-head-loss-elevation-difference-trick}}
This trick, also called the ‘control volume trick,’ or more colloquially, the ‘head loss trick,’ is incredibly useful for simplifying hydraulic systems and is used all the time in this class.

Consider the following figure:

\begin{figure}[htbp]
\centering
\capstart

\noindent\sphinxincludegraphics[width=650\sphinxpxdimen]{{head_loss_trick}.png}
\caption{A typical hydraulic system can be used to understand the head loss trick.}\label{\detokenize{Review/Review_Fluid_Mechanics:id9}}\label{\detokenize{Review/Review_Fluid_Mechanics:figure-head-loss-trick}}\end{figure}

In systems like this, where an elevation difference is causing water to flow, the elevation difference is called the \sphinxstylestrong{driving head}. In the system above, the driving head is the elevation difference between the water level and the end of the tubing. Usually, driving head is written as \(\Delta z\) or \(\Delta h\), though above it is labelled as \(h_L\). Doesn’t \(h_L\) refer to head loss though? Yes it does! Referring to \(\Delta h\) or \(\Delta z\) \sphinxstyleemphasis{IS} the head loss trick, and how it works is explained in the following paragraphs and equations.

The figure is technically violating the energy equation by saying that the elevation difference between the water in the tank and the end of the tube is \(h_L\). It implies that all of the driving head, \(\Delta z\), is lost to head loss. Since all of the energy is gone, there should not be water flowing out of the tubing. But there is. Let’s apply the energy equation across the control surfaces shown in the figure. Pressures at both ends are atmospheric and the velocity of water at the top of tank is negligible.
\begin{equation}\label{equation:Review/Review_Fluid_Mechanics:Review/Review_Fluid_Mechanics:15}
\begin{split}\cancel{ \frac{p_{1}}{\rho g} } + z_{1} + \cancel{ \frac{\bar v_{1}^2}{2g} } = \cancel{ \frac{p_{2}}{\rho g} } + z_{2} + \frac{\bar v_{2}^2}{2g} + h_L\end{split}
\end{equation}
We now get:
\begin{equation}\label{equation:Review/Review_Fluid_Mechanics:Review/Review_Fluid_Mechanics:16}
\begin{split}\Delta z = \frac{\bar v_2^2}{2g} + h_L\end{split}
\end{equation}
This equation contradicts the figure above, which says that \(\Delta z = h_L\) and neglects \(\frac{\bar v_2^2}{2g}\). The figure above is correct, however, if you apply the head loss trick. The trick incorporates the \(\frac{\bar v_2^2}{2g}\) term \sphinxstyleemphasis{into} the \(h_L\) term as a minor loss. See the math below:
\begin{equation}\label{equation:Review/Review_Fluid_Mechanics:Review/Review_Fluid_Mechanics:17}
\begin{split}\Delta z = \frac{\bar v_2^2}{2g} + h_e + h_f\end{split}
\end{equation}\begin{equation}\label{equation:Review/Review_Fluid_Mechanics:Review/Review_Fluid_Mechanics:18}
\begin{split}\Delta z = \frac{\bar v_2^2}{2g} + \left( \sum K \right) \frac{\bar v_2^2}{2g} + h_f\end{split}
\end{equation}\begin{equation}\label{equation:Review/Review_Fluid_Mechanics:Review/Review_Fluid_Mechanics:19}
\begin{split}\Delta z = \left( 1 + \sum K \right) \frac{\bar v_2^2}{2g} + h_f\end{split}
\end{equation}\begin{equation}\label{equation:Review/Review_Fluid_Mechanics:Review/Review_Fluid_Mechanics:20}
\begin{split}\Delta z = \left( \sum K \right) \frac{\bar v_2^2}{2g} + h_f\end{split}
\end{equation}
This last step incorporated the kinetic energy term of the energy equation, \(\frac{\bar v_2^2}{2g}\), into the minor loss equation by saying that its \(K\) is 1 and incorporating that 1 into \(\sum K\). From here, we reverse our steps to get \(\Delta z = h_L\), starting with \(h_e = \left( \sum K \right) \frac{\bar v_2^2}{2g}\)
\begin{equation}\label{equation:Review/Review_Fluid_Mechanics:Review/Review_Fluid_Mechanics:21}
\begin{split}\Delta z = h_e + h_f\end{split}
\end{equation}\begin{equation}\label{equation:Review/Review_Fluid_Mechanics:Review/Review_Fluid_Mechanics:22}
\begin{split}\Delta z = h_L\end{split}
\end{equation}
By applying the head loss trick, you are considering the entire flow of the fluid out of a control volume as energy lost via minor losses. This is just an algebraic trick, the only thing to remember when applying this trick is that \(\sum K\) will always be at least 1, even if there are no ‘real’ minor losses in the system.


\section{The Orifice Equation}
\label{\detokenize{Review/Review_Fluid_Mechanics:the-orifice-equation}}\label{\detokenize{Review/Review_Fluid_Mechanics:heading-the-orifice-equation}}
This equation is one that you’ll see and use again and again throughout this class. Understanding it now will be invaluable, as future concepts will use and build on this equation.


\subsection{What is a Vena Contracta?}
\label{\detokenize{Review/Review_Fluid_Mechanics:what-is-a-vena-contracta}}\label{\detokenize{Review/Review_Fluid_Mechanics:heading-what-is-a-vena-contracta}}
Before describing the equation, we must first understand the concept of a \sphinxhref{https://en.wikipedia.org/wiki/Vena\_contracta}{vena contracta}. Refer to the figure below.

\begin{figure}[htbp]
\centering
\capstart

\noindent\sphinxincludegraphics[width=650\sphinxpxdimen]{{sluice_gate_vena_contracta}.png}
\caption{This figure shows flow around a sluice gate. Since streamlines can’t make sharp turns, the flow is forced to gradually curve and contract to an area smaller than the area of the gate.}\label{\detokenize{Review/Review_Fluid_Mechanics:id10}}\label{\detokenize{Review/Review_Fluid_Mechanics:figure-sluice-gate-vena-contracta}}\end{figure}

The flow contracts as the fluid moves past the gate. This happens because the fluid can’t make a sharp turn as it tries to go around the gate, as indicated by the streamline in the figure. Instead, the most extreme streamline makes a gradual change in direction. As a result of this gradual turn, the flow contracts and the cross-sectional area the fluid is flowing decreases.

The term ‘vena contracta’ describes the phenomenon of contracting flow due to streamlines being unable to make sharp turns. \(\Pi_{vc}\) is a dimensionless ratio comparing the flow area at the point of maximal contraction, \(A_{downstream}\), and the flow area \sphinxstyleemphasis{before} the contraction, \(A_{gate}\). In the figure above, the equation for the vena contracta coefficient would be:
\begin{equation}\label{equation:Review/Review_Fluid_Mechanics:Review/Review_Fluid_Mechanics:23}
\begin{split}\Pi_{vc} = \frac{A_{downstream}}{A_{gate}}\end{split}
\end{equation}
When the most extreme turn a streamline must make is 90°, the value of the vena contracta coefficient is close to 0.62. This parameter value, 0.62, is in aide\_design as \sphinxcode{\sphinxupquote{pc.RATIO\_VC\_ORIFICE}}. The vena contracta coefficient value is a function of the flow geometry. Since the ratio always puts the most contracted area over the least contracted area, \(\Pi_{vc}\) is always less than 1.

\begin{sphinxadmonition}{important}{Important:}
\sphinxstylestrong{A vena contracta coefficient is not a minor loss coefficient.} Though the equations for the two both involve contracted and non-contracted areas, these coefficients are not the same. Minor losses coefficients imply energy loss, and vena contractas do not. Minor losses coefficients deal with flow expansions, and vena contracas deal with flow contractions. Confusing the two coefficients is common mistake that this paragraph will hopefully help you to avoid.
\end{sphinxadmonition}

\begin{sphinxadmonition}{note}{Note:}
Note that what this class calls \(\Pi_{vc}\) is often referred to as a ‘Coefficient of Contraction,’ \(C_c\), in other engineering courses and settings.
\end{sphinxadmonition}


\subsection{Origin of the Orifice Equation}
\label{\detokenize{Review/Review_Fluid_Mechanics:origin-of-the-orifice-equation}}
The orifice equation is derived from the Bernoulli equation as applied to the purple points in the following image:

\begin{figure}[htbp]
\centering
\capstart

\noindent\sphinxincludegraphics[width=650\sphinxpxdimen]{{hole_in_a_bucket}.png}
\caption{Flow through a hole in the bottom of a bucket is a great example of the orifice equation.}\label{\detokenize{Review/Review_Fluid_Mechanics:id11}}\label{\detokenize{Review/Review_Fluid_Mechanics:figure-hole-in-a-bucket}}\end{figure}

At point 1, the pressure is atmospheric and the instantaneous velocity is negligible as the water level in the bucket drops slowly. At point 2, the pressure is also atmospheric. We define the difference in elevations between the two points, \(z_1 - z_2\), to be \(\Delta h\). With these simplifications \((p_1 = \bar v_1 = p_2 = 0)\) and assumptions \((z_A - z_B = \Delta h)\), the Bernoulli equation becomes:
\begin{equation}\label{equation:Review/Review_Fluid_Mechanics:Review/Review_Fluid_Mechanics:24}
\begin{split}\Delta h = \frac{\bar v_2^2}{2g}\end{split}
\end{equation}
Substituting the continuity equation \(Q = \bar v A\) in the form of \(\bar v_2^2 = \frac{Q^2}{A_{vc}^2}\), the vena contracta coefficient in the form of \(A_{vc} = \Pi_{vc} A_{or}\) yields:
\begin{equation}\label{equation:Review/Review_Fluid_Mechanics:Review/Review_Fluid_Mechanics:25}
\begin{split}\Delta h = \frac{Q^2}{2g \Pi_{vc}^2 A_{or}^2}\end{split}
\end{equation}
Which, rearranged to solve for \(Q\) gives \sphinxstylestrong{The Orifice Equation:}
\begin{equation}\label{equation:Review/Review_Fluid_Mechanics:orifice_equation}
\begin{split}  Q = \Pi_{vc} A_{or} \sqrt{2g\Delta h}\end{split}
\end{equation}
\begin{DUlineblock}{0em}
\item[] Such that:
\item[] \(\Pi_{vc}\) = 0.62 = vena contracta coefficient, in aide\_design as \sphinxcode{\sphinxupquote{pc.RATIO\_VC\_ORIFICE}}
\item[] \(A_{or}\) = orifice area- NOT contracted flow area
\item[] \(\Delta h\) = elevation difference between orifice and water level
\end{DUlineblock}


\sphinxstrong{See also:}


\sphinxstylestrong{Equation in aide\_design:} \sphinxcode{\sphinxupquote{pc.flow\_orifice(Diam, Height, RatioVCOrifice)}} Returns flow through a horizontal orifice.




\sphinxstrong{See also:}


\sphinxstylestrong{Equation in aide\_design:} \sphinxcode{\sphinxupquote{pc.flow\_orifice\_vert(Diam, Height, RatioVCOrifice)}} Returns flow through a vertical orifice. The height parameter refers to height above the center of the orifice.



There are two configurations for an orifice in the tank holding a fluid: horizontal and vertical. These are both displayed in the figure below. The orifice equation written is for a horizontal orifice; the equation for flow through vertical orifice equation requires integration or the orifice equation across its height to return the correct flow. This is explored in the Flow Control and Measurement Examples section.

\begin{figure}[htbp]
\centering
\capstart

\noindent\sphinxincludegraphics[width=650\sphinxpxdimen]{{vertical_and_horizontal_orifices}.png}
\caption{The descriptions ‘vertical’ and ‘horizontal’ \sphinxstylestrong{apply to the orientation of the orifices,} not to the orientation of the fluid coming out of the orifices.}\label{\detokenize{Review/Review_Fluid_Mechanics:id12}}\label{\detokenize{Review/Review_Fluid_Mechanics:figure-vertical-and-horizontal-orifices}}\end{figure}


\section{Section Summary}
\label{\detokenize{Review/Review_Fluid_Mechanics:section-summary}}\label{\detokenize{Review/Review_Fluid_Mechanics:heading-fr-section-summary}}\begin{enumerate}
\item {} 
\sphinxstylestrong{Introductory Concepts:}
\begin{quote}
\begin{itemize}
\item {} 
\sphinxstylestrong{Continuity} means that the mass of a fluid is conserved as it flows, and implies a constant density. The continuity equation has two purposes:
\begin{quote}
\begin{enumerate}
\item {} 
Relating the average velocity of a fluid, \(\bar v\), to its flow rate, \(Q\), via the cross-sectional area, \(A\), that it flows through. When the fluid is flowing in a pipe, we can simply this even further to relate the flow rate and velocity to the pipe’s diameter, \(D\). The final equation below is only used for circular pipes, as it includes a pipe diameter.

\end{enumerate}
\begin{equation}\label{equation:Review/Review_Fluid_Mechanics:Review/Review_Fluid_Mechanics:26}
\begin{split}Q = \bar v A = \bar v \frac{\pi D^2}{4}\end{split}
\end{equation}\begin{enumerate}
\item {} 
Finding the average velocity or flow when the geometry of a fluid’s flow changes, as the mass of the fluid must be conserved when it transitions through flow geometries.

\end{enumerate}
\begin{equation}\label{equation:Review/Review_Fluid_Mechanics:Review/Review_Fluid_Mechanics:27}
\begin{split}Q_1 = Q_2\end{split}
\end{equation}\begin{equation}\label{equation:Review/Review_Fluid_Mechanics:Review/Review_Fluid_Mechanics:28}
\begin{split}\bar v_1 A_1 = \bar v_2 A_2\end{split}
\end{equation}\begin{equation}\label{equation:Review/Review_Fluid_Mechanics:Review/Review_Fluid_Mechanics:29}
\begin{split}\bar v_1 \frac{\pi D_1^2}{4} = \bar v_2 \frac{\pi D_2^2}{4}\end{split}
\end{equation}\end{quote}

\item {} 
\sphinxstylestrong{Laminar and Turbulent flow} describe the disorder and chaos of fluid flow. The \sphinxstylestrong{Reynolds number,} \({\rm Re}\) is used to distinguish laminar from turbulent flow. For \({\rm Re} < 2100\), flow is considered laminar. For \({\rm Re} > 2100\), flow is considered turbulent. The equations for the Reynolds number are below:

\end{itemize}
\begin{equation}\label{equation:Review/Review_Fluid_Mechanics:Review/Review_Fluid_Mechanics:30}
\begin{split}{\rm Re} = \frac{\bar vD}{\nu} = \frac{4Q}{\pi D\nu} = \frac{\rho \bar vD}{\mu}\end{split}
\end{equation}\begin{itemize}
\item {} 
\sphinxstylestrong{Control volumes vs Streamlines.} This section is quite short, a summary would simply repeat what the sections says. The section is its own summary; read it here: {\hyperref[\detokenize{Review/Review_Fluid_Mechanics:streamlines-and-control-volumes}]{\sphinxcrossref{Streamlines and Control Volumes}}}

\end{itemize}
\end{quote}

\item {} 
\sphinxstylestrong{Bernoulli vs Energy equations:} The Bernoulli equation assumes that energy is conserved throughout a streamline or control volume. The Energy equation assumes that there is energy loss, or head loss \(h_L\). This head loss is composed of major losses, \(h_{\rm{f}}\), and minor losses, \(h_e\).

\end{enumerate}
\begin{quote}

Bernoulli equation:
\begin{equation}\label{equation:Review/Review_Fluid_Mechanics:Review/Review_Fluid_Mechanics:31}
\begin{split}\frac{p_1}{\rho g} + {z_1} + \frac{\bar v_1^2}{2g} = \frac{p_2}{\rho g} + {z_2} + \frac{\bar v_2^2}{2g}\end{split}
\end{equation}
Energy equation, simplified to remove pumps, turbines, and \(\alpha\) factors:
\begin{equation}\label{equation:Review/Review_Fluid_Mechanics:Review/Review_Fluid_Mechanics:32}
\begin{split}\frac{p_{1}}{\rho g} + z_{1} + \frac{\bar v_{1}^2}{2g} = \frac{p_{2}}{\rho g} + z_{2} + \frac{\bar v_{2}^2}{2g} + h_L\end{split}
\end{equation}\end{quote}
\begin{enumerate}
\setcounter{enumi}{2}
\item {} 
\sphinxstylestrong{Major losses:} Defined as the energy loss due to shear between the walls of the pipe/flow conduit and the fluid. The Darcy-Weisbach equation is used to find major losses in both laminar and turbulent flow regimes. The equation for finding the Darcy friction factor, \(\rm{f}\), changes depending on whether the flow is laminar or turbulent. The Moody diagram is a common graphical method for finding \(\rm{f}\). During laminar flow, the Hagen-Poiseuille equation, which is just a combination of Darcy-Weisbach, Reynolds number, and \({\rm{f}} = \frac{64}{\rm{Re}}\), can be used

\end{enumerate}
\begin{quote}

Darcy-Weisbach equation:
\begin{equation}\label{equation:Review/Review_Fluid_Mechanics:Review/Review_Fluid_Mechanics:33}
\begin{split}h_{\rm{f}} = {\rm{f}} \frac{L}{D} \frac{\bar v^2}{2g}\end{split}
\end{equation}
For water treatment plant design we tend to use plant flow rate, \(Q\), as our master variable and thus we have.
\begin{equation}\label{equation:Review/Review_Fluid_Mechanics:Review/Review_Fluid_Mechanics:34}
\begin{split}h_{\rm{f}} = {\rm{f}} \frac{8}{g \pi^2} \frac{LQ^2}{D^5}\end{split}
\end{equation}
\(\rm{f}\) for laminar flow:
\begin{equation}\label{equation:Review/Review_Fluid_Mechanics:Review/Review_Fluid_Mechanics:35}
\begin{split}{\rm{f}} = \frac{64}{\rm{Re}} = \frac{16 \pi D \nu}{Q} = \frac{64 \nu}{\bar v D}\end{split}
\end{equation}
\(\rm{f}\) for turbulent flow:
\begin{equation}\label{equation:Review/Review_Fluid_Mechanics:Review/Review_Fluid_Mechanics:36}
\begin{split}{\rm{f}} = \frac{0.25} {\left[ \log \left( \frac{\epsilon }{3.7D} + \frac{5.74}{{\rm Re}^{0.9}} \right) \right]^2}\end{split}
\end{equation}
Hagen-Poiseuille equation for laminar flow:
\begin{equation}\label{equation:Review/Review_Fluid_Mechanics:Review/Review_Fluid_Mechanics:37}
\begin{split}h_{\rm{f}} = \frac{32\mu L \bar v}{\rho gD^2} = \frac{128\mu Q}{\rho g\pi D^4}\end{split}
\end{equation}\end{quote}
\begin{enumerate}
\setcounter{enumi}{3}
\item {} 
\sphinxstylestrong{Minor losses:} Defined as the energy loss due to the generation of turbulent eddies when flow expands. Once more: minor losses are caused by flow expansions. There are three forms of the minor loss equation, two of which look the same but use different coefficients (\(K^{'}\) vs \(K\)) and velocities (\(\bar v_{in}\) vs \(\bar v_{out}\)). \sphinxstyleemphasis{Make sure the coefficient you select is consistent with the velocity you use}. The third form, written in purple, is the most commonly used form of the minor loss equation.

\end{enumerate}
\begin{equation}\label{equation:Review/Review_Fluid_Mechanics:Review/Review_Fluid_Mechanics:38}
\begin{split}{\rm{ \mathbf{First \, form:} }} \quad h_e = \frac{\left( \bar v_{in}  - \bar v_{out} \right)^2}{2g}\end{split}
\end{equation}\begin{equation}\label{equation:Review/Review_Fluid_Mechanics:Review/Review_Fluid_Mechanics:39}
\begin{split}{\rm{ \mathbf{Second \, form:} }} \quad h_e = \left( 1 - \frac{A_{in}}{A_{out}} \right)^2 \, \frac{\bar v_{in}^2}{2g} \, \, = \, \, K_e^{'} \frac{\bar v_{in}^2}{2g}, \quad {\rm where} \quad K_e^{'} = \left( 1 - \frac{A_{in}}{A_{out}} \right)^2\end{split}
\end{equation}\begin{equation}\label{equation:Review/Review_Fluid_Mechanics:Review/Review_Fluid_Mechanics:40}
\begin{split}\color{purple}{
{\rm{ \mathbf{Third \, form:} }} \quad h_e = \left( \frac{A_{out}}{A_{in}} -1 \right)^2 \, \frac{\bar  v_{out}^2}{2g} \, \, = \, \, K_e \frac{\bar v_{out}^2}{2g}, \quad {\rm where} \quad K_e = \left( \frac{A_{out}}{A_{in}} - 1 \right)^2
}\end{split}
\end{equation}\begin{enumerate}
\setcounter{enumi}{4}
\item {} 
\sphinxstylestrong{Major and minor losses vary with flow:} While it is generally important to know how increasing or decreasing flow will affect head loss, it is even more important for this class to understand exactly how flow will affect head loss. As the table below shows, head loss will always be proportional to flow squared during turbulent flow. During laminar flow, however, the exponent on \(Q\) will be between 1 and 2 depending on the proportion of major to minor losses.

\end{enumerate}


\begin{savenotes}\sphinxattablestart
\centering
\sphinxcapstartof{table}
\sphinxcaption{Proportionality between head loss \(h_L\) and flow rate \(Q\) for different flow regimes and types of head loss.}\label{\detokenize{Review/Review_Fluid_Mechanics:id13}}\label{\detokenize{Review/Review_Fluid_Mechanics:table-h-q-proportionality}}
\sphinxaftercaption
\begin{tabular}[t]{|\X{10}{30}|\X{10}{30}|\X{10}{30}|}
\hline

\(h_L propto Q^?\)
&\sphinxstyletheadfamily 
Major Losses
&\sphinxstyletheadfamily 
Minor Losses
\\
\hline
Laminar
&
\(Q\)
&
\(Q^2\)
\\
\hline
Turbulent
&
\(Q^2\)
&
\(Q^2\)
\\
\hline
\end{tabular}
\par
\sphinxattableend\end{savenotes}
\begin{enumerate}
\setcounter{enumi}{5}
\item {} 
The \sphinxstylestrong{head loss trick}, also called the control volume trick, can be used to incorporate the ‘kinetic energy out’ term of the energy equation, \(\frac{\bar v_2^2}{2g}\), into head loss as a minor loss with \(K = 1\), so the minor loss equation becomes \(\left( 1 + \sum K \right) \frac{\bar v^2}{2g}\). This is used to be able to say that \(\Delta z = h_L\) and makes many equation simplifications possible in the future.

\item {} 
\sphinxstylestrong{Orifice equation and vena contractas:} The orifice equation is used to determine the flow out of an orifice given the elevation of water above the orifice. This equation introduces the concept of vena contracta, which describes flow contraction due to the inability of streamlines to make sharp turns. The equation shows that the flow out of an orifice is proportional to the square root of the driving head, \(Q \propto \sqrt{\Delta h}\). Depending on the orientation of the orifice, vertical (like a hole in the side of a bucket) or horizontal (like a hole in the bottom of a bucket), a different equation in aide\_design should be used.

\end{enumerate}
\begin{quote}

The Orifice Equation:
\begin{equation}\label{equation:Review/Review_Fluid_Mechanics:Review/Review_Fluid_Mechanics:41}
\begin{split}Q = \Pi_{vc} A_{or} \sqrt{2g\Delta h}\end{split}
\end{equation}\end{quote}


\chapter{Review: Fluid Mechanics Derivations}
\label{\detokenize{Review/Review_Fluid_Mechanics_Derivations:review-fluid-mechanics-derivations}}\label{\detokenize{Review/Review_Fluid_Mechanics_Derivations:title-review-fluid-mechanics-derivations}}\label{\detokenize{Review/Review_Fluid_Mechanics_Derivations::doc}}

\section{Minor Loss Equation}
\label{\detokenize{Review/Review_Fluid_Mechanics_Derivations:minor-loss-equation}}\label{\detokenize{Review/Review_Fluid_Mechanics_Derivations:heading-minor-loss-equation-derivation}}
This section contains the derivation of the minor loss equation using the following figure as a reference. The derivation begins with a slightly simplified energy equation across the control volume shown. Our energy equation begins with \(h_P\) and \(h_T\) having been
eliminated.

\begin{figure}[htbp]
\centering
\capstart

\noindent\sphinxincludegraphics[width=700\sphinxpxdimen]{{minor_loss_pipe}.png}
\caption{This is the system we will use to derive the minor loss equation.}\label{\detokenize{Review/Review_Fluid_Mechanics_Derivations:id1}}\label{\detokenize{Review/Review_Fluid_Mechanics_Derivations:figure-minor-loss-pipe}}\end{figure}
\begin{equation}\label{equation:Review/Review_Fluid_Mechanics_Derivations:Review/Review_Fluid_Mechanics_Derivations:0}
\begin{split}\frac{p_{in}}{\rho g} + {z_{in}} + \frac{\bar v_{in}^2}{2g} = \frac{p_{out}}{\rho g} + z_{out} + \frac{\bar v_{out}^2}{2g} + h_L\end{split}
\end{equation}
Since the elevations at the center of the \(in\) and \(out\) control surfaces are the same, we can eliminate \(z_{in}\) and \(z_{out}\). As we are considering such a small length of pipe, we will neglect the major loss component of head loss. Thus, \(h_L = h_e + \cancel{h_f}\). The following three equations are all the same, simply rearranged to solve for \(h_e\).
\begin{equation}\label{equation:Review/Review_Fluid_Mechanics_Derivations:Review/Review_Fluid_Mechanics_Derivations:1}
\begin{split}\frac{p_{in}}{\rho g} + \frac{\bar v_{in}^2}{2g} = \frac{p_{out}}{\rho g} + \frac{\bar v_{out}^2}{2g} + h_e\end{split}
\end{equation}\begin{equation}\label{equation:Review/Review_Fluid_Mechanics_Derivations:Review/Review_Fluid_Mechanics_Derivations:2}
\begin{split}\frac{p_{in} - p_{out}}{\rho g} = \frac{\bar v_{out}^2 - \bar v_{in}^2}{2g} + h_e\end{split}
\end{equation}\begin{equation}\label{equation:Review/Review_Fluid_Mechanics_Derivations:minor_loss_energy_eq}
\begin{split}  h_e = \frac{p_{in} - p_{out}}{\rho g} + \frac{\bar v_{in}^2 - \bar v_{out}^2}{2g}\end{split}
\end{equation}
This last equation has \(h_e\) as a function of four variables \((p_{in}, \, p_{out}, \, v_{in}\), and \(v_{out})\); we would like it to be a function of only one. Thus, we will invoke conservation of momentum in the horizontal direction across our control volume to remove variables. The difference in momentum from the \(in\) point to the \(out\) point is driven by the pressure difference between each end of the control volume. We will be considering the pressure at the centroid of our control surfaces, and we will neglect shear along the pipe walls. After these assumptions, our momentum equation becomes the following:
\begin{equation}\label{equation:Review/Review_Fluid_Mechanics_Derivations:Review/Review_Fluid_Mechanics_Derivations:3}
\begin{split}M_{in, \, x} + M_{out, \, x} = F_{p_{in, \, x}} + F_{p_{out, \, x}}\end{split}
\end{equation}
\begin{DUlineblock}{0em}
\item[] Such that:
\item[] \(M_{x}\) = momentum flowing through the control volume in the x-direction
\item[] \(F_{p_x}\) = force due to pressure acting on the boundaries of the control volume in the x-direction
\end{DUlineblock}

Recall that momentum is mass times velocity for solids, \(m v\), with units of \(\frac{[M][L]}{[T]}\). Since we consider water flowing through a pipe, there is not one singular mass or one singular velocity. Instead, there is a mass flow rate, or a mass per time indicated by \(\dot m = \rho Q\), which has units of \(\frac{[M]}{[T]}\). Therefore, the momentum for a fluid is \(\rho Q \bar v\). Applying the continuity equation \(Q = \bar v A\), we get to the following equation for the momentum of a fluid flowing through a pipe which we will use in this derivation, \(M = \rho \bar v^2 A\). The pressure force is simply the pressure at the centroid of the flow multiplied by the area the pressure is acting upon, \(p A\).

To ensure correct sign convention, we will make each side of the equation negative for reasons discussed shortly. Since \(\bar v_{in} > \bar v_{out}\), the left hand side will be \(M_{out} - M_{in}\) in order to be negative. The reduction in velocity from \(in\) to \(out\) causes an increase in pressure, therefore \(p_{in} - p_{out}\) is negative. With these substitutions, the conservation of momentum equation becomes as follows:
\begin{equation}\label{equation:Review/Review_Fluid_Mechanics_Derivations:Review/Review_Fluid_Mechanics_Derivations:4}
\begin{split}M_{out} - M_{in} = p_{in} - p_{out}\end{split}
\end{equation}\begin{equation}\label{equation:Review/Review_Fluid_Mechanics_Derivations:Review/Review_Fluid_Mechanics_Derivations:5}
\begin{split}\rho \bar v_{out}^2 A_{out} - \rho \bar v_{in}^2 A_{in} = p_{in} A_{out} - p_{out} A_{out}\end{split}
\end{equation}
Note that the area term attached to \(p_{in}\) is actually \(A_{out}\) instead of \(A_{in}\), as one might think. This is because \(A_{out} = A_{in}\). We chose our control volume to start a few millimeters into the larger pipe, which means that the cross-sectional area does not change over the course of the control volume.

Dividing both sides of the equation by \(A_{out} \rho g\), we obtain the following equation, which contains the very same pressure term as our adjusted energy equation above, equation \eqref{equation:Review/Review_Fluid_Mechanics_Derivations:minor_loss_energy_eq}. This is why we chose a negative sign convention.
\begin{equation}\label{equation:Review/Review_Fluid_Mechanics_Derivations:Review/Review_Fluid_Mechanics_Derivations:6}
\begin{split}\frac{p_{in} - p_{out}}{\rho g} = \frac{\bar v_{out}^2 - \bar v_{in}^2 \frac{A_{in}}{A_{out}}}{g}\end{split}
\end{equation}
Now, we combine the momentum, continuity, and adjusted energy equations:
\begin{equation}\label{equation:Review/Review_Fluid_Mechanics_Derivations:Review/Review_Fluid_Mechanics_Derivations:7}
\begin{split}{\rm{Energy \, equation:}} \,\,\,  h_e = \frac{p_{in} - p_{out}}{\rho g} + \frac{\bar v_{in}^2 - \bar v_{out}^2}{2g}\end{split}
\end{equation}\begin{equation}\label{equation:Review/Review_Fluid_Mechanics_Derivations:Review/Review_Fluid_Mechanics_Derivations:8}
\begin{split}{\rm{Momentum \, equation:}} \,\,\, \frac{p_{in} - p_{out}}{\rho g} = \frac{\bar v_{out}^2 - \bar v_{in}^2 \frac{A_{in}}{A_{out}}}{g}\end{split}
\end{equation}\begin{equation}\label{equation:Review/Review_Fluid_Mechanics_Derivations:Review/Review_Fluid_Mechanics_Derivations:9}
\begin{split}{\rm{Continuity \, equation:}} \,\,\, \frac{A_{in}}{A_{out}} = \frac{\bar v_{out}}{\bar v_{in}}\end{split}
\end{equation}
To obtain an equation for minor losses with just two variables, \(\bar v_{in}\) and \(\bar v_{out}\).
\begin{equation}\label{equation:Review/Review_Fluid_Mechanics_Derivations:Review/Review_Fluid_Mechanics_Derivations:10}
\begin{split}h_e = \frac{\bar v_{out}^2 - \bar v_{in}^2\frac{\bar v_{out}}{\bar v_{in}}}{g} + \frac{\bar v_{in}^2 - \bar v_{out}^2}{2g}\end{split}
\end{equation}
Now we will combine the two terms. The numerator and denominator of the first term, \(\frac{\bar v_{out}^2 - \bar v_{in}^2\frac{\bar v_{out}}{\bar v_{in}}}{g}\) will be multiplied by \(2\) to become \(\frac{2 \bar v_{out}^2 - 2 \bar v_{in}^2\frac{\bar v_{out}}{\bar v_{in}}}{2 g}\). The equation then looks like:
\begin{equation}\label{equation:Review/Review_Fluid_Mechanics_Derivations:Review/Review_Fluid_Mechanics_Derivations:11}
\begin{split}h_e = \frac{\bar v_{out}^2 - 2 \bar v_{in} \bar v_{out} + \bar v_{in}^2}{2g}\end{split}
\end{equation}

\subsection{Final Forms of the Minor Loss Equation}
\label{\detokenize{Review/Review_Fluid_Mechanics_Derivations:final-forms-of-the-minor-loss-equation}}\label{\detokenize{Review/Review_Fluid_Mechanics_Derivations:heading-final-minor-loss-equations}}
Factoring the numerator yields to the first ‘final’ form of the minor loss equation:
\begin{equation}\label{equation:Review/Review_Fluid_Mechanics_Derivations:Review/Review_Fluid_Mechanics_Derivations:12}
\begin{split}{\rm{ \mathbf{First \, form:} }} \quad h_e = \frac{\left( \bar v_{in}  - \bar v_{out} \right)^2}{2g}\end{split}
\end{equation}
From here, the two other forms of the minor loss equation can be derived by solving for either \(\bar v_{in}\) or \(\bar v_{out}\) using the ubiquitous continuity equation \(\bar v_{in} A_{in} = \bar v_{out} A_{out}\):
\begin{equation}\label{equation:Review/Review_Fluid_Mechanics_Derivations:Review/Review_Fluid_Mechanics_Derivations:13}
\begin{split}{\rm{ \mathbf{Second \, form:} }} \quad h_e = \left( 1 - \frac{A_{in}}{A_{out}} \right)^2 \, \frac{\bar v_{in}^2}{2g} \, \, = \, \, K_e^{'} \frac{\bar v_{in}^2}{2g}, \quad {\rm where} \quad K_e^{'} = \left( 1 - \frac{A_{in}}{A_{out}} \right)^2\end{split}
\end{equation}\begin{equation}\label{equation:Review/Review_Fluid_Mechanics_Derivations:minor_loss_equation}
\begin{split}   \color{purple}{
   {\rm{ \mathbf{Third \, form:} }} \quad h_e = \left( \frac{A_{out}}{A_{in}} -1 \right)^2 \, \frac{\bar  v_{out}^2}{2g} \, \, = \, \, K_e \frac{\bar v_{out}^2}{2g}, \quad {\rm where} \quad K_e = \left( \frac{A_{out}}{A_{in}} - 1 \right)^2
   }\end{split}
\end{equation}
\begin{sphinxadmonition}{note}{Note:}
You will often see \(K_e^{'}\) and \(K_e\) used without the \(e\) subscript, they will appear as \(K^{'}\) and \(K\).
\end{sphinxadmonition}

Being familiar with these three forms and how they are used will be of great help throughout the class. The third form is the one that is most commonly used.


\chapter{Flow Control and Measurement Introduction}
\label{\detokenize{Flow_Control_and_Measurement/FCM_Intro:flow-control-and-measurement-introduction}}\label{\detokenize{Flow_Control_and_Measurement/FCM_Intro:title-flow-control-intro}}\label{\detokenize{Flow_Control_and_Measurement/FCM_Intro::doc}}

\section{Tank with a Valve}
\label{\detokenize{Flow_Control_and_Measurement/FCM_Intro:tank-with-a-valve}}\label{\detokenize{Flow_Control_and_Measurement/FCM_Intro:heading-tank-with-a-valve}}

\subsection{Flow \protect\(Q\protect\) and Water Level \protect\(h\protect\) as a Function of Time}
\label{\detokenize{Flow_Control_and_Measurement/FCM_Intro:flow-and-water-level-as-a-function-of-time}}\label{\detokenize{Flow_Control_and_Measurement/FCM_Intro:heading-qh-as-a-function-of-t}}
Our first step is to see if we can get constant head out of a simple system. The most simple flow control system is a bucket or tank with a hole in it. This system is too coarse to provide constant head. One step above that is a bucket or tank with a valve. This is where we begin our search for constant head.

Using the setup of in the image below, we derive the following equation for flow \(Q\) through the valve as a function of time \(t\). The derivation is found here: {\hyperref[\detokenize{Flow_Control_and_Measurement/FCM_Derivations:heading-flow-for-a-tank-with-a-valve}]{\sphinxcrossref{\DUrole{std,std-ref}{ for a Tank with a Valve}}}}. You are advised to read through it if you are at all confused about this equation.
\begin{equation}\label{equation:Flow_Control_and_Measurement/FCM_Intro:Q_tank_with_valve}
\begin{split}  \frac{Q}{Q_0} = 1 - \frac{1}{2} \frac{t}{t_{Design}} \frac{h_{Tank}}{h_0}\end{split}
\end{equation}
\begin{DUlineblock}{0em}
\item[] Such that:
\item[] \(Q\) = \(Q(t)\) = flow of hypochlorite through valve at time \(t\)
\item[] \(Q_0\) = flow of hypochlorite through valve at time \(t = 0\)
\item[] \(t\) = elapsed time
\item[] \(t_{Design}\) = time it \sphinxstyleemphasis{would} take for tank to empty if flow stayed constant at \(Q_0\), which it does not
\item[] \(h_{Tank}\) = elevation of water level with reference to tank bottom at time \(t\) = 0
\item[] \(h_0\) = elevation of water level with reference to the valve at time \(t = 0\)
\end{DUlineblock}

\begin{figure}[htbp]
\centering
\capstart

\noindent\sphinxincludegraphics[width=600\sphinxpxdimen]{{hypochlorinator_variable_explanation}.png}
\caption{This figure shows the variables that are defined in the equation above.}\label{\detokenize{Flow_Control_and_Measurement/FCM_Intro:id1}}\label{\detokenize{Flow_Control_and_Measurement/FCM_Intro:figure-hypochlorinator-variable-explanation-design}}\end{figure}

This equation has historically give students some trouble, and while its nuances are explained in the derivation, they will be quickly summarized here:
\begin{itemize}
\item {} 
\(t_{Design}\) is \sphinxstylestrong{NOT} the time it takes to drain the tank. It is the time that it \sphinxstyleemphasis{would} take to drain the tank \sphinxstyleemphasis{if} the flow rate at time \(t = 0\), \(Q_0\), were the flow rate forever, which it is not. \(t_{Design}\) was used in the derivation to simplify the equation, which is why this potentially-confusing parameter exists. The actual time it takes to drain the tank lies somewhere between \(t_{Design}\) and \(2 \, t_{Design}\) and depends on the ratio \(\frac{h_{Tank}}{h_0}\).

\item {} 
\(h_{Tank}\) is not the same as \(h_{0}\). \(h_{Tank}\) is the height of water level in the tank with reference to the tank bottom. \(h_{0}\) is the water level in the tank with reference to the valve. Neither change with time, they both refer to the water level at one instance in time, \(t = 0\). Therefore, \(h_{0} \geq h_{Tank}\) is always true. If the tank is elevated far above the valve, then the \(h_{0} > > h_{Tank}\). If the valve is at the same elevation as the bottom of the tank, then \(h_{0} = h_{Tank}\). Please refer to the figure above to clarify \(h_{0}\) and \(h_{Tank}\).

\end{itemize}

We can use the proportionality \(Q \propto \sqrt{h}\), which applies to both minor losses and orifices to form a relationship between water level in the tank \(h\) and time \(t\). This proportionality comes from rearranging the minor loss equation \(h = K \frac{Q^2}{2 g A^2}\) for \(Q\) instead of \(h\). A table of proportionality between \(Q\) and math:\sphinxtitleref{h} can be found in \hyperref[\detokenize{Review/Review_Fluid_Mechanics:table-h-q-proportionality}]{Table \ref{\detokenize{Review/Review_Fluid_Mechanics:table-h-q-proportionality}}}

Using equation \eqref{equation:Flow_Control_and_Measurement/FCM_Intro:Q_tank_with_valve} and this proportionality relationship, we make the following plots. On the left, the valve is at the same elevation as the bottom of the tank, or \(h_{Tank} = h_0\). Our attempt to get a continuous flow rate out of this system is to make \(\frac{h_{Tank}}{h_0}\) very small by elevating the tank far above the valve. On the right, \(\frac{h_{Tank}}{h_0} = \frac{1}{50}\). While the plot looks great and provides essentially constant head, elevating the tank by 50 times its height is not realistic. The ‘tank with a valve’ is not a solution to the constant head problem.

\begin{figure}[htbp]
\centering
\capstart

\noindent\sphinxincludegraphics[width=600\sphinxpxdimen]{{tank_valve_play}.png}
\caption{These graphs show how manipulation of the variables in the \(Q(t)\) expression can result in effectively constant head.}\label{\detokenize{Flow_Control_and_Measurement/FCM_Intro:id2}}\label{\detokenize{Flow_Control_and_Measurement/FCM_Intro:figure-tank-valve-play}}\end{figure}


\subsection{Drain System for a Tank}
\label{\detokenize{Flow_Control_and_Measurement/FCM_Intro:drain-system-for-a-tank}}\label{\detokenize{Flow_Control_and_Measurement/FCM_Intro:heading-drain-system-for-a-tank}}
While the ‘tank with a valve’ scenario is not a good constant head solution, we can use our understanding of the system to properly design drain systems for AguaClara reactors like flocculators and sedimentation tanks, since they are just tanks with valves. The derivation for the following equation is here, along with more details on AguaClara’s pipe stub method for draining tanks: {\hyperref[\detokenize{Flow_Control_and_Measurement/FCM_Derivations:heading-diameter-and-time-tank-drain-equation}]{\sphinxcrossref{\DUrole{std,std-ref}{ and  for Tank Drain Equation}}}}. The derived ‘Tank Drain’ equation is as follows:
\begin{equation}\label{equation:Flow_Control_and_Measurement/FCM_Intro:Flow_Control_and_Measurement/FCM_Intro:0}
\begin{split}D_{Pipe} = \sqrt{ \frac{8 L_{Tank} W_{Tank}}{\pi t_{Drain}}} {\left( \frac{H_{Tank} \sum K }{2g} \right)^{\frac{1}{4}}}\end{split}
\end{equation}
The equation can also be rearranged to solve for the time it would take to drain a tank given its dimensions and a certain drain pipe size:
\begin{equation}\label{equation:Flow_Control_and_Measurement/FCM_Intro:Flow_Control_and_Measurement/FCM_Intro:1}
\begin{split}t_{Drain} =  \frac{8 L_{Tank} W_{Tank}}{\pi D_{Pipe}^2} {\left( \frac{H_{Tank} \sum K }{2g} \right)^{\frac{1}{2}}}\end{split}
\end{equation}
\begin{DUlineblock}{0em}
\item[] Such that:
\item[] \(D_{Pipe}\) = Diameter of the drain piping
\item[] \(L_{Tank}, W_{Tank}, H_{Tank}\) = Tank dimensions
\item[] \(t_{Drain}\) = Time it takes to drain the tank
\item[] \(\sum K\) = Sum of all the minor loss coefficients in the system
\end{DUlineblock}

\begin{figure}[htbp]
\centering
\capstart

\noindent\sphinxincludegraphics[width=600\sphinxpxdimen]{{pipe_stub_drainage_variables}.png}
\caption{Variables for draining a tank}\label{\detokenize{Flow_Control_and_Measurement/FCM_Intro:id3}}\label{\detokenize{Flow_Control_and_Measurement/FCM_Intro:figure-pipe-stub-drainage-variables-in-derivation}}\end{figure}


\chapter{Flow Control and Measurement Design}
\label{\detokenize{Flow_Control_and_Measurement/FCM_Design:flow-control-and-measurement-design}}\label{\detokenize{Flow_Control_and_Measurement/FCM_Design:title-flow-control-design}}\label{\detokenize{Flow_Control_and_Measurement/FCM_Design::doc}}
This section explores AguaClara’s search for constant head in chemical dosing. The term \sphinxstylestrong{constant head} means that the driving head of a system, \(\Delta z\) or \(\Delta h\), does not change over time, even as water flows through or out of the system. Constant head implies constant flow, since the energy driving the flow does not change.

The challenge of constant head in chemical dosing for water treatment plants is not \sphinxstyleemphasis{just} providing one continuous flow of chemicals; it is also varying that flow of chemicals as the flow rate through the plant changes, so that the concentration of chemicals in the raw water stays the same.


\section{Important Terms and Equations}
\label{\detokenize{Flow_Control_and_Measurement/FCM_Design:important-terms-and-equations}}\label{\detokenize{Flow_Control_and_Measurement/FCM_Design:heading-fcm-terms-eqs}}
\sphinxstylestrong{Terms:}
\begin{enumerate}
\item {} 
Dose

\item {} 
Coagulant

\item {} 
Chlorination

\item {} 
Turbidity

\item {} 
Organic Matter

\item {} 
Constant Head Tank

\item {} 
Sutro weir

\end{enumerate}

\sphinxstylestrong{Equations:}
\begin{enumerate}
\item {} 
Hagen-Poiseuille equation

\end{enumerate}


\section{AguaClara Flow Control and Measurement Technologies}
\label{\detokenize{Flow_Control_and_Measurement/FCM_Design:aguaclara-flow-control-and-measurement-technologies}}\label{\detokenize{Flow_Control_and_Measurement/FCM_Design:heading-aguaclara-flow-control-and-measurement-technologies}}
Each technology or component for this section will have five subsections:
\begin{itemize}
\item {} 
\sphinxstylestrong{What it is}

\item {} 
\sphinxstylestrong{What it does and why}

\item {} 
\sphinxstylestrong{How it works}

\item {} 
\sphinxstylestrong{Notes}

\end{itemize}

Before diving into the technologies, recall the purpose of the chemicals that we are seeking to constantly \sphinxstylestrong{dose}, and why it is important to keep a constant, specific dose. Also recall that ‘dose’ means ‘concentration of chemical’ \sphinxstyleemphasis{in the water we are trying to treat}, not in the stock tanks of the chemicals. \sphinxhref{https://en.wikipedia.org/wiki/Coagulation\_(water\_treatment)}{Coagulant} like alum, PAC, and some iron-based chemicals are used to turn small particles into bigger particles, allowing them to be captured more easily. Waters with high \sphinxhref{https://en.wikipedia.org/wiki/Turbidity}{turbidity}, indicative of a lot of particles like clay and bacteria, require more coagulant to treat effectively. Additionally, waters with a lot of \sphinxhref{https://en.wikipedia.org/wiki/Organic\_matter}{organic matter} require significantly more coagulant to treat. \sphinxhref{https://en.wikipedia.org/wiki/Water\_chlorination}{Chlorine} is used to disinfect water that has already been fully treated. A proper and consistent chlorine dose is required, as too low of a dose creates a risk of reintroduction of pathogens in the distribution system and too high of a dose increases the risk of carcinogenic \sphinxhref{https://en.wikipedia.org/wiki/Disinfection\_by-product}{disinfection byproduct} formation.

\begin{sphinxadmonition}{important}{Important:}
This section will often refer to the proportionality between flow \(Q\) and head \(\Delta h\) (recall that \(\Delta h = h_L\) after applying the head loss trick) by using the ‘proportional to’ symbol, \(\propto\). It is important to remember that it doesn’t necessarily matter whether \(Q\) or \(h_L\) goes first, \(Q \propto \sqrt{h_L}\) is equivalent to saying that \(h_L \propto Q^2\).
\end{sphinxadmonition}


\subsection{“Almost Linear” Flow Controller}
\label{\detokenize{Flow_Control_and_Measurement/FCM_Design:almost-linear-flow-controller}}\label{\detokenize{Flow_Control_and_Measurement/FCM_Design:heading-almost-linear-flow-controller}}

\subsubsection{What it is}
\label{\detokenize{Flow_Control_and_Measurement/FCM_Design:what-it-is}}
This device consists of a bottle of chemical solution, called the \sphinxstylestrong{Constant Head Tank} (CHT), a float valve to keep a solution in the CHT at a constant water level, a flexible tube starting at the bottom of the CHT, and many precisely placed and equally spaced holes in a pipe, as the image below shows. The holes in the pipe hold the other end of the tube that starts at the CHT.

Chemical solution, either coagulant or chlorine, is stored in a stock tank somewhere above the CHT. A different tube connects the stock tank to the float valve within the CHT.


\subsubsection{What it does and why}
\label{\detokenize{Flow_Control_and_Measurement/FCM_Design:what-it-does-and-why}}
This flow controller provides a constant flow of chemical solution to the water in the plant. When the end of the flexible tube is placed in a hole, the elevation difference between the water level in the bottle and the hole is set and does not change unless the tube is then placed in another hole. Thus, a constant flow is provided while the end of the tube is not moved.

As has been mentioned previously, the amount of chlorine and coagulant that must be added to the raw water changes depending on the flow rate of the plant; the change is necessary to keep the dose constant. More water flowing through the plant means more chlorine is necessary to maintain the dose of chlorine in the treated water. For coagulant, there are also other factors aside from plant flow rate that impact the required dose, including the turbidity and amount of organic matter in the water. The operator must be able to change the dose of both coagulant and chlorine quickly and easily, and they must be able to know the value of the new dose they set. The “Almost Linear” Flow Controller accomplishes this by having a large number of holes in the flow control pipe next to the CHT. This large number of holes gives the operator many options for adjusting the dose, and let them quickly change the flow of chemicals into the raw water by moving the end of the flexible tube from one hole to another.


\subsubsection{How it works}
\label{\detokenize{Flow_Control_and_Measurement/FCM_Design:how-it-works}}
The idea behind this flow controller is to have a linear relationship between \(Q\) and \(h_L\) (remember that \(h_L\) is equal to \(\Delta h\) when we apply the head loss trick), which can be written as \(Q \propto h_L\). Here, \(Q\) is the flow of chemicals out of the flexible tube, and \(h_L\) is the elevation difference between the water level in the CHT and the end of the flexible tube.

As you remember from section 1.5, the summary of Fluids Review, \(Q \propto \Delta h\), or \(\Delta h \propto Q\) as it was written in the section summary, is only true for the combination of major losses and laminar flow, which makes applicable the Hagen-Poiseuille equation. Therefore, the flow must always be laminar in the flexible tube that goes between the CHT and the holes, and major losses must far exceed minor losses.

It is easy to design for laminar flow, but the “Almost Linear” Flow Controller was unable to make major losses far exceed minor losses. The bending in the flexible tube caused a lot of minor losses which changed in magnitude depending on exactly how the tube was bent. This made the flow controller “almost linear,” but that wasn’t good enough.


\subsubsection{Notes}
\label{\detokenize{Flow_Control_and_Measurement/FCM_Design:notes}}\begin{itemize}
\item {} 
This flow controller is \sphinxstylestrong{no longer used by AguaClara.}

\item {} 
The tube connecting the CHT to the outlet of chemicals must really belong and, more importantly, \sphinxstylestrong{straight} to form a linear relationship between driving head and flow. This was not true for the “Almost Linear” Flow Controller. When you read about the Linear Chemical Flow Controller (CDC), you will be learning about the replacement to the “Almost Linear” Flow Controller’s replacement.

\end{itemize}


\subsection{Linear Flow Orifice Meter (LFOM)}
\label{\detokenize{Flow_Control_and_Measurement/FCM_Design:linear-flow-orifice-meter-lfom}}\label{\detokenize{Flow_Control_and_Measurement/FCM_Design:heading-lfom}}

\subsubsection{What it is}
\label{\detokenize{Flow_Control_and_Measurement/FCM_Design:id1}}
The LFOM is a weir shape cut into a pipe. It was meant to imitate \sphinxhref{https://confluence.cornell.edu/display/AGUACLARA/LFOM+sutro+weir+research}{the Sutro Weir} while being far easier to build. The LFOM is a pipe with rows of holes, or orifices, drilled into it. There are progressively fewer holes per row as you move up the LFOM, as the shape is meant to resemble half a parabola on each side. The size of all holes is the same, and the amount of holes per row are precisely calculated. Water in the entrance tank flows into and down the LFOM, towards the rapid mix orifice and flocculator.

\begin{figure}[htbp]
\centering
\capstart

\noindent\sphinxincludegraphics[width=600\sphinxpxdimen]{{sutro_v_lfom}.png}
\caption{On the left is a sutro weir. On the right is AguaClara’s approximation of the sutro weir’s geometery. This elegant innovation is called a linear flow orifice meter, or LFOM for short.}\label{\detokenize{Flow_Control_and_Measurement/FCM_Design:id9}}\label{\detokenize{Flow_Control_and_Measurement/FCM_Design:figure-sutro-v-lfom}}\end{figure}


\subsubsection{What it does and why}
\label{\detokenize{Flow_Control_and_Measurement/FCM_Design:id2}}
The LFOM does one thing and serves two purposes.

What it does:

\sphinxstylestrong{The LFOM creates a linear relationship between water level in the entrance tank and the flow out of the entrance tank.} \sphinxstyleemphasis{It does not control the flow through the plant}. If the LFOM were replaced with a hole in the bottom of the entrance tank, the same flow rate would go through the plant, the only difference being that the water level in the entrance tank would scale with flow squared \(h \propto Q^2\) instead of \(h \propto Q\). For example, if an LFOM has 10 rows of holes and has been designed for a plant whose maximum flow rate is 10 L/s, then the operator knows that the number of rows submerged in water is equal to the flow rate of the plant in L/s. So if the water were up to the third row of holes, there would be 3 L/s of water flowing through the plant.

Why it is useful:
\begin{enumerate}
\item {} 
Allows the operator to measure the flow through the plant quickly and easily, explained above.

\item {} 
Allows for the Linear Chemical Dose Controller, which will be explained next, to automatically adjust the flow of coagulant/chlorine into the plant as the plant flow rate changes. This means the operator would only need to adjust the flow of coagulant when there is a change in turbidity or organic matter.

\end{enumerate}


\subsubsection{How it works}
\label{\detokenize{Flow_Control_and_Measurement/FCM_Design:id3}}
This is best understood with examples. By shaping a weir differently, different relationships between \(Q\) and \(h\) are formed:
* In the case of a \sphinxhref{https://swmm5.files.wordpress.com/2016/09/image00124.jpg}{rectangular weir}, \(Q \propto h^{\frac{3}{2}}\)
* In the case of a \sphinxhref{https://swmm5.files.wordpress.com/2016/09/image0096.jpg}{v-notch weir}, \(Q \propto h^{\frac{5}{2}}\)
* In the case of a \sphinxhref{http://www.engineeringexcelspreadsheets.com/wp-content/uploads/2012/11/Sutro-Weir-Diagram1.jpg}{Sutro weir} and thus LFOM, \(Q \propto h\).


\subsubsection{Notes}
\label{\detokenize{Flow_Control_and_Measurement/FCM_Design:id4}}\begin{itemize}
\item {} 
The LFOM is not perfect. Before the water level reaches the second row of holes, the LFOM is simulating a rectangular weir, and thus \(h \not\propto Q\). The Sutro weir also experiences this problem.

\item {} 
If the water level exceeds the topmost row of the LFOM’s orifices, the linearity also breaks down. The entire LFOM begins to act like an orifice, the exponent of \(Q\) in \(h \propto Q\) becomes greater than 1. This is because the LFOM approaches orifice behavior, and for orifices, \(h \propto Q^2\).

\end{itemize}


\subsection{Linear Chemical Dose Controller (CDC)}
\label{\detokenize{Flow_Control_and_Measurement/FCM_Design:linear-chemical-dose-controller-cdc}}\label{\detokenize{Flow_Control_and_Measurement/FCM_Design:heading-linear-cdc}}
Since the Linear Chemical Dose Controller has become the standard in AguaClara, it is often simply called the Chemical Dose Controller, \sphinxstylestrong{or CDC for short}. It can be confusing to describe with words, so be sure to flip through the slides in the ‘Flow Control and Measurement’ powerpoint, as they contain very, very, helpful diagrams of the CDC.


\subsubsection{What it is}
\label{\detokenize{Flow_Control_and_Measurement/FCM_Design:id5}}
The CDC brings together the LFOM and many improvements to the “Almost Linear” Flow Controller. Let’s break it down, with the image below as a guide.
\begin{enumerate}
\item {} 
Start at the Constant Head Tank (CHT). This is the same set up as the “Almost Linear” Flow Controller. The stock tank feeds into the CHT, and the float valve makes sure that the water level in the constant head tank is always the same.

\end{enumerate}

2. Now the tubes. These fix the linearity problems that were the main problem in the “Almost Linear” Flow Controller.
* The tube connected to the bottom of the CHT is large diameter to minimize any head loss through it.
\begin{itemize}
\item {} 
The three thin, straight tubes are designed to generate a lot of major losses and to minimize any minor losses. This is to make sure that major losses far exceed any minor losses, which will ensure that the Hagen-Poiseuille equation is applicable and that flow will be directly proportional to the head, \(Q \propto \Delta h\). Why are there 3 tubes?
\begin{enumerate}
\item {} 
\sphinxstylestrong{3 short instead of 1 short} Removing 2 of the 3 tubes would mean 3 times the flow through the remaining tube. This means the velocity in the tube would be 3 times as fast. Since minor losses scale with \(v^2\) and major losses only scale with \(v\), this would increase the ratio of \(\rm{\frac{minor \, losses}{major \, losses}}\), which would break the linearity we’re trying to achieve. It would also increase the total head loss through the system, resulting in a lower maximum flow rate than before.

\item {} 
\sphinxstylestrong{1 long instead of 3 short} One tube whose length is equal to the three combined would be inconveniently long, and would suffer from the same problems as above. There would be even more head loss through the tube, since its length would be longer.

\end{enumerate}

\item {} 
The large-diameter tube on the right of the three thin, straight tubes is where the chemicals flow out. The end of the tube is connected to both a slider and a ‘drop tube.’ The drop tube allows for supercritical flow of the chemical leaving the dosing tubes; once the chemical enters the drop tube it falls freely and no longer affects the CDC system.

\end{itemize}
\begin{enumerate}
\setcounter{enumi}{2}
\item {} 
The slider rests on a lever. This lever is the critical part of the CDC, it connects the water level in the entrance tank, which is adjusted by the LFOM, to the difference in head between the CHT and the end of the dosing tube. This allows the flow of chemicals to automatically adjust to a change in the plant flow rate, maintaining a constant dose in the plant water. One end of the lever tracks the water level in the entrance tank by using a float. The counterweight on the other side of the lever is to make sure the float ‘floats,’ since this float is usually made of PVC, which is more dense than water.

\item {} 
The slider itself controls the dose of chemicals. For any given plant flow rate, the slider can be adjusted to increase or decrease the amount of chemical flowing through the plant.

\end{enumerate}

\begin{figure}[htbp]
\centering
\capstart

\noindent\sphinxincludegraphics[width=600\sphinxpxdimen]{{cdc_labelled}.png}
\caption{This is the setup of the chemical dose controller.}\label{\detokenize{Flow_Control_and_Measurement/FCM_Design:id10}}\label{\detokenize{Flow_Control_and_Measurement/FCM_Design:figure-cdc-labelled}}\end{figure}


\subsubsection{What it does and why}
\label{\detokenize{Flow_Control_and_Measurement/FCM_Design:id6}}
The CDC makes it easy and accurate to dose chemicals. The flow of chemicals automatically adjusts to changes in the plant flow rate to keep a constant dose, set by the operator. When a turbidity event occurs, the operator can change the dose of coagulant by moving the coagulant slider \sphinxstyleemphasis{lower} on the lever to increase the dose. The slider has labelled marks so the operator can record the dose accurately.


\subsubsection{How it works}
\label{\detokenize{Flow_Control_and_Measurement/FCM_Design:id7}}
A lot of design has gone into the CDC. The design equations and their derivations that the following steps are based on can be found here: {\hyperref[\detokenize{Flow_Control_and_Measurement/FCM_Derivations:heading-design-equations-for-the-cdc}]{\sphinxcrossref{\DUrole{std,std-ref}{Design Equations for the Linear Chemical Dose Controller (CDC)}}}}, and you are very, very strongly encouraged to read them.

The CDC can be designed manually using the equations from the derivation linked above or via aide\_design, using the equations found in \sphinxhref{https://github.com/AguaClara/aide\_design/blob/master/aide\_design/cdc\_functions.py}{cdc\_functions.py}. Either way, the design algorithm is roughly the same:
\begin{enumerate}
\item {} 
Calculate the maximum flow rate, \(Q_{Max, \, Tube}\), through each available dosing tube diameter \(D\) that keeps error due to minor losses below 10\% of total head loss. Recall that tubing diameter is an array, as there are many diameters available at hardware stores and suppliers. This means that for each step, there will be as many solutions as there are reasonable diameters available.

\end{enumerate}
\begin{equation}\label{equation:Flow_Control_and_Measurement/FCM_Design:Flow_Control_and_Measurement/FCM_Design:0}
\begin{split}Q_{Max, \, Tube} = \frac{\pi D^2}{4} \sqrt{\frac{2 h_L g \Pi_{Error}}{\sum{K} }}\end{split}
\end{equation}\begin{enumerate}
\setcounter{enumi}{1}
\item {} 
Calculate how much flow of chemical needs to pass through the CDC at maximum plant flow and maximum chemical dose. This depends on the concentration of chemicals in the stock tank.

\end{enumerate}
\begin{equation}\label{equation:Flow_Control_and_Measurement/FCM_Design:Flow_Control_and_Measurement/FCM_Design:1}
\begin{split}Q_{Max, \, CDC} = \frac{Q_{Plant} \cdot C_{Dose, \, Max}}{C_{StockTank}}\end{split}
\end{equation}\begin{enumerate}
\setcounter{enumi}{2}
\item {} 
Calculate the number of dosing tubes required if the tubes flow at  maximum capacity (round up)

\end{enumerate}
\begin{equation}\label{equation:Flow_Control_and_Measurement/FCM_Design:Flow_Control_and_Measurement/FCM_Design:2}
\begin{split}n_{Tubes} = {\rm ceil} \left( \frac{Q_{Max, \, CDC}}{Q_{Max, \, Tube}} \right)\end{split}
\end{equation}\begin{enumerate}
\setcounter{enumi}{3}
\item {} 
Calculate the length of dosing tube(s) that correspond to each available tube diameter.

\end{enumerate}
\begin{equation}\label{equation:Flow_Control_and_Measurement/FCM_Design:Flow_Control_and_Measurement/FCM_Design:3}
\begin{split}L_{Min} = \left( \frac{g h_L \pi D^4}{128 \nu Q_{Max}} - \frac{Q_{Max}}{16 \pi \nu} \sum{K} \right)\end{split}
\end{equation}\begin{enumerate}
\setcounter{enumi}{4}
\item {} 
Select a tube length from your array of solutions. Pick the longest dosing tube that you can, keeping in mind that the tube(s) must be able to fit in the plant and can’t be longer than the length of the plant wall it will be placed along.

\item {} 
Finally, select the dosing tube diameter and flow rate corresponding to the selected tube length.

\end{enumerate}


\subsubsection{Notes}
\label{\detokenize{Flow_Control_and_Measurement/FCM_Design:id8}}
Nothing in life is perfect, and the CDC is no exception. It has a few causes of inaccuracy which go beyond non-zero minor losses:
* Float valves are not perfect. There will still be minor fluctuations of the fluid level in the CHT which will result in imperfect dosing.
* Surface tension may resist the flow of chemicals from the dosing tube into the drop tube during low flows. Since the CDC design does not consider surface tension, this is a potential source of error.
* The lever and everything attached to it are not weightless. Changing the dose of coagulant or chlorine means moving the slider along the lever. Since the slider and tubes attached to it (drop tube, dosing tube) have mass, moving the slider means that the torque of the lever is altered. This means that the depth that the float is submerged is changed, which affects \(\Delta h\) of the system. This can be remedied by making the float’s diameter as large as possible, which makes these fluctuations small. This problem can not be avoided entirely.


\section{Section Summary}
\label{\detokenize{Flow_Control_and_Measurement/FCM_Design:section-summary}}\label{\detokenize{Flow_Control_and_Measurement/FCM_Design:heading-fcm-section-summary}}
1. \sphinxstylestrong{Tank with a valve:}
.. math:

\fvset{hllines={, ,}}%
\begin{sphinxVerbatim}[commandchars=\\\{\}]
\PYGZbs{}\PYG{n}{frac}\PYG{p}{\PYGZob{}}\PYG{n}{Q}\PYG{p}{\PYGZcb{}}\PYG{p}{\PYGZob{}}\PYG{n}{Q\PYGZus{}0}\PYG{p}{\PYGZcb{}} \PYG{o}{=} \PYG{l+m+mi}{1} \PYG{o}{\PYGZhy{}} \PYGZbs{}\PYG{n}{frac}\PYG{p}{\PYGZob{}}\PYG{l+m+mi}{1}\PYG{p}{\PYGZcb{}}\PYG{p}{\PYGZob{}}\PYG{l+m+mi}{2}\PYG{p}{\PYGZcb{}} \PYGZbs{}\PYG{n}{frac}\PYG{p}{\PYGZob{}}\PYG{n}{t}\PYG{p}{\PYGZcb{}}\PYG{p}{\PYGZob{}}\PYG{n}{t\PYGZus{}}\PYG{p}{\PYGZob{}}\PYG{n}{Design}\PYG{p}{\PYGZcb{}}\PYG{p}{\PYGZcb{}} \PYGZbs{}\PYG{n}{frac}\PYG{p}{\PYGZob{}}\PYG{n}{h\PYGZus{}}\PYG{p}{\PYGZob{}}\PYG{n}{Tank}\PYG{p}{\PYGZcb{}}\PYG{p}{\PYGZcb{}}\PYG{p}{\PYGZob{}}\PYG{n}{h\PYGZus{}0}\PYG{p}{\PYGZcb{}}
\end{sphinxVerbatim}

This equation describes flow \(Q\) as a function of time \(t\) of a fluid leaving a tank through a valve. Attempting to get this ‘tank with a valve’ system to yield constant head means raising the tank far, far above the valve that controls the flow. This is unreasonable when designing a flow control system for constant dosing, but can be used to design systems to drain a tank. See the section above for a description of the variables in the equation.
\begin{enumerate}
\setcounter{enumi}{1}
\item {} 
\sphinxstylestrong{LFOM:} The LFOM makes the water level in the entrance tank linear with respect to the flow out of the entrance tank. This is useful in measuring the flow and is a critical component in AguaClara’s chemical dosing system. The LFOM \sphinxstyleemphasis{measures} the flow through the plant, it does not \sphinxstyleemphasis{control} the flow through the plant.

\item {} 
\sphinxstylestrong{The Linear Chemical Dose Controller (CDC)} combines the:
* linear relationship between water level and flow in the entrance tank caused by the LFOM,
* linear relationship between elevation difference and flow caused by the Hagen-Poiseuille equation, which is only valid for major losses under laminar flow, and
* a lever to link the two linear relationships

\end{enumerate}

To keep the chemical dose constant by automatically adjusting the addition of coagulant and chlorine as the plant flow rate varies. Two sliders on the lever allows the operator to change the dose of coagulant and chlorine independently of the plant flow rate.


\chapter{Flow Control and Measurement Derivations}
\label{\detokenize{Flow_Control_and_Measurement/FCM_Derivations:flow-control-and-measurement-derivations}}\label{\detokenize{Flow_Control_and_Measurement/FCM_Derivations:title-flow-control-derivations}}\label{\detokenize{Flow_Control_and_Measurement/FCM_Derivations::doc}}

\section{\protect\(Q(t)\protect\) for a Tank with a Valve}
\label{\detokenize{Flow_Control_and_Measurement/FCM_Derivations:for-a-tank-with-a-valve}}\label{\detokenize{Flow_Control_and_Measurement/FCM_Derivations:heading-flow-for-a-tank-with-a-valve}}
This document contains the derivation of the flow through a tank-with-a-valve over time, \(Q(t)\). Our reference will be a simple hypochlorinator, shown in the following image. In the image, a hypochlorite solution is slowly dripping and mixing with piped source water, thereby disinfecting it. The valve is almost closed to make sure that the hypochlorite solution drips instead of flows. At the end of this document is an image which shows the variables in the final equation.

\begin{figure}[htbp]
\centering
\capstart

\noindent\sphinxincludegraphics[width=600\sphinxpxdimen]{{drip_hypochlorinator}.png}
\caption{This is a common setup for chlorinating water before distributing it to a nearby community.}\label{\detokenize{Flow_Control_and_Measurement/FCM_Derivations:id1}}\label{\detokenize{Flow_Control_and_Measurement/FCM_Derivations:figure-drip-hypochlorinator}}\end{figure}

This derivation begins by finding two equations for flow, \(Q\), through the hypochlorinator and setting them equal to each other. First, the rate of change of the volume of hypochlorite solution in the tank is equivalent to the flow out of the hypochlorinator. Since the volume of hypochlorite solution in the tank is equal to the tank’s cross-sectional area times it height, we get the following equation:
\begin{equation}\label{equation:Flow_Control_and_Measurement/FCM_Derivations:Flow_Control_and_Measurement/FCM_Derivations:0}
\begin{split}Q =  - \frac{d\rlap{-}V}{dt} = - \frac{{A_{Tank}}dh}{dt}\end{split}
\end{equation}
\begin{DUlineblock}{0em}
\item[] Such that:
\item[] \(\frac{d\rlap{-}V}{dt}\) = rate of change in volume of solution in the tank
\item[] \(\frac{dh}{dt}\) = rate of change in height of water (hypochlorite solution) level with time
\end{DUlineblock}

Our other equation for flow is the head loss equation. Since major losses are negligible for a short pipe-low flow rate system, we only need to consider minor losses. The only real minor loss in this system occurs in the almost-closed valve that is dripping the hypochlorite solution. However, we will also use the head loss trick. Therefore, the total driving head of the system \(h\) is equal to the minor losses:
\begin{equation}\label{equation:Flow_Control_and_Measurement/FCM_Derivations:Flow_Control_and_Measurement/FCM_Derivations:1}
\begin{split}h = h_e = \left( \sum K \right) \frac{Q^2}{2gA_{Valve}^2}\end{split}
\end{equation}
Bear in mind that this is the second form of the minor loss equation as described in {\hyperref[\detokenize{Review/Review_Fluid_Mechanics_Derivations:heading-final-minor-loss-equations}]{\sphinxcrossref{\DUrole{std,std-ref}{this derivation}}}}. Rearranging the minor loss equation to solve for \(Q\), it looks like this:
\begin{equation}\label{equation:Flow_Control_and_Measurement/FCM_Derivations:Flow_Control_and_Measurement/FCM_Derivations:2}
\begin{split}Q = A_{Valve} \sqrt{\frac{2 h_e g}{\sum K}}\end{split}
\end{equation}
Now we can set both equations for \(Q\) equal to each other and move them both to one side:
\begin{equation}\label{equation:Flow_Control_and_Measurement/FCM_Derivations:Flow_Control_and_Measurement/FCM_Derivations:3}
\begin{split}A_{Tank} \frac{dh}{dt} + A_{Valve} \sqrt{\frac{2gh}{\sum K}} = 0\end{split}
\end{equation}
From here, calculus and equation substitution dominate the derivation. Separating the variables of the equation immediately above, we get the following integral:
\begin{equation}\label{equation:Flow_Control_and_Measurement/FCM_Derivations:Flow_Control_and_Measurement/FCM_Derivations:4}
\begin{split}\frac{ -A_{Tank}}{{A_{Valve}} \sqrt{\frac{2g}{\sum K}} }   \int \limits_{h_0}^h \frac{dh}{\sqrt h} = \int \limits_0^t {dt}\end{split}
\end{equation}
Which, when integrated, yields:
\begin{equation}\label{equation:Flow_Control_and_Measurement/FCM_Derivations:Flow_Control_and_Measurement/FCM_Derivations:5}
\begin{split}\frac{ -A_{Tank}}{A_{Valve} \sqrt{ \frac{2g}{\sum K}} } \cdot 2 \left( \sqrt{h} - \sqrt{h_0} \right) = t\end{split}
\end{equation}
And solved for \(\sqrt{h}\) returns:
\begin{equation}\label{equation:Flow_Control_and_Measurement/FCM_Derivations:Flow_Control_and_Measurement/FCM_Derivations:6}
\begin{split}\sqrt h  = \sqrt{h_0} - t \frac{A_{Valve}}{2 A_{tank}} \sqrt {\frac{2g}{\sum K}}\end{split}
\end{equation}
At this point, the steps and equation substitutions may begin to seem unintuitive. Do not worry if you do not understand why \sphinxstyleemphasis{exactly} a particular substitution is occurring. Since we determined above that \(h_e = h\), our equation above for \(\sqrt{h}\) is also an equation for \(\sqrt{h_e}\). As such, we will plug the equation above back into the minor loss equation solved for \(Q\) from above, \(Q = A_{Valve} \sqrt{\frac{2 h_e g}{\sum K}}\), to produce:
\begin{equation}\label{equation:Flow_Control_and_Measurement/FCM_Derivations:Flow_Control_and_Measurement/FCM_Derivations:7}
\begin{split}Q = A_{Valve} \sqrt{\frac{2g}{\sum K}} \left( \sqrt{h_0}  - t \frac{A_{Valve}}{2 A_{tank}} \sqrt{\frac{2g}{\sum K}} \right)\end{split}
\end{equation}
Now we can focus on getting rid of the variables \(A_{Valve}\), \(\sum K\), and \(A_{tank}\). By using the minor loss equation once more, we can remove both \(A_{Valve}\) and \(\sum K\). Consider the initial state of the system, when the hypochlorinator is set up and starts dropping its first few drops of hypochlorite solution into the water. The initial flow rate, \(Q_0\), and elevation difference between the water level and the valve, \(h_0\), can be input into the minor loss equation, which can then be solved for \(A_{Valve}\):
\begin{equation}\label{equation:Flow_Control_and_Measurement/FCM_Derivations:Flow_Control_and_Measurement/FCM_Derivations:8}
\begin{split}A_{Valve} = \frac{Q_{0}}{ \sqrt{ \frac{2 h_0 g}{\sum K}} }\end{split}
\end{equation}
Plugging this equation for \(A_{Valve}\) into the equation for \(Q\) just above, we get the following two equations, in which the second equation is a simplified version of the first:
\begin{equation}\label{equation:Flow_Control_and_Measurement/FCM_Derivations:Flow_Control_and_Measurement/FCM_Derivations:9}
\begin{split}Q = Q_0 \frac{1}{\sqrt{h_0}} \left( \sqrt{h_0} - \frac{Q_0 t}{2 A_{Tank} \sqrt{h_0}} \right)\end{split}
\end{equation}\begin{equation}\label{equation:Flow_Control_and_Measurement/FCM_Derivations:Flow_Control_and_Measurement/FCM_Derivations:10}
\begin{split}\frac{Q}{Q_0} = 1 - \frac{t Q_0}{2 A_{Tank} h_0}\end{split}
\end{equation}
This next step will eliminate \(A_{Tank}\). However, it requires some clever manipulation that has a tendency to cause some confusion. We will define a new parameter, \(t_{Design}\), which represents the time it would take to empty the tank \sphinxstylestrong{if the initial flow rate through the valve, :math:{}`Q\_0{}`, stays constant in time}. Of course, the flow \(Q\) through the valve does not stay constant in time, which is why this derivation document exists. But imagining this hypothetical \(t_{Design}\) parameter allows us to form the following equation:
\begin{equation}\label{equation:Flow_Control_and_Measurement/FCM_Derivations:Flow_Control_and_Measurement/FCM_Derivations:11}
\begin{split}Q_0 t_{Design} = A_{Tank} h_{Tank}\end{split}
\end{equation}
This equation describes draining all the hypochlorite solution from the tank. The volume of the solution, \(A_{Tank} h_{Tank}\), is drained in \(t_{Design}\). Rearranged, the equation becomes:
\begin{equation}\label{equation:Flow_Control_and_Measurement/FCM_Derivations:Flow_Control_and_Measurement/FCM_Derivations:12}
\begin{split}\frac{Q_0}{A_{Tank}} = \frac{h_{Tank}}{t_{Design}}\end{split}
\end{equation}
\begin{DUlineblock}{0em}
\item[] Such that:
\item[] \(h_{Tank}\) = elevation of water level in the tank with reference to tank bottom at the initial state, \(t = 0\)
\end{DUlineblock}

Here lies another common source of confusion. \(h_{Tank}\) is not the same as \(h_{0}\). \(h_{Tank}\) is the height of water level in the tank with reference to the tank bottom. \(h_{0}\) is the water level in the tank with reference to the valve. Therefore, \(h_{0} \geq h_{Tank}\) is true if the valve is located at or below the bottom of the tank. If the tank is elevated far above the valve, then the \(h_{0} > > h_{Tank}\). If the valve is at the same elevation as the bottom of the tank, then \(h_{0} = h_{Tank}\). Please refer to the following image to clarify \(h_{0}\) and \(h_{Tank}\). Also note that both \(h_{Tank}\) and \(h_{0}\) are not variables, they are constants which are defined by the initial state of the hypochlorinator, when the solution just begins to flow.

\begin{figure}[htbp]
\centering
\capstart

\noindent\sphinxincludegraphics[width=600\sphinxpxdimen]{{hypochlorinator_variable_explanation}.png}
\caption{\(Q_0 =\) initial flow rate of hypochlorite solution at time \(t = 0\), \(t_{Design} =\) time it would take to drain the tank if flow was held constant at \(Q_0\)}\label{\detokenize{Flow_Control_and_Measurement/FCM_Derivations:id2}}\label{\detokenize{Flow_Control_and_Measurement/FCM_Derivations:figure-hypochlorinator-variable-explanation}}\end{figure}

Finally, our fabricated equivalence, \(\frac{Q_0}{A_{Tank}} = \frac{h_{Tank}}{t_{Design}}\) can be plugged into \(\frac{Q}{Q_0} = 1 - \frac{t Q_0}{2 A_{Tank} h_0}\) to create the highly useful equation for flow rate as a function of time for a drip hypochlorinator:
\begin{equation}\label{equation:Flow_Control_and_Measurement/FCM_Derivations:Flow_Control_and_Measurement/FCM_Derivations:13}
\begin{split}\color{purple}{
\frac{Q}{Q_0} = 1 - \frac{1}{2} \frac{t}{t_{Design}} \frac{h_{Tank}}{h_0}
}\end{split}
\end{equation}
Which can be slightly rearranged to yield:
\begin{equation}\label{equation:Flow_Control_and_Measurement/FCM_Derivations:Flow_Control_and_Measurement/FCM_Derivations:14}
\begin{split}\color{purple}{
Q(t) = Q_0 \left( 1 - \frac{1}{2} \frac{t}{t_{Design}} \frac{h_{Tank}}{h_0} \right)
}\end{split}
\end{equation}
\begin{DUlineblock}{0em}
\item[] Such that:
\item[] \(Q = Q(t)\) = flow of hypochlorite through valve at time \(t\)
\item[] \(t\) = elapsed time
\item[] \(t_{Design}\) = time it would take for tank to empty \sphinxstyleemphasis{if} flow stayed constant at \(Q_0\), which it does not
\item[] \(h_{Tank}\) = elevation of water level with reference to tank bottom
\item[] \(h_0\) = elevation of water level with reference to the valve
\end{DUlineblock}

“How does this ‘tank with a valve’ scenario differ from the ‘hole in a bucket’ scenario?” some might ask. If you are interested, you may go through the derivation on your own using the orifice equation instead of the minor loss equation for the first step. If you do so you’ll find that the equation remains almost the same, the only difference being that the \(\frac{h_{Tank}}{h_0}\) term drops out for an orifice, as \(h_{Tank} = h_0\). The big difference in the systems lies with the flexibility of having a valve. It can be tightened or loosened to change the flow rate, whereas changing the size of an orifice multiple times in a row is not recommended and is usually irreversible.


\section{\protect\(D(t)\protect\) and \protect\(t(D)\protect\) for Tank Drain Equation}
\label{\detokenize{Flow_Control_and_Measurement/FCM_Derivations:and-for-tank-drain-equation}}\label{\detokenize{Flow_Control_and_Measurement/FCM_Derivations:heading-diameter-and-time-tank-drain-equation}}
This document contains the derivation of \(D_{Pipe}\), which is the pipe diameter necessary to install in a drain system to entirely drain a tank in time \(t_{Drain}\).

First, it is necessary to understand how AguaClara tank drains work and what they look like. Many tanks, including the flocculator and entrance tank, have a hole in their bottoms which are fitted with \sphinxhref{https://www.mrpoolman.com.au/assets/thumbL/16057.jpg}{pipe couplings}. During normal operation, these couplings have pipe stubs in them, and the pipe stubs are tall enough to go above the water level in the tank and not allow water to flow into the drain. When the pipe stub is removed, the water begins to flow out of the drain, as the image below indicates. The drain pipe consists of pipe and one elbow, shown in the image.

\begin{figure}[htbp]
\centering
\capstart

\noindent\sphinxincludegraphics[width=600\sphinxpxdimen]{{pipe_stub_drainage}.png}
\caption{This is AguaClara’s alternatives to having valves.}\label{\detokenize{Flow_Control_and_Measurement/FCM_Derivations:id3}}\label{\detokenize{Flow_Control_and_Measurement/FCM_Derivations:figure-pipe-stub-drainage}}\end{figure}

While AguaClara sedimentation tanks use valves instead of pipe to begin the process of draining, the actual drain piping system is the same, pipe and an elbow. The equation that will soon be derived applies to both pipe stub and valve drains.

We will start the derivation from the following equation, which is found in an intermediate step from the ‘\(Q(t)\) {\hyperref[\detokenize{Flow_Control_and_Measurement/FCM_Derivations:heading-flow-for-a-tank-with-a-valve}]{\sphinxcrossref{\DUrole{std,std-ref}{ for a Tank with a Valve}}}}.’ While this system does not have a valve, it has other sources of minor loss and therefore the equation is still valid.
\begin{equation}\label{equation:Flow_Control_and_Measurement/FCM_Derivations:Flow_Control_and_Measurement/FCM_Derivations:15}
\begin{split}\sqrt h  = \sqrt{h_0} - t \frac{A_{Valve}}{2 A_{Tank}} \sqrt {\frac{2g}{K}}\end{split}
\end{equation}
We need to make some adjustments to this equation before proceeding, to make it applicable for this new drain-system scenario. First, we want to assume that the tank has fully drained. Thus, \(t = t_{Drain}\) and \(h = 0\). Next, we recall that the tank drain is not actually a valve, but just pipe and an elbow, so \(A_{Valve} = A_{Pipe}\). Additionally, there can be multiple points of minor loss in the drain system: the entrance from the tank into the drain pipe, the elbow, and potentially the exit of the water out of the drain pipe. When considering a sedimentation tank, the open valve required to begin drainage also has a minor loss associated with it. Therefore, it is necessary to substitute \(\sum K\) for \(K\) With these substitutions, the equation becomes:
\begin{equation}\label{equation:Flow_Control_and_Measurement/FCM_Derivations:Flow_Control_and_Measurement/FCM_Derivations:16}
\begin{split}\sqrt{h_0}  = t_{Drain} \frac{A_{Pipe}}{2 A_{Tank}} \sqrt {\frac{2g}{\sum K}}\end{split}
\end{equation}
Now, with the knowledge that \(A_{Pipe} = \frac{\pi D_{Pipe}^2}{4}\) and rearranging to solve for \(D_{Pipe}\), we obtain the following equation:
\begin{equation}\label{equation:Flow_Control_and_Measurement/FCM_Derivations:Flow_Control_and_Measurement/FCM_Derivations:17}
\begin{split}D_{Pipe} = \sqrt{ \frac{8 A_{Tank}}{\pi t_{Drain}} \sqrt{ \frac{h_0 \sum K}{2g} } }\end{split}
\end{equation}
To get the equation in terms of easily measureable tank parameters, we substitute \(L_{Tank} W_{Tank}\) for \(A_{Tank}\). To maintain consistency in variable names, we substitute \(H_{Tank}\) for \(h_0\).

\begin{sphinxadmonition}{note}{Note:}
By saying that \(h_0 = H_{Tank}\), we are making the assumption that the pipe drain is at the same elevation as the bottom of the tank. The pipe drain is actually a little lower than the bottom of the tank, but that would make the tank drain faster than \(t_{Drain}\), which is preferred. Therefore, we are designing a slight safety factor when we say that \(h_0 = H_{Tank}\).
\end{sphinxadmonition}

Finally, we arrive at the equation for drain pipe sizing:
\begin{equation}\label{equation:Flow_Control_and_Measurement/FCM_Derivations:Flow_Control_and_Measurement/FCM_Derivations:18}
\begin{split}\color{purple}{
D_{Pipe} = \sqrt{ \frac{8 L_{Tank} W_{Tank}}{\pi t_{Drain}}} \left( \frac{H_{Tank} \sum K}{2g} \right)^{\frac{1}{4}}
}\end{split}
\end{equation}
We can also easily rearrange to find the time required to drain a tank given a drain diameter:
\begin{equation}\label{equation:Flow_Control_and_Measurement/FCM_Derivations:Flow_Control_and_Measurement/FCM_Derivations:19}
\begin{split}\color{purple}{
t_{Drain} = \frac{8 L_{Tank} W_{Tank}}{\pi D_{Pipe}^2} \sqrt{ \frac{H_{Tank} \sum K}{2g} }
}\end{split}
\end{equation}
Such that the variables are as the appear in the image below.

\begin{figure}[htbp]
\centering
\capstart

\noindent\sphinxincludegraphics[width=600\sphinxpxdimen]{{pipe_stub_drainage_variables}.png}
\caption{\(L_{Tank}\) is the length of the tank which goes the page. \(K\) is the aggregate minor loss coefficient of the drain system.}\label{\detokenize{Flow_Control_and_Measurement/FCM_Derivations:id4}}\label{\detokenize{Flow_Control_and_Measurement/FCM_Derivations:figure-pipe-stub-drainage-variables}}\end{figure}


\section{Design Equations for the Linear Chemical Dose Controller (CDC)}
\label{\detokenize{Flow_Control_and_Measurement/FCM_Derivations:design-equations-for-the-linear-chemical-dose-controller-cdc}}\label{\detokenize{Flow_Control_and_Measurement/FCM_Derivations:heading-design-equations-for-the-cdc}}
This document will include the equation derivations required to design a CDC system. The most important restriction in this design process is maintaining linearity between head \(h\) and flow \(Q\), which is the entire purpose of the CDC. Recall that major losses under laminar flow scale with \(Q\) and minor losses scale with \(Q^2\) Since it is impossible to remove minor losses from the system entirely, we will simply try to make minor losses very small compared to major losses. The CDC does this by including ‘dosing tube(s),’ which are long, straight tubes designed to generate a lot of major losses. There can be one tube or multiple, depending on the design conditions.

We will use the ‘head loss trick’ that was introduced in the Fluids Review section. Therefore, the elevation difference between the water level in the constant head tank (CHT) and the end of the tube connected to the slider, \(\Delta h\), is equal to the head loss between the two points, \(h_L\). Thus, \(\Delta h = h_L = h_e + h_f\).

\begin{sphinxadmonition}{note}{Note:}
There are a lot of equations in this section, and they may quickly get confusing. They are color coded in an attempt to make them easier to follow. There are two final design equations: \(\color{purple}{\bar v_{Max}}\) and math:\sphinxtitleref{color\{purple\}\{L\_\{Min\}\}}, and they will be written in \(\color{purple}{\rm{purple \, text \, coloring}}\) to make them noticeable.
\end{sphinxadmonition}

\begin{figure}[htbp]
\centering
\capstart

\noindent\sphinxincludegraphics[width=600\sphinxpxdimen]{{CDC_derivation}.png}
\caption{Visual representation of CDC.}\label{\detokenize{Flow_Control_and_Measurement/FCM_Derivations:id5}}\label{\detokenize{Flow_Control_and_Measurement/FCM_Derivations:figure-cdc-derivation}}\end{figure}


\subsection{CDC Design Equation Derivation}
\label{\detokenize{Flow_Control_and_Measurement/FCM_Derivations:cdc-design-equation-derivation}}\label{\detokenize{Flow_Control_and_Measurement/FCM_Derivations:heading-cdc-design-equation-derivations}}
\begin{sphinxadmonition}{important}{Important:}
\sphinxstylestrong{When designing the CDC, there are a few parameters which are picked and set initially, before applying any equations. These parameters are:}
\end{sphinxadmonition}
\begin{enumerate}
\item {} 
\(D\) = tube diameter. only certain tubing diameters are manufactured (like \(\frac{x}{16}\) inch), so an array of available tube diameters is set initially.

\item {} 
\(\sum K\) = sum of minor loss coefficients for the whole system. This is also set initially, it is usually 2.

\item {} 
\(h_{L_{Max}}\) = maximum elevation difference between CHT water level and outlet of solution. This parameter is usually 20 cm.

\end{enumerate}

We begin by defining the head loss through the system \(h_L\), which is equivalent to defining the driving head \(\Delta h\). Major losses will be coded as red.
\begin{equation}\label{equation:Flow_Control_and_Measurement/FCM_Derivations:Flow_Control_and_Measurement/FCM_Derivations:20}
\begin{split}\color{red}{
  h_{\rm{f}} = \frac{128\nu LQ}{g\pi D^4}
  }\end{split}
\end{equation}
\begin{DUlineblock}{0em}
\item[] Such that:
\item[] \(\nu\) = kinematic viscosity \sphinxstyleemphasis{of the solution going through the dosing tube(s)}. This is either coagulant or chlorine
\item[] \(Q\) = flow rate through the dosing tube(s)
\item[] \(L\) = length of the dosing tube(s)
\end{DUlineblock}

\begin{sphinxadmonition}{note}{Note:}
‘Tube(s)’ is used because there may be 1 or more dosing tubes depending on the particular design.
\end{sphinxadmonition}

Minor losses are equal to:
\begin{equation}\label{equation:Flow_Control_and_Measurement/FCM_Derivations:Flow_Control_and_Measurement/FCM_Derivations:21}
\begin{split}h_e = \frac{8 Q^2}{g \pi^2 D^4} \sum{K}\end{split}
\end{equation}
Therefore, the total head loss is a function of flow, and is shown in the following equation.
\begin{equation}\label{equation:Flow_Control_and_Measurement/FCM_Derivations:Flow_Control_and_Measurement/FCM_Derivations:22}
\begin{split}h_L(Q) =
{\color{red}{
  \frac{128\nu L Q}{g \pi D^4}}} +
  \frac{8Q^2}{g \pi^2 D^4} \sum K\end{split}
\end{equation}
Blue will be used to reference \sphinxstyleemphasis{actual} head loss from now on. This is the same equation as above.
\begin{equation}\label{equation:Flow_Control_and_Measurement/FCM_Derivations:Flow_Control_and_Measurement/FCM_Derivations:23}
\begin{split}\color{blue}{
  h_L(Q) = \left( \frac{128\nu L}{g \pi D^4} + \frac{8Q}{g \pi ^2 D^4} \sum{K} \right) Q
  }\end{split}
\end{equation}
This equation is not linear with respect to flow. We can make it linear by turning the variable \(Q\) in the \(\frac{8Q}{g \pi ^2 D^4} \sum{K}\) term into a constant. To do this, we pick a maximum flow rate of coagulant/chlorine through the dose controller, \(Q_{Max}\), and put that into the term in place of \(Q\). The term becomes \(\frac{8Q_{Max}}{g \pi ^2 D^4} \sum{K}\), and our linearized model of head loss, coded as green, becomes:
\begin{equation}\label{equation:Flow_Control_and_Measurement/FCM_Derivations:Flow_Control_and_Measurement/FCM_Derivations:24}
\begin{split}\color{green}{
  h_{L_{linear}}(Q) = \left( \frac{128\nu L}{g \pi D^4} + \frac{8Q_{Max}}{g \pi ^2 D^4} \sum{K} \right) Q
  }\end{split}
\end{equation}
Here is a plot of the three colored equations above. Our goal is to minimize the minor losses in the system; to bring the red and blue curves as close as possible to the green one.

\begin{figure}[htbp]
\centering
\capstart

\noindent\sphinxincludegraphics[width=600\sphinxpxdimen]{{CDC_linearity_model}.png}
\caption{MathCAD generated graph for linearity error analysis. TODO: make this in python}\label{\detokenize{Flow_Control_and_Measurement/FCM_Derivations:id6}}\label{\detokenize{Flow_Control_and_Measurement/FCM_Derivations:figure-cdc-linearity-model}}\end{figure}


\subsubsection{Designing for the error constraint, \protect\(\Pi_{Error}\protect\)}
\label{\detokenize{Flow_Control_and_Measurement/FCM_Derivations:designing-for-the-error-constraint}}
\begin{sphinxadmonition}{important}{Important:}
The first step in the design is to make sure that major losses far exceed minor losses. This will result in an equation for the maximum velocity that can go through the dosing tube(s), \(\color{purple}{\bar v_{Max} }\).
\end{sphinxadmonition}

Minor losses will never be 0, so how much error in our linearity are we willing to accept? Let’s define a new parameter, \(\Pi_{Error}\), as the maximum amount of error we are willing to accept. We are ok with 10\% error or less, so \(\Pi_{Error} = 0.1\).
\begin{equation}\label{equation:Flow_Control_and_Measurement/FCM_Derivations:Flow_Control_and_Measurement/FCM_Derivations:25}
\begin{split}\Pi_{Error} = \frac{\color{green}{ h_{L_{linear}} } - \color{blue}{ h_L }}{\color{green}{ h_{L_{linear}} }} = 1 - \frac{\color{blue}{ h_L }}{\color{green}{ h_{L_{linear}} }}\end{split}
\end{equation}\begin{equation}\label{equation:Flow_Control_and_Measurement/FCM_Derivations:Flow_Control_and_Measurement/FCM_Derivations:26}
\begin{split}1 - \Pi_{Error} = \frac{\color{blue}{ h_L }}{\color{green}{ h_{L_{linear}} }}\end{split}
\end{equation}
Now we plug \(\color{blue}{ h_L(Q) }\) and \(\color{green}{ h_{L_{linear}} }\) back into the equation for \(1 - \Pi_{Error}\) and take the limit as \(Q \rightarrow 0\), as that is when the relative difference between actual head loss and our linear model for head loss is the greatest.
\begin{equation}\label{equation:Flow_Control_and_Measurement/FCM_Derivations:Flow_Control_and_Measurement/FCM_Derivations:27}
\begin{split}1 - \Pi_{Error} =
  \frac{ \color{blue}{
  \left( \frac{128 \nu L}{g \pi D^4} +
  \cancel{\frac{8Q}{g \pi^2 D^4} \sum{K}}
  \right) Q
  }}
  {\color{green}{
  \left( \frac{128 \nu L}{g \pi D^4} + \frac{8 Q_{Max}}{g \pi^2 D^4} \sum{K} \right) Q
  }}
  =     \frac{\left( \frac{128 \nu L}{g \pi D^4} \right)}{\left( \frac{128 \nu L}{g \pi D^4} + \frac{8 Q_{Max}}{g \pi^2 D^4} \sum{K} \right)}\end{split}
\end{equation}
The next steps are algebraic rearrangements to solve for \(L\). This \(L\) describes the \sphinxstyleemphasis{minimum} length of dosing tube necessary to meet our error constraint at \sphinxstyleemphasis{maximum} flow. Thus, we will refer to it as \(L_{Min, \, \Pi_{Error}}\).
\begin{equation}\label{equation:Flow_Control_and_Measurement/FCM_Derivations:Flow_Control_and_Measurement/FCM_Derivations:28}
\begin{split}\left( 1 - \Pi_{Error} \right)  \frac{128 \nu L}{g \pi D^4} + \left( 1 - \Pi_{Error} \right) \frac{8 Q_{Max}}{g \pi ^2 D^4} \sum{K}  =  \frac{128 \nu L}{g \pi D^4}\end{split}
\end{equation}\begin{equation}\label{equation:Flow_Control_and_Measurement/FCM_Derivations:Flow_Control_and_Measurement/FCM_Derivations:29}
\begin{split}- \Pi_{Error} \frac{128 \nu L}{g \pi D^4} + \left( 1 - \Pi_{Error} \right) \frac{8 Q_{Max}}{g \pi^2 D^4} \sum{K}  = 0\end{split}
\end{equation}\begin{equation}\label{equation:Flow_Control_and_Measurement/FCM_Derivations:Flow_Control_and_Measurement/FCM_Derivations:30}
\begin{split}L = \left( \frac{1 - \Pi_{Error}}{\Pi_{Error}} \right) \frac{Q_{Max}}{16 \nu \pi} \sum{K}\end{split}
\end{equation}\begin{equation}\label{equation:Flow_Control_and_Measurement/FCM_Derivations:Flow_Control_and_Measurement/FCM_Derivations:31}
\begin{split}L_{Min, \, \Pi_{Error}} = L = \left( \frac{1 - \Pi_{Error}}{\Pi_{Error}} \right) \frac{Q_{Max}}{16 \nu \pi} \sum{K}\end{split}
\end{equation}
\begin{DUlineblock}{0em}
\item[] Note that this equation is independent of head loss.
\end{DUlineblock}

Unfortunately, both \(L_{Min, \, \Pi_{Error}}\) and \(Q_{Max}\) are unknowns. We can plug this equation for \(L_{Min, \, \Pi_{Error}}\) back into the head loss equation at maximum flow, which is \(h_{L_{Max}} = \left( \frac{128\nu L Q_{Max}}{g \pi D^4} + \frac{8Q_{Max}^2}{g \pi ^2 D^4} \sum{K} \right)\) and rearrange for \(Q_{Max}\) to get:
\begin{equation}\label{equation:Flow_Control_and_Measurement/FCM_Derivations:Flow_Control_and_Measurement/FCM_Derivations:32}
\begin{split}Q_{Max} = \frac{\pi D^2}{4} \sqrt{\frac{2 h_{L_{Max}} g \Pi_{Error}}{\sum K }}\end{split}
\end{equation}

\sphinxstrong{See also:}


\sphinxstylestrong{Function in aide\_design} \sphinxcode{\sphinxupquote{cdc.max\_linear\_flow(Diam, HeadlossCDC, Ratio\_Error, KMinor)}} Returns the maximum flow \(Q_{Max}\) that can go through a dosing tube will making sure that linearity between head loss and flow is conserved.



From this equation for \(Q_{Max}\), we can get to our first design equation, \(\color{purple}{\bar v_{Max}}\) by using the continuity equation \(\bar v_{Max} = \frac{Q_{Max}}{\frac{\pi D^2}{4}}\)
\begin{equation}\label{equation:Flow_Control_and_Measurement/FCM_Derivations:Flow_Control_and_Measurement/FCM_Derivations:33}
\begin{split}\color{purple}{
  \bar v_{Max} = \sqrt{ \frac{2 h_L g \Pi_{Error}}{\sum{K} }}
  }\end{split}
\end{equation}

\subsubsection{Designing for the proper amount of head loss, \protect\(h_{L_{Max}}\protect\)}
\label{\detokenize{Flow_Control_and_Measurement/FCM_Derivations:designing-for-the-proper-amount-of-head-loss}}
\begin{sphinxadmonition}{important}{Important:}
The second step in the design is to make sure that the maximum head loss corresponds to the maximum flow of chemicals. This will result in an equation for the length of the dosing tube(s), \(\color{purple}{L_{Min} }\).
\end{sphinxadmonition}

We previously derived an equation for the minimum length of the dosing tube(s), \(L_{Min, \, \Pi_{Error}}\), which was the minimum length needed to ensure that our linearity constraint was met. This equation is shown again below, in red:
\begin{equation}\label{equation:Flow_Control_and_Measurement/FCM_Derivations:Flow_Control_and_Measurement/FCM_Derivations:34}
\begin{split}\color{red}{
  L_{Min, \, \Pi_{Error}} = \left( \frac{1 - \Pi_{Error}}{\Pi_{Error}} \right) \frac{Q_{Max}}{16 \nu \pi} \sum{K}
  }\end{split}
\end{equation}
This equation does not, however, account for getting to the proper amount of head loss. If we were to use this equation to design the dosing tubes, we might not end up with the proper amount of flow \(Q_{Max}\) at the maximum head loss \(h_{L{Max}}\). So we need to double check to make sure that we get our desired head loss.

First, consider the head loss at maximum flow that was used to get the equation for \(Q_{Max}\):
\begin{equation}\label{equation:Flow_Control_and_Measurement/FCM_Derivations:Flow_Control_and_Measurement/FCM_Derivations:35}
\begin{split}h_{L_{Max}} = \left( \frac{128 \nu L{Q_{Max}}}{g \pi D^4} + \frac{8 Q_{Max}^2}{g \pi^2 D^4} \sum{K} \right)\end{split}
\end{equation}
Now that we know all of the parameters in this equation except for \(L\), we can solve the equation for \(L\). This the \sphinxstyleemphasis{shortest} tube that generates our required head loss, \(h_{L_{Max}}\).
\begin{equation}\label{equation:Flow_Control_and_Measurement/FCM_Derivations:Flow_Control_and_Measurement/FCM_Derivations:36}
\begin{split}\color{green}{
   L_{Min, \, head loss} = L = \left( \frac{g h_{L_{Max}} \pi D^4}{128 \nu Q_{Max}} - \frac{Q_{Max}}{16 \pi \nu} \sum{K} \right)
   }\end{split}
\end{equation}

\sphinxstrong{See also:}


\sphinxstylestrong{Function in aide\_design:} \sphinxcode{\sphinxupquote{cdc.\_length\_cdc\_tube\_array(FlowPlant, ConcDoseMax, ConcStock, DiamTubeAvail, HeadlossCDC, temp, en\_chem, KMinor)}} Returns \(\color{purple}{L_{Min}}\), takes in the flow rate input of \sphinxstyleemphasis{plant design flow rate}.




\sphinxstrong{See also:}


\sphinxstylestrong{Function in aide\_design:} \sphinxcode{\sphinxupquote{cdc.\_len\_tube(Flow, Diam, HeadLoss, conc\_chem, temp, en\_chem, KMinor)}} Returns \(\color{purple}{L_{Min}}\), takes in the flow rate input of \sphinxstyleemphasis{max flow rate through the dosing tube(s)}.



If you decrease the max flow \(Q_{Max}\) and hold \(h_{L_{Max}}\) constant, \(\color{green}{L_{Min, \, head loss}}\) becomes larger. This means that a CDC system for a plant of 40 \(\frac{L}{s}\) must be different than one for a plant of 20 \(\frac{L}{s}\). If we want to maintain the same head loss at maximum flow in both plants, then the dosing tube(s) will need to be a lot longer for the 20 \(\frac{L}{s}\) plant.

To visualize the distinction between \(\color{red}{  L_{Min, \, \Pi_{Error}}}\) and math:\sphinxtitleref{color\{green\}\{ L\_\{Min, , head loss\}\}}, see the following plot. \(\color{green}{ L_{Min, \, head loss}}\) is discontinuous because it takes in the smallest allowable tube diameter as an input. As the chemical flow rate through the dosing tube(s) decreases, the dosing tube diameter does as well. Whenever you see a jump in the green points, that means the tubing diameter has changed.

\begin{figure}[htbp]
\centering
\capstart

\noindent\sphinxincludegraphics[width=600\sphinxpxdimen]{{CDC_length_model}.png}
\caption{CDC length modeling in MathCAD.}\label{\detokenize{Flow_Control_and_Measurement/FCM_Derivations:id7}}\label{\detokenize{Flow_Control_and_Measurement/FCM_Derivations:figure-cdc-length-model}}\end{figure}

As you can see, the head loss constraint is more limiting than the linearity constraint when designing for tube length. Therefore, the design equation for tube length is the one which accounts for head loss. This is the second and final design equation for designing the CDC:
\begin{equation}\label{equation:Flow_Control_and_Measurement/FCM_Derivations:Flow_Control_and_Measurement/FCM_Derivations:37}
\begin{split}\color{purple}{
L_{Min} = L_{Min, \, head loss} = \left( \frac{g h_{L_{Max}} \pi D^4}{128 \nu Q_{Max}} - \frac{Q_{Max}}{16 \pi \nu} \sum{K} \right)
}\end{split}
\end{equation}
The equations for \(\color{purple}{\bar v_{Max}}\) and \(\color{purple}{L_{Min}}\) are the only ones you \sphinxstylestrong{need} to manually design a CDC.


\subsubsection{CDC Dosing Tube(s) Diameter \protect\(D_{Min}\protect\) Plots}
\label{\detokenize{Flow_Control_and_Measurement/FCM_Derivations:cdc-dosing-tube-s-diameter-plots}}
Below are equations which also govern the CDC and greatly aid in understanding the physics behind it, but are not strictly necessary in design.

By rearranging \(Q_{Max} = \frac{\pi D^2}{4} \sqrt{\frac{2 h_L g \Pi_{Error}}{\sum K }}\), we can solve for \(D\) to get the \sphinxstyleemphasis{minimum} diameter we can use assuming the shortest tube possible that meets the error constraint, \(\color{red}{L_{Min, \, \Pi_{Error}}}\). If we use a diameter smaller than \(D_{Min, \, \Pi_{Error}}\), we will not be able to simultaneously reach \(Q_{Max}\) and meet the error constraint \(\Pi_{Error}\).
\begin{equation}\label{equation:Flow_Control_and_Measurement/FCM_Derivations:Flow_Control_and_Measurement/FCM_Derivations:38}
\begin{split}\color{blue}{
D_{Min, \, \Pi_{Error}} = \left[ \frac{8 Q_{Max}^2 \sum K}{\Pi_{Error} h_l g \pi^2} \right]^{\frac{1}{4}}
}\end{split}
\end{equation}
We can also find the minimum diameter needed to guarantee laminar flow, which is another critical condition in the CDC design. We can do this by combining the equation for Reynolds number at the maximum \(\rm{Re}\) for laminar flow, \({\rm{Re}}_{Max} = 2100\) with the continuity equation at maximum flow:
\begin{equation}\label{equation:Flow_Control_and_Measurement/FCM_Derivations:Flow_Control_and_Measurement/FCM_Derivations:39}
\begin{split}{\rm Re}_{Max}  = \frac{\bar v_{Max} D}{\nu}\end{split}
\end{equation}\begin{equation}\label{equation:Flow_Control_and_Measurement/FCM_Derivations:Flow_Control_and_Measurement/FCM_Derivations:40}
\begin{split}\bar v_{Max} = \frac{4 Q_{Max}}{\pi D^2}\end{split}
\end{equation}
To get:
\begin{equation}\label{equation:Flow_Control_and_Measurement/FCM_Derivations:Flow_Control_and_Measurement/FCM_Derivations:41}
\begin{split}\color{red}{
D_{Min, \, Laminar} = \frac{4 Q_{Max}}{\pi \nu {\rm{Re}}_{Max}}
}\end{split}
\end{equation}
Combined with the discrete amount of tubing sizes (shown in dark green), we can create a graph of the three diameter constraints:

\begin{figure}[htbp]
\centering
\capstart

\noindent\sphinxincludegraphics[width=600\sphinxpxdimen]{{CDC_diameter_model}.png}
\caption{CDC diameter modeling in MathCAD.}\label{\detokenize{Flow_Control_and_Measurement/FCM_Derivations:id8}}\label{\detokenize{Flow_Control_and_Measurement/FCM_Derivations:figure-cdc-diameter-model}}\end{figure}


\chapter{Rapid Mix Introduction}
\label{\detokenize{Rapid_Mix/RM_Intro:rapid-mix-introduction}}\label{\detokenize{Rapid_Mix/RM_Intro:title-rapid-mix-introduction}}\label{\detokenize{Rapid_Mix/RM_Intro::doc}}\begin{description}
\item[{This chapter is currently home for the prerequisites of successful flocculation. Those prerequisites include:}] \leavevmode\begin{itemize}
\item {} 
ensuring that the pH is in the correct range for coagulant nanoparticle formation after the coagulant has been added to the raw water.

\item {} 
increasing the coagulant dose to account for the coagulant interactions with dissolved species that effectively cover some of the coagulant nanoparticle surfaces.

\item {} 
increasing the coagulant dose to account for the available surface area of suspended particles to achieve

\item {} 
If there are multiple treatment trains, mixing the coagulant with the raw water so that parallel treatment trains receive the same concentration of coagulant (perhaps the conventional role of rapid mix)

\item {} 
transporting the coagulant nanoparticles to attach to suspended particles

\end{itemize}

\end{description}

Rapid mix is the term commonly used to describe the processes that occur between the coagulant addition to the raw water and the flocculation process. The processes that occur are not well understood and thus design guidelines have been empirical.

“In summary, little is known about rapid mix, much less any sensitivity to scale. However, the models and data reviewed suggest the need to be on the lookout for certain effects. From what is presently known, it can be speculated that since coagulant precipitation is sensitive to both micro- and macro-mixing, scale-up must consider not only energy dissipation rate, but also the reaction injection point and the contacting method.” - \sphinxhref{https://books.google.com/books/about/Mixing\_in\_coagulation\_and\_flocculation.html?id=dkFSAAAAMAAJ}{Mixing in Coagulation and Flocculation 1991 page 292}.

Although the processes have not been well characterized, the energy that is invested for rapid mix processes is significant. In many cases the amount of energy used isn’t practical for gravity powered water treatment plants. The high energy consumption of rapid mix units has led some municipal water treatment plant operators to experiment with turning off rapid mix units. They have found that at least under some conditions there is no indication that the energy used in rapid mix improved plant performance. Thus there is a need to understand the physical and chemical processes that occur when a concentrated liquid coagulant is added to water.

Rapid mix sets the stage for aggregation of both suspended particles and dissolved substances. Particle and dissolve substance aggregation is mediated by coagulant nanoparticles. The nanoparticles attach to raw water particles as well as to some dissolved species. After the nanoparticles have been mixed with the raw water and have attached to raw water particles the next process, flocculation, can begin.  {\hyperref[\detokenize{Flocculation/Floc_Intro:title-flocculation-introduction}]{\sphinxcrossref{\DUrole{std,std-ref}{Flocculation}}}} is the process of producing collisions between particles to create flocs (aggregates of particles).

Coagulant nanoparticle application includes multiple steps that must occur before the raw water particles can begin to aggregate. The sticky nanoparticles can be aluminum \((Al^{+3})\) or iron \((Fe^{+3})\) based and in either case the nanoparticles are formed from precipitated hydroxide species (\(Al(OH)_3\) or \(Fe(OH)_3\)). The series of events that are contained in the broad designation of “rapid mix” are:
\begin{enumerate}
\item {} 
Liquid coagulant stock solution with a low pH is injected into the raw water

\item {} 
Fluid Mixing: Turbulent eddies randomize the fluids (but don’t blend them)
\begin{enumerate}
\item {} 
Large scale eddies mix the coagulant with the raw water by creating large fluid deformations. This stretching and turning of the raw water and coagulant is analogous to shuffling a deck of cards. The cards are randomized, but the cards maintain their identity. The original liquids retain their chemical composition. This step must be completed before any flow splitting for parallel treatment trains.

\item {} 
Turbulent eddies disintegrate into smaller and smaller eddies.

\item {} 
At a very small scale (Inner viscous length scale) viscosity becomes significant and the kinetic energy of the eddies begins to be converted to heat by viscosity.

\end{enumerate}

\item {} 
The coagulant is blended with the raw water by molecular diffusion

\item {} 
The higher pH of the raw water causes the coagulant to begin to precipitate as \(Al_{12}AlO_4(OH)_{24}(H_2O)_{12}^{7+}\), an aluminum, Al, nanoparticle.

\item {} 
The precipitating \(Al_{13}\) molecules aggregates with other nearby \(Al_{13}\) molecules to form aluminum hydroxide nanoparticles. It is also possible that the nanoparticles are already formed in the coagulant stock suspension. Polyaluminum chloride stock solutions turn white in about a year at room temperature and this suggests that nanoparticles form in the stock solution.

\item {} 
The Al nanoparticles attach to other dissolved species and suspended particles.

\item {} 
Molecular diffusion causes some dissolved species and Al nanoparticles to aggregate.

\item {} 
Fluid shear and molecular diffusion cause Al nanoparticles with attached formerly dissolved species to collide with inorganic particles (such as clay) and organic particles (such as viruses, bacteria, and protozoans).

\end{enumerate}

These multiple steps cover a wide range of length scales and it is not clear at the onset which processes might be the rate limiting steps. We will develop time scale estimates for several of these steps to help identify which processes will likely require the most attention to design. Many of these transport processes are presumed to occur in parallel. \hyperref[\detokenize{Rapid_Mix/RM_Intro:figure-transport-length-scales}]{Fig.\@ \ref{\detokenize{Rapid_Mix/RM_Intro:figure-transport-length-scales}}} shows the range of length scales.

\begin{figure}[htbp]
\centering
\capstart

\noindent\sphinxincludegraphics[width=700\sphinxpxdimen]{{rapid_mix_macro_to_nano_scale}.png}
\caption{Transport of coagulant nanoparticles occurs over length scales ranging from meter to a fraction of a nanometer.}\label{\detokenize{Rapid_Mix/RM_Intro:id2}}\label{\detokenize{Rapid_Mix/RM_Intro:figure-transport-length-scales}}\end{figure}


\section{Chemistry of Coagulant Nanoparticles}
\label{\detokenize{Rapid_Mix/RM_Intro:chemistry-of-coagulant-nanoparticles}}\label{\detokenize{Rapid_Mix/RM_Intro:heading-chemistry-of-coagulant-nanoparticles}}\begin{description}
\item[{Aluminum based coagulants are commonly used in drinking water treatment plants. Less frequently iron based coagulants are used. These metals precipitate in water at neutral pH as \(Al(OH)_3\) or \(Fe(OH)_3\). These precipitates form nanoparticles that are sticky. The origin of that stickiness is not well known, but one significant property of both precipitates is that they are both highly polar molecules. The \sphinxhref{https://en.wikipedia.org/wiki/Electronegativity}{difference in electronegativity} (Pauling scale) between}] \leavevmode\begin{itemize}
\item {} 
Aluminum (1.61) and Oxygen (3.44) is 1.83

\item {} 
Iron (1.83) and Oxygen (3.44) is 1.61

\item {} 
Hydrogen (2.20) and Oxygen (3.44) is 1.24

\end{itemize}

\end{description}

Thus both aluminum and iron coagulants are more polar than water and it is possibly that it is their strong polarity that enables them to displace water that is bound to particles surfaces and then form bonds with that surface. In order to displace water molecules that are bound to the particles surfaces, the coagulants must have stronger bonds to particles surfaces than the polar water molecules and thus it seems likely that coagulants must be more polar than water.


\subsection{pH Effects of Adding Coagulant}
\label{\detokenize{Rapid_Mix/RM_Intro:ph-effects-of-adding-coagulant}}\label{\detokenize{Rapid_Mix/RM_Intro:heading-ph-effects-of-adding-coagulant}}
The coagulants used for drinking water treatment are acidic and thus result in a lowering of the pH of the treated water. The optimal pH for aluminum coagulant nanoparticle formation is between pH of 6.5 and 8.5. This is also the \sphinxhref{https://www.epa.gov/dwstandardsregulations/secondary-drinking-water-standards-guidance-nuisance-chemicals}{pH range set by the EPA secondary standards for drinking water}. Although many water sources are within this pH range, there are some waters with more extreme values of pH. The aluminum and iron based coagulants are also acidic and in some waters the pH may drop below the ideal range when adding the coagulant. When the pH is outside the acceptable range it is necessary to adjust the pH by adding either a base or an acid.

When aluminum sulfate (alum) to water it dissociates and then precipitates as \(Al(OH)_3\). In the process protons, \(H^+\) are released and thus the pH (\(-log[H^+]\)) decreases.
\begin{equation}\label{equation:Rapid_Mix/RM_Intro:Rapid_Mix/RM_Intro:0}
\begin{split}Al_2(SO_4)_3 + 6H_2O\rightarrow 2Al(OH)_3 + 6H^+ + 3SO_4^{-2}\end{split}
\end{equation}
The release of these protons reduces the acid neutralizing capacity, ANC, (also known as alkalinity) of the water. ANC is traditionally measured with units of mg/L of \(CaCO_3\) rather than eq/L.  \hyperref[\detokenize{Rapid_Mix/RM_Intro:table-anc-consumed-by-alum}]{Table \ref{\detokenize{Rapid_Mix/RM_Intro:table-anc-consumed-by-alum}}} shows the relationship between ANC measured as mg/L of \(CaCO_3\) and alum (\(Al_2(SO_4)_3 \cdot 14H_2O\))


\begin{savenotes}\sphinxattablestart
\raggedright
\sphinxcapstartof{table}
\sphinxcaption{Reduction in ANC caused by addition of alum.}\label{\detokenize{Rapid_Mix/RM_Intro:id3}}\label{\detokenize{Rapid_Mix/RM_Intro:table-anc-consumed-by-alum}}
\sphinxaftercaption
\begin{tabulary}{\linewidth}[t]{|T|T|T|}
\hline
\sphinxstyletheadfamily &\sphinxstyletheadfamily 
Alum
&\sphinxstyletheadfamily 
Calcium Carbonate
\\
\hline
Molecular Formula
&
\(Al_2(SO_4)_3 \cdot 14.3H_2O\)
&
\(CaCO_3\)
\\
\hline
Molecular mass
&
600 g/mole
&
100 g/mole
\\
\hline
eq/mole
&
6
&
2
\\
\hline
Molecular mass/eq
&
100 g/eq
&
50 g/eq
\\
\hline
Simple guide
&
1 mg/L alum consumes
&
0.5 mg/L calcium carbonate ANC
\\
\hline
\end{tabulary}
\par
\sphinxattableend\end{savenotes}

Low ANC waters (See section on {\hyperref[\detokenize{Rapid_Mix/RM_Intro:heading-buffering-capacity-of-natural-waters}]{\sphinxcrossref{\DUrole{std,std-ref}{Buffering Capacity of Natural Waters}}}}.) could have their ANC increased by addition of a base. A simpler approach is often to use a different coagulant that is less acidic.

Polyaluminum chloride (PACl) is another aluminum based coagulant that performs similarly to alum. PACl is manufactured by slowly titrating an acidic solution containing dissolved aluminum with a base (in the chemical plant) to produce a meta-stable and soluble polymeric aluminum. The PACl consumes less alkalinity (ANC) because it is partially neutralized by the titration with a base. In addition, the aluminum mass fraction of PACl is higher than in alum because there are no attached water molecules. The mass of PACl required for flocculation is less than for alum due largely to the higher aluminum fraction. The lower mass of PACl required is an economic benefit when shipping is a significant cost of the coagulant.

The PACl molecular formula is:
\begin{equation}\label{equation:Rapid_Mix/RM_Intro:Rapid_Mix/RM_Intro:1}
\begin{split}[Al_n(OH)_mCl_{3n-m}]_x\end{split}
\end{equation}
The extent of the PACl titration with base is defined as basicity. Basicity is the ratio of hydroxyl equivalents to aluminum equivalents. Basicity of 1 would mean that the PACl does not produce any protons when it dissolves in water. Basicity of
0 means it produces 3 protons per Al (like alum). The equation for basicity is:
\begin{equation}\label{equation:Rapid_Mix/RM_Intro:Rapid_Mix/RM_Intro:2}
\begin{split}Basicity = \left( \frac{m}{3n}\right)\end{split}
\end{equation}
The lowest basicity commercial PACl formulations are about 10\%. Most PACls are in the medium to high basicity range (50-70\%). The highest stable basicity (83\%) is aluminum chlorohydrate (\(Al_2(OH)_5Cl\)) that is useful for treating water with very low ANC.

The ANC of the aluminum coagulant can be obtained from the number of protons it releases:
\begin{equation}\label{equation:Rapid_Mix/RM_Intro:Rapid_Mix/RM_Intro:3}
\begin{split}ANC_{Al} = 3(Basicity-1)[Al] = \left(\frac{m}{n} - 3\right)[Al] = \Pi_{Al}[Al]\end{split}
\end{equation}
\begin{DUlineblock}{0em}
\item[] where:
\item[] \(\Pi_{Al}=\left(\frac{m}{n} - 3\right)\) is ANC per mole of aluminum for the given coagulant
\end{DUlineblock}

Thus the ANC of alum (with 0 hydroxides) is \(-3[Al]\). The method of calculating the \(ANC_{Al}\) will be used to calculate the amount of base that must be added to achieve a target pH.


\subsection{Buffering Capacity of Natural Waters}
\label{\detokenize{Rapid_Mix/RM_Intro:buffering-capacity-of-natural-waters}}\label{\detokenize{Rapid_Mix/RM_Intro:heading-buffering-capacity-of-natural-waters}}
When acid is added to a water containing bicarbonate, \(HCO_3^-\), one of the potential reactions is for a proton to combine with \(HCO_3^-\) to form carbonic acid, \({H_2}CO_3\). If a base is added to water the reaction will proceed in the opposite direction. Carbonic acid, \({H_2}CO_3\), is chemical indistinguishable from dissolved carbon dioxide, \(CO_{2_{aq}}\) and the total of carbonic acid and dissolved carbon dioxide is represented as \({H_2}CO_3^{\star}\). The reaction of bicarbonate to form carbonic acid removes one proton from solution and thus the concentration of protons doesn’t increase as fast as we might have first expected as acid is added to the water.

The reactions of carbonate species with protons provides pH buffering capacity that must be considered when calculating the effect of acid or base addition. Since carbonates are the dominant buffering agents in natural waters it is essential to account for their influence on pH.

The effect of acid or base addition to a water containing carbonates (or other weak acids and bases) can be modeled using the equation for {\hyperref[\detokenize{Rapid_Mix/RM_Derivations:heading-acid-neutralizing-capacity-anc-or-alkalinity}]{\sphinxcrossref{\DUrole{std,std-ref}{Acid Neutralizing Capacity}}}}.


\subsection{pH Range for Precipitation of Coagulant Nanoparticles}
\label{\detokenize{Rapid_Mix/RM_Intro:ph-range-for-precipitation-of-coagulant-nanoparticles}}\label{\detokenize{Rapid_Mix/RM_Intro:heading-ph-range-for-precipitation-of-coagulant-nanoparticles}}
A critical property of coagulants is that in order to act as an adhesive between particles they must be solid phase at neutral pH. Both Al(III) and Fe(III) have low solubility at neutral pH and thus meet this requirement. The pH region of low solubility sets the range of pH where flocculation is effective. \hyperref[\detokenize{Rapid_Mix/RM_Intro:figure-al-solubility}]{Fig.\@ \ref{\detokenize{Rapid_Mix/RM_Intro:figure-al-solubility}}} shows the solubilty of aluminum as a function of pH.

\begin{figure}[htbp]
\centering
\capstart

\noindent\sphinxincludegraphics[width=600\sphinxpxdimen]{{Al_solubility}.png}
\caption{Solubility of aluminum as a function of pH. Figure adapted from \sphinxhref{http://dx.doi.org/10.2166/aqua.2006.062}{Pernitsky and Edzwald}.}\label{\detokenize{Rapid_Mix/RM_Intro:id4}}\label{\detokenize{Rapid_Mix/RM_Intro:figure-al-solubility}}\end{figure}

Research is needed to quantify flocculation performance in continuous flow floc/floc blanket/plate settler systems as a function of pH.

The aluminum concentration range used for flocculation ranges from approximately 0.4 - 10 mg/L and is strongly influenced by the concentration of dissolved organic matter and the concentration of suspended solids. The flocculation and floc blanket capacity to produce collisions between suspended particles also influences the required aluminum concentration.


\subsection{pH Adjustment in Water Treatment Plants}
\label{\detokenize{Rapid_Mix/RM_Intro:ph-adjustment-in-water-treatment-plants}}\label{\detokenize{Rapid_Mix/RM_Intro:heading-ph-adjustment-in-water-treatment-plants}}
In drinking water treatment plant operation it is sometimes necessary to add a base (or acid) to increase (or decrease) the pH of the raw water. The added coagulant tends to reduce the pH. The carbonate system is most important in understanding how the base will adjust the pH because the reaction between carbonic acid and bicarbonate occurs around pH 6.3, the pK (negative log of the dissociation constant is the pH where that reaction is centered) for that reaction. Carbon dioxide exchange with the atmosphere is insignificant in drinking water treatment unit processes unless there is an aeration stage. Thus we can use the ANC equation for the case with no \(CO_2\) exchange with the atmosphere.

In the section, {\hyperref[\detokenize{Rapid_Mix/RM_Derivations:heading-ph-adjustment}]{\sphinxcrossref{\DUrole{std,std-ref}{pH Adjustment}}}} we evaluate the case where we add a base that will increase the ANC of the raw water and it might also increase the total carbonate concentration. We calculate how much of that base to add to reach a target pH.


\section{Fluid Mixing}
\label{\detokenize{Rapid_Mix/RM_Intro:fluid-mixing}}\label{\detokenize{Rapid_Mix/RM_Intro:id1}}
Fluid mixing is the process by which large scale eddies distribute packets of the coagulant stock throughout the raw water. The term “Rapid mix” is probably best used to describe this process. Traditional methods of achieving this fluid mixing include various methods of generating intense turbulence. Fluid mixing is able to rapidly blend the coagulant with the raw water in a matter of a few seconds. The equations describing the fluid mixing process are presented in the section on {\hyperref[\detokenize{Rapid_Mix/RM_Derivations:heading-estimates-of-time-required-for-mixing-processes}]{\sphinxcrossref{\DUrole{std,std-ref}{Estimates of time required for mixing processes}}}}.

\begin{figure}[htbp]
\centering
\capstart

\noindent\sphinxincludegraphics[width=200\sphinxpxdimen]{{Backmix}.jpg}
\caption{Backmix: a mechanical rapid mixer that has a relatively long residence time in a completely mixed flow reactor.}\label{\detokenize{Rapid_Mix/RM_Intro:id5}}\label{\detokenize{Rapid_Mix/RM_Intro:figure-backmix}}\end{figure}

\begin{figure}[htbp]
\centering
\capstart

\noindent\sphinxincludegraphics[width=400\sphinxpxdimen]{{Inline}.jpg}
\caption{Inline: a mechanical rapid mixer that has a short residence time in a completely mixed flow reactor that is often built into a pipe.}\label{\detokenize{Rapid_Mix/RM_Intro:id6}}\label{\detokenize{Rapid_Mix/RM_Intro:figure-inline}}\end{figure}

\begin{figure}[htbp]
\centering
\capstart

\noindent\sphinxincludegraphics[width=200\sphinxpxdimen]{{hydraulic_jump}.jpg}
\caption{Hydraulic jump: a hydraulic rapid mixer uses the flow expansion downstream from supercritical open channel flow.}\label{\detokenize{Rapid_Mix/RM_Intro:id7}}\label{\detokenize{Rapid_Mix/RM_Intro:figure-hydraulic-jump}}\end{figure}

The hydraulic jump in \hyperref[\detokenize{Rapid_Mix/RM_Intro:figure-hydraulic-jump}]{Fig.\@ \ref{\detokenize{Rapid_Mix/RM_Intro:figure-hydraulic-jump}}} uses a flow expansion to generate mixing in an open channel and that suggests that a flow expansion could also be used to generate mixing in a closed conduit. AguaClara rapid mix units consist of an orifice in the bottom of the Linear Flow Orifice Meter ({\hyperref[\detokenize{Flow_Control_and_Measurement/FCM_Design:heading-lfom}]{\sphinxcrossref{\DUrole{std,std-ref}{Linear Flow Orifice Meter (LFOM)}}}}) where the water enters the flocculator (see \hyperref[\detokenize{Rapid_Mix/RM_Intro:figure-rapid-mix-orifice}]{Fig.\@ \ref{\detokenize{Rapid_Mix/RM_Intro:figure-rapid-mix-orifice}}}). However, given that fluid mixing is so easy to attain it is unclear if the energy used in the rapid mix orifice is necessary.

\begin{figure}[htbp]
\centering
\capstart

\noindent\sphinxincludegraphics[width=400\sphinxpxdimen]{{Rapid_mix_orifice}.png}
\caption{The orifice creates a high velocity jet that generates mixing as it expands in the contact chamber prior to flocculation.}\label{\detokenize{Rapid_Mix/RM_Intro:id8}}\label{\detokenize{Rapid_Mix/RM_Intro:figure-rapid-mix-orifice}}\end{figure}


\subsection{Conventional Mechanical Rapid Mix}
\label{\detokenize{Rapid_Mix/RM_Intro:conventional-mechanical-rapid-mix}}\label{\detokenize{Rapid_Mix/RM_Intro:heading-conventional-mechanical-rapid-mix}}

\subsection{Maximum Velocity Gradients}
\label{\detokenize{Rapid_Mix/RM_Intro:maximum-velocity-gradients}}\label{\detokenize{Rapid_Mix/RM_Intro:heading-conventional-maximum-velocity-gradients}}
\fvset{hllines={, ,}}%
\begin{sphinxVerbatim}[commandchars=\\\{\}]
\PYG{n}{Mix\PYGZus{}HRT} \PYG{o}{=} \PYG{n}{np}\PYG{o}{.}\PYG{n}{array}\PYG{p}{(}\PYG{p}{[}\PYG{l+m+mf}{0.5}\PYG{p}{,}\PYG{l+m+mi}{15}\PYG{p}{,}\PYG{l+m+mi}{25}\PYG{p}{,}\PYG{l+m+mi}{35}\PYG{p}{,}\PYG{l+m+mi}{85}\PYG{p}{]}\PYG{p}{)}\PYG{o}{*}\PYG{n}{u}\PYG{o}{.}\PYG{n}{s}
\PYG{n}{Mix\PYGZus{}G} \PYG{o}{=} \PYG{n}{np}\PYG{o}{.}\PYG{n}{array}\PYG{p}{(}\PYG{p}{[}\PYG{l+m+mi}{4000}\PYG{p}{,}\PYG{l+m+mi}{1500}\PYG{p}{,}\PYG{l+m+mi}{950}\PYG{p}{,}\PYG{l+m+mi}{850}\PYG{p}{,}\PYG{l+m+mi}{750}\PYG{p}{]}\PYG{p}{)}\PYG{o}{/}\PYG{n}{u}\PYG{o}{.}\PYG{n}{s}
\PYG{n}{Mix\PYGZus{}CP} \PYG{o}{=} \PYG{n}{np}\PYG{o}{.}\PYG{n}{multiply}\PYG{p}{(}\PYG{n}{Mix\PYGZus{}HRT}\PYG{p}{,} \PYG{n}{np}\PYG{o}{.}\PYG{n}{sqrt}\PYG{p}{(}\PYG{n}{Mix\PYGZus{}G}\PYG{p}{)}\PYG{p}{)}
\PYG{n}{Mix\PYGZus{}Gt} \PYG{o}{=} \PYG{n}{np}\PYG{o}{.}\PYG{n}{multiply}\PYG{p}{(}\PYG{n}{Mix\PYGZus{}HRT}\PYG{p}{,} \PYG{n}{Mix\PYGZus{}G}\PYG{p}{)}
\PYG{n}{Mix\PYGZus{}EDR} \PYG{o}{=} \PYG{p}{(}\PYG{n}{Mix\PYGZus{}G}\PYG{o}{*}\PYG{o}{*}\PYG{l+m+mi}{2}\PYG{o}{*}\PYG{n}{pc}\PYG{o}{.}\PYG{n}{viscosity\PYGZus{}kinematic}\PYG{p}{(}\PYG{n}{Temperature}\PYG{p}{)}\PYG{p}{)}

\PYG{n}{fig}\PYG{p}{,} \PYG{n}{ax} \PYG{o}{=} \PYG{n}{plt}\PYG{o}{.}\PYG{n}{subplots}\PYG{p}{(}\PYG{p}{)}
\PYG{n}{ax}\PYG{o}{.}\PYG{n}{plot}\PYG{p}{(}\PYG{n}{Mix\PYGZus{}G}\PYG{o}{.}\PYG{n}{to}\PYG{p}{(}\PYG{l+m+mi}{1}\PYG{o}{/}\PYG{n}{u}\PYG{o}{.}\PYG{n}{s}\PYG{p}{)}\PYG{p}{,}\PYG{n}{Mix\PYGZus{}HRT}\PYG{o}{.}\PYG{n}{to}\PYG{p}{(}\PYG{n}{u}\PYG{o}{.}\PYG{n}{s}\PYG{p}{)}\PYG{p}{,}\PYG{l+s+s1}{\PYGZsq{}}\PYG{l+s+s1}{o}\PYG{l+s+s1}{\PYGZsq{}}\PYG{p}{)}
\PYG{n}{ax}\PYG{o}{.}\PYG{n}{yaxis}\PYG{o}{.}\PYG{n}{set\PYGZus{}major\PYGZus{}formatter}\PYG{p}{(}\PYG{n}{FormatStrFormatter}\PYG{p}{(}\PYG{l+s+s1}{\PYGZsq{}}\PYG{l+s+s1}{\PYGZpc{}}\PYG{l+s+s1}{.f}\PYG{l+s+s1}{\PYGZsq{}}\PYG{p}{)}\PYG{p}{)}
\PYG{n}{ax}\PYG{o}{.}\PYG{n}{xaxis}\PYG{o}{.}\PYG{n}{set\PYGZus{}major\PYGZus{}formatter}\PYG{p}{(}\PYG{n}{FormatStrFormatter}\PYG{p}{(}\PYG{l+s+s1}{\PYGZsq{}}\PYG{l+s+s1}{\PYGZpc{}}\PYG{l+s+s1}{.f}\PYG{l+s+s1}{\PYGZsq{}}\PYG{p}{)}\PYG{p}{)}
\PYG{n}{ax}\PYG{o}{.}\PYG{n}{set}\PYG{p}{(}\PYG{n}{xlabel}\PYG{o}{=}\PYG{l+s+s1}{\PYGZsq{}}\PYG{l+s+s1}{Velocity gradient (Hz)}\PYG{l+s+s1}{\PYGZsq{}}\PYG{p}{,} \PYG{n}{ylabel}\PYG{o}{=}\PYG{l+s+s1}{\PYGZsq{}}\PYG{l+s+s1}{Residence time (s)}\PYG{l+s+s1}{\PYGZsq{}}\PYG{p}{)}
\PYG{n}{fig}\PYG{o}{.}\PYG{n}{savefig}\PYG{p}{(}\PYG{n}{imagepath}\PYG{o}{+}\PYG{l+s+s1}{\PYGZsq{}}\PYG{l+s+s1}{Mechanical\PYGZus{}RM\PYGZus{}Gt}\PYG{l+s+s1}{\PYGZsq{}}\PYG{p}{)}
\PYG{n}{plt}\PYG{o}{.}\PYG{n}{show}\PYG{p}{(}\PYG{p}{)}
\end{sphinxVerbatim}

\begin{figure}[htbp]
\centering
\capstart

\noindent\sphinxincludegraphics[width=400\sphinxpxdimen]{{Mechanical_RM_Gt}.png}
\caption{Mechanical rapid mix units use a wide range of velocity gradients and residence times.}\label{\detokenize{Rapid_Mix/RM_Intro:id9}}\label{\detokenize{Rapid_Mix/RM_Intro:figure-mechanical-rm-gt}}\end{figure}

Conventional rapid mix units use mechanical or potential energy to generate intense turbulence to begin the mixing process. Conventional design is based on the use of \(\bar G\) (an average velocity gradient) as a design parameter. We don’t yet know what the design objective is for rapid mix and thus it isn’t clear which parameters matter. We hypothesize that both velocity gradients that cause deformation of the fluid and time for molecular diffusion are required to ultimately transport coagulant nanoparticles to the surfaces of clay particles.

The velocity gradient can be obtained from the rate at which mechanical energy is being dissipated and converted to heat by viscosity.
\begin{equation}\label{equation:Rapid_Mix/RM_Intro:Rapid_Mix/RM_Intro:4}
\begin{split}\varepsilon = G^2 \nu\end{split}
\end{equation}
where \(\varepsilon\) is the energy dissipation rate, \(G\) is the velocity gradient, and \(\nu\) is the kinematic viscosity of water. We can estimate the power input required to create a target energy dissipation rate for a conventional design by noting that power is simple the energy dissipation rate times the mass of water in the rapid mix unit.
\begin{equation}\label{equation:Rapid_Mix/RM_Intro:Rapid_Mix/RM_Intro:5}
\begin{split}P = \bar\varepsilon \rlap{\kern.08em--}V \rho\end{split}
\end{equation}\begin{equation}\label{equation:Rapid_Mix/RM_Intro:Rapid_Mix/RM_Intro:6}
\begin{split}P = \bar G^2 \nu \rlap{\kern.08em--}V \rho\end{split}
\end{equation}
We can relate reactor volume to a hydraulic residence time, \(\theta\), and volumetric flow rate, Q.
\begin{equation}\label{equation:Rapid_Mix/RM_Intro:Rapid_Mix/RM_Intro:7}
\begin{split}P = \rho \bar G^2 \nu Q \theta\end{split}
\end{equation}
This equation is perfectly useful for estimating electrical motor sizing requirements for mechanical rapid mix units. For gravity powered hydraulic rapid mix units it would be more intuitive to use the change in water surface elevation, \(\Delta h\) instead of power input.
\begin{equation}\label{equation:Rapid_Mix/RM_Intro:Rapid_Mix/RM_Intro:8}
\begin{split}P = \rho g Q \Delta h\end{split}
\end{equation}
Combining the two equations we obtain.
\begin{equation}\label{equation:Rapid_Mix/RM_Intro:Rapid_Mix/RM_Intro:9}
\begin{split}\Delta h =   \frac{G^2 \nu \theta}{g}\end{split}
\end{equation}

\begin{savenotes}\sphinxattablestart
\centering
\sphinxcapstartof{table}
\sphinxcaption{Typical values for conventional rapid mix residence time and average velocity gradients}\label{\detokenize{Rapid_Mix/RM_Intro:id10}}\label{\detokenize{Rapid_Mix/RM_Intro:table-conventional-rapid-mix-design-values}}
\sphinxaftercaption
\begin{tabulary}{\linewidth}[t]{|T|T|T|T|}
\hline
\sphinxstyletheadfamily 
Residence Time (s)
&\sphinxstyletheadfamily 
Velocity gradient G (1/s)
&\sphinxstyletheadfamily 
Energy dissipation rate (W/kg)
&\sphinxstyletheadfamily 
Equivalent height (m)
\\
\hline
0.5
&
4000
&
16
&
0.8
\\
\hline
10 - 20
&
1500
&
2.25
&
2.3 - 4.6
\\
\hline
20 - 30
&
950
&
0.9
&
1.8 - 2.8
\\
\hline
30 - 40
&
850
&
0.72
&
2.2 - 2.9
\\
\hline
40 - 130
&
750
&
0.56
&
2.3 - 7.5
\\
\hline
\end{tabulary}
\par
\sphinxattableend\end{savenotes}

From Environmental Engineering: A Design Approach by Sincero and
Sincero. 1996. page 267.

Rotating propellers can either be installed in open tanks or enclosed in pipes. From a mixing and fluids perspective it doesn’t make any difference whether the tank is open to the atmosphere or not. The parameters of interest are the rate of fluid deformation and the residence time in the mixing zone.


\subsection{Mixing time}
\label{\detokenize{Rapid_Mix/RM_Intro:mixing-time}}\label{\detokenize{Rapid_Mix/RM_Intro:heading-mixing-time}}
The time required for mixing in a turbulent environment is a function of the rate that kinetic energy is being dissipated as heat (the energy dissipation rate) and the length scale of the eddies. Given that turbulent energy is passed from large eddies to smaller and smaller eddies, the amount of energy that is being transferred at any given length scale is independent of scale. The result (see equation \eqref{equation:Rapid_Mix/RM_Derivations:eq_t_eddy}) is that the time required for mixing is dominated by the time required for the largest eddies to turn over (\hyperref[\detokenize{Rapid_Mix/RM_Intro:figure-eddy-turnover-times}]{Fig.\@ \ref{\detokenize{Rapid_Mix/RM_Intro:figure-eddy-turnover-times}}}).

\begin{figure}[htbp]
\centering
\capstart

\noindent\sphinxincludegraphics[width=400\sphinxpxdimen]{{Eddy_turnover_time}.png}
\caption{Eddy turnover times as a function of length scale for a range of energy dissipation rates.}\label{\detokenize{Rapid_Mix/RM_Intro:id11}}\label{\detokenize{Rapid_Mix/RM_Intro:figure-eddy-turnover-times}}\end{figure}

The eddy turnover times are longest for the largest eddies and this analysis suggests that it only takes a few seconds for turbulent eddies to mix from the scale of the flow down to the inner viscous length scale.

The large scale mixing time is critical for the design of water treatment plants for the case where the flow is split into multiple treatment trains after coagulant addition. In this case it is critical that the coagulant be mixed equally between all of the treatment trains and thus the mixing times shown in the previous graph represent a minimum time between where the coagulant is added and where the flow is divided into the parallel treatment trains.

It is likely this process of mixing from the scale of the flow down to the inner viscous length scale is commonly referred to as “rapid mix.” Here we showed that this mixing is indeed rapid and is really only a concern in the case where the coagulant injection point is very close to the location where the flow is split into multiple treatment trains.

Fluid deformation dominated by viscous shear and molecular diffusion finish the process of blending the coagulant nanoparticles with the water. We show in \DUrole{xref,std,std-ref}{Fluid\_Deformation\_by\_Shear} that the time required by fluid deformation and molecular diffusion to finish the blending process is approximately equal to 1/G where G is the velocity gradient. Given that velocity gradients in rapid mix units are typically greater than a thousand Hz the time required to finish the blending is approximately 1 ms.

Thus the time required for mixing the coagulant nanoparticles with the fluid typically only requires a few seconds and will be accomplished whether or not the rapid mix unit is turned on. The turbulent eddies from the water flowing a the channel or pipe between the coagulant injection point and the flocculator in most cases will be sufficient to achieve the fluid mixing. However, the step of the {\hyperref[\detokenize{Rapid_Mix/RM_Theory_and_Future_Work:heading-diffusion-and-shear-transport-coagulant-nanoparticles-to-clay}]{\sphinxcrossref{\DUrole{std,std-ref}{coagulant nanoparticles attaching to the suspended particles}}}} may be aided by the high energy of the rapid mix unit.


\section{Coagulant Nanoparticle Interactions}
\label{\detokenize{Rapid_Mix/RM_Intro:coagulant-nanoparticle-interactions}}\label{\detokenize{Rapid_Mix/RM_Intro:heading-coagulant-nanoparticle-interactions}}
Coagulant nanoparticles are sticky and can attach to suspended particles as well as to each other. Some dissolved substances also adsorb to coagulant nanoparticles. The development of models to describe these interactions has been impeded by the charge neutralization hypothesis that failed to account for the size of the coagulant nanoparticles and by the complexity of modeling all of these competing processes. Although the model describing removal of dissolved organic matter is still nascent, it is possible that a simplified approach that separates fast and slow processes will enable a sequential model.

Interactions between the various suspended and dissolved substances (see \hyperref[\detokenize{Rapid_Mix/RM_Intro:figure-particle-sizes}]{Fig.\@ \ref{\detokenize{Rapid_Mix/RM_Intro:figure-particle-sizes}}}) can occur simultaneously as soon as the coagulant is blended with the raw water. The rates of these interactions are controlled by the transport processes of fluid deformation and molecular diffusion. Molecular diffusion is fastest for small particles and fluid deformation is most effective for larger particles. Thus the fastest process is hypothesized to be the diffusion of low mass molecules to the coagulant nanoparticles. Transport of the coagulant nanoparticles to attach to suspended solids is expected to be a slower process. Transport of suspended particles to collide with other suspended particles (flocculation) is even slower.

\begin{figure}[htbp]
\centering
\capstart

\noindent\sphinxincludegraphics[width=400\sphinxpxdimen]{{Particle_sizes}.png}
\caption{The size range of particles and nanoparticles that are important in drinking water treatment ranges from approximately a nanometer (for example arsenic \(HAsO_4^{2-}\)) to thousands of nanometers for clay and protozoa.}\label{\detokenize{Rapid_Mix/RM_Intro:id12}}\label{\detokenize{Rapid_Mix/RM_Intro:figure-particle-sizes}}\end{figure}


\subsection{Dissolved Organic Matter}
\label{\detokenize{Rapid_Mix/RM_Intro:dissolved-organic-matter}}\label{\detokenize{Rapid_Mix/RM_Intro:heading-dissolved-organic-matter-and-coagulant}}
Dissolved organic matter (DOM) includes humic substances, fulvic acids, and other organic molecules. The distinction between dissolved and particulate organic matter is somewhat arbitrary and often 450 nm is used as the transition.  The dissolved organic matter could also be referred to as macromolecules or as nanoparticles.

Because of its small size the DOM has a large surface per unit mass. Water that contains high DOM concentrations requires much higher coagulant dosages to achieve effective flocculation. Removal of DOM is a high priority for drinking water treatment plants because DOM both interferes with disinfection processes and produces disinfection by products. A significant fraction of DOM can be removed by coagulant nanoparticles.


\subsection{Suspended Solids}
\label{\detokenize{Rapid_Mix/RM_Intro:suspended-solids}}\label{\detokenize{Rapid_Mix/RM_Intro:heading-suspended-solids-and-coagulant}}
Suspended solids include both organic and inorganic particles. Organic particles of concern include virus, bacteria, and protozoa. Inorganic particles include clay and other minerals. Naturally occurring suspended solids tend to have negative surface charge at neutral pH. The negative surface charge effectively prevents particle aggregation and thus these particles can remain suspended for a very long time.


\subsection{Pathogens}
\label{\detokenize{Rapid_Mix/RM_Intro:pathogens}}\label{\detokenize{Rapid_Mix/RM_Intro:heading-pathogens-and-coagulant}}
Virus particles readily attach to coagulant nanoparticles (see \sphinxhref{https://link-springer-com.proxy.library.cornell.edu/chapter/10.1007/978-3-642-76093-8\_5}{“Effects of Floc-Virus Association on Chlorine Disinfection Efficiency by Shinichiro Ohgaki and Prasang Mongkonsiri}) and this attachment makes it possible to efficiently remove virus particles by flocculation followed by sedimentation. Bacteria (cite Yolanda Brook paper when it is published) and protozoans (need reference) are also removed by flocculation by coagulant nanoparticles.


\subsection{Rate Estimates for Coagulant Nanoparticle Transport to Suspended Solids}
\label{\detokenize{Rapid_Mix/RM_Intro:rate-estimates-for-coagulant-nanoparticle-transport-to-suspended-solids}}\label{\detokenize{Rapid_Mix/RM_Intro:heading-rate-estimates-for-coagulant-nanoparticle-transport-to-suspended-solids}}
Coagulant nanoparticles require significant time to attach to the surfaces of suspended solids. The time required is estimated in {\hyperref[\detokenize{Rapid_Mix/RM_Theory_and_Future_Work:heading-diffusion-and-shear-transport-coagulant-nanoparticles-to-clay}]{\sphinxcrossref{\DUrole{std,std-ref}{Diffusion and Shear Transport Coagulant Nanoparticles to Clay}}}}. It is quite possible that this stage of the rapid mix/flocculation process has been overlooked in the past. Transport of the nanoparticles to the suspended solids is accomplished by a combination of fluid deformation and diffusion.


\section{Energy Dissipation Rate, Velocity Gradient, and Mixing}
\label{\detokenize{Rapid_Mix/RM_Intro:energy-dissipation-rate-velocity-gradient-and-mixing}}\label{\detokenize{Rapid_Mix/RM_Intro:heading-edr-g-and-mixing}}\begin{description}
\item[{In addition to the general fluids review ({\hyperref[\detokenize{Review/Review_Fluid_Mechanics:title-review-fluid-mechanics}]{\sphinxcrossref{\DUrole{std,std-ref}{Review: Fluid Mechanics}}}}), there are a few extra fluid dynamics concepts that are important to know in order to understand drinking water treatment and AguaClara’s approach to it. These concepts are primarily focused on the relationships between:}] \leavevmode\begin{itemize}
\item {} 
Turbulence

\item {} 
Viscosity

\item {} 
Shear

\item {} 
Velocity Gradients (\(G\)),which serve as a measure of fluid deformation

\item {} 
Energy Dissipation Rate (EDR, \(\varepsilon\))

\end{itemize}

\end{description}

Knowledge of these concepts and how they interact is critical to understand rapid mix, flocculation, filtration, and disinfection. These concepts and their interactions first become relevant in rapid mix, the step in which the coagulant gets added to the raw water.

The two concepts that were not covered in the previous chapter, {\hyperref[\detokenize{Review/Review_Fluid_Mechanics:title-review-fluid-mechanics}]{\sphinxcrossref{\DUrole{std,std-ref}{Review: Fluid Mechanics}}}}, are velocity gradient \(G\) and energy dissipation rate \(\varepsilon\). While these will be very thoroughly described over the course of this introduction, a brief and simple explanation is included to help get the ball rolling.


\subsection{Understanding \protect\(G\protect\) and \protect\(\varepsilon\protect\)}
\label{\detokenize{Rapid_Mix/RM_Intro:understanding-and}}
\(G\), or velocity gradient, is a measure of fluid deformation. It is defined by how quickly one point of water along one streamline moves in comparison to another point on another streamline (\(v_A\) compared to \(v_B\), for example), taking into account the distance between the streamlines, \(\Delta h\). A visual example of a velocity gradient is shown in the image below:

\begin{figure}[htbp]
\centering
\capstart

\noindent\sphinxincludegraphics[width=700\sphinxpxdimen]{{Velocity_gradient_image}.jpg}
\caption{Velocity gradients cause relative velocities of fluid elements. Those relative velocities form the basis of particle collisions that are essential for the flocculation process.}\label{\detokenize{Rapid_Mix/RM_Intro:id13}}\label{\detokenize{Rapid_Mix/RM_Intro:figure-velocity-gradient-image}}\end{figure}

\sphinxstylestrong{Note on terminology:} “Fluid deformation” is equivalent to “velocity gradient,” and the two terms can be used interchangeably. They are different ways of thinking about the same concept. Thus, \(G\) is the measure of both terms.

\(\varepsilon\), or energy dissipation rate, is the rate that the kinetic energy of the fluid is being converted to heat. EDR is a very useful concept because the last step of converting kinetic energy into heat is accomplished by viscosity (\(\nu\)). This kinetic energy being dissipated by viscosity is the energy associated with velocity gradients (\(G\)). Thus, through EDR there is a direct connection between \(\nu\) and \(G\). This connection will be further covered later on in this introduction.

As mentioned above, EDR and velocity gradients play an important role in mixing and in causing suspended particles to collide with each other, both of which are important topics in flocculation. Their use is not limited to flocculation, they are also helpful in understanding failure modes of plate settlers and terminal head loss of sand filters

We will begin by defining the concept of energy dissipation rate for a control volume. In a control volume that does not include pumps, turbines or other external energy sources or sinks, the mechanical energy lost is indicated by a change in elevation and quantified as \(g h_L\). That mechanical energy is lost in the time that the fluid is in the control volume, \(\theta\).
\begin{equation}\label{equation:Rapid_Mix/RM_Intro:Rapid_Mix/RM_Intro:10}
\begin{split}\bar\varepsilon \theta = g h_L\end{split}
\end{equation}
This equation simply states that the average rate of energy dissipation times the time over which that dissipation occurs is equal to the total lost mechanical energy. The dimensions of \(\varepsilon\) are:
\begin{equation}\label{equation:Rapid_Mix/RM_Intro:Rapid_Mix/RM_Intro:11}
\begin{split}\varepsilon = \frac{[m^3]}{[s^3]} = {\rm \frac{W}{kg}}\end{split}
\end{equation}
These dimensions can be understood as a velocity squared per time, otherwise known as a rate of kinetic energy loss (recall that kinetic energy is \({\rm Ke} = \frac{\bar v^2}{2g}\), or \({\rm Ke} \propto \bar v^2\)), or as power per unit mass, which would be \({\rm  \frac{W}{kg}}\).

Velocity gradients are central to flocculation because they cause the deformation of the fluid, and this results in particle collisions. Consider a real-world example via the image below: if everyone on a sidewalk is walking in the same direction at exactly the same velocity, then there will never be any collisions between people (top). If, however, people at one side of the sidewalk stand still and people walk progressively faster as a function of how far they are away from the zero velocity side of the sidewalk, then there will be many collisions between the pedestrians (see \hyperref[\detokenize{Rapid_Mix/RM_Intro:figure-pedestrians-on-sidewalk}]{Fig.\@ \ref{\detokenize{Rapid_Mix/RM_Intro:figure-pedestrians-on-sidewalk}}}). Indeed, the rate of collisions is proportional to the velocity gradient.

\begin{figure}[htbp]
\centering
\capstart

\noindent\sphinxincludegraphics[width=700\sphinxpxdimen]{{Pedestrians_on_sidewalk}.jpg}
\caption{Pedestrians walking on a sidewalk serve as a model for velocity gradients.}\label{\detokenize{Rapid_Mix/RM_Intro:id14}}\label{\detokenize{Rapid_Mix/RM_Intro:figure-pedestrians-on-sidewalk}}\end{figure}


\section{Common Flow Geometries that Dissipate Energy}
\label{\detokenize{Rapid_Mix/RM_Intro:common-flow-geometries-that-dissipate-energy}}
Water treatment plants at research and municipal scales deploy a wide range of flow geometries. The following list includes the flow geometries that are commonly used for mixing processes.
\begin{itemize}
\item {} 
Straight pipe (wall shear) - {[}uncommon, but included for completeness{]}

\item {} 
Coiled tube (wall shear and expansions) - {[}research scale mixing{]}

\item {} 
Series of expansions (expansions) - {[}hydraulic flocculators{]}

\item {} 
Mechanical mixing - {[}mechanical rapid mix and flocculators{]}

\item {} 
Between flat plates (wall shear) - {[}plate settlers{]}

\item {} 
Round jet - (expansion) - {[}hydraulic rapid mix{]}

\item {} 
Plane jet - (expansion) - {[}inlet into sedimentation tank{]}

\item {} 
Behind a flat plate - (expansion) - {[}mechanical mixers{]}

\end{itemize}

The following tables can serve as a convenient reference to the equations describing head loss, energy dissipation rates, and velocity gradients in various flow geometries that are commonly encountered in water treatment plants. The {\hyperref[\detokenize{Rapid_Mix/RM_Derivations:heading-equations-varying-flow-geometries}]{\sphinxcrossref{\DUrole{std,std-ref}{Equations for  and  in Varying Flow Geometries}}}} are available as a reference.


\begin{savenotes}\sphinxattablestart
\raggedright
\sphinxcapstartof{table}
\sphinxcaption{Table of equations for control volume averaged values of head loss, energy dissipation rate, and the Camp-Stein velocity gradient.}\label{\detokenize{Rapid_Mix/RM_Intro:id15}}\label{\detokenize{Rapid_Mix/RM_Intro:table-control-volume-equations}}
\sphinxaftercaption
\begin{tabular}[t]{|*{5}{\X{1}{5}|}}
\hline
\sphinxstyletheadfamily 
Geometry
&
\(h_L\)
&\sphinxstyletheadfamily 
Energy dissipation rate
&
\(G_{CS}(bar v)\)
&
\(G_{CS}(Q)\)
\\
\hline
Straight pipe
&
\(h_{{\rm f}} = {{\rm f}} \frac{L}{D} \frac{\bar v^2}{2g}\)
&
\(\bar\varepsilon = \frac{{\rm f}}{2} \frac{\bar v^3}{D}\)
&
\(G_{CS} = \left(\frac{{\rm f}}{2\nu} \frac{\bar v^3}{D} \right)^\frac{1}{2}\)
&
\(G_{CS} = \left(\frac{\rm{32f}}{ \pi^3\nu} \frac{Q^3}{D^7} \right)^\frac{1}{2}\)
\\
\hline
Straight pipe laminar
&
\(h_{{\rm f}} = \frac{32\nu L\bar v}{ g D^2}\)
&
\(\bar\varepsilon =32\nu \left( \frac{\bar v}{D} \right)^2\)
&
\(G_{CS} =4\sqrt2 \frac{\bar v}{D}\)
&
\(G_{CS} =\frac{16\sqrt2}{\pi} \frac{Q}{D^3}\)
\\
\hline
Parallel plates laminar
&
\(h_{{\rm f}} = 12\frac{ \nu L \bar v }{gS^2}\)
&
\(\bar\varepsilon = 12 \nu \left(\frac{ \bar v}{S} \right)^2\)
&
\(G_{CS} = 2\sqrt{3}\frac{ \bar v}{S}\)
&\begin{itemize}
\item {} 
\end{itemize}
\\
\hline
Coiled tube laminar
&
\(h_{L_{coil}} = \frac{32\nu L\bar v}{ g D^2} \left[ 1 + 0.033\left(log_{10}De\right)^4 \right]\)
&
\(\bar\varepsilon = 32\nu \left( \frac{\bar v}{D} \right)^2 \left[ 1 + 0.033\left(log_{10}De\right)^4 \right]\)
&
\(G_{CS_{coil}} = 4\sqrt2 \frac{\bar v}{D}\left[ 1 + 0.033\left(log_{10}De\right)^4 \right]^\frac{1}{2}\)
&\begin{itemize}
\item {} 
\end{itemize}
\\
\hline
Expansions
&
\(h_e = K\frac{\bar v_{out}^2}{2g}\)
&
\(\bar\varepsilon = K\frac{\bar v_{out}^3}{2H}\)
&
\(G_{CS} = \bar v_{out}\sqrt{\frac{K\bar v_{out}}{2H\nu}}\)
&\begin{itemize}
\item {} 
\end{itemize}
\\
\hline
\end{tabular}
\par
\sphinxattableend\end{savenotes}

The equations used to convert between columns in the table above are:
\begin{equation}\label{equation:Rapid_Mix/RM_Intro:Rapid_Mix/RM_Intro:12}
\begin{split}\bar\varepsilon = \frac{gh_{\rm{L}}}{\theta} \qquad\qquad
G_{CS} = \sqrt{\frac{\bar \varepsilon}{\nu}} \qquad\qquad
\bar v=\frac{4Q}{\pi D}\end{split}
\end{equation}
Note that the velocity gradient is independent of viscosity (and hence temperature) for laminar flow. This is because the total amount of fluid deformation is simply based on geometry. The no slip condition, the diameter, and the length of the flow passage set the total fluid deformation. Of course, if temperature decreases and viscosity increases the amount of energy required to push the fluid through the flow passage will increase (head loss is proportional to viscosity for laminar flow).

For turbulent flow and for flow expansions the amount of fluid deformation decreases as the viscosity increases and the total energy required to send the flow through the reactor is almost independent of the viscosity. The “almost” is because for wall shear under turbulent conditions there is a small effect of viscosity that is buried inside the friction factor.


\begin{savenotes}\sphinxattablestart
\raggedright
\sphinxcapstartof{table}
\sphinxcaption{Equations for maximum (wall) energy dissipation rates and wall velocity gradients.}\label{\detokenize{Rapid_Mix/RM_Intro:id16}}\label{\detokenize{Rapid_Mix/RM_Intro:table-edr-g-max-equations}}
\sphinxaftercaption
\begin{tabular}[t]{|*{3}{\X{1}{3}|}}
\hline
\sphinxstyletheadfamily 
Geometry
&\sphinxstyletheadfamily 
Energy dissipation rate at the wall
&\sphinxstyletheadfamily 
Velocity gradient at the wall
\\
\hline
Straight pipe
&
\(\varepsilon_{wall} = \frac{1}{\nu}\left({\rm f}  \frac{\bar v^2}{8} \right)^2\)
&
\(G_{wall} ={\rm f}  \frac{\bar v^2}{8\nu}\)
\\
\hline
Straight pipe laminar
&
\(\varepsilon_{wall} = \left(\frac{8\bar v}{D} \right)^2 \nu\)
&
\(G_{wall} =  \frac{8\bar v}{D}\)
\\
\hline
parallel plates laminar
&
\(\varepsilon_{wall} = 36\left( \frac{\bar v}{S}\right)^2 \nu\)
&
\(G_{wall} = \frac{6 \bar v}{S}\)
\\
\hline
Coiled pipe laminar
&\begin{itemize}
\item {} 
\end{itemize}
&
\(G_{CS_{wall_{coil}}} ={\rm f} \left[ 1 + 0.033\left(log_{10}De\right)^4 \right]\frac{\bar v^2}{8\nu}\)
\\
\hline
\end{tabular}
\par
\sphinxattableend\end{savenotes}


\begin{savenotes}\sphinxattablestart
\raggedright
\sphinxcapstartof{table}
\sphinxcaption{Equations for maximum energy dissipation rates and velocity gradients for flow expansions.}\label{\detokenize{Rapid_Mix/RM_Intro:id17}}\label{\detokenize{Rapid_Mix/RM_Intro:table-edr-g-equations}}
\sphinxaftercaption
\begin{tabulary}{\linewidth}[t]{|T|T|T|T|}
\hline
\sphinxstyletheadfamily 
Geometry
&
\(Pi_{Jet}\)
&\sphinxstyletheadfamily 
Maximum energy dissipation rate
&\sphinxstyletheadfamily 
Maximum velocity gradient
\\
\hline
Round jet
&
0.08
&
\(\varepsilon_{Max} = \Pi_{JetRound}\frac{   \bar v_{Jet} ^3}{D_{Jet}}\)
&
\(G_{Max} = \bar v_{Jet} \sqrt{\frac{\Pi_{JetRound} \bar v_{Jet} }{\nu D_{Jet}}}\)
\\
\hline
Plane jet
&
0.0124
&
\(\varepsilon_{Max} = \Pi_{JetPlane} \frac{  \bar v_{Jet} ^3}{S_{Jet}}\)
&
\(G_{Max} = \bar v_{Jet}\sqrt{\frac{\Pi_{JetPlane} \bar v_{Jet}}{\nu S_{Jet}}}\)
\\
\hline
Behind a flat plate
&
0.04
&
\(\varepsilon _{Max} = \Pi_{Plate}\frac{\bar v^3}{W_{Plate}}\)
&
\(G_{Max} = \bar v\sqrt{\frac{\Pi_{Plate} \bar v}{\nu W_{Plate}}}\)
\\
\hline
\end{tabulary}
\par
\sphinxattableend\end{savenotes}

For mechanical mixing where an impeller or other stirring device is adding shaft work to a control volume we have
\begin{equation}\label{equation:Rapid_Mix/RM_Intro:Rapid_Mix/RM_Intro:13}
\begin{split}\bar\varepsilon = \frac{P}{m} = \frac{P}{\rho \rlap{-}V}\end{split}
\end{equation}
\begin{DUlineblock}{0em}
\item[] where
\item[] \(P\) = power input into the control volume
\item[] \(m\) = mass of fluid in the control volume
\item[] \(\rlap{-}V\) = volume of the control volume
\item[] \(\rho\) = density of the fluid
\end{DUlineblock}

The Camp-Stein velocity gradient for a mechanically mixed reactor is
\begin{equation}\label{equation:Rapid_Mix/RM_Intro:Rapid_Mix/RM_Intro:14}
\begin{split}G_{CS} = \sqrt{\frac{P}{\rho \nu \rlap{-}V}}\end{split}
\end{equation}

\chapter{Rapid Mix Design}
\label{\detokenize{Rapid_Mix/RM_Design:rapid-mix-design}}\label{\detokenize{Rapid_Mix/RM_Design:title-rapid-mix-design}}\label{\detokenize{Rapid_Mix/RM_Design::doc}}
As of 2018 the design for AguaClara rapid mix units has been based on the goal of achieving a target energy dissipation rate. This in turn was based on the assumption that it was important to rapidly mix the coagulant with the water, perhaps to minimize the self-aggregation of coagulant nanoparticles. We don’t yet have any experimental evidence that rapid mixing is important and it is quite likely that the energy dissipation rate found in the hydraulic flocculator is sufficient to provide the required mixing.

The design requirements for fluid mixing of the coagulant is an area that needs research. If the goal is to achieve a velocity gradient for a number of seconds, then this design will be the same as that developed in the flocculator section. Until we have clear guidance on the goal of rapid mix we will not provide a detailed design here.


\chapter{Rapid Mix Derivations}
\label{\detokenize{Rapid_Mix/RM_Derivations:rapid-mix-derivations}}\label{\detokenize{Rapid_Mix/RM_Derivations:title-rapid-mix-derivations}}\label{\detokenize{Rapid_Mix/RM_Derivations::doc}}

\section{Carbonate reactions, buffering, and pH}
\label{\detokenize{Rapid_Mix/RM_Derivations:carbonate-reactions-buffering-and-ph}}\label{\detokenize{Rapid_Mix/RM_Derivations:heading-carbonate-reactions-buffering-and-ph}}
Carbonates provide the majority of the buffering for drinking water as long as the pH is close to neutral. These equations provide a basis to calculate how much base or acid must be added to a natural water to achieve a target pH.


\subsection{Carbonic Acid and Bicarbonate}
\label{\detokenize{Rapid_Mix/RM_Derivations:carbonic-acid-and-bicarbonate}}\label{\detokenize{Rapid_Mix/RM_Derivations:heading-carbonic-acid-and-bicarbonate}}\begin{equation}\label{equation:Rapid_Mix/RM_Derivations:carbonate}
\begin{split}{H_2}CO_3^{\star} \overset {K_1} \longleftrightarrow {H^+} + HCO_3^-\end{split}
\end{equation}\begin{description}
\item[{Where:}] \leavevmode
\begin{DUlineblock}{0em}
\item[] \(K_1\) is the dissociation constant defined below.
\end{DUlineblock}

\end{description}
\begin{equation}\label{equation:Rapid_Mix/RM_Derivations:Rapid_Mix/RM_Derivations:0}
\begin{split}{K_1} = \frac{{\left[ {{H^ + }} \right]\left[ {HCO_3^ - } \right]}}{{\left[ {{H_2}CO_3^{\star} } \right]}}\end{split}
\end{equation}
Where the {[} {]} indicates concentration in mole/L. We will use the p function, \(p(x)=-log_{10}(x)\), to define the dissociation constant.
\begin{equation}\label{equation:Rapid_Mix/RM_Derivations:Rapid_Mix/RM_Derivations:1}
\begin{split}p{K_1} = 6.3\end{split}
\end{equation}
At the point of equal concentrations of bicarbonate and carbonic acid the dissociation constant, \(K_1\), is equal to the hydrogen ion concentration, \(H^ +\). Thus we have equal concentrations at \(p{K_1} = pH\). This reaction is “centered” at pH = 6.3 and thus there is maximum buffering due to this reaction at pH = 6.3.


\subsection{Bicarbonate and Carbonate}
\label{\detokenize{Rapid_Mix/RM_Derivations:bicarbonate-and-carbonate}}\label{\detokenize{Rapid_Mix/RM_Derivations:heading-bicarbonate-and-carbonate}}\begin{equation}\label{equation:Rapid_Mix/RM_Derivations:Rapid_Mix/RM_Derivations:2}
\begin{split}HCO_3^ - \overset {{K_2}} \longleftrightarrow {H^ + } + CO_3^{ - 2}\end{split}
\end{equation}\begin{equation}\label{equation:Rapid_Mix/RM_Derivations:Rapid_Mix/RM_Derivations:3}
\begin{split}{K_2} = \frac{{\left[ {{H^ + }} \right]\left[ {CO_3^{ - 2}} \right]}}{{\left[ {HCO_3^ - } \right]}}\end{split}
\end{equation}\begin{equation}\label{equation:Rapid_Mix/RM_Derivations:Rapid_Mix/RM_Derivations:4}
\begin{split}p{K_2} = 10.3\end{split}
\end{equation}
Thus the carbonate system also provides buffering around pH 10.3.


\subsection{Total Concentration of Carbonates}
\label{\detokenize{Rapid_Mix/RM_Derivations:total-concentration-of-carbonates}}\label{\detokenize{Rapid_Mix/RM_Derivations:heading-total-concentration-of-carbonates}}
The total concentration of carbonate species is given by
\begin{equation}\label{equation:Rapid_Mix/RM_Derivations:Rapid_Mix/RM_Derivations:5}
\begin{split}{C_T} = \left[ {{H_2}CO_3^{\star} } \right] + \left[ {HCO_3^ - } \right] + \left[ {CO_3^{ - 2}} \right]\end{split}
\end{equation}
Where: \({C_T}\) is the total concentration of carbonates.

The total concentration of carbonates, \({C_T}\), is useful because it is conservative (in a closed system) even though the individual species concentrations change as pH changes.


\subsection{Alpha Notation}
\label{\detokenize{Rapid_Mix/RM_Derivations:alpha-notation}}\label{\detokenize{Rapid_Mix/RM_Derivations:heading-alpha-notation}}
The alpha notation is used to show the concentration dependence on pH and to make the equations simpler.
\begin{equation}\label{equation:Rapid_Mix/RM_Derivations:Rapid_Mix/RM_Derivations:6}
\begin{split}\left[ {{H_2}CO_3^{\star} } \right] = {\alpha_0}{C_T}\end{split}
\end{equation}\begin{equation}\label{equation:Rapid_Mix/RM_Derivations:Rapid_Mix/RM_Derivations:7}
\begin{split}\left[ {HCO_3^-} \right] = {\alpha_1}{C_T}\end{split}
\end{equation}\begin{equation}\label{equation:Rapid_Mix/RM_Derivations:Rapid_Mix/RM_Derivations:8}
\begin{split}\left[ {CO_3^{-2}} \right] = {\alpha_2}{C_T}\end{split}
\end{equation}
The alphas sum to 1 because each \(\alpha\) is the fraction of the carbonates corresponding to that species. The alphas are each a function of the proton concentration and the dissociation constants of the carbonate reactions.
\begin{equation}\label{equation:Rapid_Mix/RM_Derivations:Rapid_Mix/RM_Derivations:9}
\begin{split}{\alpha_{\text{0}}} = \frac{1}{{1 + \frac{{{K_1}}}{{[{H^ + }]}} + \frac{{{K_1}{K_2}}}{{{{[{H^ + }]}^2}}}}}\end{split}
\end{equation}\begin{equation}\label{equation:Rapid_Mix/RM_Derivations:Rapid_Mix/RM_Derivations:10}
\begin{split}{\alpha_{\text{0}}} = \frac{1}{{1 + \frac{{{K_1}}}{{[{H^ + }]}}\left( {1 + \frac{{{K_2}}}{{[{H^ + }]}}} \right)}}\end{split}
\end{equation}\begin{equation}\label{equation:Rapid_Mix/RM_Derivations:Rapid_Mix/RM_Derivations:11}
\begin{split}{\alpha_{\text{1}}} = \frac{1}{{\frac{{[{{\rm H}^ + }]}}{{{{\rm K}_1}}} + 1 + \frac{{{{\rm K}_2}}}{{[{{\rm H}^ + }]}}}}\end{split}
\end{equation}\begin{equation}\label{equation:Rapid_Mix/RM_Derivations:Rapid_Mix/RM_Derivations:12}
\begin{split}{\alpha_{\text{2}}} = \frac{1}{{\frac{{{{[{{\rm H}^ + }]}^2}}}{{{{\rm K}_1}{{\rm K}_2}}} + \frac{{[{{\rm H}^ + }]}}{{{{\rm K}_2}}} + 1}}\end{split}
\end{equation}\begin{equation}\label{equation:Rapid_Mix/RM_Derivations:Rapid_Mix/RM_Derivations:13}
\begin{split}{\alpha_{\text{2}}} = \frac{1}{{1 + \frac{{[{{\rm H}^ + }]}}{{{{\rm K}_2}}}\left( {1 + \frac{{[{{\rm H}^ + }]}}{{{{\rm K}_1}}}} \right)}}\end{split}
\end{equation}

\subsection{Acid Neutralizing Capacity (ANC) or Alkalinity}
\label{\detokenize{Rapid_Mix/RM_Derivations:acid-neutralizing-capacity-anc-or-alkalinity}}\label{\detokenize{Rapid_Mix/RM_Derivations:heading-acid-neutralizing-capacity-anc-or-alkalinity}}
Acid neutralizing capacity or alkalinity is the ability of a water sample to react with and neutralize an input of acid. The units of ANC are equivalents (or protons) per liter. Bicarbonate, \(HCO_3^-\), can react with one proton, \(H^+\), and thus each mole of \(HCO_3^-\) provides one equivalent per liter of ANC. The other terms in the equation have similar explanations.
\begin{equation}\label{equation:Rapid_Mix/RM_Derivations:Rapid_Mix/RM_Derivations:14}
\begin{split}{\text{ANC}} = [HCO_3^ - {\text{] + 2[CO}}_3^{ - 2}{\text{] + [O}}{{\text{H}}^{\text{ - }}}{\text{] - [}}{{\text{H}}^{\text{ + }}}{\text{]}}\end{split}
\end{equation}
Note that carbonic acid and dissolved carbon dioxide are not in the ANC equation because they have no ability to neutralize protons.

We can write the ANC equation using alpha notation
\begin{equation}\label{equation:Rapid_Mix/RM_Derivations:Rapid_Mix/RM_Derivations:15}
\begin{split}ANC = {C_T}({\alpha_1} + 2{\alpha_2}) + \frac{{{K_w}}}{{\left[ {{H^ + }} \right]}} - \left[ {{H^ + }} \right]\end{split}
\end{equation}
For completeness we include acid neutralizing capacity for the case where the system is in equilibrium with atmospheric carbon dioxide,
\(CO_2\).
\begin{equation}\label{equation:Rapid_Mix/RM_Derivations:Rapid_Mix/RM_Derivations:16}
\begin{split}ANC_{atm\,equilibrium} = \frac{{{P{C{O_2}}}{K_H}}}{{{\alpha_0}}}({\alpha_1} + 2{\alpha_2}) + \frac{{{K_w}}}{{\left[ {{H^ + }} \right]}} - \left[ {{H^ + }} \right]\end{split}
\end{equation}

\subsection{pH Adjustment}
\label{\detokenize{Rapid_Mix/RM_Derivations:ph-adjustment}}\label{\detokenize{Rapid_Mix/RM_Derivations:heading-ph-adjustment}}
The final ANC, \(ANC_1\), after base addition and aluminum coagulant addition is given by
\begin{equation}\label{equation:Rapid_Mix/RM_Derivations:Rapid_Mix/RM_Derivations:17}
\begin{split}ANC_1 = ANC_0 + \Pi_{base}C_B + \Pi_{Al}C_{Al}\end{split}
\end{equation}
\begin{DUlineblock}{0em}
\item[] where:
\item[] \(ANC_0\) is the initial acid neutralizing capacity of the water sample.
\item[] \(ANC_1\) is the final acid neutralizing capacity of the mixture after the base and aluminum coagulant is added.
\item[] \(C_B\) is concentration of base in mole/liter
\item[] \(\Pi_{base}\) is ANC per mole of base
\item[] \(C_{Al}\) is the concentration of coagulant in mole of aluminum/liter
\item[] \(\Pi_{Al}\) is ANC per mole of aluminum
\end{DUlineblock}

The final carbonate concentration is given by
\begin{equation}\label{equation:Rapid_Mix/RM_Derivations:Rapid_Mix/RM_Derivations:18}
\begin{split}C_{T_1} ={C_{T_0}}+ \Pi_{CO_3^{-2}}C_B\end{split}
\end{equation}
\begin{DUlineblock}{0em}
\item[] where:
\item[] \(C_{T_1}\) is the final total carbonate concentration of the mixture after the base is added.
\item[] \(\Pi_{CO_3^{-2}}\) is mole of carbonate per mole of base (0 for \(NaOH\) and 1 for \(Na_2CO_3\))
\end{DUlineblock}

Substituting these values into the ANC equation we obtain
\begin{equation}\label{equation:Rapid_Mix/RM_Derivations:Rapid_Mix/RM_Derivations:19}
\begin{split}ANC_0 + \Pi_{base}C_B + \Pi_{Al}C_{Al} = ({C_{T_0}}+ \Pi_{CO_3^{-2}}C_B)({\alpha_1} + 2{\alpha_2}) +  \frac{{{K_w}}}{{\left[ {{H^ + }} \right]}} - \left[ {{H^ + }} \right]\end{split}
\end{equation}
Now we solve for \(C_B\), the concentration of base that must be added to reach a target pH.
\begin{equation}\label{equation:Rapid_Mix/RM_Derivations:Rapid_Mix/RM_Derivations:20}
\begin{split}(\Pi_{base} -\Pi_{CO_3^{-2}}({\alpha_1} + 2{\alpha_2}) )C_B= {C_{T_0}}({\alpha_1} + 2{\alpha_2}) +  \frac{{{K_w}}}{{\left[ {{H^ + }} \right]}} - \left[ {{H^ + }} \right] - ANC_0 - \Pi_{Al}C_{Al}\end{split}
\end{equation}\begin{equation}\label{equation:Rapid_Mix/RM_Derivations:Base_for_pH_Adjust}
\begin{split}C_B= \frac{{C_{T_0}}({\alpha_1} + 2{\alpha_2}) +  \frac{{{K_w}}}{{\left[ {{H^ + }} \right]}} - \left[ {{H^ + }} \right] - ANC_0 - \Pi_{Al}C_{Al}}{\Pi_{base} -\Pi_{CO_3^{-2}}({\alpha_1} + 2{\alpha_2})}\end{split}
\end{equation}
Note that the equations above can also be used for the case where acid is added to reduce the pH. In that case \(\Pi_{base}\) will have a negative value.

An example using this equation to find the required amount of base addition is given in {\hyperref[\detokenize{Rapid_Mix/RM_Examples:heading-example-ph-adjustment}]{\sphinxcrossref{\DUrole{std,std-ref}{Example: pH Adjustment}}}}.


\section{Equations for \protect\(\varepsilon\protect\) and \protect\(G\protect\) in Varying Flow Geometries}
\label{\detokenize{Rapid_Mix/RM_Derivations:equations-for-and-in-varying-flow-geometries}}\label{\detokenize{Rapid_Mix/RM_Derivations:heading-equations-varying-flow-geometries}}
Estimation of velocity gradients for various flow geometries is the basis for the design of rapid mix, flocculators, and plate settlers. Thus, our goal is to define the velocity gradients consistently across a range of possible flow regimes. There are three approaches to calculating the average velocity gradient within a control volume. 1) Use the Navier Stokes equations and solve for the spatially averaged velocity gradient. 1) Use Computational Fluid Dynamics (CFD) to solve for the spatially averaged velocity gradient. 1) Use the total mechanical energy loss in the control volume to calculate the energy dissipation rate. Estimate the velocity gradient directly from the energy dissipation rate, \(G_{CS} = \sqrt{\frac{\bar\varepsilon}{\nu}}\), as defined by Camp and Stein in 1943 (Camp, T. R., and Stein, P. C. (1943) ‘‘Velocity Gradients and Hydraulic Work in Fluid Motion,’’ J. Boston Soc. Civil Eng., 30, 203\textendash{}221.).

The first approach would be ideal but is difficult in practice because Navier Stokes solutions are only available for limited geometries and laminar flow. CFD could be used but is difficult to use as a general engineering design approach given the large number of geometries that are used in drinking water treatment plants. For these reasons we will use the control volume approach to estimate the average velocity gradient. This method incorrectly assumes that the energy dissipation rate is completely uniform in the control volume and hence the velocity gradient is also uniform. This method results in an over estimation of the velocity gradient.
The Camp-Stein estimate of \(G_{CS}\) is based on a control volume where the velocity gradient is uniform. Consider a layer of fluid of depth \(H\) and apply a velocity, \(v\) at the top of the fluid. The velocity gradient, \(G\), is thus \(\frac{v}{H}\) everywhere in the fluid. The force required to move the top of the fluid at velocity v can be obtained from the required shear, \(\tau\). From Newtons Law of Friction we have
\begin{equation}\label{equation:Rapid_Mix/RM_Derivations:Rapid_Mix/RM_Derivations:21}
\begin{split}\tau = \mu \frac{v}{H} = \mu G = \nu\rho G\end{split}
\end{equation}
Where \(\tau\) is the force required per unit plan view area. The power per unit area required to move the fluid at velocity \(v\) is \(\tau v\). The mass per unit area is \(\rho H\). Thus the energy dissipation rate or the power per mass is
\begin{equation}\label{equation:Rapid_Mix/RM_Derivations:Rapid_Mix/RM_Derivations:22}
\begin{split}\varepsilon = \frac{P}{m} = \frac{\tau v}{\rho H} = \frac{\nu \rho G v}{\rho H} = \nu G^2\end{split}
\end{equation}
This equation has no approximations, but has one very important assumption. We derived this equation for a control volume where the velocity gradient was \sphinxstylestrong{uniform}. The reactors and control volumes that we will be using as we design water treatment plants will \sphinxstylestrong{not} have uniform velocity gradients. Indeed, several of the water treatment processes will be turbulent and thus the velocity gradients in the fluid will vary in both space and time. Even in laminar flow in a pipe the velocity gradient is far from uniform with high velocity gradients at the wall and zero velocity gradient at the center of the pipe.

We’d like to know if we can apply the previous equation
\begin{equation}\label{equation:Rapid_Mix/RM_Derivations:Rapid_Mix/RM_Derivations:23}
\begin{split}\varepsilon = \nu G^2\end{split}
\end{equation}
to the case where the energy dissipation rate and velocity gradients are nonuniform by simply introducing average values of both quantities.
\begin{equation}\label{equation:Rapid_Mix/RM_Derivations:Rapid_Mix/RM_Derivations:24}
\begin{split}\bar\varepsilon \overset{?}{=} \nu \bar G^2\end{split}
\end{equation}
We will test this option with a simple case. Consider a hypothetical reactor (case 2) that is 4 times as large in plan view area as the uniform velocity gradient case explored above (case 1). In addition, assume that 3/4 of the reactor has a velocity gradient of zero. The average energy dissipation rate for case 1 is
\begin{equation}\label{equation:Rapid_Mix/RM_Derivations:Rapid_Mix/RM_Derivations:25}
\begin{split}\bar \varepsilon_1 = \frac{P_1}{m_1} =  \nu \bar G_1^2\end{split}
\end{equation}
The average energy dissipation rate for case 2 is
\begin{equation}\label{equation:Rapid_Mix/RM_Derivations:Rapid_Mix/RM_Derivations:26}
\begin{split}\bar \varepsilon_2 = \frac{P_1}{4m_1} = \frac{\bar \varepsilon_1}{4}\end{split}
\end{equation}
This makes sense because we are putting in the same amount of energy into a control volume that is 4 times bigger.

Now we calculate the velocity gradients. As previously determined,
\begin{equation}\label{equation:Rapid_Mix/RM_Derivations:Rapid_Mix/RM_Derivations:27}
\begin{split}\bar G_1 = \sqrt{\frac{\bar\varepsilon_1}{\nu}}\end{split}
\end{equation}
The average velocity gradient in the second control volume is simply the volume weighted average
\begin{equation}\label{equation:Rapid_Mix/RM_Derivations:Rapid_Mix/RM_Derivations:28}
\begin{split}\bar G_2 = \bar G_1\frac{1}{4}+ 0 \frac{3}{4}\end{split}
\end{equation}
where 1/4 of the case 2 control volume has the same velocity gradient as the case 1 control volume and 3/4 of the control volume has a velocity gradient of 0. The Camp Stein method would suggest that \(\bar G_2\) is equal to
\begin{equation}\label{equation:Rapid_Mix/RM_Derivations:Rapid_Mix/RM_Derivations:29}
\begin{split}\bar G_2 \overset{?}{=} \sqrt{\frac{\bar\varepsilon_2}{\nu}}= \sqrt{\frac{\bar\varepsilon_1}{4\nu}}\end{split}
\end{equation}
Now we check to see if the Camp Stein method of estimating the average velocity gradient, \(\bar G\), is correct.
\begin{equation}\label{equation:Rapid_Mix/RM_Derivations:Rapid_Mix/RM_Derivations:30}
\begin{split}\bar G_2 = \frac{\bar G_1}{4} \neq \sqrt{\frac{\bar\varepsilon_1}{4\nu}} =  \frac{\bar G_1}{2}\end{split}
\end{equation}
Given that the energy dissipation rate is proportional to the square of the velocity gradient the mean of the energy dissipation rate is \sphinxstylestrong{not} proportional to the mean of the velocity gradient. Thus the Camp Stein method of calculating the average velocity gradient is not correct except in the case of uniform velocity gradient. The Camp Stein equation is dimensionally correct and could be corrected by adding a dimensionless constant \(\Pi_{CS}\) that is a function of the energy dissipation rate distribution within the control volume.
\begin{equation}\label{equation:Rapid_Mix/RM_Derivations:Rapid_Mix/RM_Derivations:31}
\begin{split}\bar G =\Pi_{CS}\sqrt{\frac{\bar\varepsilon}{\nu}}\end{split}
\end{equation}
where \(\Pi_{CS}\) is 1 for a uniform velocity gradient and is less than one for non uniform velocity gradients. We can think \(\Pi_{CS}\) as a measure of the efficiency of using energy to deform the fluid. We can calculate \(\Pi_{CS}\) for cases where we have either a Navier Stokes or a computation fluid dynamics estimate of \(\bar G\).

The conventional approach to design of flocculators uses the Camp Stein definition of
\begin{equation}\label{equation:Rapid_Mix/RM_Derivations:Rapid_Mix/RM_Derivations:32}
\begin{split}G_{CS} = \sqrt{\frac{\bar\varepsilon}{\nu}}\end{split}
\end{equation}
where \(G_{CS}\) is \sphinxstylestrong{not} the average velocity gradient, but is larger than the average velocity gradient by a factor of \(\Pi_{CS}\). Thus we have
\begin{equation}\label{equation:Rapid_Mix/RM_Derivations:Rapid_Mix/RM_Derivations:33}
\begin{split}G_{CS} = \Pi_{CS}\bar G\end{split}
\end{equation}
Use of the Camp Stein velocity gradient in design of mixing units and flocculators results in an error when applying results from one reactor to another. If the energy dissipation rate distribution within the reactors is different, then \(\Pi_{CS}\) will be different for the two reactors and the actual average velocity gradient, \(\bar G\) will be different for the two reactors.

Given that energy is used more efficiently to produce velocity gradients if the velocity gradients are uniform, our goal is to design mixing and flocculation units that have relatively uniform velocity gradients. If all of our reactors at both research scale and municipal scale have similar values of \(\Pi_{CS}\), then we can use the Camp Stein definition of \(G_{CS}\) and not introduce any significant errors. It will not be reasonable, however, to expect similar performance based on similar values of \(G_{CS}\) if one reactor has relatively uniform energy dissipation rates and the other reactor has zones with very high energy dissipation rates and zones with very low energy dissipation rates.

We will demonstrate later that mechanically mixed reactors typically have a much wider range of energy dissipation rates than do well designed hydraulically mixed reactors. Thus comparisons between mechanically mixed and hydraulically mixed reactors must account for differences in \(\Pi_{CS}\).

We will use the Camp Stein definition \(G_{CS} = \sqrt{\frac{\bar\varepsilon}{\nu}}\) as the design parameter of convenience in this textbook.


\section{Estimates of time required for mixing processes}
\label{\detokenize{Rapid_Mix/RM_Derivations:estimates-of-time-required-for-mixing-processes}}\label{\detokenize{Rapid_Mix/RM_Derivations:heading-estimates-of-time-required-for-mixing-processes}}

\subsection{Turbulent Large Scale Eddies}
\label{\detokenize{Rapid_Mix/RM_Derivations:turbulent-large-scale-eddies}}\label{\detokenize{Rapid_Mix/RM_Derivations:heading-turbulent-large-scale-eddies}}
The first step in mixing is at the scale of the largest eddies. The largest eddies are limited in size by the smallest dimension normal to the direction of flow. Thus in a pipe the dimension of the largest eddies is set by the pipe diameter. In a open channel the dimension of the largest eddies is usually the water depth although it could be the width of the channel for the case of a narrow, deep channel.

Eddy turnover time, \(t_{eddy}\), is the time it takes for the eddy to travel a distance equal to its length-scale. Thus the eddy turnover time provides a good estimate of the time required for mixing to occur at the length scale of the eddy. We assume that the energy of the large eddy is dissipated into smaller length scales in the time \(t_{eddy}\):
\begin{equation}\label{equation:Rapid_Mix/RM_Derivations:Rapid_Mix/RM_Derivations:34}
\begin{split}t_{eddy} \approx \frac{L_{eddy}}{v_{eddy}}\end{split}
\end{equation}
The rate of energy loss to smaller scales is
\begin{equation}\label{equation:Rapid_Mix/RM_Derivations:Rapid_Mix/RM_Derivations:35}
\begin{split}\bar\varepsilon \approx\frac{v_{eddy}^2}{t_{eddy}}\end{split}
\end{equation}
Combining the two equations
\begin{equation}\label{equation:Rapid_Mix/RM_Derivations:Rapid_Mix/RM_Derivations:36}
\begin{split}\bar\varepsilon \approx\frac{v_{eddy}^3}{L_{eddy}}\end{split}
\end{equation}
We can use this equation to estimate the eddy velocity given an energy dissipation rate.
\begin{equation}\label{equation:Rapid_Mix/RM_Derivations:Rapid_Mix/RM_Derivations:37}
\begin{split}v_{eddy} \approx \left( \bar\varepsilon \, L_{eddy} \right)^\frac{1}{3}\end{split}
\end{equation}
Now we can solve for the eddy turnover time which is a measure of the mixing time at the eddy scale.
\begin{equation}\label{equation:Rapid_Mix/RM_Derivations:Rapid_Mix/RM_Derivations:38}
\begin{split}t_{eddy} \approx \frac{L_{eddy}}{\left( \bar\varepsilon \, L_{eddy} \right)^\frac{1}{3}} \approx \left( \frac{L_{eddy}^2}{ \bar\varepsilon }\right)^\frac{1}{3}\end{split}
\end{equation}
This provides a simple insight that the time required for an eddy to turn over scales with the size of the eddy raised to the 2/3 power. Thus large eddies take more time to turn over than do small eddies. Thus if we calculate the time required for large scale mixing using the dimension of the eddies, it will provide a reasonable estimate of the total time for mixing because mixing at all smaller scales requires much less time. A notable exception to this is the case of mixing in rivers. Rivers are usually shallow and wide. The largest eddies in a river are limited by the depth of the river. Mixing over the width of the river takes much longer than vertical mixing because multiple eddies are required to transport a substance from one side of the river to the other.

We can use the eddy velocity to estimate how long it will take for an eddy to cross the smallest dimension of flow. Eddy velocity is \(v_{eddy} \approx \left( \bar\varepsilon \, L_{eddy} \right)^\frac{1}{3}\).
The “\(\approx\)” indicates that this relationship is the same order of magnitude.

Chemical injection into the center of a pipe is common in drinking water treatment plants. We can develop equations to estimate the distance required for full mixing with the fluid in the pipe. In a pipe we have
\begin{equation}\label{equation:Rapid_Mix/RM_Derivations:Rapid_Mix/RM_Derivations:39}
\begin{split}v_{eddy} \approx \left( \bar\varepsilon \, D \right)^\frac{1}{3}\end{split}
\end{equation}
For a long straight pipe
\(\bar\varepsilon = \frac{{\rm f}}{2} \frac{\bar v^3}{D}\) (Equation \eqref{equation:Rapid_Mix/RM_Derivations:eq_EDR_straight_pipe}) and thus we can obtain the ratio between mean velocity and the velocity of the large scale eddies.
\begin{equation}\label{equation:Rapid_Mix/RM_Derivations:Rapid_Mix/RM_Derivations:40}
\begin{split}v_{eddy} \approx \left( \frac{{\rm f}}{2} \frac{\bar v^3}{D} \, D \right)^\frac{1}{3}\end{split}
\end{equation}\begin{equation}\label{equation:Rapid_Mix/RM_Derivations:Rapid_Mix/RM_Derivations:41}
\begin{split}\frac{v_{eddy}}{\bar v} \approx \left( \frac{{\rm f}}{2}   \right)^\frac{1}{3}\end{split}
\end{equation}
Given a friction factor of 0.02, the eddy velocity is approximately 20\% of the mean velocity. We can use this ratio to estimate how many pipe diameters downstream from an injection point will the coagulant be mixed across the diameter of the pipe.
\begin{align}\label{equation:Rapid_Mix/RM_Derivations:Rapid_Mix/RM_Derivations:42}\!\begin{aligned}
:label: mixing_pipe_diameters\\
N_{D_{pipe}} \approx \frac{\bar v}{v_{eddy}} \approx \left(\frac{2}{{\rm f}} \right)^\frac{1}{3}\\
\end{aligned}\end{align}
Where \(N_{D_{pipe}}\) is the distance in number of pipe diameters downstream of the injection point where complete mixing will have occurred. This estimate is a minimum distance and a factor of safety of 2 or more would reasonably be applied. In addition it is best practice to inject the coagulant in the center of the pipe. Injecting the coagulant at the side of the pipe will require considerably greater distance downstream for mixing across the pipe.

\fvset{hllines={, ,}}%
\begin{sphinxVerbatim}[commandchars=\\\{\}]
\PYG{n+nb}{print}\PYG{p}{(}\PYG{p}{(}\PYG{l+m+mf}{0.02}\PYG{o}{/}\PYG{l+m+mi}{2}\PYG{p}{)}\PYG{o}{*}\PYG{o}{*}\PYG{p}{(}\PYG{l+m+mi}{1}\PYG{o}{/}\PYG{l+m+mi}{3}\PYG{p}{)}\PYG{p}{)}
\end{sphinxVerbatim}


\subsection{Inner Viscous Length Scale}
\label{\detokenize{Rapid_Mix/RM_Derivations:inner-viscous-length-scale}}\label{\detokenize{Rapid_Mix/RM_Derivations:heading-inner-viscous-length-scale}}
The smallest scale at which inertia containing eddies causes mixing is set by the final damping of inertia by viscosity. Turbulence occurs when fluid inertia is too large to be damped by viscosity. The ratio of inertia to viscosity is given by the Reynolds number, \(\rm Re\):
\begin{equation}\label{equation:Rapid_Mix/RM_Derivations:Rapid_Mix/RM_Derivations:43}
\begin{split}{\rm{Re}} = \frac{\bar vD}{\nu}\end{split}
\end{equation}
Flows with high Reynolds numbers are turbulent (inertia dominated) and with low Reynolds are laminar (viscosity dominated). The transition Reynolds number is a function of the flow geometry and the velocity and length scale that are used to characterize the flow. In all turbulent flows there is a length scale at which inertia finally loses to viscosity. The scale where viscosity wins is some multiple of the Kolmogorov length scale, which is defined as:
\begin{equation}\label{equation:Rapid_Mix/RM_Derivations:Rapid_Mix/RM_Derivations:44}
\begin{split}\eta_K = \left( \frac{\nu^3}{\varepsilon} \right)^{\frac{1}{4}}\end{split}
\end{equation}
where \(\eta_K\) is the Kolmogorov length scale. At the Kolmogorov length scale viscosity completely dampens the inertia of the eddies and effectively “kills” the turbulence.

The length scale at which most of the kinetic energy contained in the small eddies is dissipated by viscosity is the inner viscous length scale, \(\lambda_v\), which is about \sphinxhref{http://dimotakis.caltech.edu/pdf/Dimotakis\_JFM2000.pdf}{50 times larger than} the
Kolmogorov length scale. Thus we have
\begin{equation}\label{equation:Rapid_Mix/RM_Derivations:eq_inner_viscous_length}
\begin{split}\lambda_\nu = \Pi_{K\nu}\left( \frac{\nu^3}{\varepsilon} \right)^{\frac{1}{4}}\end{split}
\end{equation}
where \(\Pi_{K\nu} = 50\)

At length scales larger than the inner viscous length scale, \(\lambda_v\), the dominant transport mechanism is by turbulent eddies. At length scales smaller than \(\lambda_v\) the dominant transport mechanism is fluid deformation due to shear. If the flow regime is completely laminar such as in a small diameter tube flocculator, then the dominant transport mechanism is fluid deformation due to shear at length scales all the way up to the diameter of the tubing.

The dividing line between eddy transport and fluid deformation controlled by viscosity can be calculated as a function of the energy dissipation rate using \eqref{equation:Rapid_Mix/RM_Derivations:eq_inner_viscous_length}.

\fvset{hllines={, ,}}%
\begin{sphinxVerbatim}[commandchars=\\\{\}]
\PYG{l+s+sd}{\PYGZdq{}\PYGZdq{}\PYGZdq{} importing \PYGZdq{}\PYGZdq{}\PYGZdq{}}
\PYG{k+kn}{from} \PYG{n+nn}{aide\PYGZus{}design}\PYG{n+nn}{.}\PYG{n+nn}{play} \PYG{k}{import}\PYG{o}{*}
\PYG{k+kn}{from} \PYG{n+nn}{aguaclara\PYGZus{}research}\PYG{n+nn}{.}\PYG{n+nn}{play} \PYG{k}{import}\PYG{o}{*}
\PYG{k+kn}{import} \PYG{n+nn}{aguaclara\PYGZus{}research}\PYG{n+nn}{.}\PYG{n+nn}{floc\PYGZus{}model} \PYG{k}{as} \PYG{n+nn}{fm}
\PYG{k+kn}{import} \PYG{n+nn}{matplotlib}\PYG{n+nn}{.}\PYG{n+nn}{pyplot} \PYG{k}{as} \PYG{n+nn}{plt}
\PYG{k+kn}{from} \PYG{n+nn}{matplotlib}\PYG{n+nn}{.}\PYG{n+nn}{ticker} \PYG{k}{import} \PYG{n}{FormatStrFormatter}
\PYG{n}{imagepath} \PYG{o}{=} \PYG{l+s+s1}{\PYGZsq{}}\PYG{l+s+s1}{Rapid\PYGZus{}Mix/Images/}\PYG{l+s+s1}{\PYGZsq{}}
\PYG{n}{EDR\PYGZus{}array} \PYG{o}{=} \PYG{n}{np}\PYG{o}{.}\PYG{n}{logspace}\PYG{p}{(}\PYG{l+m+mi}{0}\PYG{p}{,}\PYG{l+m+mi}{4}\PYG{p}{,}\PYG{n}{num}\PYG{o}{=}\PYG{l+m+mi}{50}\PYG{p}{)}\PYG{o}{*}\PYG{n}{u}\PYG{o}{.}\PYG{n}{mW}\PYG{o}{/}\PYG{n}{u}\PYG{o}{.}\PYG{n}{kg}
\PYG{n}{Temperature} \PYG{o}{=} \PYG{l+m+mi}{20}\PYG{o}{*}\PYG{n}{u}\PYG{o}{.}\PYG{n}{degC}
\PYG{k}{def} \PYG{n+nf}{Inner\PYGZus{}viscous}\PYG{p}{(}\PYG{n}{EDR}\PYG{p}{,} \PYG{n}{Temperature}\PYG{p}{)}\PYG{p}{:}
    \PYG{k}{return} \PYG{n}{fm}\PYG{o}{.}\PYG{n}{RATIO\PYGZus{}KOLMOGOROV} \PYG{o}{*} \PYG{n}{fm}\PYG{o}{.}\PYG{n}{eta\PYGZus{}kolmogorov}\PYG{p}{(}\PYG{n}{EDR}\PYG{p}{,} \PYG{n}{Temperature}\PYG{p}{)}

\PYG{n}{fig}\PYG{p}{,} \PYG{n}{ax} \PYG{o}{=} \PYG{n}{plt}\PYG{o}{.}\PYG{n}{subplots}\PYG{p}{(}\PYG{p}{)}
\PYG{n}{ax}\PYG{o}{.}\PYG{n}{semilogx}\PYG{p}{(}\PYG{n}{EDR\PYGZus{}array}\PYG{o}{.}\PYG{n}{to}\PYG{p}{(}\PYG{n}{u}\PYG{o}{.}\PYG{n}{mW}\PYG{o}{/}\PYG{n}{u}\PYG{o}{.}\PYG{n}{kg}\PYG{p}{)}\PYG{p}{,}\PYG{n}{Inner\PYGZus{}viscous}\PYG{p}{(}\PYG{n}{EDR\PYGZus{}array}\PYG{p}{,} \PYG{n}{Temperature}\PYG{p}{)}\PYG{o}{.}\PYG{n}{to}\PYG{p}{(}\PYG{n}{u}\PYG{o}{.}\PYG{n}{mm}\PYG{p}{)}\PYG{p}{)}
\PYG{n}{ax}\PYG{o}{.}\PYG{n}{yaxis}\PYG{o}{.}\PYG{n}{set\PYGZus{}major\PYGZus{}formatter}\PYG{p}{(}\PYG{n}{FormatStrFormatter}\PYG{p}{(}\PYG{l+s+s1}{\PYGZsq{}}\PYG{l+s+s1}{\PYGZpc{}}\PYG{l+s+s1}{.f}\PYG{l+s+s1}{\PYGZsq{}}\PYG{p}{)}\PYG{p}{)}
\PYG{n}{ax}\PYG{o}{.}\PYG{n}{xaxis}\PYG{o}{.}\PYG{n}{set\PYGZus{}major\PYGZus{}formatter}\PYG{p}{(}\PYG{n}{FormatStrFormatter}\PYG{p}{(}\PYG{l+s+s1}{\PYGZsq{}}\PYG{l+s+s1}{\PYGZpc{}}\PYG{l+s+s1}{.f}\PYG{l+s+s1}{\PYGZsq{}}\PYG{p}{)}\PYG{p}{)}
\PYG{n}{ax}\PYG{o}{.}\PYG{n}{set}\PYG{p}{(}\PYG{n}{xlabel}\PYG{o}{=}\PYG{l+s+s1}{\PYGZsq{}}\PYG{l+s+s1}{Energy dissipation rate (W/kg)}\PYG{l+s+s1}{\PYGZsq{}}\PYG{p}{,} \PYG{n}{ylabel}\PYG{o}{=}\PYG{l+s+s1}{\PYGZsq{}}\PYG{l+s+s1}{Inner viscous length scale (mm)}\PYG{l+s+s1}{\PYGZsq{}}\PYG{p}{)}
\PYG{n}{ax}\PYG{o}{.}\PYG{n}{text}\PYG{p}{(}\PYG{l+m+mi}{30}\PYG{p}{,} \PYG{l+m+mi}{6}\PYG{p}{,} \PYG{l+s+s1}{\PYGZsq{}}\PYG{l+s+s1}{Eddies cause mixing}\PYG{l+s+s1}{\PYGZsq{}}\PYG{p}{,} \PYG{n}{fontsize}\PYG{o}{=}\PYG{l+m+mi}{12}\PYG{p}{,}\PYG{n}{rotation}\PYG{o}{=}\PYG{o}{\PYGZhy{}}\PYG{l+m+mi}{30}\PYG{p}{)}
\PYG{n}{ax}\PYG{o}{.}\PYG{n}{text}\PYG{p}{(}\PYG{l+m+mi}{1}\PYG{p}{,} \PYG{l+m+mi}{5}\PYG{p}{,} \PYG{l+s+s1}{\PYGZsq{}}\PYG{l+s+s1}{Shear and diffusion cause mixing}\PYG{l+s+s1}{\PYGZsq{}}\PYG{p}{,} \PYG{n}{fontsize}\PYG{o}{=}\PYG{l+m+mi}{12}\PYG{p}{,}\PYG{n}{rotation}\PYG{o}{=}\PYG{o}{\PYGZhy{}}\PYG{l+m+mi}{30}\PYG{p}{)}
\PYG{n}{fig}\PYG{o}{.}\PYG{n}{savefig}\PYG{p}{(}\PYG{n}{imagepath}\PYG{o}{+}\PYG{l+s+s1}{\PYGZsq{}}\PYG{l+s+s1}{Inner\PYGZus{}viscous\PYGZus{}vs\PYGZus{}EDR}\PYG{l+s+s1}{\PYGZsq{}}\PYG{p}{)}
\PYG{n}{plt}\PYG{o}{.}\PYG{n}{show}\PYG{p}{(}\PYG{p}{)}
\end{sphinxVerbatim}

\begin{figure}[htbp]
\centering
\capstart

\noindent\sphinxincludegraphics[width=400\sphinxpxdimen]{{Inner_viscous_vs_EDR}.png}
\caption{Eddies can cause fluid mixing down to the scale of a few millimeters for energy dissipation rates used in rapid mix units and flocculators.}\label{\detokenize{Rapid_Mix/RM_Derivations:id3}}\label{\detokenize{Rapid_Mix/RM_Derivations:figure-inner-viscous-vs-edr}}\end{figure}


\subsection{Turbulent Mixing Time as a Function of Scale}
\label{\detokenize{Rapid_Mix/RM_Derivations:turbulent-mixing-time-as-a-function-of-scale}}\label{\detokenize{Rapid_Mix/RM_Derivations:heading-mixing-time-as-a-function-of-scale}}
We are searching for the rate limiting step in the mixing process as we transition from the scale of the flow down to the scale of the coagulant nanoparticles. We can estimate the time required for eddies to mix at their length scales by assuming that the eddies pass all of their energy to smaller scales in the time it takes for an eddy to travel the distance equal to the length scale of the eddy. This time is known as the \sphinxstylestrong{{}`eddy turnover time \textless{}http://ceeserver.cee.cornell.edu/eac20/cee637/handouts/TURBFLOW\_1.pdf\textgreater{}{}`\_\_},
\(t_{eddy}\). {\hyperref[\detokenize{Rapid_Mix/RM_Derivations:heading-estimates-of-time-required-for-mixing-processes}]{\sphinxcrossref{\DUrole{std,std-ref}{The derivation for the equation below is found here}}}}.
\begin{equation}\label{equation:Rapid_Mix/RM_Derivations:eq_t_eddy}
\begin{split}t_{eddy} \approx \left( \frac{L_{eddy}^2}{ \bar\varepsilon }\right)^\frac{1}{3}\end{split}
\end{equation}
We can plot the eddy turnover time as a function of scale from the inner viscous length scale up to the scale of the flow. We will discover whether large scale mixing by eddies is faster or slower than small scale mixing by eddies.

\fvset{hllines={, ,}}%
\begin{sphinxVerbatim}[commandchars=\\\{\}]
\PYG{k+kn}{from} \PYG{n+nn}{aide\PYGZus{}design}\PYG{n+nn}{.}\PYG{n+nn}{play} \PYG{k}{import}\PYG{o}{*}
\PYG{n}{EDR\PYGZus{}graph} \PYG{o}{=} \PYG{n}{np}\PYG{o}{.}\PYG{n}{array}\PYG{p}{(}\PYG{p}{[}\PYG{l+m+mf}{0.01}\PYG{p}{,}\PYG{l+m+mf}{0.1}\PYG{p}{,}\PYG{l+m+mi}{1}\PYG{p}{,}\PYG{l+m+mi}{10} \PYG{p}{]}\PYG{p}{)}\PYG{o}{*}\PYG{n}{u}\PYG{o}{.}\PYG{n}{W}\PYG{o}{/}\PYG{n}{u}\PYG{o}{.}\PYG{n}{kg}
\PYG{n}{Temperature}
\PYG{l+s+sd}{\PYGZdq{}\PYGZdq{}\PYGZdq{}Use the highest EDR to estimate the smallest length scale\PYGZdq{}\PYGZdq{}\PYGZdq{}}
\PYG{n}{Inner\PYGZus{}viscous\PYGZus{}graph} \PYG{o}{=} \PYG{n}{Inner\PYGZus{}viscous}\PYG{p}{(}\PYG{n}{EDR\PYGZus{}graph}\PYG{p}{[}\PYG{l+m+mi}{2}\PYG{p}{]}\PYG{p}{,} \PYG{n}{Temperature}\PYG{p}{)}
\PYG{n}{Inner\PYGZus{}viscous\PYGZus{}graph}
\PYG{n}{L\PYGZus{}flow} \PYG{o}{=} \PYG{l+m+mf}{0.5}\PYG{o}{*}\PYG{n}{u}\PYG{o}{.}\PYG{n}{m}
\PYG{n}{L\PYGZus{}scale} \PYG{o}{=} \PYG{n}{np}\PYG{o}{.}\PYG{n}{logspace}\PYG{p}{(}\PYG{n}{np}\PYG{o}{.}\PYG{n}{log10}\PYG{p}{(}\PYG{n}{Inner\PYGZus{}viscous\PYGZus{}graph}\PYG{o}{.}\PYG{n}{magnitude}\PYG{p}{)}\PYG{p}{,}\PYG{n}{np}\PYG{o}{.}\PYG{n}{log10}\PYG{p}{(}\PYG{n}{L\PYGZus{}flow}\PYG{o}{.}\PYG{n}{magnitude}\PYG{p}{)}\PYG{p}{,}\PYG{l+m+mi}{50}\PYG{p}{)}
\PYG{n}{L\PYGZus{}scale}
\PYG{n}{imagepath} \PYG{o}{=} \PYG{l+s+s1}{\PYGZsq{}}\PYG{l+s+s1}{Rapid\PYGZus{}Mix/Images/}\PYG{l+s+s1}{\PYGZsq{}}
\PYG{n}{fig}\PYG{p}{,} \PYG{n}{ax} \PYG{o}{=} \PYG{n}{plt}\PYG{o}{.}\PYG{n}{subplots}\PYG{p}{(}\PYG{p}{)}
\PYG{k}{for} \PYG{n}{i} \PYG{o+ow}{in} \PYG{n+nb}{range}\PYG{p}{(}\PYG{n+nb}{len}\PYG{p}{(}\PYG{n}{EDR\PYGZus{}graph}\PYG{p}{)}\PYG{p}{)}\PYG{p}{:}
  \PYG{n}{ax}\PYG{o}{.}\PYG{n}{semilogx}\PYG{p}{(}\PYG{n}{L\PYGZus{}scale}\PYG{p}{,}\PYG{p}{(}\PYG{p}{(}\PYG{n}{L\PYGZus{}scale}\PYG{o}{*}\PYG{o}{*}\PYG{l+m+mi}{2}\PYG{o}{/}\PYG{n}{EDR\PYGZus{}graph}\PYG{p}{[}\PYG{n}{i}\PYG{p}{]}\PYG{p}{)}\PYG{o}{*}\PYG{o}{*}\PYG{p}{(}\PYG{l+m+mi}{1}\PYG{o}{/}\PYG{l+m+mi}{3}\PYG{p}{)}\PYG{p}{)}\PYG{o}{.}\PYG{n}{to\PYGZus{}base\PYGZus{}units}\PYG{p}{(}\PYG{p}{)}\PYG{p}{)}

\PYG{n}{ax}\PYG{o}{.}\PYG{n}{legend}\PYG{p}{(}\PYG{n}{EDR\PYGZus{}graph}\PYG{p}{)}

\PYG{c+c1}{\PYGZsh{}ax.yaxis.set\PYGZus{}major\PYGZus{}formatter(FormatStrFormatter(\PYGZsq{}\PYGZpc{}.f\PYGZsq{}))}
\PYG{c+c1}{\PYGZsh{}ax.xaxis.set\PYGZus{}major\PYGZus{}formatter(FormatStrFormatter(\PYGZsq{}\PYGZpc{}.f\PYGZsq{}))}
\PYG{n}{ax}\PYG{o}{.}\PYG{n}{set}\PYG{p}{(}\PYG{n}{xlabel}\PYG{o}{=}\PYG{l+s+s1}{\PYGZsq{}}\PYG{l+s+s1}{Length (m)}\PYG{l+s+s1}{\PYGZsq{}}\PYG{p}{,} \PYG{n}{ylabel}\PYG{o}{=}\PYG{l+s+s1}{\PYGZsq{}}\PYG{l+s+s1}{Eddy turnover time (s)}\PYG{l+s+s1}{\PYGZsq{}}\PYG{p}{)}
\PYG{n}{fig}\PYG{o}{.}\PYG{n}{savefig}\PYG{p}{(}\PYG{n}{imagepath}\PYG{o}{+}\PYG{l+s+s1}{\PYGZsq{}}\PYG{l+s+s1}{Eddy\PYGZus{}turnover\PYGZus{}time}\PYG{l+s+s1}{\PYGZsq{}}\PYG{p}{)}
\PYG{n}{plt}\PYG{o}{.}\PYG{n}{show}\PYG{p}{(}\PYG{p}{)}
\end{sphinxVerbatim}

\begin{figure}[htbp]
\centering
\capstart

\noindent\sphinxincludegraphics[width=400\sphinxpxdimen]{{Eddy_turnover_time}.png}
\caption{Small eddies turn in less time than large eddies. This is why the mixing at the largest scale dominates the mixing time.}\label{\detokenize{Rapid_Mix/RM_Derivations:id4}}\label{\detokenize{Rapid_Mix/RM_Derivations:figure-eddy-turnover-time}}\end{figure}


\subsection{Shear-Diffusion Transport}
\label{\detokenize{Rapid_Mix/RM_Derivations:shear-diffusion-transport}}\label{\detokenize{Rapid_Mix/RM_Derivations:heading-shear-diffusion-transport}}
After the first few seconds in which mixing occurs from the length scale of the flow down to the inner viscous length scale the next step in the transport process is blending of the coagulant uniformly with the raw water. At the end of the turbulent transport the coagulant stock has been stretched out into thin bands throughout the raw water, but the two fluids are not actually blended together by turbulence. The blending is accomplished by a combination of fluid deformation controlled by viscous shear and then by molecular diffusion.


\subsection{Fluid Deformation by Shear}
\label{\detokenize{Rapid_Mix/RM_Derivations:fluid-deformation-by-shear}}\label{\detokenize{Rapid_Mix/RM_Derivations:heading-fluid-deformation-by-shear}}
The time scale for fluid deformation is \(1/G\) where \(G\) is the velocity gradient. This simple relationship is because the velocity of fluid deformation is proportional to the length scale and thus the time to travel any given distance in a linear velocity gradient is always the same. Velocity gradients in conventional mechanized rapid mix units are order 1000 Hz and thus the time for fluid deformation to blur concentration gradients is approximately 1 ms. This confirms the idea that blending the coagulant with the raw water is actually a very fast process with the slowest phase being the transport by turbulent eddies at the scale of reactor.

The full time required for fluid deformation to achieve blending down to the scale where molecular diffusion takes over is likely a multiple of 1/G where the multiple is determined by the number of different directions that the fluid must be sheared in to reach close to uniform blending. However, even multiplying 1/G by a factor of 10 still results in very rapid mixing.


\subsection{Einstein’s Diffusion Equation}
\label{\detokenize{Rapid_Mix/RM_Derivations:einsteins-diffusion-equation}}\label{\detokenize{Rapid_Mix/RM_Derivations:heading-einstein-diffusion-equation}}
The final step of mixing is mediated by molecular diffusion. We can estimate the length scale at which fluid shear and diffusion provide transport at the same rate. Einstein’s diffusion equation is
\begin{equation}\label{equation:Rapid_Mix/RM_Derivations:Rapid_Mix/RM_Derivations:45}
\begin{split}D_{Diffusion} = \frac{k_B T}{3 \pi \mu d_P}\end{split}
\end{equation}
where \(k_B\) is the Boltzmann constant and \(d_P\) is the diameter of the particle that is diffusion in a fluid with viscosity \(\nu\) and density \(\rho\). The diffusion coefficient \(D_{Diffusion}\) has dimensions of \(\frac{[L^2]}{[T]}\) and can be understood as the velocity of the particle multiplied by the length of the mean free path. From dimensional analysis the time for diffusion to blur a concentration gradient over a length scale, \(L_{Diffusion}\) is
\begin{equation}\label{equation:Rapid_Mix/RM_Derivations:Rapid_Mix/RM_Derivations:46}
\begin{split}t_{Diffusion} \approx \frac{L_{Diffusion}^2}{D_{Diffusion}}\end{split}
\end{equation}
The shear time scale is \(1/G\) and thus we can solve for the length scale at which diffusion and shear have equivalent transport rates.
\begin{equation}\label{equation:Rapid_Mix/RM_Derivations:Rapid_Mix/RM_Derivations:47}
\begin{split}1/G \approx t_{Diffusion} \approx \frac{L_{Diffusion}^2}{D_{Diffusion}}\end{split}
\end{equation}
Substitute Einstein’s diffusion equation and solve for the length scale that transitions between shear and diffusion transport.
\begin{equation}\label{equation:Rapid_Mix/RM_Derivations:Rapid_Mix/RM_Derivations:48}
\begin{split}L_{Diffusion}^{Shear} \approx \sqrt{\frac{k_B T}{3 G \pi \mu  d_P}}\end{split}
\end{equation}
\fvset{hllines={, ,}}%
\begin{sphinxVerbatim}[commandchars=\\\{\}]
\PYG{k+kn}{from} \PYG{n+nn}{aide\PYGZus{}design}\PYG{n+nn}{.}\PYG{n+nn}{play} \PYG{k}{import}\PYG{o}{*}
\PYG{k+kn}{from} \PYG{n+nn}{aguaclara\PYGZus{}research}\PYG{n+nn}{.}\PYG{n+nn}{play} \PYG{k}{import}\PYG{o}{*}
\PYG{k+kn}{import} \PYG{n+nn}{aguaclara\PYGZus{}research}\PYG{n+nn}{.}\PYG{n+nn}{floc\PYGZus{}model} \PYG{k}{as} \PYG{n+nn}{fm}
\PYG{k}{def} \PYG{n+nf}{L\PYGZus{}Shear\PYGZus{}Diffusion}\PYG{p}{(}\PYG{n}{G}\PYG{p}{,}\PYG{n}{Temperature}\PYG{p}{,}\PYG{n}{d\PYGZus{}particle}\PYG{p}{)}\PYG{p}{:}
  \PYG{k}{return} \PYG{n}{np}\PYG{o}{.}\PYG{n}{sqrt}\PYG{p}{(}\PYG{p}{(}\PYG{n}{u}\PYG{o}{.}\PYG{n}{boltzmann\PYGZus{}constant}\PYG{o}{*}\PYG{n}{Temperature}\PYG{o}{/}
  \PYG{p}{(}\PYG{l+m+mi}{3} \PYG{o}{*} \PYG{n}{G} \PYG{o}{*}  \PYG{n}{np}\PYG{o}{.}\PYG{n}{pi} \PYG{o}{*}\PYG{n}{pc}\PYG{o}{.}\PYG{n}{viscosity\PYGZus{}dynamic}\PYG{p}{(}\PYG{n}{Temperature}\PYG{p}{)}\PYG{o}{*} \PYG{n}{d\PYGZus{}particle}\PYG{p}{)}\PYG{p}{)}\PYG{o}{.}\PYG{n}{to\PYGZus{}base\PYGZus{}units}\PYG{p}{(}\PYG{p}{)}\PYG{p}{)}

\PYG{n}{G} \PYG{o}{=} \PYG{n}{np}\PYG{o}{.}\PYG{n}{arange}\PYG{p}{(}\PYG{l+m+mi}{10}\PYG{p}{,}\PYG{l+m+mi}{5000}\PYG{p}{)}\PYG{o}{*}\PYG{n}{u}\PYG{o}{.}\PYG{n}{Hz}
\PYG{n}{d\PYGZus{}particle} \PYG{o}{=} \PYG{n}{fm}\PYG{o}{.}\PYG{n}{PACl}\PYG{o}{.}\PYG{n}{Diameter}\PYG{o}{*}\PYG{n}{u}\PYG{o}{.}\PYG{n}{m}
\PYG{n}{Temperature}\PYG{o}{=}\PYG{l+m+mi}{20}\PYG{o}{*}\PYG{n}{u}\PYG{o}{.}\PYG{n}{degC}
\PYG{n}{x} \PYG{o}{=} \PYG{p}{(}\PYG{n}{L\PYGZus{}Shear\PYGZus{}Diffusion}\PYG{p}{(}\PYG{n}{G}\PYG{p}{,}\PYG{n}{Temperature}\PYG{p}{,}\PYG{n}{d\PYGZus{}particle}\PYG{p}{)}\PYG{p}{)}\PYG{o}{.}\PYG{n}{to}\PYG{p}{(}\PYG{n}{u}\PYG{o}{.}\PYG{n}{nm}\PYG{p}{)}
\PYG{n}{imagepath} \PYG{o}{=} \PYG{l+s+s1}{\PYGZsq{}}\PYG{l+s+s1}{Rapid\PYGZus{}Mix/Images/}\PYG{l+s+s1}{\PYGZsq{}}
\PYG{n}{fig}\PYG{p}{,} \PYG{n}{ax} \PYG{o}{=} \PYG{n}{plt}\PYG{o}{.}\PYG{n}{subplots}\PYG{p}{(}\PYG{p}{)}
\PYG{n}{ax}\PYG{o}{.}\PYG{n}{semilogx}\PYG{p}{(}\PYG{n}{G}\PYG{p}{,}\PYG{n}{x}\PYG{p}{)}
\PYG{n}{ax}\PYG{o}{.}\PYG{n}{set}\PYG{p}{(}\PYG{n}{xlabel}\PYG{o}{=}\PYG{l+s+s1}{\PYGZsq{}}\PYG{l+s+s1}{Velocity gradient (Hz)}\PYG{l+s+s1}{\PYGZsq{}}\PYG{p}{,} \PYG{n}{ylabel}\PYG{o}{=}\PYG{l+s+s1}{\PYGZsq{}}\PYG{l+s+s1}{Length scale (nm)}\PYG{l+s+s1}{\PYGZsq{}}\PYG{p}{)}
\PYG{n}{fig}\PYG{o}{.}\PYG{n}{savefig}\PYG{p}{(}\PYG{n}{imagepath}\PYG{o}{+}\PYG{l+s+s1}{\PYGZsq{}}\PYG{l+s+s1}{Shear\PYGZus{}diffusion\PYGZus{}length\PYGZus{}scale}\PYG{l+s+s1}{\PYGZsq{}}\PYG{p}{)}
\PYG{n}{plt}\PYG{o}{.}\PYG{n}{show}\PYG{p}{(}\PYG{p}{)}
\end{sphinxVerbatim}

\begin{figure}[htbp]
\centering
\capstart

\noindent\sphinxincludegraphics[width=400\sphinxpxdimen]{{Shear_diffusion_length_scale}.png}
\caption{The length scale at which diffusion becomes the dominant transport mechanism for coagulant nanoparticles as a function of the velocity gradient. The time scale for the final diffusion to achieve blending of the nanoparticles with the water is simply 1/G.}\label{\detokenize{Rapid_Mix/RM_Derivations:id5}}\label{\detokenize{Rapid_Mix/RM_Derivations:figure-shear-diffusion-length-scale}}\end{figure}

Molecular diffusion finishes the blending process by transporting the coagulant nanoparticles the last few hundred nanometers. The entire mixing process from the coagulant injection point to uniform blending with the raw water takes only a few seconds.
\begin{description}
\item[{We have demonstrated that all of the steps for mixing of the coagulant nanoparticles with the raw water are very fast. Compared with the time required for flocculation, 10s to 1000s of seconds, the time required for this mixing (blending the nanoparticles uniformly with the water) is insignificant. The remaining steps are:}] \leavevmode
1. Molecular diffusion causes some dissolved species and Al nanoparticles to aggregate.
1. Fluid shear and molecular diffusion cause Al nanoparticles with attached formerly dissolved species to collide with inorganic particles (such as clay) and organic particles (such as viruses, bacteria, and protozoans).

\end{description}

The time scale for the fluid shear and molecular diffusion to cause coagulant nanoparticles to collide with particles is estimated in {\hyperref[\detokenize{Rapid_Mix/RM_Theory_and_Future_Work:heading-diffusion-and-shear-transport-coagulant-nanoparticles-to-clay}]{\sphinxcrossref{\DUrole{std,std-ref}{Diffusion and Shear Transport Coagulant Nanoparticles to Clay}}}}.

Below are the derivations for the equations that appear in \hyperref[\detokenize{Rapid_Mix/RM_Intro:table-control-volume-equations}]{Table \ref{\detokenize{Rapid_Mix/RM_Intro:table-control-volume-equations}}} containing equations for \(G\), \(\varepsilon\), and \(h_L\).


\subsection{Straight pipe (wall shear)}
\label{\detokenize{Rapid_Mix/RM_Derivations:straight-pipe-wall-shear}}\label{\detokenize{Rapid_Mix/RM_Derivations:heading-straight-pipe-wall-shear}}
The average energy dissipation rate, \(\bar\varepsilon\), in a control volume with residence time \(\theta\) is
\begin{equation}\label{equation:Rapid_Mix/RM_Derivations:eq_EDR_straight_pipe}
\begin{split} \bar\varepsilon = \frac{gh_{\rm{L}}}{\theta}\end{split}
\end{equation}
The residence time can be expressed as a function of length and average velocity.
\begin{equation}\label{equation:Rapid_Mix/RM_Derivations:Rapid_Mix/RM_Derivations:49}
\begin{split}\theta = \frac{L}{\bar v}\end{split}
\end{equation}
For straight pipe flow the only head loss is due to wall shear and thus we have the Darcy Weisbach equation.
\begin{equation}\label{equation:Rapid_Mix/RM_Derivations:Rapid_Mix/RM_Derivations:50}
\begin{split}h_{{\rm f}} = {{\rm f}} \frac{L}{D} \frac{\bar v^2}{2g}\end{split}
\end{equation}
Combining the 3 previous equations we obtain the energy dissipation rate for pipe flow
\begin{equation}\label{equation:Rapid_Mix/RM_Derivations:Rapid_Mix/RM_Derivations:51}
\begin{split}\bar\varepsilon = \frac{{\rm f}}{2} \frac{\bar v^3}{D}\end{split}
\end{equation}
The average velocity gradient was defined by Camp and Stein as
\begin{equation}\label{equation:Rapid_Mix/RM_Derivations:Rapid_Mix/RM_Derivations:52}
\begin{split}G_{CS} = \sqrt{\frac{\bar \varepsilon}{\nu}}\end{split}
\end{equation}
where this approximation neglects the fact that square root of an average is not the same as the average of the square roots.
\begin{equation}\label{equation:Rapid_Mix/RM_Derivations:Rapid_Mix/RM_Derivations:53}
\begin{split}G_{CS} = \left(\frac{{\rm f}}{2\nu} \frac{\bar v^3}{D} \right)^\frac{1}{2}\end{split}
\end{equation}
or in terms of flow rate, we have:
\begin{equation}\label{equation:Rapid_Mix/RM_Derivations:Rapid_Mix/RM_Derivations:54}
\begin{split}G_{CS} = \left(\frac{\rm{32f}}{ \pi^3\nu} \frac{Q^3}{D^7} \right)^\frac{1}{2}\end{split}
\end{equation}

\subsection{Straight Pipe Laminar}
\label{\detokenize{Rapid_Mix/RM_Derivations:straight-pipe-laminar}}\label{\detokenize{Rapid_Mix/RM_Derivations:heading-straight-pipe-laminar}}
Laboratory scale apparatus is often limited to laminar flow where viscosity effects dominate. The equations describing laminar flow conditions always include viscosity. For the case of laminar flow in a straight pipe, we have:
\begin{equation}\label{equation:Rapid_Mix/RM_Derivations:Rapid_Mix/RM_Derivations:55}
\begin{split}{\rm f} = \frac{64}{Re}\end{split}
\end{equation}
Reynolds number is defined as
\begin{equation}\label{equation:Rapid_Mix/RM_Derivations:Rapid_Mix/RM_Derivations:56}
\begin{split}Re= \frac{\bar vD}{\nu}\end{split}
\end{equation}
The Darcy Weisbach head loss equation simplifies to the Hagen\textendash{}Poiseuille equation for the case of laminar flow.
\begin{equation}\label{equation:Rapid_Mix/RM_Derivations:Rapid_Mix/RM_Derivations:57}
\begin{split}h_{{\rm f}} = \frac{32\nu L\bar v}{gD^2}\end{split}
\end{equation}
and thus the energy dissipation rate in a straight pipe under conditions of laminar flow is
\begin{equation}\label{equation:Rapid_Mix/RM_Derivations:Rapid_Mix/RM_Derivations:58}
\begin{split}\bar\varepsilon =32\nu \left( \frac{\bar v}{D} \right)^2\end{split}
\end{equation}
The Camp-Stein velocity gradient in a long straight laminar flow tube is thus
\begin{equation}\label{equation:Rapid_Mix/RM_Derivations:Rapid_Mix/RM_Derivations:59}
\begin{split}G_{CS}^2 =32 \left( \frac{\bar v}{D} \right)^2\end{split}
\end{equation}\begin{equation}\label{equation:Rapid_Mix/RM_Derivations:Rapid_Mix/RM_Derivations:60}
\begin{split}G_{CS} =4\sqrt2 \frac{\bar v}{D}\end{split}
\end{equation}
Our estimate of \(G_{CS}\) based on \(\bar \varepsilon\) is an overestimate because it assumes that the energy dissipation is completely uniform through the control volume. The true spatial average velocity gradient, \(\bar G\), for laminar flow in a pipe is (\sphinxhref{https://doi.org/10.1016/0009-2509(81)80126-1}{Gregory, 1981}),
\begin{equation}\label{equation:Rapid_Mix/RM_Derivations:Rapid_Mix/RM_Derivations:61}
\begin{split}\bar G = \frac{8}{3}\frac{\bar v}{D}\end{split}
\end{equation}
Our estimate of \(G_{CS}\) for the case of laminar flow in a pipe is too high by a factor of \(\frac{3}{\sqrt2}\).

As a function of flow rate we have
\begin{equation}\label{equation:Rapid_Mix/RM_Derivations:Rapid_Mix/RM_Derivations:62}
\begin{split}\bar v=\frac{Q}{A} = \frac{4Q}{\pi D^2}\end{split}
\end{equation}\begin{equation}\label{equation:Rapid_Mix/RM_Derivations:Rapid_Mix/RM_Derivations:63}
\begin{split}G_{CS} =\frac{16\sqrt2}{\pi} \frac{Q}{D^3}\end{split}
\end{equation}

\subsection{Parallel Plates Laminar}
\label{\detokenize{Rapid_Mix/RM_Derivations:parallel-plates-laminar}}\label{\detokenize{Rapid_Mix/RM_Derivations:heading-parallel-plates-laminar}}
Flow between parallel plates occurs in plate settlers in the sedimentation tank. We will derive the velocity gradient at the wall using the Navier Stokes equation.

\begin{figure}[htbp]
\centering
\capstart

\noindent\sphinxincludegraphics[width=700\sphinxpxdimen]{{Parallel_Plate_schematic}.png}
\caption{A fluid flowing from left to right due to a pressure gradient results in wall shear on the parallel plates. This flow profile is for the case when \(\frac{dp}{dx}\) is negative.}\label{\detokenize{Rapid_Mix/RM_Derivations:id6}}\label{\detokenize{Rapid_Mix/RM_Derivations:figure-parallel-plate-schematic}}\end{figure}

We start with the Navier-Stokes equation written for flow in the x direction.
\begin{equation}\label{equation:Rapid_Mix/RM_Derivations:Rapid_Mix/RM_Derivations:64}
\begin{split}\frac{y^2}{2} \frac{dp}{dx} + Ay + B = \mu u\end{split}
\end{equation}
where \(u\) is the velocity in the x direction.

Apply the no slip condition at bottom plate.
\begin{equation}\label{equation:Rapid_Mix/RM_Derivations:Rapid_Mix/RM_Derivations:65}
\begin{split}u=0 \quad at \quad y=0\end{split}
\end{equation}
Thus the constant \(B=0\).

Apply the no slip condition at top plate.
\begin{equation}\label{equation:Rapid_Mix/RM_Derivations:Rapid_Mix/RM_Derivations:66}
\begin{split}u=0 \quad at \quad y=S\end{split}
\end{equation}
Thus the constant \(A = \frac{- S}{2} \frac{dp}{dx}\)

Substitute the values for constants \(A\) and \(B\) into the original equation.
\begin{equation}\label{equation:Rapid_Mix/RM_Derivations:Rapid_Mix/RM_Derivations:67}
\begin{split}\frac{y^2}{2} \frac{dp}{dx} - \frac{S}{2} \frac{dp}{dx} y = \mu \,u\end{split}
\end{equation}
Simply the equation to obtain
\begin{equation}\label{equation:Rapid_Mix/RM_Derivations:Rapid_Mix/RM_Derivations:68}
\begin{split}u = \frac{y \left( y - S \right)}{2 \mu} \frac{dp}{dx}\end{split}
\end{equation}
We need a relationship between average velocity and \(\frac{dp}{dx}\). We can obtain this by integrating from 0 to
\(S\).
\begin{equation}\label{equation:Rapid_Mix/RM_Derivations:Rapid_Mix/RM_Derivations:69}
\begin{split}{\bar v } = \frac{q}{S}
= \frac{1}{S}\int\limits_0^S u dy
= \frac{1}{S} \int\limits_0^S
\left(
  \frac{y^2 - S y}{2 \mu} \left( \frac{dp}{dx} \right)
\right) dy\end{split}
\end{equation}\begin{equation}\label{equation:Rapid_Mix/RM_Derivations:Rapid_Mix/RM_Derivations:70}
\begin{split}\bar v = - \frac{S^2}{12 \mu} \frac{dp}{dx}\end{split}
\end{equation}
Solving for \(\frac{dp}{dx}\)
\begin{equation}\label{equation:Rapid_Mix/RM_Derivations:Rapid_Mix/RM_Derivations:71}
\begin{split}\frac{dp}{dx} = - \frac{12 \mu \bar v}{S^2}\end{split}
\end{equation}
From the Navier Stokes equation after integrating once we get
\begin{equation}\label{equation:Rapid_Mix/RM_Derivations:Rapid_Mix/RM_Derivations:72}
\begin{split}\mu \,\left( \frac{du}{dy} \right) = y \frac{dp}{dx} + A\end{split}
\end{equation}
Substituting our boundary condition,
\(A = \frac{- S}{2} \frac{dp}{dx}\) we obtain
\begin{equation}\label{equation:Rapid_Mix/RM_Derivations:Rapid_Mix/RM_Derivations:73}
\begin{split}\frac{du}{dy}_{y = 0} = - \frac{S}{2 \mu} \frac{dp}{dx}\end{split}
\end{equation}
Substituting the result for \(\frac{dp}{dx}\) we obtain
\begin{equation}\label{equation:Rapid_Mix/RM_Derivations:Rapid_Mix/RM_Derivations:74}
\begin{split}\frac{du}{dy}_{y = 0} = \frac{6 \bar v}{S}\end{split}
\end{equation}
Therefore in velocity gradient notation we have
\begin{equation}\label{equation:Rapid_Mix/RM_Derivations:Rapid_Mix/RM_Derivations:75}
\begin{split}G_{wall} = \frac{6 \bar v}{S}\end{split}
\end{equation}
The energy dissipation rate at the wall
\begin{equation}\label{equation:Rapid_Mix/RM_Derivations:Rapid_Mix/RM_Derivations:76}
\begin{split}\varepsilon_{wall} = G_{wall}^2 \nu\end{split}
\end{equation}\begin{equation}\label{equation:Rapid_Mix/RM_Derivations:Rapid_Mix/RM_Derivations:77}
\begin{split}\varepsilon_{wall} = \left( \frac{6 \bar v}{S}\right)^2 \nu\end{split}
\end{equation}
Head loss due to shear on the plates is obtained from a force balance on a control volume between two parallel plates as shown in \hyperref[\detokenize{Rapid_Mix/RM_Derivations:figure-parallel-plate-schematic}]{Fig.\@ \ref{\detokenize{Rapid_Mix/RM_Derivations:figure-parallel-plate-schematic}}}.

A force balance on a control volume gives
\begin{equation}\label{equation:Rapid_Mix/RM_Derivations:Rapid_Mix/RM_Derivations:78}
\begin{split}2 \tau L W = -\Delta P W S\end{split}
\end{equation}\begin{equation}\label{equation:Rapid_Mix/RM_Derivations:Rapid_Mix/RM_Derivations:79}
\begin{split}\Delta P = -\frac{2 \tau L}{S}\end{split}
\end{equation}
The equation relating shear and velocity gradient is
\begin{equation}\label{equation:Rapid_Mix/RM_Derivations:Rapid_Mix/RM_Derivations:80}
\begin{split}\tau = \nu \rho \frac{du}{dy} = \nu \rho G\end{split}
\end{equation}
The velocity gradient at the wall is
\begin{equation}\label{equation:Rapid_Mix/RM_Derivations:Rapid_Mix/RM_Derivations:81}
\begin{split}G_{wall} = \frac{6 \bar v}{S}\end{split}
\end{equation}\begin{equation}\label{equation:Rapid_Mix/RM_Derivations:Rapid_Mix/RM_Derivations:82}
\begin{split}\tau  = \nu \rho \frac{6 \bar v}{S}\end{split}
\end{equation}
Substituting into the force balance equation
\begin{equation}\label{equation:Rapid_Mix/RM_Derivations:Rapid_Mix/RM_Derivations:83}
\begin{split}\Delta P = -\frac{2 \nu \rho 6 \bar v L}{S^2}\end{split}
\end{equation}
The head loss for horizontal flow at uniform velocity simplifies too
\begin{equation}\label{equation:Rapid_Mix/RM_Derivations:Rapid_Mix/RM_Derivations:84}
\begin{split}h_{{\rm f}} = \frac{-\Delta P}{\rho g}\end{split}
\end{equation}\begin{equation}\label{equation:Rapid_Mix/RM_Derivations:Rapid_Mix/RM_Derivations:85}
\begin{split}h_{{\rm f}} = 12\frac{ \nu \bar v L}{gS^2}\end{split}
\end{equation}
The average energy dissipation rate is
\begin{equation}\label{equation:Rapid_Mix/RM_Derivations:Rapid_Mix/RM_Derivations:86}
\begin{split}\bar\varepsilon = \frac{gh_{\rm{L}}}{\theta}\end{split}
\end{equation}\begin{equation}\label{equation:Rapid_Mix/RM_Derivations:Rapid_Mix/RM_Derivations:87}
\begin{split}\bar\varepsilon = 12 \nu \left(\frac{  \bar v}{S} \right)^2\end{split}
\end{equation}
The Camp-Stein velocity gradient for laminar flow between parallel plates is
\begin{equation}\label{equation:Rapid_Mix/RM_Derivations:Rapid_Mix/RM_Derivations:88}
\begin{split}G_{CS} = 2\sqrt{3}\frac{  \bar v}{S}\end{split}
\end{equation}

\subsection{Coiled tubes (laminar flow)}
\label{\detokenize{Rapid_Mix/RM_Derivations:coiled-tubes-laminar-flow}}\label{\detokenize{Rapid_Mix/RM_Derivations:heading-coiled-tubes-laminar-flow}}
Coiled tubes are used as flocculators at laboratory scale. The one shown below is a doubled coil. A single coil would only go around one cylinder

{}` \textless{}\sphinxurl{https://confluence.cornell.edu/display/AGUACLARA/Laminar+Tube+Floc?preview=/10422268/258146480/ReportLaminarTubeFlocSpring2014.pdf}\textgreater{}{}`\_\_

\begin{figure}[htbp]
\centering
\capstart

\noindent\sphinxincludegraphics[width=500\sphinxpxdimen]{{Coiled_tube_flocculator}.jpg}
\caption{The double coiled laminar flow flocculator creates secondary currents that oscillate in direction. This may be helpful in creating much more mixing than would occur in a straight laminar flow pipe.}\label{\detokenize{Rapid_Mix/RM_Derivations:id7}}\label{\detokenize{Rapid_Mix/RM_Derivations:figure-coiled-tube-flocculator}}\end{figure}

The ratio of the coiled to straight friction factors is given by \sphinxhref{https://doi.org/10.1021/i260069a017}{Mishra and Gupta}

The Dean number is defined as:
\begin{equation}\label{equation:Rapid_Mix/RM_Derivations:Rapid_Mix/RM_Derivations:89}
\begin{split}De = Re\left(\frac{D}{D_c}\right)^\frac{1}{2}\end{split}
\end{equation}
where \(D\) is the inner diameter of the tube and \(D_c\) is the diameter of the coil. Note that the tubing coils are actually helixes and that for the tubing diameters and coil diameters used for flocculators that the helix doesn’t significantly change the radius of curvature.
\begin{equation}\label{equation:Rapid_Mix/RM_Derivations:Rapid_Mix/RM_Derivations:90}
\begin{split}\frac{{\rm f}_{coil}}{{\rm f}} = 1 + 0.033\left(log_{10}De\right)^4\end{split}
\end{equation}\begin{equation}\label{equation:Rapid_Mix/RM_Derivations:Rapid_Mix/RM_Derivations:91}
\begin{split}h_{L_{coil}} = h_{{\rm f}} \left[ 1 + 0.033\left(log_{10}De\right)^4 \right]\end{split}
\end{equation}
where \(h_{{\rm f}} = \frac{32\nu L\bar v}{ g D^2}\). Note that we switch from major losses to total head loss here because the head loss from flowing around the coil is no longer simply due to shear on the
wall.
\begin{equation}\label{equation:Rapid_Mix/RM_Derivations:Rapid_Mix/RM_Derivations:92}
\begin{split}h_{L_{coil}} = \frac{32\nu L\bar v}{ g D^2} \left[ 1 + 0.033\left(log_{10}De\right)^4 \right]\end{split}
\end{equation}
The average energy dissipation rate is
\begin{equation}\label{equation:Rapid_Mix/RM_Derivations:Rapid_Mix/RM_Derivations:93}
\begin{split}\bar\varepsilon = 32\nu \left( \frac{\bar v}{D} \right)^2 \left[ 1 + 0.033\left(log_{10}De\right)^4 \right]\end{split}
\end{equation}
The average velocity gradient is proportional to the square root of the head loss and thus we obtain
\begin{equation}\label{equation:Rapid_Mix/RM_Derivations:Rapid_Mix/RM_Derivations:94}
\begin{split}G_{CS_{coil}} = G_{CS}\left[ 1 + 0.033\left(log_{10}De\right)^4  \right]^\frac{1}{2}\end{split}
\end{equation}
where \(G_{CS} =4\sqrt2 \frac{\bar v}{D}\) for laminar flow in a straight pipe.
\begin{equation}\label{equation:Rapid_Mix/RM_Derivations:Rapid_Mix/RM_Derivations:95}
\begin{split}G_{CS_{coil}} = 4\sqrt2 \frac{\bar v}{D}\left[ 1 + 0.033\left(log_{10}De\right)^4  \right]^\frac{1}{2}\end{split}
\end{equation}

\subsection{Expansions}
\label{\detokenize{Rapid_Mix/RM_Derivations:expansions}}\label{\detokenize{Rapid_Mix/RM_Derivations:heading-expansions}}
The average energy dissipation rate for a flow expansion really only has meaning if there is a defined control volume where the mechanical energy is lost. Hydraulic flocculators provide such a case because the same flow expansion is repeated and thus the mechanical energy loss can be assumed to happen in the volume associated with one flow expansion. In this case we have
\begin{equation}\label{equation:Rapid_Mix/RM_Derivations:Rapid_Mix/RM_Derivations:96}
\begin{split}h_e =  K\frac{\bar v_{out}^2}{2g}\end{split}
\end{equation}
In this equation \(K\) represents the fraction of the kinetic energy that is dissipated.

If we define the length of the control volume (in the direction of flow) as \(H\) then the residence time is
\begin{equation}\label{equation:Rapid_Mix/RM_Derivations:Rapid_Mix/RM_Derivations:97}
\begin{split}\theta = \frac{H}{\bar v}\end{split}
\end{equation}\begin{equation}\label{equation:Rapid_Mix/RM_Derivations:Rapid_Mix/RM_Derivations:98}
\begin{split}\bar\varepsilon = \frac{gh_{\rm{e}}}{\theta}\end{split}
\end{equation}
Combining the previous equations we obtain
\begin{equation}\label{equation:Rapid_Mix/RM_Derivations:Rapid_Mix/RM_Derivations:99}
\begin{split}\bar\varepsilon = K\frac{\bar v_{out}^3}{2H}\end{split}
\end{equation}\begin{equation}\label{equation:Rapid_Mix/RM_Derivations:Rapid_Mix/RM_Derivations:100}
\begin{split}G_{CS} = \sqrt{\frac{\bar \varepsilon}{\nu}}\end{split}
\end{equation}\begin{equation}\label{equation:Rapid_Mix/RM_Derivations:Rapid_Mix/RM_Derivations:101}
\begin{split}G_{CS} = \bar v_{out}\sqrt{\frac{K\bar v_{out}}{2H\nu}}\end{split}
\end{equation}

\section{Maximum velocity gradients}
\label{\detokenize{Rapid_Mix/RM_Derivations:maximum-velocity-gradients}}\label{\detokenize{Rapid_Mix/RM_Derivations:heading-maximum-velocity-gradients}}

\subsection{Straight pipe (major losses)}
\label{\detokenize{Rapid_Mix/RM_Derivations:straight-pipe-major-losses}}\label{\detokenize{Rapid_Mix/RM_Derivations:heading-straight-pipe-major-losses}}
The maximum velocity gradient in pipe flow occurs at the wall. This is true for both laminar and turbulent flow. In either case a force balance on a control volume of pipe gives us the wall shear and the wall shear can then be used to estimate the velocity gradient at the wall.

\begin{figure}[htbp]
\centering
\capstart

\noindent\sphinxincludegraphics[width=400\sphinxpxdimen]{{pipe_pressure_shear_force_balance}.png}
\caption{A fluid flowing from left to right due to a pressure gradient results in wall shear.}\label{\detokenize{Rapid_Mix/RM_Derivations:id8}}\label{\detokenize{Rapid_Mix/RM_Derivations:figure-pipe-pressure-shear-force-balance}}\end{figure}

A force balance for the case of steady flow in a round pipe requires that sum of the forces in the x direction must equal zero. Given a pipe with diameter, D, and length, L, we obtain
\begin{equation}\label{equation:Rapid_Mix/RM_Derivations:Rapid_Mix/RM_Derivations:102}
\begin{split}\left(P_{in}- P_{out}\right)\frac{\pi D^2}{4} = \tau_{wall} \pi D L\end{split}
\end{equation}\begin{equation}\label{equation:Rapid_Mix/RM_Derivations:Rapid_Mix/RM_Derivations:103}
\begin{split}-\Delta P\frac{D}{4} = \tau_{wall} L\end{split}
\end{equation}
For this control volume the energy equation simplifies to
\begin{equation}\label{equation:Rapid_Mix/RM_Derivations:Rapid_Mix/RM_Derivations:104}
\begin{split}-\Delta P=\rho g h_{{\rm f}}\end{split}
\end{equation}
The relationship between shear and velocity gradient is
\begin{equation}\label{equation:Rapid_Mix/RM_Derivations:Rapid_Mix/RM_Derivations:105}
\begin{split}\tau_{wall} = \mu \frac{du}{dy}_{wall} = \nu \rho G_{wall}\end{split}
\end{equation}
Combining the energy equation, the force balance, and the relationship between shear and velocity gradient we obtain
\begin{equation}\label{equation:Rapid_Mix/RM_Derivations:Rapid_Mix/RM_Derivations:106}
\begin{split}\rho g h_{{\rm f}}\frac{D}{4} = \nu \rho G_{wall} L\end{split}
\end{equation}\begin{equation}\label{equation:Rapid_Mix/RM_Derivations:Rapid_Mix/RM_Derivations:107}
\begin{split}G_{wall} = \frac{g h_{{\rm f}}D}{4\nu L}\end{split}
\end{equation}
This equation is valid for both laminar flow. For turbulent flow it is necessary to make the approximation that wall shear perpendicular to the direction of flow is insignificant in increasing the magnitude of the wall shear. We can substitute the Darcy Weisbach equation for head loss to obtain
\begin{equation}\label{equation:Rapid_Mix/RM_Derivations:Rapid_Mix/RM_Derivations:108}
\begin{split}G_{wall} ={\rm f}  \frac{\bar v^2}{8\nu}\end{split}
\end{equation}
The energy dissipation rate at the wall is
\begin{equation}\label{equation:Rapid_Mix/RM_Derivations:Rapid_Mix/RM_Derivations:109}
\begin{split}\varepsilon_{wall} = G_{wall}^2 \nu\end{split}
\end{equation}\begin{equation}\label{equation:Rapid_Mix/RM_Derivations:Rapid_Mix/RM_Derivations:110}
\begin{split}\varepsilon_{wall} = \frac{1}{\nu}\left({\rm f}  \frac{\bar v^2}{8} \right)^2\end{split}
\end{equation}
For laminar flow we can substitute \({\rm f} = \frac{64}{{\rm Re}}\) and the definition of the Reynolds number to obtain
\begin{equation}\label{equation:Rapid_Mix/RM_Derivations:Rapid_Mix/RM_Derivations:111}
\begin{split}G_{wall} =  \frac{8\bar v}{D}\end{split}
\end{equation}
This equation is useful for finding the velocity gradient at the wall of a tube settler.

The energy dissipation rate at the wall is
\begin{equation}\label{equation:Rapid_Mix/RM_Derivations:Rapid_Mix/RM_Derivations:112}
\begin{split}\varepsilon_{wall} = G_{wall}^2 \nu\end{split}
\end{equation}\begin{equation}\label{equation:Rapid_Mix/RM_Derivations:Rapid_Mix/RM_Derivations:113}
\begin{split}\varepsilon_{wall} = \left(\frac{8\bar v}{D} \right)^2 \nu\end{split}
\end{equation}

\subsection{Coiled tubes (laminar flow)}
\label{\detokenize{Rapid_Mix/RM_Derivations:heading-coiled-tubes-laminar-flow-1}}\label{\detokenize{Rapid_Mix/RM_Derivations:id1}}
The shear on the wall of a coiled tube is not uniform. The outside of the curve has a higher velocity gradient than the inside of the curve and there are secondary currents that results in wall shear that is not purely in the locally defined upstream direction. We do not have a precise equation for the wall shear. The best we can do currently is define an average wall shear in the locally defined direction of flow by combining
\(G_{{CS}_{wall_{coil}}} =\rm{f_{coil}} \frac{\bar v^2}{8\nu}\) and
\({\rm f}_{coil} = {\rm f} \left[ 1 + 0.033\left(log_{10}De\right)^4 \right]\)
to obtain
\begin{equation}\label{equation:Rapid_Mix/RM_Derivations:Rapid_Mix/RM_Derivations:114}
\begin{split}G_{{CS}_{wall_{coil}}} ={\rm f} \left[ 1 + 0.033 \left(log_{10}De \right)^4 \right]  \frac{\bar v^2}{8\nu}\end{split}
\end{equation}

\subsection{Expansions}
\label{\detokenize{Rapid_Mix/RM_Derivations:heading-expansions-1}}\label{\detokenize{Rapid_Mix/RM_Derivations:id2}}
Flow expansions are used intentionally or unavoidable in multiple locations in hydraulically optimized water treatment plants. Rapid mix and hydraulic flocculation use flow expansions to generate fluid mixing and collisions between particles.


\subsection{Round Jet}
\label{\detokenize{Rapid_Mix/RM_Derivations:round-jet}}\label{\detokenize{Rapid_Mix/RM_Derivations:heading-round-jet}}
\sphinxhref{https://doi.org/10.1016/0009-2509(95)00049-B}{Baldyga, et al. 1995}
\begin{equation}\label{equation:Rapid_Mix/RM_Derivations:Rapid_Mix/RM_Derivations:115}
\begin{split}\varepsilon_{Centerline} = \frac{50 D_{Jet}^3 \bar v_{Jet}^3}{ \left( x - 2 D_{Jet} \right)^4}\end{split}
\end{equation}\begin{equation}\label{equation:Rapid_Mix/RM_Derivations:Rapid_Mix/RM_Derivations:116}
\begin{split}\varepsilon_{Max} = \frac{\left( \frac{50}{\left( 5 \right)^4} \right) \bar v_{Jet}^3}{D_{Jet}}\end{split}
\end{equation}\begin{equation}\label{equation:Rapid_Mix/RM_Derivations:Rapid_Mix/RM_Derivations:117}
\begin{split}\varepsilon_{Max} = \Pi_{JetRound} \frac{\bar v_{Jet} ^3}{D_{Jet}}\end{split}
\end{equation}\begin{equation}\label{equation:Rapid_Mix/RM_Derivations:Rapid_Mix/RM_Derivations:118}
\begin{split}\Pi_{JetRound} = 0.08\end{split}
\end{equation}
The maximum velocity gradient in a jet is thus
\begin{equation}\label{equation:Rapid_Mix/RM_Derivations:Rapid_Mix/RM_Derivations:119}
\begin{split}G_{Max} = \bar v_{Jet} \sqrt{\frac{\Pi_{JetRound} \bar v_{Jet} }{\nu D_{Jet}}}\end{split}
\end{equation}
Below we plot the Baldyga et al. equation for the energy dissipation rate as a function of distance from the discharge location for the case of a round jet that is discharging into a large tank.

\begin{figure}[htbp]
\centering
\capstart

\noindent\sphinxincludegraphics[width=400\sphinxpxdimen]{{Jet_centerline_EDR}.png}
\caption{The centerline energy dissipation rate downstream from a round jet. The distance downstream is measured in units of jet diameters. The energy dissipation rate between the jet and 7 jet diameters is developing as the shear between the stationary fluid and the jet propagates toward the center of the jet and turbulence is generated.}\label{\detokenize{Rapid_Mix/RM_Derivations:id9}}\label{\detokenize{Rapid_Mix/RM_Derivations:figure-jet-centerline-edr}}\end{figure}


\subsection{Plane Jet}
\label{\detokenize{Rapid_Mix/RM_Derivations:plane-jet}}\label{\detokenize{Rapid_Mix/RM_Derivations:heading-plane-jet}}
Plane jets occur in hydraulic flocculators and in the sedimentation tank inlet jet system. We haven’t been able to find a literature estimate of the maximum energy dissipation rate in a plane jet. Original measurements of a plane turbulent jet have been made by \sphinxhref{http://dx.doi.org/10.1115/1.3627309}{Heskestad in 1965} and it may be possible to use that data to get a better estimate of \$:raw-latex:\sphinxtitleref{Pi}\_\{JetPlane\} \$ from that source.
\begin{equation}\label{equation:Rapid_Mix/RM_Derivations:Rapid_Mix/RM_Derivations:120}
\begin{split}\Pi_{\bar \epsilon}^{\epsilon_{Max}} = \frac{\varepsilon_{Max}}{\bar \varepsilon}\end{split}
\end{equation}\begin{equation}\label{equation:Rapid_Mix/RM_Derivations:Rapid_Mix/RM_Derivations:121}
\begin{split}\varepsilon_{Max} = \Pi_{JetPlane}  \frac{  \bar v_{Jet} ^3}{S_{Jet}}\end{split}
\end{equation}
The maximum velocity gradient is thus
\begin{equation}\label{equation:Rapid_Mix/RM_Derivations:Rapid_Mix/RM_Derivations:122}
\begin{split}G_{Max} = \bar v_{Jet}\sqrt{\frac{\Pi_{JetPlane} \bar v_{Jet}}{\nu S_{Jet}}}\end{split}
\end{equation}\begin{equation}\label{equation:Rapid_Mix/RM_Derivations:Rapid_Mix/RM_Derivations:123}
\begin{split}\bar v = \frac{Q}{SW}\end{split}
\end{equation}\begin{equation}\label{equation:Rapid_Mix/RM_Derivations:Rapid_Mix/RM_Derivations:124}
\begin{split}\bar v_{Jet} = \frac{\bar v}{\Pi_{VCBaffle}}\end{split}
\end{equation}\begin{equation}\label{equation:Rapid_Mix/RM_Derivations:Rapid_Mix/RM_Derivations:125}
\begin{split}S_{Jet} = S \Pi_{VCBaffle}\end{split}
\end{equation}
The average hydraulic residence time for the fluid between two baffles
is
\begin{equation}\label{equation:Rapid_Mix/RM_Derivations:Rapid_Mix/RM_Derivations:126}
\begin{split}\theta_B = \frac{H}{\bar v}\end{split}
\end{equation}
where \(H\) is the depth of water. Substituting into the equation for \(\varepsilon_{Max}\) to get the equation in terms of the average velocity \(\bar v\) and flow dimension \(S\)
\begin{equation}\label{equation:Rapid_Mix/RM_Derivations:Rapid_Mix/RM_Derivations:127}
\begin{split}\varepsilon_{Max}= \frac{\Pi_{JetPlane}}{S \Pi_{VCBaffle}} \left( \frac{ \bar v}{\Pi_{VCBaffle}} \right)^3\end{split}
\end{equation}
From the control volume analysis the average energy dissipation rate is
\begin{equation}\label{equation:Rapid_Mix/RM_Derivations:Rapid_Mix/RM_Derivations:128}
\begin{split}\bar \varepsilon = K \frac{\bar v^2}{2} \frac{1}{\theta_B} = \frac{K}{2} \frac{\bar v^3}{H_e}\end{split}
\end{equation}
where \(K\) is the minor loss coefficient for flow around the end of a baffle with a \(180^\circ\) turn.

Substitute the values for \(\bar \varepsilon\) and
\(\varepsilon_{Max}\) to obtain the ratio,
\(\Pi_{\bar \epsilon}^{\epsilon_{Max}}\)
\begin{equation}\label{equation:Rapid_Mix/RM_Derivations:Rapid_Mix/RM_Derivations:129}
\begin{split}\Pi_{\bar \epsilon}^{\epsilon_{Max}} = \frac{\Pi_{JetPlane}}{\Pi_{VCBaffle}^4} \frac{2 H_e}{K S}\end{split}
\end{equation}
\(\Pi_{\bar \varepsilon}^{\varepsilon_{Max}}\) has a value of 2 for
\(H_e/S <5\) (CFD analysis and \sphinxhref{https://search-proquest-com.proxy.library.cornell.edu/docview/1943098053?accountid=10267}{Haarhoff, 2001})
The transition value for \(H_e/S\) is at 5 (from CFD analysis, our weakest assumption).

We also have that \(\Pi_{\bar \varepsilon}^{\varepsilon_{Max}}\) has a value of
\(\frac{\Pi_{JetPlane}}{\Pi_{VCBaffle}^4} \frac{2 H_e}{K S}\) for
\(H_e/S>5\). Thus we can solve for \(\Pi_{JetPlane}\) at
\(H_e/S=5\)
\begin{equation}\label{equation:Rapid_Mix/RM_Derivations:Rapid_Mix/RM_Derivations:130}
\begin{split}\Pi_{JetPlane} = \left(
  \Pi_{\bar \epsilon}^{\epsilon_{Max}} \Pi_{VCBaffle}^4 \frac{K}{2} \frac{S}{H_e}
  \right)\end{split}
\end{equation}\begin{equation}\label{equation:Rapid_Mix/RM_Derivations:Rapid_Mix/RM_Derivations:131}
\begin{split}\Pi_{JetPlane} = 0.0124\end{split}
\end{equation}
\fvset{hllines={, ,}}%
\begin{sphinxVerbatim}[commandchars=\\\{\}]
\PYG{n}{x}\PYG{o}{=}\PYG{n}{con}\PYG{o}{.}\PYG{n}{RATIO\PYGZus{}VC\PYGZus{}ORIFICE}\PYG{o}{*}\PYG{o}{*}\PYG{l+m+mi}{2}
\PYG{n}{Ratio\PYGZus{}Jet\PYGZus{}Plane} \PYG{o}{=} \PYG{l+m+mi}{2}\PYG{o}{*}\PYG{n}{con}\PYG{o}{.}\PYG{n}{RATIO\PYGZus{}VC\PYGZus{}ORIFICE}\PYG{o}{*}\PYG{o}{*}\PYG{l+m+mi}{8} \PYG{o}{*} \PYG{n}{con}\PYG{o}{.}\PYG{n}{K\PYGZus{}MINOR\PYGZus{}FLOC\PYGZus{}BAFFLE}\PYG{o}{/}\PYG{l+m+mi}{2}\PYG{o}{/}\PYG{l+m+mi}{5}
\PYG{n}{Ratio\PYGZus{}Jet\PYGZus{}Plane}

\PYG{n}{con}\PYG{o}{.}\PYG{n}{RATIO\PYGZus{}VC\PYGZus{}ORIFICE}\PYG{o}{*}\PYG{o}{*}\PYG{l+m+mi}{8}\PYG{o}{*}\PYG{n}{con}\PYG{o}{.}\PYG{n}{K\PYGZus{}MINOR\PYGZus{}FLOC\PYGZus{}BAFFLE}\PYG{o}{/}\PYG{n}{Ratio\PYGZus{}Jet\PYGZus{}Plane}
\end{sphinxVerbatim}


\subsection{Behind a flat plate}
\label{\detokenize{Rapid_Mix/RM_Derivations:behind-a-flat-plate}}\label{\detokenize{Rapid_Mix/RM_Derivations:heading-behind-a-flat-plate}}
A flat plate normal to the direction of flow could be used in a hydraulic flocculator. In vertical flow flocculators it would create a space where flocs can settle and thus it is not a recommended design.

The impellers used in mechanical flocculators could be modeled as a rotating flat plate. The energy dissipation rate in the wake behind the flat plate is often quite high in mechanical flocculators and this may be responsible for breaking previously formed flocs.

Ariane Walker-Horn modeled the flat plate using Fluent in 2015.

\begin{figure}[htbp]
\centering
\capstart

\noindent\sphinxincludegraphics[width=600\sphinxpxdimen]{{CFD_Flat_Plate}.png}
\caption{The energy dissipation rate and streamlines for a 1 m wide plate in two dimensional flow with an approach velocity of \(1 m/s\). The maximum energy dissipation rate was approximately \(0.04 W/kg\).}\label{\detokenize{Rapid_Mix/RM_Derivations:id10}}\label{\detokenize{Rapid_Mix/RM_Derivations:figure-cfd-flat-plate}}\end{figure}
\begin{equation}\label{equation:Rapid_Mix/RM_Derivations:Rapid_Mix/RM_Derivations:132}
\begin{split}\varepsilon _{Max} = \Pi_{Plate}\frac{\bar v^3}{W_{Plate}}\end{split}
\end{equation}
The maximum velocity gradient is thus
\begin{equation}\label{equation:Rapid_Mix/RM_Derivations:Rapid_Mix/RM_Derivations:133}
\begin{split}G_{Max} = \bar v\sqrt{\frac{\Pi_{Plate} \bar v}{\nu W_{Plate}}}\end{split}
\end{equation}\begin{equation}\label{equation:Rapid_Mix/RM_Derivations:Rapid_Mix/RM_Derivations:134}
\begin{split}\Pi_{Plate} = \frac{ \left( \varepsilon_{Max} W_{Plate} \right)}{\bar v^3}\end{split}
\end{equation}
\fvset{hllines={, ,}}%
\begin{sphinxVerbatim}[commandchars=\\\{\}]
\PYG{l+s+sd}{\PYGZdq{}\PYGZdq{}\PYGZdq{}CFD analysis setup used by Ariane Walker\PYGZhy{}Horn in 2015\PYGZdq{}\PYGZdq{}\PYGZdq{}}
\PYG{n}{EDR\PYGZus{}Max} \PYG{o}{=} \PYG{l+m+mf}{0.04}\PYG{o}{*}\PYG{n}{u}\PYG{o}{.}\PYG{n}{W}\PYG{o}{/}\PYG{n}{u}\PYG{o}{.}\PYG{n}{kg}
\PYG{n}{v} \PYG{o}{=} \PYG{l+m+mi}{1}\PYG{o}{*}\PYG{n}{u}\PYG{o}{.}\PYG{n}{m}\PYG{o}{/}\PYG{n}{u}\PYG{o}{.}\PYG{n}{s}
\PYG{n}{W} \PYG{o}{=} \PYG{l+m+mi}{1}\PYG{o}{*}\PYG{n}{u}\PYG{o}{.}\PYG{n}{m}
\PYG{n}{Ratio\PYGZus{}Jet\PYGZus{}Plate} \PYG{o}{=} \PYG{p}{(}\PYG{n}{EDR\PYGZus{}Max} \PYG{o}{*} \PYG{n}{W}\PYG{o}{/}\PYG{n}{v}\PYG{o}{*}\PYG{o}{*}\PYG{l+m+mi}{3}\PYG{p}{)}\PYG{o}{.}\PYG{n}{to\PYGZus{}base\PYGZus{}units}\PYG{p}{(}\PYG{p}{)}
\PYG{n+nb}{print}\PYG{p}{(}\PYG{n}{Ratio\PYGZus{}Jet\PYGZus{}Plate}\PYG{p}{)}
\end{sphinxVerbatim}


\chapter{Rapid Mix Appendix C: Examples}
\label{\detokenize{Rapid_Mix/RM_Examples:rapid-mix-appendix-c-examples}}\label{\detokenize{Rapid_Mix/RM_Examples:title-rapid-mix-examples}}\label{\detokenize{Rapid_Mix/RM_Examples::doc}}

\section{Example: pH Adjustment}
\label{\detokenize{Rapid_Mix/RM_Examples:example-ph-adjustment}}\label{\detokenize{Rapid_Mix/RM_Examples:heading-example-ph-adjustment}}
Find the required dose of several bases to raise the pH at the Manzaragua Water Treatment Plant
The Mazaragua AguaClara plant consists of two 1 L/s plants operating in parallel. The plant is located in the municipality of Guinope, the department of El Paraiso, Honduras.

\begin{figure}[htbp]
\centering
\capstart

\noindent\sphinxincludegraphics[width=700\sphinxpxdimen]{{Manzaragua_WTP}.jpg}
\caption{Manzaragua water treatment plant using two of the AguaClara 1 L/s plants in parallel.}\label{\detokenize{Rapid_Mix/RM_Examples:id1}}\label{\detokenize{Rapid_Mix/RM_Examples:manzaragua-wtp}}\end{figure}

The plant performed very poorly from the first day of operation. The first attempted fix was to double the flocculator residence time by increasing the number of flocculator pipes (3 inch diameter by 1.5 m long) from 12 to 24. This improved performance, but the plant continued to perform poorly. A raw water sample was analyzed on May 30, 2018 and the following results were obtained.

\begin{figure}[htbp]
\centering
\capstart

\noindent\sphinxincludegraphics[width=700\sphinxpxdimen]{{Manzaragua_Water_Analysis}.jpg}
\caption{Water quality analysis for Manzaragua.}\label{\detokenize{Rapid_Mix/RM_Examples:id2}}\label{\detokenize{Rapid_Mix/RM_Examples:figure-manzaragua-water-analysis}}\end{figure}


\begin{savenotes}\sphinxattablestart
\centering
\sphinxcapstartof{table}
\sphinxcaption{Manzaragua water quality analysis}\label{\detokenize{Rapid_Mix/RM_Examples:id3}}\label{\detokenize{Rapid_Mix/RM_Examples:table-manzaragua-water-quality-analysis}}
\sphinxaftercaption
\begin{tabular}[t]{|\X{20}{80}|\X{20}{80}|\X{20}{80}|\X{20}{80}|}
\hline
\sphinxstyletheadfamily 
Parameter
&\sphinxstyletheadfamily 
Units
&\sphinxstyletheadfamily 
Standard
&\sphinxstyletheadfamily 
Results
\\
\hline
Turbidity
&
NTU
&
5
&
71
\\
\hline
Color
&
color units
&
15
&
150
\\
\hline
pH
&
pH
&
6.5 - 8.5
&
5.91
\\
\hline
Conductivity
&
\(\mu s/cm\)
&
400
&
69.15
\\
\hline
Alkalinity
&
\(mg/L\) as \(CaCO_3\)
&\begin{itemize}
\item {} 
\end{itemize}
&
24.5
\\
\hline
Bicarbonates
&
\(mg/L\) as \(CaCO_3\)
&\begin{itemize}
\item {} 
\end{itemize}
&
24.5
\\
\hline
Carbonates
&
\(mg/L\) as \(CaCO_3\)
&\begin{itemize}
\item {} 
\end{itemize}
&
0
\\
\hline
Hardness
&
\(mg/L\) as \(CaCO_3\)
&
400
&
15.68
\\
\hline
\end{tabular}
\par
\sphinxattableend\end{savenotes}

This water has high color which suggests a high concentration of dissolved organic matter. The pH is a clear problem because the pH is too low for the coagulant nanoparticles to precipitate. As the water sample pH of 5.91 a significant fraction of the coagulant will remain soluble.

Our goal is to determine how much base will need to be added to raise the pH. We do not have data on the \sphinxstyleemphasis{optimal} pH for treating high color water with PACl and so we will use pH 7 as the target.

At circumneutral pH (pH close to 7) the buffering capacity of the water is dominated by carbonate chemistry and specifically by the equilibrium between \({H_2}CO_3^{\star}\) and \(HCO_3^-\) . We will use the acid neutralizing capacity (reported as calcium carbonate alkalinity) and the pH from the sample analysis to estimate the total concentration of carbonates. We will not use the sample analysis carbonate concentrations because they can not be precisely correct.

We will find the amount of base that must be added using \eqref{equation:Rapid_Mix/RM_Derivations:Base_for_pH_Adjust}.


\begin{savenotes}\sphinxattablestart
\centering
\sphinxcapstartof{table}
\sphinxcaption{ANC and carbonate values for several bases and acids}\label{\detokenize{Rapid_Mix/RM_Examples:id4}}\label{\detokenize{Rapid_Mix/RM_Examples:table-anc-and-carbonate-values-for-several-bases-and-acids}}
\sphinxaftercaption
\begin{tabular}[t]{|\X{20}{60}|\X{20}{60}|\X{20}{60}|}
\hline
\sphinxstyletheadfamily 
Base/Acid
&
\(Pi_{ANC}\)
&
\(Pi_{CO_3^{-2}}\)
\\
\hline
\(Na_2CO_3\) or \(CaCO_3\)
&
2
&
1
\\
\hline
\(NaHCO_3\)
&
1
&
1
\\
\hline
\(NaOH\)
&
1
&
0
\\
\hline
\(HCl\) or \(HNO_3\)
&
-1
&
0
\\
\hline
\(H_2SO_4\)
&
-2
&
0
\\
\hline
\end{tabular}
\par
\sphinxattableend\end{savenotes}

For \(Na_2CO_3\) * \(\Pi_{ANC}\) = 2 because we are adding
\(CO_3^{-2}\) which is multiplied by two in the ANC equation because
\(CO_3^{-2}\) can react with two protons. * \(\Pi_{CO_3^{-2}}\)
= 1 because there is one mole of \(CO_3\) per mole of
\(Na_2CO_3\)

Below is the code used to calculate the required base addition.

\fvset{hllines={, ,}}%
\begin{sphinxVerbatim}[commandchars=\\\{\}]
\PYG{k+kn}{from} \PYG{n+nn}{aide\PYGZus{}design}\PYG{n+nn}{.}\PYG{n+nn}{play} \PYG{k}{import}\PYG{o}{*}
\PYG{k+kn}{from} \PYG{n+nn}{aguaclara\PYGZus{}research}\PYG{n+nn}{.}\PYG{n+nn}{play} \PYG{k}{import}\PYG{o}{*}
\PYG{k+kn}{import} \PYG{n+nn}{aguaclara\PYGZus{}research}\PYG{n+nn}{.}\PYG{n+nn}{Environmental\PYGZus{}Processes\PYGZus{}Analysis} \PYG{k}{as} \PYG{n+nn}{epa}

\PYG{l+s+sd}{\PYGZdq{}\PYGZdq{}\PYGZdq{}define molecular weights\PYGZdq{}\PYGZdq{}\PYGZdq{}}
\PYG{n}{m\PYGZus{}Ca} \PYG{o}{=} \PYG{l+m+mf}{40.078}\PYG{o}{*}\PYG{n}{u}\PYG{o}{.}\PYG{n}{g}\PYG{o}{/}\PYG{n}{u}\PYG{o}{.}\PYG{n}{mol}
\PYG{n}{m\PYGZus{}C} \PYG{o}{=} \PYG{l+m+mf}{12.011}\PYG{o}{*}\PYG{n}{u}\PYG{o}{.}\PYG{n}{g}\PYG{o}{/}\PYG{n}{u}\PYG{o}{.}\PYG{n}{mol}
\PYG{n}{m\PYGZus{}O} \PYG{o}{=} \PYG{l+m+mf}{15.999}\PYG{o}{*}\PYG{n}{u}\PYG{o}{.}\PYG{n}{g}\PYG{o}{/}\PYG{n}{u}\PYG{o}{.}\PYG{n}{mol}
\PYG{n}{m\PYGZus{}Na} \PYG{o}{=} \PYG{l+m+mf}{22.99}\PYG{o}{*}\PYG{n}{u}\PYG{o}{.}\PYG{n}{g}\PYG{o}{/}\PYG{n}{u}\PYG{o}{.}\PYG{n}{mol}
\PYG{n}{m\PYGZus{}H} \PYG{o}{=} \PYG{l+m+mf}{1.008}\PYG{o}{*}\PYG{n}{u}\PYG{o}{.}\PYG{n}{g}\PYG{o}{/}\PYG{n}{u}\PYG{o}{.}\PYG{n}{mol}
\PYG{n}{m\PYGZus{}CaCO3} \PYG{o}{=} \PYG{n}{m\PYGZus{}Ca}\PYG{o}{+}\PYG{n}{m\PYGZus{}C}\PYG{o}{+}\PYG{l+m+mi}{3}\PYG{o}{*}\PYG{n}{m\PYGZus{}O}
\PYG{n}{m\PYGZus{}Na2CO3} \PYG{o}{=} \PYG{l+m+mi}{2}\PYG{o}{*}\PYG{n}{m\PYGZus{}Na}\PYG{o}{+}\PYG{n}{m\PYGZus{}C}\PYG{o}{+}\PYG{l+m+mi}{3}\PYG{o}{*}\PYG{n}{m\PYGZus{}O}
\PYG{n}{m\PYGZus{}NaHCO3} \PYG{o}{=} \PYG{n}{m\PYGZus{}Na}\PYG{o}{+}\PYG{n}{m\PYGZus{}H}\PYG{o}{+}\PYG{n}{m\PYGZus{}C}\PYG{o}{+}\PYG{l+m+mi}{3}\PYG{o}{*}\PYG{n}{m\PYGZus{}O}
\PYG{n}{m\PYGZus{}NaOH} \PYG{o}{=} \PYG{n}{m\PYGZus{}Na}\PYG{o}{+}\PYG{n}{m\PYGZus{}O}\PYG{o}{+}\PYG{n}{m\PYGZus{}H}

\PYG{l+s+sd}{\PYGZdq{}\PYGZdq{}\PYGZdq{}Raw water characteristics\PYGZdq{}\PYGZdq{}\PYGZdq{}}
\PYG{n}{pH\PYGZus{}0} \PYG{o}{=} \PYG{l+m+mf}{5.91}
\PYG{n}{ANC\PYGZus{}0} \PYG{o}{=} \PYG{p}{(}\PYG{l+m+mf}{24.5} \PYG{o}{*} \PYG{n}{u}\PYG{o}{.}\PYG{n}{mg}\PYG{o}{/}\PYG{n}{u}\PYG{o}{.}\PYG{n}{L}\PYG{o}{/}\PYG{n}{m\PYGZus{}CaCO3}\PYG{p}{)}\PYG{o}{.}\PYG{n}{to}\PYG{p}{(}\PYG{n}{u}\PYG{o}{.}\PYG{n}{mmol}\PYG{o}{/}\PYG{n}{u}\PYG{o}{.}\PYG{n}{L}\PYG{p}{)}
\PYG{n}{ANC\PYGZus{}0}

\PYG{k}{def} \PYG{n+nf}{total\PYGZus{}carbonates\PYGZus{}closed}\PYG{p}{(}\PYG{n}{pH}\PYG{p}{,} \PYG{n}{ANC}\PYG{p}{)}\PYG{p}{:}
    \PYG{l+s+sd}{\PYGZdq{}\PYGZdq{}\PYGZdq{}This function calculates total carbonates for a closed system given pH and ANC}

\PYG{l+s+sd}{    Parameters}
\PYG{l+s+sd}{    \PYGZhy{}\PYGZhy{}\PYGZhy{}\PYGZhy{}\PYGZhy{}\PYGZhy{}\PYGZhy{}\PYGZhy{}\PYGZhy{}\PYGZhy{}}
\PYG{l+s+sd}{    pH : float}
\PYG{l+s+sd}{        pH of the sample}
\PYG{l+s+sd}{    ANC: float}
\PYG{l+s+sd}{        acid neutralizing capacity of the sample}
\PYG{l+s+sd}{    Returns}
\PYG{l+s+sd}{    \PYGZhy{}\PYGZhy{}\PYGZhy{}\PYGZhy{}\PYGZhy{}\PYGZhy{}\PYGZhy{}}
\PYG{l+s+sd}{    The total carbonates of the sample}
\PYG{l+s+sd}{    Examples}
\PYG{l+s+sd}{    \PYGZhy{}\PYGZhy{}\PYGZhy{}\PYGZhy{}\PYGZhy{}\PYGZhy{}\PYGZhy{}\PYGZhy{}}
\PYG{l+s+sd}{    \PYGZgt{}\PYGZgt{}\PYGZgt{} total\PYGZus{}carbonates\PYGZus{}closed(1*u.mmol/u.L,8)}
\PYG{l+s+sd}{    1.017 mole/liter}
\PYG{l+s+sd}{    \PYGZdq{}\PYGZdq{}\PYGZdq{}}
    \PYG{k}{return} \PYG{p}{(}\PYG{n}{ANC} \PYG{o}{\PYGZhy{}} \PYG{n}{epa}\PYG{o}{.}\PYG{n}{Kw}\PYG{o}{/}\PYG{n}{epa}\PYG{o}{.}\PYG{n}{invpH}\PYG{p}{(}\PYG{n}{pH}\PYG{p}{)} \PYG{o}{+} \PYG{n}{epa}\PYG{o}{.}\PYG{n}{invpH}\PYG{p}{(}\PYG{n}{pH}\PYG{p}{)}\PYG{p}{)} \PYG{o}{/} \PYG{p}{(}\PYG{n}{epa}\PYG{o}{.}\PYG{n}{alpha1\PYGZus{}carbonate}\PYG{p}{(}\PYG{n}{pH}\PYG{p}{)} \PYG{o}{+} \PYG{l+m+mi}{2} \PYG{o}{*} \PYG{n}{epa}\PYG{o}{.}\PYG{n}{alpha2\PYGZus{}carbonate}\PYG{p}{(}\PYG{n}{pH}\PYG{p}{)}\PYG{p}{)}


\PYG{n}{CT\PYGZus{}0} \PYG{o}{=} \PYG{n}{total\PYGZus{}carbonates\PYGZus{}closed}\PYG{p}{(}\PYG{n}{pH\PYGZus{}0}\PYG{p}{,}\PYG{n}{ANC\PYGZus{}0}\PYG{p}{)}


\PYG{l+s+sd}{\PYGZdq{}\PYGZdq{}\PYGZdq{} calculate the amount of base that must be added to reach a target pH\PYGZdq{}\PYGZdq{}\PYGZdq{}}

\PYG{k}{def} \PYG{n+nf}{pH\PYGZus{}adjust}\PYG{p}{(}\PYG{n}{pH\PYGZus{}0}\PYG{p}{,}\PYG{n}{ANC\PYGZus{}0}\PYG{p}{,}\PYG{n}{Pi\PYGZus{}base}\PYG{p}{,}\PYG{n}{Pi\PYGZus{}CO3}\PYG{p}{,}\PYG{n}{Pi\PYGZus{}Al}\PYG{p}{,}\PYG{n}{C\PYGZus{}Al}\PYG{p}{,}\PYG{n}{pH\PYGZus{}target}\PYG{p}{)}\PYG{p}{:}
  \PYG{l+s+sd}{\PYGZdq{}\PYGZdq{}\PYGZdq{}This function calculates the required base (or acid) to adjust the pH to a target value. The buffering capacity is assumed to be completely due to carbonate species. The initial carbonate concentration is calculated based on the initial pH and the initial ANC.}

\PYG{l+s+sd}{  Parameters}
\PYG{l+s+sd}{  \PYGZhy{}\PYGZhy{}\PYGZhy{}\PYGZhy{}\PYGZhy{}\PYGZhy{}\PYGZhy{}\PYGZhy{}\PYGZhy{}\PYGZhy{}}
\PYG{l+s+sd}{  pH\PYGZus{}0: float}
\PYG{l+s+sd}{      pH of the sample}
\PYG{l+s+sd}{  ANC\PYGZus{}0: float}
\PYG{l+s+sd}{      acid neutralizing capacity (Alkalinity) of the sample in eq/L.}
\PYG{l+s+sd}{  Pi\PYGZus{}base: float}
\PYG{l+s+sd}{    equivalents of ANC per mole of base (or acid)}
\PYG{l+s+sd}{  Pi\PYGZus{}CO3: float}
\PYG{l+s+sd}{    mole of carbonate per mole of base (or acid)}
\PYG{l+s+sd}{  Pi\PYGZus{}Al : float}
\PYG{l+s+sd}{    equivalents of ANC per mole of aluminum coagulant}
\PYG{l+s+sd}{  C\PYGZus{}Al}
\PYG{l+s+sd}{    concentration of aluminum coagulant in moles/L}
\PYG{l+s+sd}{  pH\PYGZus{}target: float}
\PYG{l+s+sd}{    pH goal}
\PYG{l+s+sd}{  Returns}
\PYG{l+s+sd}{  \PYGZhy{}\PYGZhy{}\PYGZhy{}\PYGZhy{}\PYGZhy{}\PYGZhy{}\PYGZhy{}}
\PYG{l+s+sd}{  The required concentration of base (or acid) in millimoles/L}
\PYG{l+s+sd}{  Examples}
\PYG{l+s+sd}{  \PYGZhy{}\PYGZhy{}\PYGZhy{}\PYGZhy{}\PYGZhy{}\PYGZhy{}\PYGZhy{}\PYGZhy{}}
\PYG{l+s+sd}{  \PYGZgt{}\PYGZgt{}\PYGZgt{} pH\PYGZus{}adjust(5.91,0.2*u.mmol/u.L,1,1,0,0,7)}
\PYG{l+s+sd}{  2.2892822041250924 millimole/liter}
\PYG{l+s+sd}{  \PYGZgt{}\PYGZgt{}\PYGZgt{} pH\PYGZus{}adjust(7,0.2*u.mmol/u.L,1,1,0,0,0,0,7)}
\PYG{l+s+sd}{  0.0 millimole/liter}
\PYG{l+s+sd}{  \PYGZgt{}\PYGZgt{}\PYGZgt{} pH\PYGZus{}adjust(7,0*u.mmol/u.L,1,0,\PYGZhy{}3,1*u.mmol/u.L,7)}
\PYG{l+s+sd}{  3.0 millimole/liter}
\PYG{l+s+sd}{  \PYGZdq{}\PYGZdq{}\PYGZdq{}}
  \PYG{n}{CT\PYGZus{}0} \PYG{o}{=} \PYG{n}{total\PYGZus{}carbonates\PYGZus{}closed}\PYG{p}{(}\PYG{n}{pH\PYGZus{}0}\PYG{p}{,}\PYG{n}{ANC\PYGZus{}0}\PYG{p}{)}
  \PYG{n}{B\PYGZus{}num} \PYG{o}{=} \PYG{n}{CT\PYGZus{}0} \PYG{o}{*} \PYG{p}{(}\PYG{n}{epa}\PYG{o}{.}\PYG{n}{alpha1\PYGZus{}carbonate}\PYG{p}{(}\PYG{n}{pH\PYGZus{}target}\PYG{p}{)} \PYG{o}{+} \PYG{l+m+mi}{2} \PYG{o}{*} \PYG{n}{epa}\PYG{o}{.}\PYG{n}{alpha2\PYGZus{}carbonate}\PYG{p}{(}\PYG{n}{pH\PYGZus{}target}\PYG{p}{)}\PYG{p}{)} \PYG{o}{+} \PYG{n}{epa}\PYG{o}{.}\PYG{n}{Kw}\PYG{o}{/}\PYG{n}{epa}\PYG{o}{.}\PYG{n}{invpH}\PYG{p}{(}\PYG{n}{pH\PYGZus{}target}\PYG{p}{)} \PYG{o}{\PYGZhy{}} \PYG{n}{epa}\PYG{o}{.}\PYG{n}{invpH}\PYG{p}{(}\PYG{n}{pH\PYGZus{}target}\PYG{p}{)} \PYG{o}{\PYGZhy{}} \PYG{n}{ANC\PYGZus{}0} \PYG{o}{\PYGZhy{}} \PYG{n}{Pi\PYGZus{}Al}\PYG{o}{*}\PYG{n}{C\PYGZus{}Al}
  \PYG{n}{B\PYGZus{}den} \PYG{o}{=} \PYG{n}{Pi\PYGZus{}base} \PYG{o}{\PYGZhy{}} \PYG{n}{Pi\PYGZus{}CO3}\PYG{o}{*}\PYG{p}{(}\PYG{n}{epa}\PYG{o}{.}\PYG{n}{alpha1\PYGZus{}carbonate}\PYG{p}{(}\PYG{n}{pH\PYGZus{}target}\PYG{p}{)} \PYG{o}{+} \PYG{l+m+mi}{2} \PYG{o}{*} \PYG{n}{epa}\PYG{o}{.}\PYG{n}{alpha2\PYGZus{}carbonate}\PYG{p}{(}\PYG{n}{pH\PYGZus{}target}\PYG{p}{)}\PYG{p}{)}
  \PYG{k}{return} \PYG{p}{(}\PYG{n}{B\PYGZus{}num}\PYG{o}{/}\PYG{n}{B\PYGZus{}den}\PYG{p}{)}\PYG{o}{.}\PYG{n}{to}\PYG{p}{(}\PYG{n}{u}\PYG{o}{.}\PYG{n}{mmol}\PYG{o}{/}\PYG{n}{u}\PYG{o}{.}\PYG{n}{L}\PYG{p}{)}


\PYG{l+s+sd}{\PYGZdq{}\PYGZdq{}\PYGZdq{}target pH\PYGZdq{}\PYGZdq{}\PYGZdq{}}
\PYG{n}{pH\PYGZus{}target} \PYG{o}{=} \PYG{l+m+mi}{7}

\PYG{n}{Pi\PYGZus{}base\PYGZus{}Na2CO3} \PYG{o}{=} \PYG{l+m+mi}{2}
\PYG{n}{Pi\PYGZus{}CO3\PYGZus{}Na2CO3} \PYG{o}{=} \PYG{l+m+mi}{1}

\PYG{n}{Pi\PYGZus{}base\PYGZus{}NaHCO3} \PYG{o}{=} \PYG{l+m+mi}{1}
\PYG{n}{Pi\PYGZus{}CO3\PYGZus{}NaHCO3} \PYG{o}{=} \PYG{l+m+mi}{1}

\PYG{n}{Pi\PYGZus{}base\PYGZus{}NaOH} \PYG{o}{=} \PYG{l+m+mi}{1}
\PYG{n}{Pi\PYGZus{}CO3\PYGZus{}NaOH} \PYG{o}{=} \PYG{l+m+mi}{0}

\PYG{n}{C\PYGZus{}Na2CO3} \PYG{o}{=} \PYG{n}{pH\PYGZus{}adjust}\PYG{p}{(}\PYG{n}{pH\PYGZus{}0}\PYG{p}{,}\PYG{n}{ANC\PYGZus{}0}\PYG{p}{,}\PYG{n}{Pi\PYGZus{}base\PYGZus{}Na2CO3}\PYG{p}{,}\PYG{n}{Pi\PYGZus{}CO3\PYGZus{}Na2CO3}\PYG{p}{,}\PYG{l+m+mi}{0}\PYG{p}{,}\PYG{l+m+mi}{0}\PYG{p}{,}\PYG{n}{pH\PYGZus{}target}\PYG{p}{)}

\PYG{n}{C\PYGZus{}NaHCO3} \PYG{o}{=} \PYG{n}{pH\PYGZus{}adjust}\PYG{p}{(}\PYG{n}{pH\PYGZus{}0}\PYG{p}{,}\PYG{n}{ANC\PYGZus{}0}\PYG{p}{,}\PYG{n}{Pi\PYGZus{}base\PYGZus{}NaHCO3}\PYG{p}{,}\PYG{n}{Pi\PYGZus{}CO3\PYGZus{}NaHCO3}\PYG{p}{,}\PYG{l+m+mi}{0}\PYG{p}{,}\PYG{l+m+mi}{0}\PYG{p}{,}\PYG{n}{pH\PYGZus{}target}\PYG{p}{)}
\PYG{n}{C\PYGZus{}NaOH} \PYG{o}{=} \PYG{n}{pH\PYGZus{}adjust}\PYG{p}{(}\PYG{n}{pH\PYGZus{}0}\PYG{p}{,}\PYG{n}{ANC\PYGZus{}0}\PYG{p}{,}\PYG{n}{Pi\PYGZus{}base\PYGZus{}NaOH}\PYG{p}{,}\PYG{n}{Pi\PYGZus{}CO3\PYGZus{}NaOH}\PYG{p}{,}\PYG{l+m+mi}{0}\PYG{p}{,}\PYG{l+m+mi}{0}\PYG{p}{,}\PYG{n}{pH\PYGZus{}target}\PYG{p}{)}

\PYG{l+s+sd}{\PYGZdq{}\PYGZdq{}\PYGZdq{}Display results in a pandas table\PYGZdq{}\PYGZdq{}\PYGZdq{}}
\PYG{n}{base} \PYG{o}{=} \PYG{p}{[}\PYG{l+s+s2}{\PYGZdq{}}\PYG{l+s+s2}{NaOH}\PYG{l+s+s2}{\PYGZdq{}}\PYG{p}{,}\PYG{l+s+s2}{\PYGZdq{}}\PYG{l+s+s2}{NaHCO3}\PYG{l+s+s2}{\PYGZdq{}}\PYG{p}{,}\PYG{l+s+s2}{\PYGZdq{}}\PYG{l+s+s2}{Na2CO3}\PYG{l+s+s2}{\PYGZdq{}}\PYG{p}{]}
\PYG{n}{myindex} \PYG{o}{=} \PYG{p}{[}\PYG{l+s+s2}{\PYGZdq{}}\PYG{l+s+s2}{[mmoles/L]}\PYG{l+s+s2}{\PYGZdq{}}\PYG{p}{,}\PYG{l+s+s2}{\PYGZdq{}}\PYG{l+s+s2}{[mg/L]}\PYG{l+s+s2}{\PYGZdq{}}\PYG{p}{]}
\PYG{n}{row1} \PYG{o}{=} \PYG{p}{[}\PYG{n}{C\PYGZus{}Na2CO3}\PYG{o}{.}\PYG{n}{magnitude}\PYG{p}{,}\PYG{n}{C\PYGZus{}NaHCO3}\PYG{o}{.}\PYG{n}{magnitude}\PYG{p}{,}\PYG{n}{C\PYGZus{}NaOH}\PYG{o}{.}\PYG{n}{magnitude}\PYG{p}{]}
\PYG{n}{row2} \PYG{o}{=} \PYG{p}{[}\PYG{p}{(}\PYG{n}{C\PYGZus{}Na2CO3}\PYG{o}{*}\PYG{n}{m\PYGZus{}Na2CO3}\PYG{p}{)}\PYG{o}{.}\PYG{n}{to}\PYG{p}{(}\PYG{n}{u}\PYG{o}{.}\PYG{n}{mg}\PYG{o}{/}\PYG{n}{u}\PYG{o}{.}\PYG{n}{L}\PYG{p}{)}\PYG{o}{.}\PYG{n}{magnitude}\PYG{p}{,}\PYG{p}{(}\PYG{n}{C\PYGZus{}NaHCO3}\PYG{o}{*}\PYG{n}{m\PYGZus{}NaHCO3}\PYG{p}{)}\PYG{o}{.}\PYG{n}{to}\PYG{p}{(}\PYG{n}{u}\PYG{o}{.}\PYG{n}{mg}\PYG{o}{/}\PYG{n}{u}\PYG{o}{.}\PYG{n}{L}\PYG{p}{)}\PYG{o}{.}\PYG{n}{magnitude}\PYG{p}{,}\PYG{p}{(}\PYG{n}{C\PYGZus{}NaOH}\PYG{o}{*}\PYG{n}{m\PYGZus{}NaOH}\PYG{p}{)}\PYG{o}{.}\PYG{n}{to}\PYG{p}{(}\PYG{n}{u}\PYG{o}{.}\PYG{n}{mg}\PYG{o}{/}\PYG{n}{u}\PYG{o}{.}\PYG{n}{L}\PYG{p}{)}\PYG{o}{.}\PYG{n}{magnitude}\PYG{p}{]}
\PYG{n}{df} \PYG{o}{=} \PYG{n}{pd}\PYG{o}{.}\PYG{n}{DataFrame}\PYG{p}{(}\PYG{p}{[}\PYG{n}{row1}\PYG{p}{,}\PYG{n}{row2}\PYG{p}{]}\PYG{p}{,}\PYG{n}{index}\PYG{o}{=}\PYG{n}{myindex}\PYG{p}{,}\PYG{n}{columns}\PYG{o}{=}\PYG{n}{base}\PYG{p}{)}
\PYG{n+nb}{print}\PYG{p}{(}\PYG{n}{df}\PYG{o}{.}\PYG{n}{round}\PYG{p}{(}\PYG{l+m+mi}{2}\PYG{p}{)}\PYG{p}{)}

\PYG{l+s+sd}{\PYGZdq{}\PYGZdq{}\PYGZdq{}Graph the base concentration required as a function of the target pH\PYGZdq{}\PYGZdq{}\PYGZdq{}}
\PYG{n}{pH\PYGZus{}graph} \PYG{o}{=} \PYG{n}{np}\PYG{o}{.}\PYG{n}{linspace}\PYG{p}{(}\PYG{l+m+mi}{6}\PYG{p}{,}\PYG{l+m+mi}{7}\PYG{p}{,}\PYG{l+m+mi}{50}\PYG{p}{)}
\PYG{n}{C\PYGZus{}Na2CO3} \PYG{o}{=} \PYG{n}{pH\PYGZus{}adjust}\PYG{p}{(}\PYG{n}{pH\PYGZus{}0}\PYG{p}{,}\PYG{n}{ANC\PYGZus{}0}\PYG{p}{,}\PYG{n}{Pi\PYGZus{}base\PYGZus{}Na2CO3}\PYG{p}{,}\PYG{n}{Pi\PYGZus{}CO3\PYGZus{}Na2CO3}\PYG{p}{,}\PYG{l+m+mi}{0}\PYG{p}{,}\PYG{l+m+mi}{0}\PYG{p}{,}\PYG{n}{pH\PYGZus{}graph}\PYG{p}{)}
\PYG{n}{C\PYGZus{}NaHCO3} \PYG{o}{=} \PYG{n}{pH\PYGZus{}adjust}\PYG{p}{(}\PYG{n}{pH\PYGZus{}0}\PYG{p}{,}\PYG{n}{ANC\PYGZus{}0}\PYG{p}{,}\PYG{n}{Pi\PYGZus{}base\PYGZus{}NaHCO3}\PYG{p}{,}\PYG{n}{Pi\PYGZus{}CO3\PYGZus{}NaHCO3}\PYG{p}{,}\PYG{l+m+mi}{0}\PYG{p}{,}\PYG{l+m+mi}{0}\PYG{p}{,}\PYG{n}{pH\PYGZus{}graph}\PYG{p}{)}
\PYG{n}{C\PYGZus{}NaOH} \PYG{o}{=} \PYG{n}{pH\PYGZus{}adjust}\PYG{p}{(}\PYG{n}{pH\PYGZus{}0}\PYG{p}{,}\PYG{n}{ANC\PYGZus{}0}\PYG{p}{,}\PYG{n}{Pi\PYGZus{}base\PYGZus{}NaOH}\PYG{p}{,}\PYG{n}{Pi\PYGZus{}CO3\PYGZus{}NaOH}\PYG{p}{,}\PYG{l+m+mi}{0}\PYG{p}{,}\PYG{l+m+mi}{0}\PYG{p}{,}\PYG{n}{pH\PYGZus{}graph}\PYG{p}{)}

\PYG{n}{fig}\PYG{p}{,} \PYG{n}{ax} \PYG{o}{=} \PYG{n}{plt}\PYG{o}{.}\PYG{n}{subplots}\PYG{p}{(}\PYG{p}{)}

\PYG{n}{ax}\PYG{o}{.}\PYG{n}{plot}\PYG{p}{(}\PYG{n}{pH\PYGZus{}graph}\PYG{p}{,}\PYG{n}{C\PYGZus{}NaHCO3}\PYG{p}{)}
\PYG{n}{ax}\PYG{o}{.}\PYG{n}{plot}\PYG{p}{(}\PYG{n}{pH\PYGZus{}graph}\PYG{p}{,}\PYG{n}{C\PYGZus{}Na2CO3}\PYG{p}{)}
\PYG{n}{ax}\PYG{o}{.}\PYG{n}{plot}\PYG{p}{(}\PYG{n}{pH\PYGZus{}graph}\PYG{p}{,}\PYG{n}{C\PYGZus{}NaOH}\PYG{p}{)}
\PYG{n}{imagepath} \PYG{o}{=} \PYG{l+s+s1}{\PYGZsq{}}\PYG{l+s+s1}{Rapid\PYGZus{}Mix/Images/}\PYG{l+s+s1}{\PYGZsq{}}
\PYG{n}{ax}\PYG{o}{.}\PYG{n}{set}\PYG{p}{(}\PYG{n}{xlabel}\PYG{o}{=}\PYG{l+s+s1}{\PYGZsq{}}\PYG{l+s+s1}{pH target}\PYG{l+s+s1}{\PYGZsq{}}\PYG{p}{,} \PYG{n}{ylabel}\PYG{o}{=}\PYG{l+s+s1}{\PYGZsq{}}\PYG{l+s+s1}{Base concentration (mmole/L)}\PYG{l+s+s1}{\PYGZsq{}}\PYG{p}{)}
\PYG{n}{ax}\PYG{o}{.}\PYG{n}{legend}\PYG{p}{(}\PYG{p}{[}\PYG{l+s+s2}{\PYGZdq{}}\PYG{l+s+s2}{sodium bicarbonate}\PYG{l+s+s2}{\PYGZdq{}}\PYG{p}{,}\PYG{l+s+s2}{\PYGZdq{}}\PYG{l+s+s2}{sodium carbonate}\PYG{l+s+s2}{\PYGZdq{}}\PYG{p}{,}\PYG{l+s+s2}{\PYGZdq{}}\PYG{l+s+s2}{sodium hydroxide}\PYG{l+s+s2}{\PYGZdq{}}\PYG{p}{]}\PYG{p}{)}
\PYG{n}{fig}\PYG{o}{.}\PYG{n}{savefig}\PYG{p}{(}\PYG{n}{imagepath}\PYG{o}{+}\PYG{l+s+s1}{\PYGZsq{}}\PYG{l+s+s1}{mole\PYGZus{}base\PYGZus{}for\PYGZus{}target\PYGZus{}pH}\PYG{l+s+s1}{\PYGZsq{}}\PYG{p}{)}
\PYG{n}{plt}\PYG{o}{.}\PYG{n}{show}\PYG{p}{(}\PYG{p}{)}

\PYG{n}{fig}\PYG{p}{,} \PYG{n}{ax} \PYG{o}{=} \PYG{n}{plt}\PYG{o}{.}\PYG{n}{subplots}\PYG{p}{(}\PYG{p}{)}
\PYG{n}{ax}\PYG{o}{.}\PYG{n}{plot}\PYG{p}{(}\PYG{n}{pH\PYGZus{}graph}\PYG{p}{,}\PYG{p}{(}\PYG{n}{C\PYGZus{}Na2CO3}\PYG{o}{*}\PYG{n}{m\PYGZus{}Na2CO3}\PYG{p}{)}\PYG{o}{.}\PYG{n}{to}\PYG{p}{(}\PYG{n}{u}\PYG{o}{.}\PYG{n}{mg}\PYG{o}{/}\PYG{n}{u}\PYG{o}{.}\PYG{n}{L}\PYG{p}{)}\PYG{p}{)}
\PYG{n}{ax}\PYG{o}{.}\PYG{n}{plot}\PYG{p}{(}\PYG{n}{pH\PYGZus{}graph}\PYG{p}{,}\PYG{p}{(}\PYG{n}{C\PYGZus{}NaOH}\PYG{o}{*}\PYG{n}{m\PYGZus{}NaOH}\PYG{p}{)}\PYG{o}{.}\PYG{n}{to}\PYG{p}{(}\PYG{n}{u}\PYG{o}{.}\PYG{n}{mg}\PYG{o}{/}\PYG{n}{u}\PYG{o}{.}\PYG{n}{L}\PYG{p}{)}\PYG{p}{)}
\PYG{n}{ax}\PYG{o}{.}\PYG{n}{set}\PYG{p}{(}\PYG{n}{xlabel}\PYG{o}{=}\PYG{l+s+s1}{\PYGZsq{}}\PYG{l+s+s1}{pH target}\PYG{l+s+s1}{\PYGZsq{}}\PYG{p}{,} \PYG{n}{ylabel}\PYG{o}{=}\PYG{l+s+s1}{\PYGZsq{}}\PYG{l+s+s1}{Base concentration (mg/L)}\PYG{l+s+s1}{\PYGZsq{}}\PYG{p}{)}
\PYG{n}{ax}\PYG{o}{.}\PYG{n}{legend}\PYG{p}{(}\PYG{p}{[}\PYG{l+s+s2}{\PYGZdq{}}\PYG{l+s+s2}{sodium carbonate}\PYG{l+s+s2}{\PYGZdq{}}\PYG{p}{,}\PYG{l+s+s2}{\PYGZdq{}}\PYG{l+s+s2}{sodium hydroxide}\PYG{l+s+s2}{\PYGZdq{}}\PYG{p}{]}\PYG{p}{)}
\PYG{n}{fig}\PYG{o}{.}\PYG{n}{savefig}\PYG{p}{(}\PYG{n}{imagepath}\PYG{o}{+}\PYG{l+s+s1}{\PYGZsq{}}\PYG{l+s+s1}{mg\PYGZus{}base\PYGZus{}for\PYGZus{}target\PYGZus{}pH}\PYG{l+s+s1}{\PYGZsq{}}\PYG{p}{)}
\PYG{n}{plt}\PYG{o}{.}\PYG{n}{show}\PYG{p}{(}\PYG{p}{)}
\end{sphinxVerbatim}

The analysis reveals that the choice of base matters. The most efficient (on a mass or mole basis) base is \(NaOH\) because it doesn’t add any carbonates that don’t fully react with the hydrogen ions. The decision about which base to use will be influenced by economics, operator safety, and by whether additional carbonate buffering simplifies plant operation with changing raw water quality.


\begin{savenotes}\sphinxattablestart
\centering
\sphinxcapstartof{table}
\sphinxcaption{Calcium base.}\label{\detokenize{Rapid_Mix/RM_Examples:id5}}\label{\detokenize{Rapid_Mix/RM_Examples:table-calcium-bases}}
\sphinxaftercaption
\begin{tabular}[t]{|\X{20}{60}|\X{20}{60}|\X{20}{60}|}
\hline
\sphinxstyletheadfamily 
Chemical name
&\sphinxstyletheadfamily 
common name
&\sphinxstyletheadfamily 
Chemcal formula
\\
\hline
calcium carbonate
&
limestone or chalk
&
\(CaCO_3\)
\\
\hline
calcium hydroxide
&
slaked lime or hydrated lime
&
\(Ca(OH)_2\)
\\
\hline
calcium oxide
&
quicklime
&
\(CaO\)
\\
\hline
\end{tabular}
\par
\sphinxattableend\end{savenotes}

The calcium bases are relatively inexpensive and have the disadvantage of lower solubility than sodium bases. Calcium carbonate has a low solubility, carbon dioxide is present in the atmosphere, and thus calcium carbonate precipitation limits the concentration that can be used for chemical feeds.

\begin{figure}[htbp]
\centering
\capstart

\noindent\sphinxincludegraphics[width=700\sphinxpxdimen]{{mole_base_for_target_pH}.png}
\caption{Dose of three bases (in mole/L) required to achieve a target pH for the Manzaragua water. Carbonates provide more buffering and less change in the pH compared with \(NaOH\).}\label{\detokenize{Rapid_Mix/RM_Examples:id6}}\label{\detokenize{Rapid_Mix/RM_Examples:figure-mole-base-for-target-ph}}\end{figure}

\begin{figure}[htbp]
\centering
\capstart

\noindent\sphinxincludegraphics[width=700\sphinxpxdimen]{{mg_base_for_target_pH}.png}
\caption{Dose of two bases (in mg/L) required to achieve a target pH for the Manzaragua water. Carbonates provide more buffering and less change in the pH compared with \(NaOH\).}\label{\detokenize{Rapid_Mix/RM_Examples:id7}}\label{\detokenize{Rapid_Mix/RM_Examples:figure-mg-base-for-target-ph}}\end{figure}

The required dose for each of the bases is summarized below.


\begin{savenotes}\sphinxattablestart
\centering
\sphinxcapstartof{table}
\sphinxcaption{Dose of each base required to change the pH of the Manzaragua water to 7.}\label{\detokenize{Rapid_Mix/RM_Examples:id8}}\label{\detokenize{Rapid_Mix/RM_Examples:table-base-table}}
\sphinxaftercaption
\begin{tabular}[t]{|\X{20}{80}|\X{20}{80}|\X{20}{80}|\X{20}{80}|}
\hline
\sphinxstyletheadfamily 
units
&
\(NaOH\)
&
\(NaHCO_3\)
&
\(Na_2CO_3\)
\\
\hline
{[}mmoles/L{]}
&
0.45
&
2.8
&
0.53
\\
\hline
{[}mg/L{]}
&
47.21
&
235.0
&
21.19
\\
\hline
\end{tabular}
\par
\sphinxattableend\end{savenotes}


\section{LFOM and coagulant injection sizing}
\label{\detokenize{Rapid_Mix/RM_Examples:lfom-and-coagulant-injection-sizing}}\label{\detokenize{Rapid_Mix/RM_Examples:heading-lfom-and-coag-injection-sizing}}
A water treatment plant that is treating 120 L/s of water injects the coagulant into the middle of the pipe that delivers the raw water to the plant and then splits the flow into 2 parallel treatment trains for subsequent flocculation. The pipe is PVC 24 inch nominal pipe diameter SDR 26. The water temperature is \(0^{\circ}C\). Estimate the minimum distance between the injection point and the flow split.

We will use a {\hyperref[\detokenize{Flow_Control_and_Measurement/FCM_Design:heading-lfom}]{\sphinxcrossref{\DUrole{std,std-ref}{linear flow orifice meter}}}} with 20 cm of head loss. The first step is to determine the diameter of the LFOM.

\fvset{hllines={, ,}}%
\begin{sphinxVerbatim}[commandchars=\\\{\}]
\PYG{l+s+sd}{\PYGZdq{}\PYGZdq{}\PYGZdq{} importing \PYGZdq{}\PYGZdq{}\PYGZdq{}}
\PYG{k+kn}{from} \PYG{n+nn}{aide\PYGZus{}design}\PYG{n+nn}{.}\PYG{n+nn}{play} \PYG{k}{import}\PYG{o}{*}
\PYG{k+kn}{from} \PYG{n+nn}{aguaclara\PYGZus{}research}\PYG{n+nn}{.}\PYG{n+nn}{play} \PYG{k}{import}\PYG{o}{*}
\PYG{k+kn}{import} \PYG{n+nn}{aguaclara\PYGZus{}research}\PYG{n+nn}{.}\PYG{n+nn}{floc\PYGZus{}model} \PYG{k}{as} \PYG{n+nn}{fm}
\PYG{k+kn}{import} \PYG{n+nn}{matplotlib}\PYG{n+nn}{.}\PYG{n+nn}{pyplot} \PYG{k}{as} \PYG{n+nn}{plt}
\PYG{k+kn}{from} \PYG{n+nn}{matplotlib}\PYG{n+nn}{.}\PYG{n+nn}{ticker} \PYG{k}{import} \PYG{n}{FormatStrFormatter}
\PYG{n}{imagepath} \PYG{o}{=} \PYG{l+s+s1}{\PYGZsq{}}\PYG{l+s+s1}{AguaClara Water Treatment Plant Design/Rapid Mix/Images/}\PYG{l+s+s1}{\PYGZsq{}}

\PYG{n}{Q\PYGZus{}plant} \PYG{o}{=} \PYG{l+m+mi}{120} \PYG{o}{*} \PYG{n}{u}\PYG{o}{.}\PYG{n}{L}\PYG{o}{/}\PYG{n}{u}\PYG{o}{.}\PYG{n}{s}
\PYG{n}{HL\PYGZus{}LFOM} \PYG{o}{=} \PYG{l+m+mi}{20} \PYG{o}{*} \PYG{n}{u}\PYG{o}{.}\PYG{n}{cm}
\PYG{n}{Pi\PYGZus{}LFOM\PYGZus{}safety} \PYG{o}{=} \PYG{l+m+mf}{1.2}
\PYG{n}{SDR\PYGZus{}LFOM} \PYG{o}{=} \PYG{l+m+mi}{26}
\PYG{k+kn}{from} \PYG{n+nn}{aide\PYGZus{}design}\PYG{n+nn}{.}\PYG{n+nn}{unit\PYGZus{}process\PYGZus{}design} \PYG{k}{import} \PYG{n}{lfom} \PYG{k}{as} \PYG{n}{lfom}
\PYG{n}{ND\PYGZus{}LFOM} \PYG{o}{=} \PYG{n}{lfom}\PYG{o}{.}\PYG{n}{nom\PYGZus{}diam\PYGZus{}lfom\PYGZus{}pipe}\PYG{p}{(}\PYG{n}{Q\PYGZus{}plant}\PYG{p}{,}\PYG{n}{HL\PYGZus{}LFOM}\PYG{p}{)}
\PYG{n+nb}{print}\PYG{p}{(}\PYG{n}{ND\PYGZus{}LFOM}\PYG{p}{,} \PYG{l+s+s1}{\PYGZsq{}}\PYG{l+s+s1}{(}\PYG{l+s+s1}{\PYGZsq{}}\PYG{p}{,}\PYG{n}{ND\PYGZus{}LFOM}\PYG{o}{.}\PYG{n}{to}\PYG{p}{(}\PYG{n}{u}\PYG{o}{.}\PYG{n}{cm}\PYG{p}{)}\PYG{p}{,} \PYG{l+s+s1}{\PYGZsq{}}\PYG{l+s+s1}{)}\PYG{l+s+s1}{\PYGZsq{}}\PYG{p}{)}

\PYG{n}{L\PYGZus{}flow} \PYG{o}{=} \PYG{n}{pipe}\PYG{o}{.}\PYG{n}{ID\PYGZus{}SDR}\PYG{p}{(}\PYG{n}{ND\PYGZus{}LFOM}\PYG{p}{,}\PYG{n}{SDR\PYGZus{}LFOM}\PYG{p}{)}
\PYG{n}{L\PYGZus{}flow}
\end{sphinxVerbatim}

The LFOM requires a 24 inch diameter pipe.


\section{Example problem: Energy dissipation rate in a straight pipe}
\label{\detokenize{Rapid_Mix/RM_Examples:example-problem-energy-dissipation-rate-in-a-straight-pipe}}\begin{description}
\item[{Solution scheme}] \leavevmode
1) Calculate the friction factor
1) Use \sphinxcode{\sphinxupquote{mixing\_pipe\_diameters}} to estimate the mixing length in pipe diameters
1) Convert to pipe length in meters.

\end{description}

\fvset{hllines={, ,}}%
\begin{sphinxVerbatim}[commandchars=\\\{\}]
\PYG{k+kn}{from} \PYG{n+nn}{aide\PYGZus{}design}\PYG{n+nn}{.}\PYG{n+nn}{play} \PYG{k}{import}\PYG{o}{*}
\PYG{n}{T\PYGZus{}water}\PYG{o}{=}\PYG{l+m+mi}{0}\PYG{o}{*}\PYG{n}{u}\PYG{o}{.}\PYG{n}{degC}
\PYG{n}{Pipe\PYGZus{}roughness} \PYG{o}{=} \PYG{n}{mat}\PYG{o}{.}\PYG{n}{PIPE\PYGZus{}ROUGH\PYGZus{}PVC}
\PYG{n}{Pipe\PYGZus{}roughness}
\PYG{n}{Nu\PYGZus{}water} \PYG{o}{=} \PYG{n}{pc}\PYG{o}{.}\PYG{n}{viscosity\PYGZus{}kinematic}\PYG{p}{(}\PYG{n}{T\PYGZus{}water}\PYG{p}{)}
\PYG{n}{Q\PYGZus{}pipe} \PYG{o}{=} \PYG{l+m+mi}{120} \PYG{o}{*} \PYG{n}{u}\PYG{o}{.}\PYG{n}{L}\PYG{o}{/}\PYG{n}{u}\PYG{o}{.}\PYG{n}{s}
\PYG{n}{ND\PYGZus{}pipe} \PYG{o}{=} \PYG{l+m+mi}{24}\PYG{o}{*}\PYG{n}{u}\PYG{o}{.}\PYG{n}{inch}
\PYG{n}{SDR\PYGZus{}pipe} \PYG{o}{=} \PYG{l+m+mi}{26}
\PYG{n}{ID\PYGZus{}pipe} \PYG{o}{=} \PYG{n}{pipe}\PYG{o}{.}\PYG{n}{ID\PYGZus{}SDR}\PYG{p}{(}\PYG{n}{ND\PYGZus{}pipe}\PYG{p}{,}\PYG{n}{SDR\PYGZus{}pipe}\PYG{p}{)}
\PYG{n}{f\PYGZus{}pipe} \PYG{o}{=} \PYG{n}{pc}\PYG{o}{.}\PYG{n}{fric}\PYG{p}{(}\PYG{n}{Q\PYGZus{}pipe}\PYG{p}{,}\PYG{n}{ID\PYGZus{}pipe}\PYG{p}{,}\PYG{n}{Nu\PYGZus{}water}\PYG{p}{,}\PYG{n}{Pipe\PYGZus{}roughness}\PYG{p}{)}
\PYG{n}{N\PYGZus{}pipe\PYGZus{}diameters} \PYG{o}{=} \PYG{p}{(}\PYG{l+m+mi}{2}\PYG{o}{/}\PYG{n}{f\PYGZus{}pipe}\PYG{p}{)}\PYG{o}{*}\PYG{o}{*}\PYG{p}{(}\PYG{l+m+mi}{1}\PYG{o}{/}\PYG{l+m+mi}{3}\PYG{p}{)}
\PYG{n}{N\PYGZus{}pipe\PYGZus{}diameters}
\PYG{l+s+sd}{\PYGZdq{}\PYGZdq{}\PYGZdq{}The minimum length for mixing is thus\PYGZdq{}\PYGZdq{}\PYGZdq{}}
\PYG{n}{L\PYGZus{}mixing} \PYG{o}{=} \PYG{n}{ID\PYGZus{}pipe}\PYG{o}{*}\PYG{n}{N\PYGZus{}pipe\PYGZus{}diameters}
\PYG{n+nb}{print}\PYG{p}{(}\PYG{l+s+s1}{\PYGZsq{}}\PYG{l+s+s1}{The minimum distance required for mixing across the diameter of the pipe is }\PYG{l+s+s1}{\PYGZsq{}}\PYG{p}{,}\PYG{n}{L\PYGZus{}mixing}\PYG{o}{.}\PYG{n}{to\PYGZus{}base\PYGZus{}units}\PYG{p}{(}\PYG{p}{)}\PYG{p}{)}
\PYG{n}{v\PYGZus{}lfom} \PYG{o}{=} \PYG{p}{(}\PYG{n}{Q\PYGZus{}plant}\PYG{o}{/}\PYG{n}{pc}\PYG{o}{.}\PYG{n}{area\PYGZus{}circle}\PYG{p}{(}\PYG{n}{pipe}\PYG{o}{.}\PYG{n}{ID\PYGZus{}SDR}\PYG{p}{(}\PYG{n}{ND\PYGZus{}LFOM}\PYG{p}{,}\PYG{n}{SDR\PYGZus{}LFOM}\PYG{p}{)}\PYG{p}{)}\PYG{p}{)}\PYG{o}{.}\PYG{n}{to\PYGZus{}base\PYGZus{}units}\PYG{p}{(}\PYG{p}{)}
\PYG{n+nb}{print}\PYG{p}{(}\PYG{n}{v\PYGZus{}lfom}\PYG{p}{)}
\PYG{n}{t\PYGZus{}mixing} \PYG{o}{=} \PYG{p}{(}\PYG{n}{L\PYGZus{}mixing}\PYG{o}{/}\PYG{n}{v\PYGZus{}lfom}\PYG{p}{)}\PYG{o}{.}\PYG{n}{to}\PYG{p}{(}\PYG{n}{u}\PYG{o}{.}\PYG{n}{s}\PYG{p}{)}
\PYG{n}{t\PYGZus{}mixing}
\end{sphinxVerbatim}

The previous analysis provides a minimum distance for sufficient mixing so that equal mass flux of coagulant will end up in both treatment trains. This assumes that the coagulant was injected in the pipe centerline. Injection at the wall of the pipe is a poor practice and would require many more pipe diameters because it takes significant time for the coagulant to be mixed out of the slower fluid at the wall. The time required for mixing at the scale of the flow in the plant is thus accomplished in a few seconds. This ends up being the fastest part of the transport of the coagulant nanoparticles on their way to attachment to the clay particles.  Next we will determine a typical flow rate of coagulant. \sphinxstylestrong{Aluminum} concentrations for polyaluminum chloride (PACl) typically range from 1 to 10 mg/L. The maximum PACl stock solution concentration is about 70 g/L as \sphinxstylestrong{Al}.

\fvset{hllines={, ,}}%
\begin{sphinxVerbatim}[commandchars=\\\{\}]
\PYG{n}{C\PYGZus{}PACl\PYGZus{}stock} \PYG{o}{=} \PYG{l+m+mi}{70} \PYG{o}{*}\PYG{n}{u}\PYG{o}{.}\PYG{n}{g}\PYG{o}{/}\PYG{n}{u}\PYG{o}{.}\PYG{n}{L}
\PYG{n}{C\PYGZus{}PACl\PYGZus{}dose\PYGZus{}max} \PYG{o}{=} \PYG{l+m+mi}{10} \PYG{o}{*} \PYG{n}{u}\PYG{o}{.}\PYG{n}{mg}\PYG{o}{/}\PYG{n}{u}\PYG{o}{.}\PYG{n}{L}
\PYG{n}{Q\PYGZus{}PACl\PYGZus{}max} \PYG{o}{=} \PYG{p}{(}\PYG{n}{Q\PYGZus{}plant}\PYG{o}{*}\PYG{n}{C\PYGZus{}PACl\PYGZus{}dose\PYGZus{}max}\PYG{o}{/}\PYG{n}{C\PYGZus{}PACl\PYGZus{}stock}\PYG{p}{)}\PYG{o}{.}\PYG{n}{to}\PYG{p}{(}\PYG{n}{u}\PYG{o}{.}\PYG{n}{mL}\PYG{o}{/}\PYG{n}{u}\PYG{o}{.}\PYG{n}{s}\PYG{p}{)}
\PYG{n+nb}{print}\PYG{p}{(}\PYG{n}{Q\PYGZus{}PACl\PYGZus{}max}\PYG{p}{)}
\end{sphinxVerbatim}

We can estimate the diameter of the injection port by setting the kinetic energy loss where the coagulant is injected into the main flow to be large enough to exceed the pressure fluctuations downstream of the LFOM. The amount of energy we invest in injecting the coagulant into the raw water is a compromise between having to raise the entire chemical feed system including the stock tanks to increase the potential energy and a goal of not having pressure fluctuations inside the LFOM pipe cause flow oscillations in the chemical dosing tube. Thus our goal is to have the kinetic energy at the injection point be large compared with the expected pressure fluctuations in the LFOM. Given that the head loss through the LFOM is often 20 cm, we expect the pressure fluctuations from turbulence to be a small fraction of that head loss. Thus we set the kinetic energy to be equivalent to 2 cm.

\fvset{hllines={, ,}}%
\begin{sphinxVerbatim}[commandchars=\\\{\}]
\PYG{n}{HL\PYGZus{}Coag\PYGZus{}injection} \PYG{o}{=} \PYG{l+m+mi}{2} \PYG{o}{*} \PYG{n}{u}\PYG{o}{.}\PYG{n}{cm}
\PYG{n}{v\PYGZus{}Coag\PYGZus{}injection} \PYG{o}{=} \PYG{p}{(}\PYG{p}{(}\PYG{l+m+mi}{2} \PYG{o}{*} \PYG{n}{u}\PYG{o}{.}\PYG{n}{gravity} \PYG{o}{*} \PYG{n}{HL\PYGZus{}Coag\PYGZus{}injection}\PYG{p}{)}\PYG{o}{*}\PYG{o}{*}\PYG{l+m+mf}{0.5}\PYG{p}{)}\PYG{o}{.}\PYG{n}{to}\PYG{p}{(}\PYG{n}{u}\PYG{o}{.}\PYG{n}{m}\PYG{o}{/}\PYG{n}{u}\PYG{o}{.}\PYG{n}{s}\PYG{p}{)}
\PYG{n+nb}{print}\PYG{p}{(}\PYG{n}{v\PYGZus{}Coag\PYGZus{}injection}\PYG{p}{)}
\PYG{n}{D\PYGZus{}Coag\PYGZus{}injection\PYGZus{}min} \PYG{o}{=} \PYG{n}{pc}\PYG{o}{.}\PYG{n}{diam\PYGZus{}circle}\PYG{p}{(}\PYG{n}{Q\PYGZus{}PACl\PYGZus{}max}\PYG{o}{/}\PYG{n}{v\PYGZus{}Coag\PYGZus{}injection}\PYG{p}{)}
\PYG{n+nb}{print}\PYG{p}{(}\PYG{n}{D\PYGZus{}Coag\PYGZus{}injection\PYGZus{}min}\PYG{o}{.}\PYG{n}{to}\PYG{p}{(}\PYG{n}{u}\PYG{o}{.}\PYG{n}{mm}\PYG{p}{)}\PYG{p}{)}
\end{sphinxVerbatim}


\chapter{Rapid Mix Theory and Future Work}
\label{\detokenize{Rapid_Mix/RM_Theory_and_Future_Work:rapid-mix-theory-and-future-work}}\label{\detokenize{Rapid_Mix/RM_Theory_and_Future_Work:title-rapid-mix-theory-and-future-work}}\label{\detokenize{Rapid_Mix/RM_Theory_and_Future_Work::doc}}
Our understanding of rapid mix is currently quite speculative. This is an area that requires substantial research. We have anecdotal evidence that the process of transporting coagulant nanoparticles to suspended particle surfaces may be a slow, rate-limiting process. Dissolved organic matter may influence the rate of coagulant nanoparticle transport by effectively increasing the size of the coagulant nanoparticles and thus reducing the diffusion rate.

Developing a fundamental understanding of the mixing and transport processes that occur between coagulant addition and flocculation is a very high priority for the AguaClara program.


\section{Diffusion and Shear Transport Coagulant Nanoparticles to Clay}
\label{\detokenize{Rapid_Mix/RM_Theory_and_Future_Work:diffusion-and-shear-transport-coagulant-nanoparticles-to-clay}}\label{\detokenize{Rapid_Mix/RM_Theory_and_Future_Work:heading-diffusion-and-shear-transport-coagulant-nanoparticles-to-clay}}
The time required for shear and diffusion to transport coagulant nanoparticles to clay has previously been assumed to be a rapid process.
.. todo:: Find references for time required for coagulant attachment to suspended particles.

Our analysis suggests that this critical step may require significant time especially given our effort to reduce the time allotted for flocculation.
\begin{itemize}
\item {} 
Turbulent eddies, viscous shear, and diffusion blends the coagulant with the raw water sufficiently ({\hyperref[\detokenize{Rapid_Mix/RM_Intro:heading-mixing-time}]{\sphinxcrossref{\DUrole{std,std-ref}{in a few seconds}}}}) so that the coagulant precipitates and forms nanoparticles.

\item {} 
Dissolved organic molecules diffuse to the coagulant nanoparticles and adhere to the nanoparticle surface.

\item {} 
The coagulant nanoparticles are transported to suspended particle surfaces by a combination of diffusion and fluid shear.

\end{itemize}

The following is a very preliminary estimate of the time required for attachment of the nanoparticles to the clay particles. This analysis includes multiple simplifying assumptions and there is a reasonable possibility that some of those assumptions are wrong. However, the core assumptions that coagulant nanoparticles are transported to clay particles by a combination of fluid deformation (shear) and molecular diffusion is reasonable.

The following analysis is similar to the collision analysis that was developed in the AguaClara flocculation model.

The volume of the suspension that is cleared of nanoparticles is proportional to a collision area defined by a ring around the clay particle with width of the diameter of the nanoparticle diffusion band. This diffusion band is the length scale over which diffusion is able to transport coagulant particles to the clay surface during the time that the nanoparticles are sliding past the clay particle.
\begin{equation}\label{equation:Rapid_Mix/RM_Theory_and_Future_Work:Rapid_Mix/RM_Theory_and_Future_Work:0}
\begin{split}\propto \pi \, d_{Clay} \, L_{Diff_{NC}}\end{split}
\end{equation}
The volume cleared is proportional to time
\begin{equation}\label{equation:Rapid_Mix/RM_Theory_and_Future_Work:Rapid_Mix/RM_Theory_and_Future_Work:1}
\begin{split}\propto t\end{split}
\end{equation}
The volume cleared is proportional to the relative velocity between clay and nanoparticles. This scaling
\begin{equation}\label{equation:Rapid_Mix/RM_Theory_and_Future_Work:Rapid_Mix/RM_Theory_and_Future_Work:2}
\begin{split}\propto v_r\end{split}
\end{equation}\begin{equation}\label{equation:Rapid_Mix/RM_Theory_and_Future_Work:Rapid_Mix/RM_Theory_and_Future_Work:3}
\begin{split}\bar v_{\rm{Cleared}} = \pi  d_{Clay} \, L_{Diff_{NC}}  v_r  t\end{split}
\end{equation}
Use dimensional analysis to get a relative velocity for the long range transport controlled by shear.
\begin{equation}\label{equation:Rapid_Mix/RM_Theory_and_Future_Work:Rapid_Mix/RM_Theory_and_Future_Work:4}
\begin{split}v_r = f \left( \varepsilon ,\nu ,\Lambda_{Clay} \right)\end{split}
\end{equation}\begin{equation}\label{equation:Rapid_Mix/RM_Theory_and_Future_Work:Rapid_Mix/RM_Theory_and_Future_Work:5}
\begin{split}v_r = \Lambda_{Clay} f \left( \varepsilon ,\nu \right)\end{split}
\end{equation}\begin{equation}\label{equation:Rapid_Mix/RM_Theory_and_Future_Work:Rapid_Mix/RM_Theory_and_Future_Work:6}
\begin{split}v_r \approx \Lambda_{Clay} G\end{split}
\end{equation}\begin{equation}\label{equation:Rapid_Mix/RM_Theory_and_Future_Work:Rapid_Mix/RM_Theory_and_Future_Work:7}
\begin{split}\Lambda_{Clay} = [L]
\, \, \, \, \, \, \,
\varepsilon = \frac{[L]^2}{[T]^3}
\, \, \, \, \, \, \,
\nu = \frac{[L]^2}{[T]}\end{split}
\end{equation}

\section{Collision Rates}
\label{\detokenize{Rapid_Mix/RM_Theory_and_Future_Work:collision-rates}}\label{\detokenize{Rapid_Mix/RM_Theory_and_Future_Work:heading-collision-rates}}\begin{equation}\label{equation:Rapid_Mix/RM_Theory_and_Future_Work:Rapid_Mix/RM_Theory_and_Future_Work:8}
\begin{split}{\rlap{\kern.08em--}V_{\rm{Cleared}}} \approx \pi d_{Clay} L_{Diff_{NC}} v_r t_c\end{split}
\end{equation}
Where \(\rlap{\kern.08em--}V_{Occupied} = \Lambda_{Clay}^3\). Solve for \(t_c\):
\begin{equation}\label{equation:Rapid_Mix/RM_Theory_and_Future_Work:Rapid_Mix/RM_Theory_and_Future_Work:9}
\begin{split}t_c = \frac{\Lambda_{NC}^3}{\pi d_{Clay} L_{Diff_{NC}} v_r}\end{split}
\end{equation}
This is the average time for a clay particle to have the entire volume of water that it occupies sweep past the clay particle.
\(v_r \approx \Lambda_{Clay} G\)
\begin{equation}\label{equation:Rapid_Mix/RM_Theory_and_Future_Work:Rapid_Mix/RM_Theory_and_Future_Work:10}
\begin{split}t_c = \frac{\Lambda_{Clay}^3}{\pi d_{Clay} L_{Diff_{NC}} \Lambda_{Clay} G}\end{split}
\end{equation}
Where \(t_c = \frac{dN_c}{dt}\):
\begin{equation}\label{equation:Rapid_Mix/RM_Theory_and_Future_Work:Rapid_Mix/RM_Theory_and_Future_Work:11}
\begin{split}dN_c = \pi d_{Clay} L_{Diff_{NC}}{\Lambda^{-2}_{Clay}} G dt\end{split}
\end{equation}

\subsection{Collision Rate and Particle Removal}
\label{\detokenize{Rapid_Mix/RM_Theory_and_Future_Work:collision-rate-and-particle-removal}}\label{\detokenize{Rapid_Mix/RM_Theory_and_Future_Work:heading-collision-rate-and-particle-removal}}
A fraction of the remaining coagulant nanoparticles are removed during the time required for one sweep past the clay particle.
\begin{equation}\label{equation:Rapid_Mix/RM_Theory_and_Future_Work:Rapid_Mix/RM_Theory_and_Future_Work:12}
\begin{split}\frac{dn_{NC}}{ - k \, n_{NC}} = dN_c\end{split}
\end{equation}\begin{equation}\label{equation:Rapid_Mix/RM_Theory_and_Future_Work:Rapid_Mix/RM_Theory_and_Future_Work:13}
\begin{split}\frac{dn_{NC}}{ - k \, n_{NC}} = \pi d_{Clay} L_{Diff_{NC}}{\Lambda^{-2}_{Clay}} G dt\end{split}
\end{equation}

\subsection{Integrate the coagulant transport model}
\label{\detokenize{Rapid_Mix/RM_Theory_and_Future_Work:integrate-the-coagulant-transport-model}}\label{\detokenize{Rapid_Mix/RM_Theory_and_Future_Work:heading-integrate-the-coagulant-transport-model}}
Integrate from the initial coagulant nanoparticle concentration to the concentration at time t.
\begin{equation}\label{equation:Rapid_Mix/RM_Theory_and_Future_Work:Rapid_Mix/RM_Theory_and_Future_Work:14}
\begin{split}\int \limits_{n_{NC_0}}^{n_{NC}} n_{NC}^{- 1} \, dn_{NC}  =  - \pi d_{Clay} L_{Diff_{NC}} \Lambda^{-2}_{Clay} G \, k  \int \limits_0^t {dt}\end{split}
\end{equation}
Use pC notation to be consistent with how we describe removal efficiency of other contaminants.
\begin{equation}\label{equation:Rapid_Mix/RM_Theory_and_Future_Work:Rapid_Mix/RM_Theory_and_Future_Work:15}
\begin{split}2.3 p C_{NC} = \pi d_{Clay}\,  L_{Diff_{NC}}\,  \Lambda^{-2}_{Clay}\,  G k  t\end{split}
\end{equation}
Solve for the time required to reach a target efficiency of application of coagulant nanoparticles to clay.
\begin{equation}\label{equation:Rapid_Mix/RM_Theory_and_Future_Work:Rapid_Mix/RM_Theory_and_Future_Work:16}
\begin{split}t_{coagulant, \, application} = \frac{2.3p C_{NC} \, \Lambda_{Clay}^2}{\pi G k \, d_{Clay}\,  L_{Diff_{NC}} }\end{split}
\end{equation}

\subsubsection{Coagulant nanoparticle application}
\label{\detokenize{Rapid_Mix/RM_Theory_and_Future_Work:coagulant-nanoparticle-application}}\begin{equation}\label{equation:Rapid_Mix/RM_Theory_and_Future_Work:Rapid_Mix/RM_Theory_and_Future_Work:17}
\begin{split}\Delta h =   \frac{G^2 \nu \theta}{g}\end{split}
\end{equation}
Replace \(\theta\) with t.
\begin{equation}\label{equation:Rapid_Mix/RM_Theory_and_Future_Work:Rapid_Mix/RM_Theory_and_Future_Work:18}
\begin{split}\Delta h =  \frac{G^2 \nu}{g} \frac{2.3p C_{NC} \, \Lambda_{Clay}^2}{\pi G k \, d_{Clay}\,  L_{Diff_{NC}} }\end{split}
\end{equation}\begin{equation}\label{equation:Rapid_Mix/RM_Theory_and_Future_Work:Rapid_Mix/RM_Theory_and_Future_Work:19}
\begin{split}L_{Diff} \approx \left( \frac{2k_B T d_{Clay}}{3 \pi \,\mu  \, d_{NC} G}\right)^\frac{1}{3}\end{split}
\end{equation}\begin{equation}\label{equation:Rapid_Mix/RM_Theory_and_Future_Work:Rapid_Mix/RM_Theory_and_Future_Work:20}
\begin{split}\Delta h =  \frac{G^2 \nu}{g} \frac{2.3p C_{NC} \, \Lambda_{Clay}^2}{\pi G k \, d_{Clay}} \left( \frac{3 \pi \,\mu  \, d_{NC} G}{2k_B T d_{Clay}}\right)^\frac{1}{3}\end{split}
\end{equation}
Solve for the velocity gradient.
\begin{equation}\label{equation:Rapid_Mix/RM_Theory_and_Future_Work:Rapid_Mix/RM_Theory_and_Future_Work:21}
\begin{split}\Delta h =  \frac{G^\frac{4}{3} \nu}{g} \frac{2.3p C_{NC} \, \Lambda_{Clay}^2}{\pi k \, d_{Clay}} \left( \frac{3 \pi \,\mu  \, d_{NC} }{2k_B T d_{Clay}}\right)^\frac{1}{3}\end{split}
\end{equation}\begin{equation}\label{equation:Rapid_Mix/RM_Theory_and_Future_Work:Rapid_Mix/RM_Theory_and_Future_Work:22}
\begin{split}G_{coagulant, \, application} =  d_{Clay}\left(\frac{\pi k \,g\Delta h }{2.3p C_{NC} \, \Lambda_{Clay}^2 \nu} \right)^\frac{3}{4} \left( \frac{2k_B T }{3 \pi \,\mu  \, d_{NC} }\right)^\frac{1}{4}\end{split}
\end{equation}

\subsection{Diffusion band thickness}
\label{\detokenize{Rapid_Mix/RM_Theory_and_Future_Work:diffusion-band-thickness}}
The time required for shear to transport all of the fluid past the clay so that diffusion can transport the coagulant nanoparticles to the clay surface is significant.
\begin{equation}\label{equation:Rapid_Mix/RM_Theory_and_Future_Work:Rapid_Mix/RM_Theory_and_Future_Work:23}
\begin{split}D_{Diffusion} = \frac{k_B T}{3 \pi \, \mu \, d_P}\end{split}
\end{equation}\begin{equation}\label{equation:Rapid_Mix/RM_Theory_and_Future_Work:Rapid_Mix/RM_Theory_and_Future_Work:24}
\begin{split}L_{Diff} \approx \sqrt{D_{Diffusion} t_{Diffusion}}\end{split}
\end{equation}
The time for nanoparticles to diffuse through the boundary layer around the clay particle is equal to the distance they travel around the clay particle divided by their velocity. The distance they travel scales with \(d_{Clay}\) and their average velocity scales with the thickness of the diffusion layer/2 * the velocity gradient.
\begin{equation}\label{equation:Rapid_Mix/RM_Theory_and_Future_Work:Rapid_Mix/RM_Theory_and_Future_Work:25}
\begin{split}t_{Diffusion} = \frac{ 2d_{Clay}} {L_{Diff} G}\end{split}
\end{equation}\begin{equation}\label{equation:Rapid_Mix/RM_Theory_and_Future_Work:Rapid_Mix/RM_Theory_and_Future_Work:26}
\begin{split}L_{Diff} \approx \left( \frac{2k_B T d_{Clay}}{3 \pi \,\mu  \, d_{NC} G}\right)^\frac{1}{3}\end{split}
\end{equation}
Let’s estimate the thickness of the diffusion band

\fvset{hllines={, ,}}%
\begin{sphinxVerbatim}[commandchars=\\\{\}]
\PYG{n}{T\PYGZus{}graph} \PYG{o}{=} \PYG{n}{np}\PYG{o}{.}\PYG{n}{linspace}\PYG{p}{(}\PYG{l+m+mi}{0}\PYG{p}{,}\PYG{l+m+mi}{30}\PYG{p}{,}\PYG{l+m+mi}{4}\PYG{p}{)}\PYG{o}{*}\PYG{n}{u}\PYG{o}{.}\PYG{n}{degC}
\PYG{n}{G} \PYG{o}{=} \PYG{n}{np}\PYG{o}{.}\PYG{n}{arange}\PYG{p}{(}\PYG{l+m+mi}{50}\PYG{p}{,}\PYG{l+m+mi}{5000}\PYG{p}{,}\PYG{l+m+mi}{50}\PYG{p}{)}\PYG{o}{*}\PYG{n}{u}\PYG{o}{.}\PYG{n}{Hz}

\PYG{k}{def} \PYG{n+nf}{L\PYGZus{}Diff}\PYG{p}{(}\PYG{n}{Temperature}\PYG{p}{,}\PYG{n}{G}\PYG{p}{)}\PYG{p}{:}
  \PYG{k}{return} \PYG{p}{(}\PYG{p}{(}\PYG{p}{(}\PYG{l+m+mi}{2}\PYG{o}{*}\PYG{n}{u}\PYG{o}{.}\PYG{n}{boltzmann\PYGZus{}constant}\PYG{o}{*}\PYG{n}{Temperature} \PYG{o}{*} \PYG{n}{fm}\PYG{o}{.}\PYG{n}{Clay}\PYG{o}{.}\PYG{n}{Diameter}\PYG{o}{*}\PYG{n}{u}\PYG{o}{.}\PYG{n}{m}\PYG{p}{)}\PYG{o}{/}\PYG{p}{(}\PYG{l+m+mi}{3} \PYG{o}{*} \PYG{n}{np}\PYG{o}{.}\PYG{n}{pi} \PYG{o}{*}\PYG{n}{pc}\PYG{o}{.}\PYG{n}{viscosity\PYGZus{}dynamic}\PYG{p}{(}\PYG{n}{Temperature}\PYG{p}{)}\PYG{o}{*} \PYG{p}{(}\PYG{n}{fm}\PYG{o}{.}\PYG{n}{PACl}\PYG{o}{.}\PYG{n}{Diameter}\PYG{o}{*}\PYG{n}{u}\PYG{o}{.}\PYG{n}{m}\PYG{p}{)}\PYG{o}{*}\PYG{n}{G}\PYG{p}{)}\PYG{p}{)}\PYG{o}{*}\PYG{o}{*}\PYG{p}{(}\PYG{l+m+mi}{1}\PYG{o}{/}\PYG{l+m+mi}{3}\PYG{p}{)}\PYG{p}{)}\PYG{o}{.}\PYG{n}{to\PYGZus{}base\PYGZus{}units}\PYG{p}{(}\PYG{p}{)}

\PYG{n}{fig}\PYG{p}{,} \PYG{n}{ax} \PYG{o}{=} \PYG{n}{plt}\PYG{o}{.}\PYG{n}{subplots}\PYG{p}{(}\PYG{p}{)}
\PYG{k}{for} \PYG{n}{i} \PYG{o+ow}{in} \PYG{n+nb}{range}\PYG{p}{(}\PYG{n+nb}{len}\PYG{p}{(}\PYG{n}{T\PYGZus{}graph}\PYG{p}{)}\PYG{p}{)}\PYG{p}{:}
  \PYG{n}{ax}\PYG{o}{.}\PYG{n}{semilogx}\PYG{p}{(}\PYG{n}{G}\PYG{p}{,}\PYG{n}{L\PYGZus{}Diff}\PYG{p}{(}\PYG{n}{T\PYGZus{}graph}\PYG{p}{[}\PYG{n}{i}\PYG{p}{]}\PYG{p}{,}\PYG{n}{G}\PYG{p}{)}\PYG{o}{.}\PYG{n}{to}\PYG{p}{(}\PYG{n}{u}\PYG{o}{.}\PYG{n}{nm}\PYG{p}{)}\PYG{p}{)}

\PYG{n}{ax}\PYG{o}{.}\PYG{n}{legend}\PYG{p}{(}\PYG{n}{T\PYGZus{}graph}\PYG{p}{)}
\PYG{n}{ax}\PYG{o}{.}\PYG{n}{yaxis}\PYG{o}{.}\PYG{n}{set\PYGZus{}major\PYGZus{}formatter}\PYG{p}{(}\PYG{n}{FormatStrFormatter}\PYG{p}{(}\PYG{l+s+s1}{\PYGZsq{}}\PYG{l+s+s1}{\PYGZpc{}}\PYG{l+s+s1}{.f}\PYG{l+s+s1}{\PYGZsq{}}\PYG{p}{)}\PYG{p}{)}
\PYG{n}{ax}\PYG{o}{.}\PYG{n}{xaxis}\PYG{o}{.}\PYG{n}{set\PYGZus{}major\PYGZus{}formatter}\PYG{p}{(}\PYG{n}{FormatStrFormatter}\PYG{p}{(}\PYG{l+s+s1}{\PYGZsq{}}\PYG{l+s+s1}{\PYGZpc{}}\PYG{l+s+s1}{.f}\PYG{l+s+s1}{\PYGZsq{}}\PYG{p}{)}\PYG{p}{)}
\PYG{n}{ax}\PYG{o}{.}\PYG{n}{set}\PYG{p}{(}\PYG{n}{xlabel}\PYG{o}{=}\PYG{l+s+s1}{\PYGZsq{}}\PYG{l+s+s1}{Velocity gradient (Hz)}\PYG{l+s+s1}{\PYGZsq{}}\PYG{p}{,} \PYG{n}{ylabel}\PYG{o}{=}\PYG{l+s+s1}{\PYGZsq{}}\PYG{l+s+s1}{Diffusion band thickness (\PYGZdl{}nm\PYGZdl{})}\PYG{l+s+s1}{\PYGZsq{}}\PYG{p}{)}
\PYG{n}{fig}\PYG{o}{.}\PYG{n}{savefig}\PYG{p}{(}\PYG{n}{imagepath}\PYG{o}{+}\PYG{l+s+s1}{\PYGZsq{}}\PYG{l+s+s1}{Diffusion\PYGZus{}band\PYGZus{}thickness}\PYG{l+s+s1}{\PYGZsq{}}\PYG{p}{)}
\PYG{n}{plt}\PYG{o}{.}\PYG{n}{show}\PYG{p}{(}\PYG{p}{)}
\end{sphinxVerbatim}

\begin{figure}[htbp]
\centering
\capstart

\noindent\sphinxincludegraphics[width=400\sphinxpxdimen]{{Diffusion_band_thickness}.png}
\caption{Molecular diffusion band thickness as a function of velocity gradient. This length scale marks the transition between transport by fluid deformation and by diffusion.}\label{\detokenize{Rapid_Mix/RM_Theory_and_Future_Work:id1}}\label{\detokenize{Rapid_Mix/RM_Theory_and_Future_Work:figure-diffusion-band-thickness}}\end{figure}

Using the equation for \(L_{Diff}\) above, we can solve for  the time required to reach a target efficiency of application of coagulant nanoparticles to clay:
\begin{equation}\label{equation:Rapid_Mix/RM_Theory_and_Future_Work:Rapid_Mix/RM_Theory_and_Future_Work:27}
\begin{split}t_{coagulant, \, application} = \frac{2.3p C_{NC} \, \Lambda_{Clay}^2}{\pi G k \, d_{Clay}\,  L_{Diff_{NC}} }\end{split}
\end{equation}
The time required for the coagulant to be transported to clay surfaces is strongly dependent on the turbidity as indicated by the average spacing of clay particles, \(\Lambda_{Clay}\). As turbidity increases the spacing between clay particles decreases and the time required for shear to transport coagulant nanoparticles to the clay decreases. Increasing the shear also results in faster transport of the coagulant nanoparticles to clay surfaces. The times required are strongly influenced by the size of the coagulant nanoparticles because larger nanoparticles diffuse more slowly.

Below we estimate the time required to achieve 80\% attachment of nanoparticles in a 10 NTU clay suspension.

\fvset{hllines={, ,}}%
\begin{sphinxVerbatim}[commandchars=\\\{\}]
\PYG{l+s+sd}{\PYGZdq{}\PYGZdq{}\PYGZdq{}I needed to attach units to material properties due to a bug in floc\PYGZus{}model. This will need to be fixed when floc\PYGZus{}model is updated.\PYGZdq{}\PYGZdq{}\PYGZdq{}}
\PYG{k}{def} \PYG{n+nf}{Nano\PYGZus{}coag\PYGZus{}attach\PYGZus{}time}\PYG{p}{(}\PYG{n}{pC\PYGZus{}NC}\PYG{p}{,}\PYG{n}{C\PYGZus{}clay}\PYG{p}{,}\PYG{n}{G}\PYG{p}{,}\PYG{n}{Temperature}\PYG{p}{)}\PYG{p}{:}
  \PYG{l+s+sd}{\PYGZdq{}\PYGZdq{}\PYGZdq{}We assume that 70\PYGZpc{} of nanoparticles attach in the average time for one collision.\PYGZdq{}\PYGZdq{}\PYGZdq{}}
  \PYG{n}{k\PYGZus{}nano} \PYG{o}{=} \PYG{l+m+mi}{1}\PYG{o}{\PYGZhy{}}\PYG{n}{np}\PYG{o}{.}\PYG{n}{exp}\PYG{p}{(}\PYG{o}{\PYGZhy{}}\PYG{l+m+mi}{1}\PYG{p}{)}
  \PYG{n}{num}\PYG{o}{=}\PYG{l+m+mf}{2.3}\PYG{o}{*}\PYG{n}{pC\PYGZus{}NC}\PYG{o}{*}\PYG{p}{(}\PYG{n}{fm}\PYG{o}{.}\PYG{n}{sep\PYGZus{}dist\PYGZus{}clay}\PYG{p}{(}\PYG{n}{C\PYGZus{}clay}\PYG{p}{,}\PYG{n}{fm}\PYG{o}{.}\PYG{n}{Clay}\PYG{p}{)}\PYG{p}{)}\PYG{o}{*}\PYG{o}{*}\PYG{l+m+mi}{2}
  \PYG{n}{den} \PYG{o}{=} \PYG{n}{np}\PYG{o}{.}\PYG{n}{pi} \PYG{o}{*} \PYG{n}{G}\PYG{o}{*} \PYG{n}{k\PYGZus{}nano} \PYG{o}{*} \PYG{n}{fm}\PYG{o}{.}\PYG{n}{Clay}\PYG{o}{.}\PYG{n}{Diameter}\PYG{o}{*}\PYG{n}{u}\PYG{o}{.}\PYG{n}{m} \PYG{o}{*} \PYG{n}{L\PYGZus{}Diff}\PYG{p}{(}\PYG{n}{Temperature}\PYG{p}{,}\PYG{n}{G}\PYG{p}{)}
  \PYG{k}{return} \PYG{p}{(}\PYG{n}{num}\PYG{o}{/}\PYG{n}{den}\PYG{p}{)}\PYG{o}{.}\PYG{n}{to\PYGZus{}base\PYGZus{}units}\PYG{p}{(}\PYG{p}{)}

\PYG{n}{C\PYGZus{}Al} \PYG{o}{=} \PYG{l+m+mi}{2} \PYG{o}{*} \PYG{n}{u}\PYG{o}{.}\PYG{n}{mg}\PYG{o}{/}\PYG{n}{u}\PYG{o}{.}\PYG{n}{L}
\PYG{n}{C\PYGZus{}clay} \PYG{o}{=} \PYG{l+m+mi}{10} \PYG{o}{*} \PYG{n}{u}\PYG{o}{.}\PYG{n}{NTU}
\PYG{n}{pC\PYGZus{}NC} \PYG{o}{=} \PYG{o}{\PYGZhy{}}\PYG{n}{np}\PYG{o}{.}\PYG{n}{log10}\PYG{p}{(}\PYG{l+m+mi}{1}\PYG{o}{\PYGZhy{}}\PYG{l+m+mf}{0.8}\PYG{p}{)}
\PYG{l+s+sd}{\PYGZdq{}\PYGZdq{}\PYGZdq{}apply 80\PYGZpc{} of the coagulant nanoparticles to the clay\PYGZdq{}\PYGZdq{}\PYGZdq{}}

\PYG{n}{G} \PYG{o}{=} \PYG{n}{np}\PYG{o}{.}\PYG{n}{arange}\PYG{p}{(}\PYG{l+m+mi}{50}\PYG{p}{,}\PYG{l+m+mi}{5000}\PYG{p}{,}\PYG{l+m+mi}{10}\PYG{p}{)}\PYG{o}{*}\PYG{n}{u}\PYG{o}{.}\PYG{n}{Hz}

\PYG{n}{fig}\PYG{p}{,} \PYG{n}{ax} \PYG{o}{=} \PYG{n}{plt}\PYG{o}{.}\PYG{n}{subplots}\PYG{p}{(}\PYG{p}{)}

\PYG{k}{for} \PYG{n}{i} \PYG{o+ow}{in} \PYG{n+nb}{range}\PYG{p}{(}\PYG{n+nb}{len}\PYG{p}{(}\PYG{n}{T\PYGZus{}graph}\PYG{p}{)}\PYG{p}{)}\PYG{p}{:}
  \PYG{n}{ax}\PYG{o}{.}\PYG{n}{semilogx}\PYG{p}{(}\PYG{n}{G}\PYG{p}{,}\PYG{n}{Nano\PYGZus{}coag\PYGZus{}attach\PYGZus{}time}\PYG{p}{(}\PYG{n}{pC\PYGZus{}NC}\PYG{p}{,}\PYG{n}{C\PYGZus{}clay}\PYG{p}{,}\PYG{n}{G}\PYG{p}{,}\PYG{n}{T\PYGZus{}graph}\PYG{p}{[}\PYG{n}{i}\PYG{p}{]}\PYG{p}{)}\PYG{p}{)}

\PYG{n}{ax}\PYG{o}{.}\PYG{n}{semilogx}\PYG{p}{(}\PYG{n}{Mix\PYGZus{}G}\PYG{o}{.}\PYG{n}{to}\PYG{p}{(}\PYG{l+m+mi}{1}\PYG{o}{/}\PYG{n}{u}\PYG{o}{.}\PYG{n}{s}\PYG{p}{)}\PYG{p}{,}\PYG{n}{Mix\PYGZus{}HRT}\PYG{o}{.}\PYG{n}{to}\PYG{p}{(}\PYG{n}{u}\PYG{o}{.}\PYG{n}{s}\PYG{p}{)}\PYG{p}{,}\PYG{l+s+s1}{\PYGZsq{}}\PYG{l+s+s1}{o}\PYG{l+s+s1}{\PYGZsq{}}\PYG{p}{)}
\PYG{n}{ax}\PYG{o}{.}\PYG{n}{legend}\PYG{p}{(}\PYG{p}{[}\PYG{o}{*}\PYG{n}{T\PYGZus{}graph}\PYG{p}{,} \PYG{l+s+s2}{\PYGZdq{}}\PYG{l+s+s2}{Conventional rapid mix}\PYG{l+s+s2}{\PYGZdq{}}\PYG{p}{]}\PYG{p}{)}
\PYG{l+s+sd}{\PYGZdq{}\PYGZdq{}\PYGZdq{}* is used to unpack T\PYGZus{}graph so that units are preserved when adding another legend item.\PYGZdq{}\PYGZdq{}\PYGZdq{}}
\PYG{n}{ax}\PYG{o}{.}\PYG{n}{yaxis}\PYG{o}{.}\PYG{n}{set\PYGZus{}major\PYGZus{}formatter}\PYG{p}{(}\PYG{n}{FormatStrFormatter}\PYG{p}{(}\PYG{l+s+s1}{\PYGZsq{}}\PYG{l+s+s1}{\PYGZpc{}}\PYG{l+s+s1}{.f}\PYG{l+s+s1}{\PYGZsq{}}\PYG{p}{)}\PYG{p}{)}
\PYG{n}{ax}\PYG{o}{.}\PYG{n}{xaxis}\PYG{o}{.}\PYG{n}{set\PYGZus{}major\PYGZus{}formatter}\PYG{p}{(}\PYG{n}{FormatStrFormatter}\PYG{p}{(}\PYG{l+s+s1}{\PYGZsq{}}\PYG{l+s+s1}{\PYGZpc{}}\PYG{l+s+s1}{.f}\PYG{l+s+s1}{\PYGZsq{}}\PYG{p}{)}\PYG{p}{)}
\PYG{n}{ax}\PYG{o}{.}\PYG{n}{set}\PYG{p}{(}\PYG{n}{xlabel}\PYG{o}{=}\PYG{l+s+s1}{\PYGZsq{}}\PYG{l+s+s1}{Velocity gradient (Hz)}\PYG{l+s+s1}{\PYGZsq{}}\PYG{p}{,} \PYG{n}{ylabel}\PYG{o}{=}\PYG{l+s+s1}{\PYGZsq{}}\PYG{l+s+s1}{Nanoparticle attachment time (s)}\PYG{l+s+s1}{\PYGZsq{}}\PYG{p}{)}
\PYG{n}{fig}\PYG{o}{.}\PYG{n}{savefig}\PYG{p}{(}\PYG{n}{imagepath}\PYG{o}{+}\PYG{l+s+s1}{\PYGZsq{}}\PYG{l+s+s1}{Coag\PYGZus{}attach\PYGZus{}time}\PYG{l+s+s1}{\PYGZsq{}}\PYG{p}{)}
\PYG{n}{plt}\PYG{o}{.}\PYG{n}{show}\PYG{p}{(}\PYG{p}{)}
\end{sphinxVerbatim}

\begin{figure}[htbp]
\centering
\capstart

\noindent\sphinxincludegraphics[width=400\sphinxpxdimen]{{Coag_attach_time}.png}
\caption{An estimate of the time required for 80\% of the coagulant nanoparticles to attach to clay particles given a raw water turbidity of 10 NTU.}\label{\detokenize{Rapid_Mix/RM_Theory_and_Future_Work:id2}}\label{\detokenize{Rapid_Mix/RM_Theory_and_Future_Work:figure-coag-attach-time}}\end{figure}


\subsection{Energy Tradeoff for Coagulant Transport}
\label{\detokenize{Rapid_Mix/RM_Theory_and_Future_Work:energy-tradeoff-for-coagulant-transport}}\label{\detokenize{Rapid_Mix/RM_Theory_and_Future_Work:heading-energy-tradeoff-for-coagulant-transport}}\begin{equation}\label{equation:Rapid_Mix/RM_Theory_and_Future_Work:Rapid_Mix/RM_Theory_and_Future_Work:28}
\begin{split}\Delta h =   \frac{G^2 \nu \theta}{g}\end{split}
\end{equation}
\fvset{hllines={, ,}}%
\begin{sphinxVerbatim}[commandchars=\\\{\}]
\PYG{n}{Nano\PYGZus{}attach\PYGZus{}time} \PYG{o}{=} \PYG{n}{Nano\PYGZus{}coag\PYGZus{}attach\PYGZus{}time}\PYG{p}{(}\PYG{n}{pC\PYGZus{}NC}\PYG{p}{,}\PYG{n}{C\PYGZus{}clay}\PYG{p}{,}\PYG{n}{G}\PYG{p}{,}\PYG{n}{Temperature}\PYG{p}{)}

\PYG{k}{def} \PYG{n+nf}{HL\PYGZus{}coag\PYGZus{}attach}\PYG{p}{(}\PYG{n}{pC\PYGZus{}NC}\PYG{p}{,}\PYG{n}{C\PYGZus{}clay}\PYG{p}{,}\PYG{n}{G}\PYG{p}{,}\PYG{n}{Temperature}\PYG{p}{)}\PYG{p}{:}
  \PYG{k}{return} \PYG{p}{(}\PYG{n}{G}\PYG{o}{*}\PYG{o}{*}\PYG{l+m+mi}{2}\PYG{o}{*}\PYG{n}{pc}\PYG{o}{.}\PYG{n}{viscosity\PYGZus{}kinematic}\PYG{p}{(}\PYG{n}{Temperature}\PYG{p}{)}\PYG{o}{*}\PYG{n}{Nano\PYGZus{}attach\PYGZus{}time}\PYG{o}{/}\PYG{n}{u}\PYG{o}{.}\PYG{n}{gravity}\PYG{p}{)}\PYG{o}{.}\PYG{n}{to}\PYG{p}{(}\PYG{n}{u}\PYG{o}{.}\PYG{n}{cm}\PYG{p}{)}

\PYG{n}{fig}\PYG{p}{,} \PYG{n}{ax} \PYG{o}{=} \PYG{n}{plt}\PYG{o}{.}\PYG{n}{subplots}\PYG{p}{(}\PYG{p}{)}

\PYG{k}{for} \PYG{n}{i} \PYG{o+ow}{in} \PYG{n+nb}{range}\PYG{p}{(}\PYG{n+nb}{len}\PYG{p}{(}\PYG{n}{T\PYGZus{}graph}\PYG{p}{)}\PYG{p}{)}\PYG{p}{:}
  \PYG{n}{ax}\PYG{o}{.}\PYG{n}{loglog}\PYG{p}{(}\PYG{n}{G}\PYG{p}{,}\PYG{n}{HL\PYGZus{}coag\PYGZus{}attach}\PYG{p}{(}\PYG{n}{pC\PYGZus{}NC}\PYG{p}{,}\PYG{n}{C\PYGZus{}clay}\PYG{p}{,}\PYG{n}{G}\PYG{p}{,}\PYG{n}{T\PYGZus{}graph}\PYG{p}{[}\PYG{n}{i}\PYG{p}{]}\PYG{p}{)}\PYG{p}{)}

\PYG{n}{ax}\PYG{o}{.}\PYG{n}{legend}\PYG{p}{(}\PYG{n}{T\PYGZus{}graph}\PYG{p}{)}
\PYG{n}{ax}\PYG{o}{.}\PYG{n}{yaxis}\PYG{o}{.}\PYG{n}{set\PYGZus{}major\PYGZus{}formatter}\PYG{p}{(}\PYG{n}{FormatStrFormatter}\PYG{p}{(}\PYG{l+s+s1}{\PYGZsq{}}\PYG{l+s+s1}{\PYGZpc{}}\PYG{l+s+s1}{.f}\PYG{l+s+s1}{\PYGZsq{}}\PYG{p}{)}\PYG{p}{)}
\PYG{n}{ax}\PYG{o}{.}\PYG{n}{xaxis}\PYG{o}{.}\PYG{n}{set\PYGZus{}major\PYGZus{}formatter}\PYG{p}{(}\PYG{n}{FormatStrFormatter}\PYG{p}{(}\PYG{l+s+s1}{\PYGZsq{}}\PYG{l+s+s1}{\PYGZpc{}}\PYG{l+s+s1}{.f}\PYG{l+s+s1}{\PYGZsq{}}\PYG{p}{)}\PYG{p}{)}
\PYG{n}{ax}\PYG{o}{.}\PYG{n}{set}\PYG{p}{(}\PYG{n}{xlabel}\PYG{o}{=}\PYG{l+s+s1}{\PYGZsq{}}\PYG{l+s+s1}{Velocity gradient (Hz)}\PYG{l+s+s1}{\PYGZsq{}}\PYG{p}{,} \PYG{n}{ylabel}\PYG{o}{=}\PYG{l+s+s1}{\PYGZsq{}}\PYG{l+s+s1}{Head loss (cm)}\PYG{l+s+s1}{\PYGZsq{}}\PYG{p}{)}
\PYG{n}{fig}\PYG{o}{.}\PYG{n}{savefig}\PYG{p}{(}\PYG{n}{imagepath}\PYG{o}{+}\PYG{l+s+s1}{\PYGZsq{}}\PYG{l+s+s1}{Coag\PYGZus{}attach\PYGZus{}head\PYGZus{}loss}\PYG{l+s+s1}{\PYGZsq{}}\PYG{p}{)}
\PYG{n}{plt}\PYG{o}{.}\PYG{n}{show}\PYG{p}{(}\PYG{p}{)}
\end{sphinxVerbatim}

\begin{figure}[htbp]
\centering
\capstart

\noindent\sphinxincludegraphics[width=400\sphinxpxdimen]{{Coag_attach_head_loss}.png}
\caption{The total energy required to attach coagulant nanoparticles to raw water inorganic particles increases rapidly with the velocity gradient used in the rapid mix process.}\label{\detokenize{Rapid_Mix/RM_Theory_and_Future_Work:id3}}\label{\detokenize{Rapid_Mix/RM_Theory_and_Future_Work:figure-coag-attach-head-loss}}\end{figure}

There is an economic tradeoff between reactor volume and energy input. The reactor volume results in a higher capital cost and the energy input requires both higher operating costs and higher capital costs. This provides an opportunity to optimize rapid mix design once we have a confirmed model characterizing the process.

The total potential energy used to operate an AguaClara plant is approximately 2 m. This represents the difference in elevation between where the raw water enters the plant and where the filtered water exits the plant. If we assume that the rapid mix energy budget is a fraction of that total and thus for subsequent analysis we will assume somewhat arbitrarily that the energy available to attach the coagulant nanoparticles to the raw water particles is 50 cm.

We solve the coagulant transport model,
\(t_{coagulant, \, application} = \frac{2.3p C_{NC} \, \Lambda_{Clay}^2}{\pi G k \, d_{Clay}\, L_{Diff_{NC}} }\),
for G given a head loss.
\begin{equation}\label{equation:Rapid_Mix/RM_Theory_and_Future_Work:Rapid_Mix/RM_Theory_and_Future_Work:29}
\begin{split}G_{coagulant, \, application} =  d_{Clay}\left(\frac{\pi k \,g\Delta h }{2.3p C_{NC} \, \Lambda_{Clay}^2 \nu} \right)^\frac{3}{4} \left( \frac{2k_B T }{3 \pi \,\mu  \, d_{NC} }\right)^\frac{1}{4}\end{split}
\end{equation}
\fvset{hllines={, ,}}%
\begin{sphinxVerbatim}[commandchars=\\\{\}]
\PYG{l+s+sd}{\PYGZdq{}\PYGZdq{}\PYGZdq{}find G for target head loss\PYGZdq{}\PYGZdq{}\PYGZdq{}}
\PYG{n}{HL\PYGZus{}nano\PYGZus{}transport} \PYG{o}{=} \PYG{n}{np}\PYG{o}{.}\PYG{n}{linspace}\PYG{p}{(}\PYG{l+m+mi}{10}\PYG{p}{,}\PYG{l+m+mi}{100}\PYG{p}{,}\PYG{l+m+mi}{10}\PYG{p}{)}\PYG{o}{*}\PYG{n}{u}\PYG{o}{.}\PYG{n}{cm}
\PYG{k}{def} \PYG{n+nf}{G\PYGZus{}max\PYGZus{}head\PYGZus{}loss}\PYG{p}{(}\PYG{n}{pC\PYGZus{}NC}\PYG{p}{,}\PYG{n}{C\PYGZus{}clay}\PYG{p}{,}\PYG{n}{HL\PYGZus{}nano\PYGZus{}transport}\PYG{p}{,}\PYG{n}{Temperature}\PYG{p}{)}\PYG{p}{:}
  \PYG{n}{k\PYGZus{}nano} \PYG{o}{=} \PYG{l+m+mi}{1}\PYG{o}{\PYGZhy{}}\PYG{n}{np}\PYG{o}{.}\PYG{n}{exp}\PYG{p}{(}\PYG{o}{\PYGZhy{}}\PYG{l+m+mi}{1}\PYG{p}{)}
  \PYG{n}{num} \PYG{o}{=} \PYG{n}{u}\PYG{o}{.}\PYG{n}{gravity} \PYG{o}{*} \PYG{n}{HL\PYGZus{}nano\PYGZus{}transport} \PYG{o}{*} \PYG{n}{np}\PYG{o}{.}\PYG{n}{pi} \PYG{o}{*} \PYG{n}{k\PYGZus{}nano}
  \PYG{n}{den}\PYG{o}{=} \PYG{l+m+mf}{2.3} \PYG{o}{*} \PYG{n}{pC\PYGZus{}NC} \PYG{o}{*} \PYG{p}{(}\PYG{n}{fm}\PYG{o}{.}\PYG{n}{sep\PYGZus{}dist\PYGZus{}clay}\PYG{p}{(}\PYG{n}{C\PYGZus{}clay}\PYG{p}{,}\PYG{n}{fm}\PYG{o}{.}\PYG{n}{Clay}\PYG{p}{)}\PYG{p}{)}\PYG{o}{*}\PYG{o}{*}\PYG{l+m+mi}{2} \PYG{o}{*} \PYG{n}{pc}\PYG{o}{.}\PYG{n}{viscosity\PYGZus{}kinematic}\PYG{p}{(}\PYG{n}{Temperature}\PYG{p}{)}
  \PYG{n}{num2} \PYG{o}{=} \PYG{l+m+mi}{2} \PYG{o}{*} \PYG{n}{u}\PYG{o}{.}\PYG{n}{boltzmann\PYGZus{}constant} \PYG{o}{*} \PYG{n}{Temperature}
  \PYG{n}{den2} \PYG{o}{=} \PYG{l+m+mi}{3} \PYG{o}{*} \PYG{n}{np}\PYG{o}{.}\PYG{n}{pi} \PYG{o}{*} \PYG{n}{pc}\PYG{o}{.}\PYG{n}{viscosity\PYGZus{}dynamic}\PYG{p}{(}\PYG{n}{Temperature}\PYG{p}{)} \PYG{o}{*} \PYG{p}{(}\PYG{n}{fm}\PYG{o}{.}\PYG{n}{PACl}\PYG{o}{.}\PYG{n}{Diameter}\PYG{o}{*}\PYG{n}{u}\PYG{o}{.}\PYG{n}{m}\PYG{p}{)}
  \PYG{k}{return} \PYG{n}{fm}\PYG{o}{.}\PYG{n}{Clay}\PYG{o}{.}\PYG{n}{Diameter}\PYG{o}{*}\PYG{n}{u}\PYG{o}{.}\PYG{n}{m}\PYG{o}{*}\PYG{p}{(}\PYG{p}{(}\PYG{p}{(}\PYG{p}{(}\PYG{n}{num}\PYG{o}{/}\PYG{n}{den}\PYG{p}{)}\PYG{o}{*}\PYG{o}{*}\PYG{p}{(}\PYG{l+m+mi}{3}\PYG{p}{)} \PYG{o}{*} \PYG{p}{(}\PYG{n}{num2}\PYG{o}{/}\PYG{n}{den2}\PYG{p}{)}\PYG{p}{)}\PYG{o}{.}\PYG{n}{to\PYGZus{}base\PYGZus{}units}\PYG{p}{(}\PYG{p}{)}\PYG{p}{)}\PYG{o}{*}\PYG{o}{*}\PYG{p}{(}\PYG{l+m+mi}{1}\PYG{o}{/}\PYG{l+m+mi}{4}\PYG{p}{)}\PYG{p}{)}
\PYG{l+s+sd}{\PYGZdq{}\PYGZdq{}\PYGZdq{}Note the use of to\PYGZus{}base\PYGZus{}units BEFORE raising to the fractional power.}
\PYG{l+s+sd}{This prevents a rounding error in the unit exponent.\PYGZdq{}\PYGZdq{}\PYGZdq{}}

\PYG{n}{G\PYGZus{}max} \PYG{o}{=} \PYG{n}{G\PYGZus{}max\PYGZus{}head\PYGZus{}loss}\PYG{p}{(}\PYG{n}{pC\PYGZus{}NC}\PYG{p}{,}\PYG{n}{C\PYGZus{}clay}\PYG{p}{,}\PYG{l+m+mi}{20}\PYG{o}{*}\PYG{n}{u}\PYG{o}{.}\PYG{n}{cm}\PYG{p}{,}\PYG{n}{Temperature}\PYG{p}{)}
\PYG{n+nb}{print}\PYG{p}{(}\PYG{n}{G\PYGZus{}max}\PYG{p}{)}

\PYG{l+s+sd}{\PYGZdq{}\PYGZdq{}\PYGZdq{}The time required?\PYGZdq{}\PYGZdq{}\PYGZdq{}}
\PYG{n}{Nano\PYGZus{}attach\PYGZus{}time} \PYG{o}{=} \PYG{n}{Nano\PYGZus{}coag\PYGZus{}attach\PYGZus{}time}\PYG{p}{(}\PYG{n}{pC\PYGZus{}NC}\PYG{p}{,}\PYG{n}{C\PYGZus{}clay}\PYG{p}{,}\PYG{n}{G\PYGZus{}max}\PYG{p}{,}\PYG{n}{Temperature}\PYG{p}{)}
\PYG{n+nb}{print}\PYG{p}{(}\PYG{n}{Nano\PYGZus{}attach\PYGZus{}time}\PYG{p}{)}
\PYG{n+nb}{print}\PYG{p}{(}\PYG{n}{G\PYGZus{}max}\PYG{o}{*}\PYG{n}{Nano\PYGZus{}attach\PYGZus{}time}\PYG{p}{)}
\end{sphinxVerbatim}

According to the analysis above, the maximum velocity gradient that can be used to achieve 80\% coagulant nanoparticle attachment using only 20 cm of head loss is 142 Hz. This requires a residence time of 100 seconds. These model results must be experimentally verified and it is very likely that the model will need to be modified.

The analysis of the time required for shear and diffusion to transport the coagulant nanoparticles the last few millimeters suggests that it is this last step that requires the most time. Indeed, the time required for coagulant nanoparticle attachment to raw water particles is comparable to the time that will be required for the next step in the processs, flocculation.


\section{Coagulant Attachment Mechanism}
\label{\detokenize{Rapid_Mix/RM_Theory_and_Future_Work:coagulant-attachment-mechanism}}\label{\detokenize{Rapid_Mix/RM_Theory_and_Future_Work:heading-coagulant-attachment-mechanism}}
We do not yet understand the origin of the bonds that form between coagulant nanoparticles, between a coagulant nanoparticle and suspended particles, and between coagulant nanoparticles and dissolved organic molecules. Historically the role of the coagulant was assumed to be to reduce the repulsive force between particles so that the particles could get close enough for Van der Waals forces to hold the particles together.
\begin{itemize}
\item {} 
Surface charge neutralization hypothesis
\begin{itemize}
\item {} 
coagulant nanoparticles attach to each other

\item {} 
\end{itemize}

\item {} 
Polar bonds
\begin{itemize}
\item {} 
Electronegativity reveals that the aluminum - oxygen bond is more polar than the hydrogen - oxygen bond

\item {} 
The bond between a coagulant nanoparticle and a clay surface can potentially be stronger than the bond between a water molecule and the clay surface.

\end{itemize}

\end{itemize}
\phantomsection\label{\detokenize{Rapid_Mix/RM_Theory_and_Future_Work:heading-conventional-mechanical-rapid-mix}}

\chapter{Flocculation  Introduction}
\label{\detokenize{Flocculation/Floc_Intro:flocculation-introduction}}\label{\detokenize{Flocculation/Floc_Intro:title-flocculation-introduction}}\label{\detokenize{Flocculation/Floc_Intro::doc}}

\section{Flocculation}
\label{\detokenize{Flocculation/Floc_Intro:flocculation}}
Flocculation transform inorganic (clays such as \sphinxhref{https://www.sciencedirect.com/science/article/pii/S0048969708010103}{kaolinite, smectite, etc. and metallic oxy-hydroxides such as goethite and gibbsite}) and organic (viruses, bacteria and protozoa) primary particles into flocs (particle aggregates). Flocculation doesn’t remove any particles from suspension. Instead it causes particle aggregation and then floc blankets, lamellar sedimentation, and sand filtration will be used to separate those flocs from the water. Sedimentation can remove flocs more easily than it can remove primary particles because flocs have a higher terminal sedimentation velocity. Floc blankets and sand filtration rely primarily on capture based on interception and interception is much more efficient when the particles are larger. Thus the purpose of flocculation is to join \sphinxstylestrong{all} of the primary particles together into flocs.

It is also possible that a difference in a physical property between primary particles and flocs plays a role in enhanced removal of flocs in floc blankets and filters. For example, the many relatively weak connection points between the primary particles in the flocs enables the flocs to deform. It is possible that deformation plays an important role right at the moment of collision. Presumably the bond strength required to lock the colliding particles together is less if the particles can deform as they are colliding.


\subsection{Primary particles can’t attach to large flocs}
\label{\detokenize{Flocculation/Floc_Intro:primary-particles-cant-attach-to-large-flocs}}
One of the mysteries of flocculation has been why it is such a slow process and yet it appears to be a very rapid process. Plant operators observe that with high raw water turbidities that they can see flocculation progressing after about 0.5 minutes of flocculation. We can estimate the collision potential, \(G\theta\) that corresponds to making visible flocs.
\begin{equation}\label{equation:Flocculation/Floc_Intro:Flocculation/Floc_Intro:0}
\begin{split}\bar G = \sqrt{ \frac{g h_e}{\theta \nu}}\end{split}
\end{equation}
\fvset{hllines={, ,}}%
\begin{sphinxVerbatim}[commandchars=\\\{\}]
\PYG{k+kn}{from} \PYG{n+nn}{aide\PYGZus{}design}\PYG{n+nn}{.}\PYG{n+nn}{play} \PYG{k}{import}\PYG{o}{*}
\PYG{k+kn}{from} \PYG{n+nn}{aguaclara\PYGZus{}research}\PYG{n+nn}{.}\PYG{n+nn}{play} \PYG{k}{import}\PYG{o}{*}
\PYG{k+kn}{from} \PYG{n+nn}{pytexit} \PYG{k}{import} \PYG{n}{py2tex}
\PYG{k+kn}{from} \PYG{n+nn}{sympy} \PYG{k}{import}\PYG{o}{*}
\PYG{k+kn}{from} \PYG{n+nn}{scipy}\PYG{n+nn}{.}\PYG{n+nn}{optimize} \PYG{k}{import} \PYG{n}{root}
\PYG{k+kn}{from} \PYG{n+nn}{scipy}\PYG{n+nn}{.}\PYG{n+nn}{optimize} \PYG{k}{import} \PYG{n}{brentq}
\PYG{k+kn}{import} \PYG{n+nn}{pandas} \PYG{k}{as} \PYG{n+nn}{pd}
\PYG{n}{HL\PYGZus{}floc} \PYG{o}{=} \PYG{l+m+mi}{43}\PYG{o}{*}\PYG{n}{u}\PYG{o}{.}\PYG{n}{cm}
\PYG{n}{HRT} \PYG{o}{=} \PYG{l+m+mi}{8} \PYG{o}{*} \PYG{n}{u}\PYG{o}{.}\PYG{n}{min}
\PYG{n}{Temperature} \PYG{o}{=}\PYG{l+m+mi}{20} \PYG{o}{*} \PYG{n}{u}\PYG{o}{.}\PYG{n}{degC}
\PYG{n}{G\PYGZus{}floc} \PYG{o}{=} \PYG{p}{(}\PYG{p}{(}\PYG{n}{pc}\PYG{o}{.}\PYG{n}{gravity}\PYG{o}{*}\PYG{n}{HL\PYGZus{}floc}\PYG{o}{/}\PYG{p}{(}\PYG{n}{HRT}\PYG{o}{*}\PYG{n}{pc}\PYG{o}{.}\PYG{n}{viscosity\PYGZus{}kinematic}\PYG{p}{(}\PYG{n}{Temperature}\PYG{p}{)}\PYG{p}{)}\PYG{p}{)}\PYG{o}{*}\PYG{o}{*}\PYG{l+m+mf}{0.5}\PYG{p}{)}\PYG{o}{.}\PYG{n}{to\PYGZus{}base\PYGZus{}units}\PYG{p}{(}\PYG{p}{)}
\PYG{n+nb}{print}\PYG{p}{(}\PYG{n}{G\PYGZus{}floc}\PYG{p}{)}
\PYG{n}{Gt\PYGZus{}floc} \PYG{o}{=} \PYG{n}{G\PYGZus{}floc}\PYG{o}{*}\PYG{n}{HRT}
\PYG{n}{HRT\PYGZus{}floc\PYGZus{}visible} \PYG{o}{=} \PYG{l+m+mf}{0.5}\PYG{o}{*}\PYG{n}{u}\PYG{o}{.}\PYG{n}{min}
\PYG{n}{Gt\PYGZus{}floc\PYGZus{}visible} \PYG{o}{=} \PYG{p}{(}\PYG{n}{G\PYGZus{}floc}\PYG{o}{*}\PYG{n}{HRT\PYGZus{}floc\PYGZus{}visible}\PYG{p}{)}\PYG{o}{.}\PYG{n}{to\PYGZus{}base\PYGZus{}units}\PYG{p}{(}\PYG{p}{)}
\PYG{n+nb}{print}\PYG{p}{(}\PYG{n}{Gt\PYGZus{}floc\PYGZus{}visible}\PYG{p}{)}
\end{sphinxVerbatim}

Here initial flocculation is visible at a \(G\theta\) of less than 3000. Given that flocculation is visible at this low collision potential, it is unclear why recommended \(G\theta\) are as high as 100,000. This is one of the great mysteries that motivated the search for a flocculation model that was based on hypotheses that were consistent with laboratory and field observations.


\section{History}
\label{\detokenize{Flocculation/Floc_Intro:history}}\begin{description}
\item[{The mechanism of particle-particle aggregation was thought to be controlled by an average surface charge. Apparently no one was able to develop a model of how that mechanism would influence particle attachment efficiency and the result was that no predictive models for flocculation were developed. There were several observations that were at odds with conventional explanations of flocculation.}] \leavevmode
1. Efficient flocculation at coagulant dosages that led to positive surface charge. This led to a second flocculation mechanism that was called “sweep floc” and that was used to describe any observations that didn’t fit the charge neutralization flocculation hypotheses
1. Flocculation time for highly turbid suspensions was expected to proceed very rapidly and produce very low turbidity settled water. This expectation was not observed and led to the hypothesis that flocs were continually breaking up and producing primary particles or at least very small flocs.
1. The floc break up hypotheses led to the expectation that high turbidity suspensions would have significantly higher settled water turbidity than low turbidity suspensions. This expectation was also not observed.

\end{description}

Evidence that the charge neutralization hypothesis doesn’t explain flocculation of surface waters has been accumulating for decades. \sphinxstyleemphasis{Sweep} flocculation has been proposed as an alternative “mechanism” that described common observations that didn’t fit the charge neutralization hypothesis. However, similar to the charge neutralization hypothesis, the \sphinxstyleemphasis{sweep} hypothesis didn’t result in the development of predictive equations to describe the process.

For example, in 1992 Ching, Tanaka, and Elimelech published their research on \sphinxhref{https://doi.org/10.1016/0043-1354(94)90007-8}{Dynamics of coagulation of kaolin particles with ferric chloride}. They found
that the electrophoretic mobility which is a measure of the clay particle surface charge was never neutralized at pH 7.8 and was neutralized at \(10\mu M\) at pH 6.0. The results were interpreted by the authors to mean that some combination of sweep floc and charge patchiness was responsible for the observed results.

See \hyperref[\detokenize{Flocculation/Floc_Intro:figure-ching-electrophoretic-mobility-vs-ferric-chloride}]{Fig.\@ \ref{\detokenize{Flocculation/Floc_Intro:figure-ching-electrophoretic-mobility-vs-ferric-chloride}}} for a typical mountain view.

\begin{figure}[htbp]
\centering
\capstart

\noindent\sphinxincludegraphics[width=300\sphinxpxdimen]{{Ching_Electrophoretic_Mobility_vs_Ferric_Chloride}.png}
\caption{\sphinxtitleref{Electrophoretic\_Mobility for final pH (after coagulant addition) of 6.0 and 7.8 as a function of :math:{}`FeCl\_3} dose \textless{}\sphinxurl{https://doi.org/10.1016/0043-1354(94)90007-8}\textgreater{}{}`\_\_}\label{\detokenize{Flocculation/Floc_Intro:id1}}\label{\detokenize{Flocculation/Floc_Intro:figure-ching-electrophoretic-mobility-vs-ferric-chloride}}\end{figure}

\begin{figure}[htbp]
\centering
\capstart

\noindent\sphinxincludegraphics[width=300\sphinxpxdimen]{{Ching_Residual_Turbidity_vs_Ferric_Chloride}.png}
\caption{\sphinxhref{https://doi.org/10.1016/0043-1354(94)90007-8}{The settled water turbidity was almost independent of pH even though the electrophoretic mobility was quite different for the two pH values tested}.}\label{\detokenize{Flocculation/Floc_Intro:id2}}\label{\detokenize{Flocculation/Floc_Intro:figure-ching-residual-turbidity-vs-ferric-chloride}}\end{figure}

\sphinxhref{https://doi.org/10.1016/0043-1354(94)90007-8}{At pH 6.0 the ferric hydroxide precipitates are positively charged and at pH 7.8 they are close to neutral}. Thus it is apparent that neutralization of the clay surface charge can not explain
these results.

\sphinxhref{https://doi.org/10.1016/0043-1354(94)90007-8}{Figure x. Settled water turbidity (jar tests) for final pH (after coagulant addition) of 6.0 and 7.8.}

Electrostatic charge neutralization hypothesis The coagulant precipitate self aggregates \textendash{} this is inconsistent with the positive charge that the electrostatic hypothesis asserts will prevent aggregation * Electrostatic repulsion extends only a few nm from the surface of a particle \textendash{} and the coagulant adhesive nanoparticles are many times larger than the reach of the repulsive electrostatic force. The hypothesis that London van der Waals forces result in attachment neglects to account for the presence of water in the system. Water molecules will also be attracted to surfaces by London van der Waals forces and thus there will be competition between the coagulant and water. Thus eliminating repulsion is NOT sufficient to produce a bond between the particles. (see \sphinxhref{https://vtechworks.lib.vt.edu/bitstream/handle/10919/30137/Chapter1.pdf?sequence=9}{hydration repulsion, page 21}) {}` “The theory of DLP was a great step forward in that it appeared to circumvent the whole intractable problem of many body forces through its use of measured bulk dielectric response functions. However, it must be stressed again that it is a perturbation theory. That is, it depends on the assumption that an intervening liquid between interacting surfaces has bulk liquid properties up to a molecular distance from the surfaces. This is thermodynamically inconsistent, being equivalent to the statement that surface energies (or alternatively, the positions of the Gibbs dividing surfaces) are changed infinitesimally with distance of separation. This limits the theory to large distances (Young\textendash{}Laplace vs. Poisson again) where large is undefined.” \textless{}\sphinxurl{https://doi.org/10.1016/S0001-8686(99)00008-1}\textgreater{}{}`\_\_

\fvset{hllines={, ,}}%
\begin{sphinxVerbatim}[commandchars=\\\{\}]
\PYG{c+c1}{\PYGZsh{} \PYGZpc{}\PYGZpc{}}
\PYG{c+c1}{\PYGZsh{}Assumptions}
\PYG{n}{Pi\PYGZus{}VC} \PYG{o}{=} \PYG{o}{.}\PYG{l+m+mi}{62} \PYG{c+c1}{\PYGZsh{}Vena contracta coefficient of an orifice}
\PYG{n}{Ke} \PYG{o}{=} \PYG{p}{(}\PYG{p}{(}\PYG{l+m+mi}{1}\PYG{o}{/}\PYG{n}{Pi\PYGZus{}VC}\PYG{o}{*}\PYG{o}{*}\PYG{l+m+mi}{2}\PYG{p}{)}\PYG{o}{\PYGZhy{}}\PYG{l+m+mi}{1}\PYG{p}{)}\PYG{o}{*}\PYG{o}{*}\PYG{l+m+mi}{2} \PYG{c+c1}{\PYGZsh{}expansion coefficient}

\PYG{c+c1}{\PYGZsh{}Functions to calculate key parameters}

\PYG{k}{def} \PYG{n+nf}{Gave}\PYG{p}{(}\PYG{n}{G\PYGZus{}theta}\PYG{p}{,}\PYG{n}{h\PYGZus{}floc}\PYG{p}{,}\PYG{n}{Temp}\PYG{p}{)}\PYG{p}{:}
    \PYG{l+s+sd}{\PYGZdq{}\PYGZdq{}\PYGZdq{}Calculates average G given target minimum collision potential, total headloss, and design temperature}
\PYG{l+s+sd}{    equation from flocculation slides\PYGZdq{}\PYGZdq{}\PYGZdq{}}
    \PYG{n}{G\PYGZus{}ave} \PYG{o}{=} \PYG{p}{(}\PYG{n}{pc}\PYG{o}{.}\PYG{n}{gravity}\PYG{o}{*}\PYG{n}{h\PYGZus{}floc}\PYG{o}{/}\PYG{p}{(}\PYG{n}{G\PYGZus{}theta}\PYG{o}{*}\PYG{n}{pc}\PYG{o}{.}\PYG{n}{viscosity\PYGZus{}kinematic}\PYG{p}{(}\PYG{n}{Temp}\PYG{p}{)}\PYG{p}{)}\PYG{p}{)}\PYG{o}{.}\PYG{n}{to}\PYG{p}{(}\PYG{l+m+mi}{1}\PYG{o}{/}\PYG{n}{u}\PYG{o}{.}\PYG{n}{s}\PYG{p}{)}
    \PYG{k}{return} \PYG{n}{G\PYGZus{}ave}

\PYG{k}{def} \PYG{n+nf}{restime}\PYG{p}{(}\PYG{n}{G\PYGZus{}theta}\PYG{p}{,}\PYG{n}{G\PYGZus{}ave}\PYG{p}{)}\PYG{p}{:}
    \PYG{l+s+sd}{\PYGZdq{}\PYGZdq{}\PYGZdq{}Calculates residence time given collision potential and average G}
\PYG{l+s+sd}{    equation from flocculation slides\PYGZdq{}\PYGZdq{}\PYGZdq{}}
    \PYG{n}{theta} \PYG{o}{=} \PYG{n}{G\PYGZus{}theta}\PYG{o}{/}\PYG{n}{G\PYGZus{}ave}
    \PYG{k}{return} \PYG{n}{theta}


\PYG{k}{def} \PYG{n+nf}{Dpipe}\PYG{p}{(}\PYG{n}{Ke}\PYG{p}{,}\PYG{n}{Pi\PYGZus{}HS}\PYG{p}{,}\PYG{n}{Q}\PYG{p}{,}\PYG{n}{G\PYGZus{}ave}\PYG{p}{,}\PYG{n}{Temp}\PYG{p}{,}\PYG{n}{SDR}\PYG{p}{)}\PYG{p}{:}
    \PYG{l+s+sd}{\PYGZdq{}\PYGZdq{}\PYGZdq{}Calculates the actual inner diameter of the pipe}
\PYG{l+s+sd}{    equation from flocculation slides\PYGZdq{}\PYGZdq{}\PYGZdq{}}
    \PYG{n}{D\PYGZus{}pipe} \PYG{o}{=} \PYG{p}{(}\PYG{p}{(}\PYG{n}{Ke}\PYG{o}{/}\PYG{p}{(}\PYG{l+m+mi}{2}\PYG{o}{*}\PYG{n}{Pi\PYGZus{}HS}\PYG{o}{*}\PYG{n}{pc}\PYG{o}{.}\PYG{n}{viscosity\PYGZus{}kinematic}\PYG{p}{(}\PYG{n}{Temp}\PYG{p}{)}\PYG{o}{*}\PYG{n}{G\PYGZus{}ave}\PYG{o}{*}\PYG{o}{*}\PYG{l+m+mi}{2}\PYG{p}{)}\PYG{p}{)}\PYG{o}{*}\PYG{p}{(}\PYG{l+m+mi}{4}\PYG{o}{*}\PYG{n}{Q}\PYG{o}{.}\PYG{n}{to}\PYG{p}{(}\PYG{n}{u}\PYG{o}{.}\PYG{n}{m}\PYG{o}{*}\PYG{o}{*}\PYG{l+m+mi}{3}\PYG{o}{/}\PYG{n}{u}\PYG{o}{.}\PYG{n}{s}\PYG{p}{)}\PYG{o}{/}\PYG{n}{np}\PYG{o}{.}\PYG{n}{pi}\PYG{p}{)}\PYG{o}{*}\PYG{o}{*}\PYG{l+m+mi}{3}\PYG{p}{)}\PYG{o}{*}\PYG{o}{*}\PYG{p}{(}\PYG{l+m+mi}{1}\PYG{o}{/}\PYG{l+m+mi}{7}\PYG{p}{)}
    \PYG{k}{return} \PYG{n}{D\PYGZus{}pipe}

\PYG{k}{def} \PYG{n+nf}{Keactual}\PYG{p}{(}\PYG{n}{ID\PYGZus{}pipe}\PYG{p}{,}\PYG{n}{G\PYGZus{}ave}\PYG{p}{,}\PYG{n}{Temp}\PYG{p}{,}\PYG{n}{Pi\PYGZus{}HS}\PYG{p}{,}\PYG{n}{Q}\PYG{p}{)}\PYG{p}{:}
    \PYG{l+s+sd}{\PYGZdq{}\PYGZdq{}\PYGZdq{}estimates actual expansion coefficient given the actual inner diameter and other relevant inputs}
\PYG{l+s+sd}{    equation from flocculation slides\PYGZdq{}\PYGZdq{}\PYGZdq{}}
    \PYG{n}{Ke\PYGZus{}actual} \PYG{o}{=} \PYG{n}{np}\PYG{o}{.}\PYG{n}{pi}\PYG{o}{*}\PYG{o}{*}\PYG{l+m+mi}{3}\PYG{o}{*}\PYG{n}{ID\PYGZus{}pipe}\PYG{o}{*}\PYG{o}{*}\PYG{l+m+mi}{7}\PYG{o}{*}\PYG{n}{G\PYGZus{}ave}\PYG{o}{*}\PYG{o}{*}\PYG{l+m+mi}{2}\PYG{o}{*}\PYG{n}{pc}\PYG{o}{.}\PYG{n}{viscosity\PYGZus{}kinematic}\PYG{p}{(}\PYG{n}{Temp}\PYG{p}{)}\PYG{o}{*}\PYG{n}{Pi\PYGZus{}HS}\PYG{o}{/}\PYG{p}{(}\PYG{l+m+mi}{32}\PYG{o}{*}\PYG{n}{Q}\PYG{o}{.}\PYG{n}{to}\PYG{p}{(}\PYG{n}{u}\PYG{o}{.}\PYG{n}{m}\PYG{o}{*}\PYG{o}{*}\PYG{l+m+mi}{3}\PYG{o}{/}\PYG{n}{u}\PYG{o}{.}\PYG{n}{s}\PYG{p}{)}\PYG{o}{*}\PYG{o}{*}\PYG{l+m+mi}{3}\PYG{p}{)}
    \PYG{k}{return} \PYG{n}{Ke\PYGZus{}actual}



\PYG{k}{def} \PYG{n+nf}{Aorifice}\PYG{p}{(}\PYG{n}{ID\PYGZus{}pipe}\PYG{p}{,}\PYG{n}{Ke\PYGZus{}actual}\PYG{p}{,}\PYG{n}{Temp}\PYG{p}{,}\PYG{n}{Q}\PYG{p}{)}\PYG{p}{:}
    \PYG{l+s+sd}{\PYGZdq{}\PYGZdq{}\PYGZdq{}Calculates the orifice area given pipe inner diameter, expansion coefficient, Temperature, and flow\PYGZdq{}\PYGZdq{}\PYGZdq{}}
    \PYG{n}{A1} \PYG{o}{=} \PYG{p}{(}\PYG{n}{pc}\PYG{o}{.}\PYG{n}{area\PYGZus{}circle}\PYG{p}{(}\PYG{n}{ID\PYGZus{}pipe}\PYG{p}{)}\PYG{p}{)}\PYG{o}{.}\PYG{n}{to}\PYG{p}{(}\PYG{n}{u}\PYG{o}{.}\PYG{n}{cm}\PYG{o}{*}\PYG{o}{*}\PYG{l+m+mi}{2}\PYG{p}{)}\PYG{o}{.}\PYG{n}{magnitude} \PYG{c+c1}{\PYGZsh{}Pipe area}
    \PYG{n}{Nu} \PYG{o}{=} \PYG{n}{pc}\PYG{o}{.}\PYG{n}{viscosity\PYGZus{}kinematic}\PYG{p}{(}\PYG{n}{Temp}\PYG{p}{)} \PYG{c+c1}{\PYGZsh{}kinematic viscocity}
    \PYG{n}{Re} \PYG{o}{=} \PYG{n}{pc}\PYG{o}{.}\PYG{n}{re\PYGZus{}pipe}\PYG{p}{(}\PYG{n}{Q}\PYG{p}{,}\PYG{n}{ID\PYGZus{}pipe}\PYG{p}{,}\PYG{n}{Nu}\PYG{p}{)} \PYG{c+c1}{\PYGZsh{}reynolds number}

    \PYG{k}{def} \PYG{n+nf}{f\PYGZus{}orif}\PYG{p}{(}\PYG{n}{A2}\PYG{p}{,}\PYG{n}{A1}\PYG{p}{,}\PYG{n}{Ke\PYGZus{}actual}\PYG{p}{,}\PYG{n}{Re}\PYG{p}{)}\PYG{p}{:} \PYG{c+c1}{\PYGZsh{}root of this function is the orifice area}
        \PYG{k}{return} \PYG{p}{(}\PYG{l+m+mf}{2.72}\PYG{o}{+}\PYG{p}{(}\PYG{n}{A2}\PYG{o}{/}\PYG{n}{A1}\PYG{p}{)}\PYG{o}{*}\PYG{p}{(}\PYG{l+m+mi}{4000}\PYG{o}{/}\PYG{n}{Re}\PYG{p}{)}\PYG{p}{)}\PYG{o}{*}\PYG{p}{(}\PYG{l+m+mi}{1}\PYG{o}{\PYGZhy{}}\PYG{n}{A2}\PYG{o}{/}\PYG{n}{A1}\PYG{p}{)}\PYG{o}{*}\PYG{p}{(}\PYG{p}{(}\PYG{n}{A1}\PYG{o}{/}\PYG{n}{A2}\PYG{p}{)}\PYG{o}{*}\PYG{o}{*}\PYG{l+m+mi}{2}\PYG{o}{\PYGZhy{}}\PYG{l+m+mi}{1}\PYG{p}{)}\PYG{o}{\PYGZhy{}}\PYG{n}{Ke\PYGZus{}actual}

    \PYG{n}{A\PYGZus{}orifice} \PYG{o}{=} \PYG{p}{(}\PYG{n}{brentq}\PYG{p}{(}\PYG{k}{lambda} \PYG{n}{A2}\PYG{p}{:} \PYG{n}{f\PYGZus{}orif}\PYG{p}{(}\PYG{n}{A2}\PYG{p}{,}\PYG{n}{A1}\PYG{p}{,}\PYG{n}{Ke\PYGZus{}actual}\PYG{p}{,}\PYG{n}{Re}\PYG{p}{)}\PYG{p}{,} \PYG{o}{\PYGZhy{}}\PYG{l+m+mi}{1}\PYG{p}{,} \PYG{l+m+mi}{2}\PYG{o}{*}\PYG{n}{A1}\PYG{p}{)}\PYG{p}{)}\PYG{o}{*}\PYG{n}{u}\PYG{o}{.}\PYG{n}{cm}\PYG{o}{*}\PYG{o}{*}\PYG{l+m+mi}{2} \PYG{c+c1}{\PYGZsh{}numerical optimization}

    \PYG{k}{return} \PYG{n}{A\PYGZus{}orifice}


\PYG{k}{def} \PYG{n+nf}{eave}\PYG{p}{(}\PYG{n}{G\PYGZus{}ave}\PYG{p}{,}\PYG{n}{Temp}\PYG{p}{)}\PYG{p}{:}
    \PYG{l+s+sd}{\PYGZdq{}\PYGZdq{}\PYGZdq{}Calculates the average energy dissipation rate\PYGZdq{}\PYGZdq{}\PYGZdq{}}
    \PYG{n}{e\PYGZus{}ave} \PYG{o}{=} \PYG{p}{(}\PYG{n}{pc}\PYG{o}{.}\PYG{n}{viscosity\PYGZus{}kinematic}\PYG{p}{(}\PYG{n}{Temp}\PYG{p}{)}\PYG{o}{*}\PYG{n}{G\PYGZus{}ave}\PYG{o}{*}\PYG{o}{*}\PYG{l+m+mi}{2}\PYG{p}{)}\PYG{o}{.}\PYG{n}{to}\PYG{p}{(}\PYG{n}{u}\PYG{o}{.}\PYG{n}{mW}\PYG{o}{/}\PYG{n}{u}\PYG{o}{.}\PYG{n}{kg}\PYG{p}{)}
    \PYG{k}{return} \PYG{n}{e\PYGZus{}ave}

\PYG{k}{def} \PYG{n+nf}{Hchip}\PYG{p}{(}\PYG{n}{A\PYGZus{}orifice}\PYG{p}{,}\PYG{n}{ID\PYGZus{}pipe}\PYG{p}{)}\PYG{p}{:}
    \PYG{l+s+sd}{\PYGZdq{}\PYGZdq{}\PYGZdq{}This function calculates the height of the chip based on the orifice area and pipe diameter}
\PYG{l+s+sd}{    The function uses numerical optimization to solve the transcendental equation\PYGZdq{}\PYGZdq{}\PYGZdq{}}
    \PYG{n}{A\PYGZus{}flow} \PYG{o}{=} \PYG{n}{A\PYGZus{}orifice}\PYG{o}{.}\PYG{n}{magnitude} \PYG{c+c1}{\PYGZsh{}orifice area stripped of units}
    \PYG{n}{r}\PYG{o}{=}\PYG{p}{(}\PYG{n}{ID\PYGZus{}pipe}\PYG{o}{/}\PYG{l+m+mi}{2}\PYG{p}{)}\PYG{o}{.}\PYG{n}{magnitude} \PYG{c+c1}{\PYGZsh{}radius stripped of units}
    \PYG{n}{c} \PYG{o}{=} \PYG{n}{A\PYGZus{}flow}\PYG{o}{/}\PYG{n}{r}\PYG{o}{*}\PYG{o}{*}\PYG{l+m+mi}{2} \PYG{c+c1}{\PYGZsh{}left hand side of equation}

    \PYG{k}{def} \PYG{n+nf}{f}\PYG{p}{(}\PYG{n}{a}\PYG{p}{,}\PYG{n}{c}\PYG{p}{)}\PYG{p}{:} \PYG{c+c1}{\PYGZsh{}roots of this function are theta}
        \PYG{k}{return} \PYG{n}{a}\PYG{o}{\PYGZhy{}}\PYG{n}{sin}\PYG{p}{(}\PYG{n}{a}\PYG{p}{)}\PYG{o}{*}\PYG{n}{cos}\PYG{p}{(}\PYG{n}{a}\PYG{p}{)}\PYG{o}{\PYGZhy{}}\PYG{n}{c}

    \PYG{n}{theta} \PYG{o}{=} \PYG{n}{brentq}\PYG{p}{(}\PYG{k}{lambda} \PYG{n}{a}\PYG{p}{:} \PYG{n}{f}\PYG{p}{(}\PYG{n}{a}\PYG{p}{,}\PYG{n}{c}\PYG{p}{)}\PYG{p}{,} \PYG{l+m+mi}{0}\PYG{p}{,} \PYG{l+m+mi}{13}\PYG{p}{)} \PYG{c+c1}{\PYGZsh{}numerical optimization}
    \PYG{n}{r\PYGZus{}u} \PYG{o}{=} \PYG{n}{r}\PYG{o}{*}\PYG{n}{u}\PYG{o}{.}\PYG{n}{cm} \PYG{c+c1}{\PYGZsh{}radius with units}
    \PYG{n}{y} \PYG{o}{=} \PYG{n}{r\PYGZus{}u} \PYG{o}{\PYGZhy{}} \PYG{n}{r\PYGZus{}u}\PYG{o}{*}\PYG{n}{np}\PYG{o}{.}\PYG{n}{cos}\PYG{p}{(}\PYG{n}{theta}\PYG{p}{)} \PYG{c+c1}{\PYGZsh{}height of orifice}

    \PYG{n}{H\PYGZus{}chip} \PYG{o}{=} \PYG{n}{ID\PYGZus{}pipe}\PYG{o}{\PYGZhy{}}\PYG{n}{y} \PYG{c+c1}{\PYGZsh{}height of chip}
    \PYG{k}{return} \PYG{n}{H\PYGZus{}chip}

\PYG{k}{def} \PYG{n+nf}{Cost\PYGZus{}Length}\PYG{p}{(}\PYG{n}{L\PYGZus{}pipe}\PYG{p}{,}\PYG{n}{ND\PYGZus{}pipe}\PYG{p}{)}\PYG{p}{:}
    \PYG{l+s+sd}{\PYGZdq{}\PYGZdq{}\PYGZdq{}This function calculates the total cost of the system and the total length of the system\PYGZdq{}\PYGZdq{}\PYGZdq{}}
    \PYG{c+c1}{\PYGZsh{}Length of pipe and number of fittings needed}
    \PYG{n}{OD\PYGZus{}pipe} \PYG{o}{=} \PYG{n}{pipe}\PYG{o}{.}\PYG{n}{OD}\PYG{p}{(}\PYG{n}{ND\PYGZus{}pipe}\PYG{p}{)}
    \PYG{n}{Total\PYGZus{}Pipe} \PYG{o}{=} \PYG{n}{L\PYGZus{}pipe} \PYG{o}{+} \PYG{o}{.}\PYG{l+m+mi}{5}\PYG{o}{*}\PYG{n}{u}\PYG{o}{.}\PYG{n}{m}
    \PYG{n}{Number\PYGZus{}T} \PYG{o}{=} \PYG{n}{np}\PYG{o}{.}\PYG{n}{ceil}\PYG{p}{(}\PYG{n}{Total\PYGZus{}Pipe}\PYG{o}{.}\PYG{n}{magnitude}\PYG{p}{)}
    \PYG{n}{Number\PYGZus{}Elbow} \PYG{o}{=} \PYG{n}{np}\PYG{o}{.}\PYG{n}{ceil}\PYG{p}{(}\PYG{n}{Total\PYGZus{}Pipe}\PYG{o}{.}\PYG{n}{magnitude}\PYG{p}{)}

    \PYG{k}{if} \PYG{n}{ND\PYGZus{}pipe}\PYG{o}{.}\PYG{n}{magnitude} \PYG{o}{==} \PYG{l+m+mi}{3}\PYG{p}{:}
        \PYG{n}{Cost\PYGZus{}T} \PYG{o}{=} \PYG{l+m+mf}{3.94}\PYG{o}{*}\PYG{n}{u}\PYG{o}{.}\PYG{n}{dollar}
        \PYG{n}{Cost\PYGZus{}Elbow} \PYG{o}{=} \PYG{l+m+mf}{3.53}\PYG{o}{*}\PYG{n}{u}\PYG{o}{.}\PYG{n}{dollar}
        \PYG{n}{Cost\PYGZus{}Pipe} \PYG{o}{=} \PYG{p}{(}\PYG{l+m+mf}{17.14}\PYG{o}{/}\PYG{l+m+mi}{10}\PYG{o}{*}\PYG{p}{(}\PYG{n}{u}\PYG{o}{.}\PYG{n}{dollar}\PYG{o}{/}\PYG{n}{u}\PYG{o}{.}\PYG{n}{foot}\PYG{p}{)}\PYG{p}{)}\PYG{o}{.}\PYG{n}{to}\PYG{p}{(}\PYG{n}{u}\PYG{o}{.}\PYG{n}{dollar}\PYG{o}{/}\PYG{n}{u}\PYG{o}{.}\PYG{n}{m}\PYG{p}{)}
        \PYG{n}{Cost\PYGZus{}Valve} \PYG{o}{=} \PYG{l+m+mi}{10}\PYG{o}{*}\PYG{n}{u}\PYG{o}{.}\PYG{n}{dollar}
        \PYG{n}{Width\PYGZus{}T} \PYG{o}{=} \PYG{p}{(}\PYG{l+m+mf}{3.99}\PYG{o}{*}\PYG{n}{u}\PYG{o}{.}\PYG{n}{inch}\PYG{p}{)}\PYG{o}{.}\PYG{n}{to}\PYG{p}{(}\PYG{n}{u}\PYG{o}{.}\PYG{n}{cm}\PYG{p}{)}
        \PYG{n}{Width\PYGZus{}Elbow} \PYG{o}{=} \PYG{p}{(}\PYG{l+m+mf}{3.97}\PYG{o}{*}\PYG{n}{u}\PYG{o}{.}\PYG{n}{inch}\PYG{p}{)}\PYG{o}{.}\PYG{n}{to}\PYG{p}{(}\PYG{n}{u}\PYG{o}{.}\PYG{n}{cm}\PYG{p}{)}


    \PYG{k}{if} \PYG{n}{ND\PYGZus{}pipe}\PYG{o}{.}\PYG{n}{magnitude} \PYG{o}{==}\PYG{l+m+mi}{4}\PYG{p}{:}
        \PYG{n}{Cost\PYGZus{}T} \PYG{o}{=} \PYG{l+m+mf}{7.16}\PYG{o}{*}\PYG{n}{u}\PYG{o}{.}\PYG{n}{dollar}
        \PYG{n}{Cost\PYGZus{}Elbow} \PYG{o}{=} \PYG{l+m+mf}{5.40}\PYG{o}{*}\PYG{n}{u}\PYG{o}{.}\PYG{n}{dollar}
        \PYG{n}{Cost\PYGZus{}Pipe} \PYG{o}{=} \PYG{p}{(}\PYG{l+m+mf}{21.5}\PYG{o}{/}\PYG{l+m+mi}{10}\PYG{o}{*}\PYG{p}{(}\PYG{n}{u}\PYG{o}{.}\PYG{n}{dollar}\PYG{o}{/}\PYG{n}{u}\PYG{o}{.}\PYG{n}{foot}\PYG{p}{)}\PYG{p}{)}\PYG{o}{.}\PYG{n}{to}\PYG{p}{(}\PYG{n}{u}\PYG{o}{.}\PYG{n}{dollar}\PYG{o}{/}\PYG{n}{u}\PYG{o}{.}\PYG{n}{m}\PYG{p}{)}
        \PYG{n}{Cost\PYGZus{}Valve} \PYG{o}{=} \PYG{l+m+mi}{10}\PYG{o}{*}\PYG{n}{u}\PYG{o}{.}\PYG{n}{dollar}
        \PYG{n}{Width\PYGZus{}T} \PYG{o}{=} \PYG{p}{(}\PYG{l+m+mf}{5.06}\PYG{o}{*}\PYG{n}{u}\PYG{o}{.}\PYG{n}{inch}\PYG{p}{)}\PYG{o}{.}\PYG{n}{to}\PYG{p}{(}\PYG{n}{u}\PYG{o}{.}\PYG{n}{cm}\PYG{p}{)}
        \PYG{n}{Width\PYGZus{}Elbow} \PYG{o}{=} \PYG{p}{(}\PYG{l+m+mf}{5.06}\PYG{o}{*}\PYG{n}{u}\PYG{o}{.}\PYG{n}{inch}\PYG{p}{)}\PYG{o}{.}\PYG{n}{to}\PYG{p}{(}\PYG{n}{u}\PYG{o}{.}\PYG{n}{cm}\PYG{p}{)}

    \PYG{k}{if} \PYG{n}{ND\PYGZus{}pipe}\PYG{o}{.}\PYG{n}{magnitude} \PYG{o}{==}\PYG{l+m+mi}{6}\PYG{p}{:}
        \PYG{n}{Cost\PYGZus{}T} \PYG{o}{=} \PYG{l+m+mf}{7.16}\PYG{o}{*}\PYG{n}{u}\PYG{o}{.}\PYG{n}{dollar}
        \PYG{n}{Cost\PYGZus{}Elbow} \PYG{o}{=} \PYG{l+m+mf}{5.40}\PYG{o}{*}\PYG{n}{u}\PYG{o}{.}\PYG{n}{dollar}
        \PYG{n}{Cost\PYGZus{}Pipe} \PYG{o}{=} \PYG{p}{(}\PYG{l+m+mf}{21.5}\PYG{o}{/}\PYG{l+m+mi}{10}\PYG{o}{*}\PYG{p}{(}\PYG{n}{u}\PYG{o}{.}\PYG{n}{dollar}\PYG{o}{/}\PYG{n}{u}\PYG{o}{.}\PYG{n}{foot}\PYG{p}{)}\PYG{p}{)}\PYG{o}{.}\PYG{n}{to}\PYG{p}{(}\PYG{n}{u}\PYG{o}{.}\PYG{n}{dollar}\PYG{o}{/}\PYG{n}{u}\PYG{o}{.}\PYG{n}{m}\PYG{p}{)}
        \PYG{n}{Cost\PYGZus{}Valve} \PYG{o}{=} \PYG{l+m+mi}{10}\PYG{o}{*}\PYG{n}{u}\PYG{o}{.}\PYG{n}{dollar}
        \PYG{n}{Width\PYGZus{}T} \PYG{o}{=} \PYG{p}{(}\PYG{l+m+mf}{5.06}\PYG{o}{*}\PYG{n}{u}\PYG{o}{.}\PYG{n}{inch}\PYG{p}{)}\PYG{o}{.}\PYG{n}{to}\PYG{p}{(}\PYG{n}{u}\PYG{o}{.}\PYG{n}{cm}\PYG{p}{)}
        \PYG{n}{Width\PYGZus{}Elbow} \PYG{o}{=} \PYG{p}{(}\PYG{l+m+mf}{5.06}\PYG{o}{*}\PYG{n}{u}\PYG{o}{.}\PYG{n}{inch}\PYG{p}{)}\PYG{o}{.}\PYG{n}{to}\PYG{p}{(}\PYG{n}{u}\PYG{o}{.}\PYG{n}{cm}\PYG{p}{)}


    \PYG{n}{Total\PYGZus{}Cost} \PYG{o}{=} \PYG{n}{Cost\PYGZus{}Pipe}\PYG{o}{*}\PYG{n}{Total\PYGZus{}Pipe} \PYG{o}{+} \PYG{n}{Cost\PYGZus{}T}\PYG{o}{*}\PYG{n}{Number\PYGZus{}T} \PYG{o}{+} \PYG{n}{Cost\PYGZus{}Elbow}\PYG{o}{*}\PYG{n}{Number\PYGZus{}Elbow} \PYG{o}{+} \PYG{n}{Cost\PYGZus{}Valve}\PYG{o}{*}\PYG{n}{Number\PYGZus{}Elbow}
    \PYG{n}{Floor\PYGZus{}Length} \PYG{o}{=} \PYG{n}{Number\PYGZus{}T}\PYG{o}{*}\PYG{p}{(}\PYG{n}{Width\PYGZus{}T}\PYG{o}{+}\PYG{n}{Width\PYGZus{}Elbow}\PYG{o}{\PYGZhy{}}\PYG{n}{OD\PYGZus{}pipe}\PYG{p}{)}\PYG{o}{.}\PYG{n}{to}\PYG{p}{(}\PYG{n}{u}\PYG{o}{.}\PYG{n}{m}\PYG{p}{)}
    \PYG{n}{Output}\PYG{o}{=}\PYG{p}{[}\PYG{n}{Total\PYGZus{}Cost}\PYG{p}{,}\PYG{n}{Floor\PYGZus{}Length}\PYG{p}{]}
    \PYG{k}{return} \PYG{n}{Output}
\end{sphinxVerbatim}

\fvset{hllines={, ,}}%
\begin{sphinxVerbatim}[commandchars=\\\{\}]
\PYG{c+c1}{\PYGZsh{}Inputs}
\PYG{n}{D\PYGZus{}Sed} \PYG{o}{=} \PYG{l+m+mf}{2.5}\PYG{o}{*}\PYG{n}{u}\PYG{o}{.}\PYG{n}{cm}
\PYG{n}{A\PYGZus{}Sed} \PYG{o}{=} \PYG{n}{pc}\PYG{o}{.}\PYG{n}{area\PYGZus{}circle}\PYG{p}{(}\PYG{n}{D\PYGZus{}Sed}\PYG{p}{)}
\PYG{n}{v\PYGZus{}Sed} \PYG{o}{=} \PYG{l+m+mi}{2}\PYG{o}{*}\PYG{n}{u}\PYG{o}{.}\PYG{n}{mm}\PYG{o}{/}\PYG{n}{u}\PYG{o}{.}\PYG{n}{s}
\PYG{n}{Q} \PYG{o}{=} \PYG{p}{(}\PYG{n}{v\PYGZus{}Sed}\PYG{o}{*}\PYG{n}{A\PYGZus{}Sed}\PYG{p}{)}\PYG{o}{.}\PYG{n}{to}\PYG{p}{(}\PYG{n}{u}\PYG{o}{.}\PYG{n}{mL}\PYG{o}{/}\PYG{n}{u}\PYG{o}{.}\PYG{n}{s}\PYG{p}{)}
\PYG{n+nb}{print}\PYG{p}{(}\PYG{l+s+s1}{\PYGZsq{}}\PYG{l+s+s1}{The flow rate is}\PYG{l+s+s1}{\PYGZsq{}}\PYG{p}{,}\PYG{n}{Q}\PYG{p}{)}

\PYG{n}{Temp} \PYG{o}{=} \PYG{l+m+mi}{15}\PYG{o}{*}\PYG{n}{u}\PYG{o}{.}\PYG{n}{degC}
\PYG{n}{h\PYGZus{}floc} \PYG{o}{=} \PYG{l+m+mi}{50}\PYG{o}{*}\PYG{n}{u}\PYG{o}{.}\PYG{n}{cm} \PYG{c+c1}{\PYGZsh{}standard for Aguaclara plants}
\PYG{n}{G\PYGZus{}theta} \PYG{o}{=} \PYG{l+m+mi}{20000} \PYG{c+c1}{\PYGZsh{}standard for Aguaclara plants}
\PYG{n}{Pi\PYGZus{}HS} \PYG{o}{=} \PYG{l+m+mi}{6}  \PYG{c+c1}{\PYGZsh{}\PYGZsh{}3\PYGZhy{}6 is a good range, more research needed}
\PYG{n}{SDR} \PYG{o}{=} \PYG{l+m+mi}{41} \PYG{c+c1}{\PYGZsh{}Standard ratio}
\end{sphinxVerbatim}

\fvset{hllines={, ,}}%
\begin{sphinxVerbatim}[commandchars=\\\{\}]
\PYG{c+c1}{\PYGZsh{}Calculate G average using functions listed above and given inputs}
\PYG{n}{G\PYGZus{}ave} \PYG{o}{=} \PYG{n}{Gave}\PYG{p}{(}\PYG{n}{G\PYGZus{}theta}\PYG{p}{,}\PYG{n}{h\PYGZus{}floc}\PYG{p}{,}\PYG{n}{Temp}\PYG{p}{)}
\PYG{n}{theta} \PYG{o}{=} \PYG{n}{restime}\PYG{p}{(}\PYG{n}{G\PYGZus{}theta}\PYG{p}{,}\PYG{n}{G\PYGZus{}ave}\PYG{p}{)}
\PYG{n}{e\PYGZus{}ave} \PYG{o}{=} \PYG{n}{eave}\PYG{p}{(}\PYG{n}{G\PYGZus{}ave}\PYG{p}{,}\PYG{n}{Temp}\PYG{p}{)}
\PYG{n+nb}{print}\PYG{p}{(}\PYG{l+s+s1}{\PYGZsq{}}\PYG{l+s+s1}{The average G value is }\PYG{l+s+s1}{\PYGZsq{}}\PYG{p}{,}\PYG{n}{G\PYGZus{}ave}\PYG{p}{)}
\PYG{n+nb}{print}\PYG{p}{(}\PYG{l+s+s1}{\PYGZsq{}}\PYG{l+s+s1}{The residence time in the flocculator is }\PYG{l+s+s1}{\PYGZsq{}}\PYG{p}{,}\PYG{n}{theta}\PYG{p}{)}
\PYG{n+nb}{print}\PYG{p}{(}\PYG{l+s+s1}{\PYGZsq{}}\PYG{l+s+s1}{The average energy dissipation rate is }\PYG{l+s+s1}{\PYGZsq{}}\PYG{p}{,} \PYG{n}{e\PYGZus{}ave}\PYG{p}{)}
\end{sphinxVerbatim}

\fvset{hllines={, ,}}%
\begin{sphinxVerbatim}[commandchars=\\\{\}]
\PYG{c+c1}{\PYGZsh{}Calculate the pipe diameter, both inner and nominal and determine area of pipe using inner diameter output}
\PYG{n}{D\PYGZus{}pipe} \PYG{o}{=} \PYG{p}{(}\PYG{n}{Dpipe}\PYG{p}{(}\PYG{n}{Ke}\PYG{p}{,}\PYG{n}{Pi\PYGZus{}HS}\PYG{p}{,}\PYG{n}{Q}\PYG{p}{,}\PYG{n}{G\PYGZus{}ave}\PYG{p}{,}\PYG{n}{Temp}\PYG{p}{,}\PYG{n}{SDR}\PYG{p}{)}\PYG{p}{)}\PYG{o}{.}\PYG{n}{to}\PYG{p}{(}\PYG{n}{u}\PYG{o}{.}\PYG{n}{cm}\PYG{p}{)}
\PYG{c+c1}{\PYGZsh{}Calculate nominal diameter of pipe}
\PYG{n}{ND\PYGZus{}pipe} \PYG{o}{=} \PYG{n}{pipe}\PYG{o}{.}\PYG{n}{ND\PYGZus{}SDR\PYGZus{}available}\PYG{p}{(}\PYG{n}{D\PYGZus{}pipe}\PYG{p}{,}\PYG{n}{SDR}\PYG{p}{)}
\PYG{c+c1}{\PYGZsh{}Calculate nominal diameter of pipe}
\PYG{n}{ID\PYGZus{}pipe} \PYG{o}{=} \PYG{n}{pipe}\PYG{o}{.}\PYG{n}{ID\PYGZus{}SDR}\PYG{p}{(}\PYG{n}{ND\PYGZus{}pipe}\PYG{p}{,}\PYG{n}{SDR}\PYG{p}{)}\PYG{o}{.}\PYG{n}{to}\PYG{p}{(}\PYG{n}{u}\PYG{o}{.}\PYG{n}{cm}\PYG{p}{)}

\PYG{n}{ID\PYGZus{}pipe} \PYG{o}{=} \PYG{l+m+mi}{5}\PYG{o}{*}\PYG{n}{u}\PYG{o}{.}\PYG{n}{mm}
\PYG{c+c1}{\PYGZsh{}Calculate inner diameter of pipe}
\PYG{n}{A\PYGZus{}pipe} \PYG{o}{=} \PYG{p}{(}\PYG{n}{pc}\PYG{o}{.}\PYG{n}{area\PYGZus{}circle}\PYG{p}{(}\PYG{n}{ID\PYGZus{}pipe}\PYG{p}{)}\PYG{p}{)}\PYG{o}{.}\PYG{n}{to}\PYG{p}{(}\PYG{n}{u}\PYG{o}{.}\PYG{n}{cm}\PYG{o}{*}\PYG{o}{*}\PYG{l+m+mi}{2}\PYG{p}{)}

\PYG{n+nb}{print}\PYG{p}{(}\PYG{l+s+s1}{\PYGZsq{}}\PYG{l+s+s1}{The ideal inner diameter of the pipe would be }\PYG{l+s+s1}{\PYGZsq{}}\PYG{p}{,}\PYG{n}{D\PYGZus{}pipe}\PYG{p}{)}
\PYG{n+nb}{print}\PYG{p}{(}\PYG{l+s+s1}{\PYGZsq{}}\PYG{l+s+s1}{The nominal diameter of the pipe is }\PYG{l+s+s1}{\PYGZsq{}}\PYG{p}{,}\PYG{n}{ND\PYGZus{}pipe}\PYG{p}{,} \PYG{l+s+s1}{\PYGZsq{}}\PYG{l+s+s1}{, and the inner diameter is }\PYG{l+s+s1}{\PYGZsq{}}\PYG{p}{,} \PYG{n}{ID\PYGZus{}pipe}\PYG{p}{)}
\PYG{n+nb}{print}\PYG{p}{(}\PYG{l+s+s1}{\PYGZsq{}}\PYG{l+s+s1}{The area of the pipe is }\PYG{l+s+s1}{\PYGZsq{}}\PYG{p}{,} \PYG{n}{A\PYGZus{}pipe}\PYG{p}{)}
\end{sphinxVerbatim}

\fvset{hllines={, ,}}%
\begin{sphinxVerbatim}[commandchars=\\\{\}]
\PYG{c+c1}{\PYGZsh{}Calculate the actual Ke as a result of the calculated inner pipe diameter}
\PYG{n}{Ke\PYGZus{}actual} \PYG{o}{=} \PYG{p}{(}\PYG{n}{Keactual}\PYG{p}{(}\PYG{n}{ID\PYGZus{}pipe}\PYG{p}{,}\PYG{n}{G\PYGZus{}ave}\PYG{p}{,}\PYG{n}{Temp}\PYG{p}{,}\PYG{n}{Pi\PYGZus{}HS}\PYG{p}{,}\PYG{n}{Q}\PYG{p}{)}\PYG{p}{)}\PYG{o}{.}\PYG{n}{to}\PYG{p}{(}\PYG{n}{u}\PYG{o}{.}\PYG{n}{dimensionless}\PYG{p}{)}
\PYG{n+nb}{print}\PYG{p}{(}\PYG{l+s+s1}{\PYGZsq{}}\PYG{l+s+s1}{The initial expansion minor loss coefficient was }\PYG{l+s+s1}{\PYGZsq{}}\PYG{p}{,}\PYG{n}{Ke}\PYG{p}{)}
\PYG{n+nb}{print}\PYG{p}{(}\PYG{l+s+s1}{\PYGZsq{}}\PYG{l+s+s1}{The actual expansion minor loss coefficient is }\PYG{l+s+s1}{\PYGZsq{}}\PYG{p}{,}\PYG{n}{Ke\PYGZus{}actual}\PYG{p}{)}
\end{sphinxVerbatim}

\fvset{hllines={, ,}}%
\begin{sphinxVerbatim}[commandchars=\\\{\}]
\PYG{c+c1}{\PYGZsh{}Calculate the orifice area}
\PYG{n}{A\PYGZus{}orifice} \PYG{o}{=} \PYG{n}{Aorifice}\PYG{p}{(}\PYG{n}{ID\PYGZus{}pipe}\PYG{p}{,}\PYG{n}{Ke\PYGZus{}actual}\PYG{p}{,}\PYG{n}{Temp}\PYG{p}{,}\PYG{n}{Q}\PYG{p}{)}
\PYG{n+nb}{print}\PYG{p}{(}\PYG{l+s+s1}{\PYGZsq{}}\PYG{l+s+s1}{The orifice area is }\PYG{l+s+s1}{\PYGZsq{}}\PYG{p}{,}\PYG{n}{A\PYGZus{}orifice}\PYG{p}{)}
\end{sphinxVerbatim}

\fvset{hllines={, ,}}%
\begin{sphinxVerbatim}[commandchars=\\\{\}]
\PYG{c+c1}{\PYGZsh{} The following line of code needs to be removed once the orifice area equation is corrected.}

\PYG{n}{H\PYGZus{}chip} \PYG{o}{=} \PYG{n}{Hchip}\PYG{p}{(}\PYG{n}{A\PYGZus{}orifice}\PYG{p}{,}\PYG{n}{ID\PYGZus{}pipe}\PYG{p}{)}
\PYG{n+nb}{print}\PYG{p}{(}\PYG{l+s+s1}{\PYGZsq{}}\PYG{l+s+s1}{The height of the chip is }\PYG{l+s+s1}{\PYGZsq{}}\PYG{p}{,} \PYG{n}{H\PYGZus{}chip}\PYG{p}{)}
\end{sphinxVerbatim}

\fvset{hllines={, ,}}%
\begin{sphinxVerbatim}[commandchars=\\\{\}]
\PYG{c+c1}{\PYGZsh{}Calculate average velocity}
\PYG{n}{v\PYGZus{}avg} \PYG{o}{=} \PYG{p}{(}\PYG{n}{Q}\PYG{o}{/}\PYG{n}{pc}\PYG{o}{.}\PYG{n}{area\PYGZus{}circle}\PYG{p}{(}\PYG{n}{ID\PYGZus{}pipe}\PYG{p}{)}\PYG{p}{)}\PYG{o}{.}\PYG{n}{to}\PYG{p}{(}\PYG{n}{u}\PYG{o}{.}\PYG{n}{m}\PYG{o}{/}\PYG{n}{u}\PYG{o}{.}\PYG{n}{s}\PYG{p}{)} \PYG{c+c1}{\PYGZsh{}first calculate average velocity}
\PYG{n+nb}{print}\PYG{p}{(}\PYG{l+s+s1}{\PYGZsq{}}\PYG{l+s+s1}{The average velocity is }\PYG{l+s+s1}{\PYGZsq{}}\PYG{p}{,}\PYG{n}{v\PYGZus{}avg}\PYG{p}{)}

\PYG{c+c1}{\PYGZsh{}Calculate pipe length}
\PYG{n}{L\PYGZus{}pipe} \PYG{o}{=} \PYG{p}{(}\PYG{n}{v\PYGZus{}avg}\PYG{o}{*}\PYG{n}{theta}\PYG{p}{)}\PYG{o}{.}\PYG{n}{to}\PYG{p}{(}\PYG{n}{u}\PYG{o}{.}\PYG{n}{m}\PYG{p}{)} \PYG{c+c1}{\PYGZsh{}then multiply velocity by residence time to get the required length of pipe}
\PYG{n+nb}{print}\PYG{p}{(}\PYG{l+s+s1}{\PYGZsq{}}\PYG{l+s+s1}{The length of the pipe is }\PYG{l+s+s1}{\PYGZsq{}}\PYG{p}{,}\PYG{n}{L\PYGZus{}pipe}\PYG{p}{)}
\end{sphinxVerbatim}

\#references \sphinxhref{https://www.iwapublishing.com/news/coagulation-and-flocculation-water-and-wastewater-treatment}{Coagulation and Flocculation in Water and Wastewater Treatment},
iwapublishing


\chapter{Flocculation Design}
\label{\detokenize{Flocculation/Floc_Design:flocculation-design}}\label{\detokenize{Flocculation/Floc_Design:title-flocculation-design}}\label{\detokenize{Flocculation/Floc_Design::doc}}
Welcome to the \sphinxstylestrong{fourth} summary sheet of CEE 4540! These documents will be guides and references for you throughout the semester. Since
Professor Monroe’s class time is limited, so too is the amount of material he can fit on the slides while ensuring that they remain
understandable. Thus, these summary sheets will supplement the powerpoints by going into further detail on the course concepts
introduced in the slides.

Equations, universal constants, and other helpful goodies can be found in the \sphinxhref{https://github.com/AguaClara/aide\_design/tree/master/aide\_design}{aide\_design repository on GitHub}. Most equations and constants you find in these summary sheets will already have been coded into aide\_design, and will be shown here in the following format:

\begin{DUlineblock}{0em}
\item[] Variable: \sphinxcode{\sphinxupquote{pc.gravity}}
\item[] Function: \sphinxcode{\sphinxupquote{pc.area\_circle(DiamCircle)}}.
\end{DUlineblock}

The letters before the \sphinxcode{\sphinxupquote{.}}, in this case \sphinxcode{\sphinxupquote{pc}}, indicate the file within aide\_design where the variable or function can be found. In the examples above, \sphinxcode{\sphinxupquote{pc.gravity}} and \sphinxcode{\sphinxupquote{pc.area\_circle(DiamCircle)}} show that the variable \sphinxcode{\sphinxupquote{gravity}} and function \sphinxcode{\sphinxupquote{area\_circle(DiamCicle)}} are located inside the \sphinxhref{https://github.com/AguaClara/aide\_design/blob/master/aide\_design/physchem.py}{physchem.py} (\sphinxcode{\sphinxupquote{pc}}) file. You are strongly recommended to look up any aide\_design equations you plan to use within in their aide\_design file before using them, even if they are given here in this summary sheet. This is because each equation has comments in its original file describing what the specific conditions are to using it.

\sphinxstylestrong{Important Note:} This chapter introduces uncertainty and empirical design. Some of the parameters used to design AguaClara flocculators are based on what has been shown to work in the field, as opposed to having been derived scientifically. To make sure that the reader is aware of these concepts and parameters that don’t yet have a thorough basis in research, they will be highlighted in red when they appear.


\section{Hydraulic Flocculators, the AguaClara Approach}
\label{\detokenize{Flocculation/Floc_Design:hydraulic-flocculators-the-aguaclara-approach}}

\subsection{Important Terms}
\label{\detokenize{Flocculation/Floc_Design:important-terms}}\begin{enumerate}
\item {} 
Collision potential

\item {} 
Energy dissipation rate

\item {} 
Baffle

\item {} 
Baffle module

\item {} 
Baffle space

\item {} 
Obstacle

\end{enumerate}


\subsection{Important Equations}
\label{\detokenize{Flocculation/Floc_Design:important-equations}}\begin{enumerate}
\item {} 
Minor Loss equation

\end{enumerate}


\section{Introduction to Hydraulic Flocculation}
\label{\detokenize{Flocculation/Floc_Design:introduction-to-hydraulic-flocculation}}
The reason that flocculation is widely used in water treatment is because of sedimentation. Sedimentation is the process that actually removes particles like clay, dirt, organic matter, and bacteria from water. As you learned in the
{\hyperref[\detokenize{Introduction/Introduction:heading-treatment-trains}]{\sphinxcrossref{\DUrole{std,std-ref}{introduction on treatment trains}}}}, sedimentation is the process of particles ‘falling’ because they have a higher density then the water, and its governing equation is:
\begin{equation}\label{equation:Flocculation/Floc_Design:Flocculation/Floc_Design:0}
\begin{split}\bar v_t = \frac{D_{particle}^2 g}{18 \nu} \frac{\rho_p - \rho_w}{\rho_w}\end{split}
\end{equation}
\begin{DUlineblock}{0em}
\item[] Such that:
\item[] \(\bar v_t\) = terminal velocity of a particle, its downwards speed if it were in quiescent (still) water
\item[] \(D_{particle}\) = particle diameter
\item[] \(\rho\) = density. The \(p\) subscript stands for particle, while \(w\) stands for water
\end{DUlineblock}

To increase \(\bar v_t\) and make sedimentation more efficient, flocculation aims to increase the diameter \(d\) of the particles. This is done by applying a coagulant to the dirty water and helping the coagulant to stick evenly to all particles during Rapid Mix \sphinxstylestrong{(DOUBLE CHECK THAT THIS IS IN RAPID MIX ONCE RAPID MIX IS WRITTEN)}. Being covered in coagulant allows the particles to collide, merge, and grow bigger during flocculation.
Our goal in designing a flocculator is to facilitate particle collisions. How can we do this?


\subsection{Collision Potential, \protect\(\bar G \theta\protect\), and Energy Dissipation Rate, \protect\(\varepsilon\protect\)}
\label{\detokenize{Flocculation/Floc_Design:collision-potential-and-energy-dissipation-rate}}
\sphinxstylestrong{Collision potential :math:{}`(bar G theta){}`} is a term with a very straightforward name. It represents the magnitude of potential particle collisions in a fluid. It is a \sphinxstyleemphasis{dimensionless} parameter which is often used as a performance metric for flocculators; big \(\bar G \theta\) values indicate lots of collisions (good) while small values indicate fewer collisions (not so good). \sphinxstylestrong{AguaClara flocculators usually aim for a collision potential of :math:{}`(bar G theta) = 37,000{}`}, which has worked well in AguaClara plants historically. However, this value may change as research continues. The value for collision potential is obtained by multiplying \(\bar G\), a parameter for average fluid shear with units of \(\frac{1}{[T]}\), and \(\theta\) , the residence time of water in the flocculator, with units of :\([T]\) . \(\theta\) is intuitive to measure, calculate, and understand. \(\bar G\) is a bit more difficult. First, an intuitive explanation. See \hyperref[\detokenize{Flocculation/Floc_Design:figure-g-velocity-profile}]{Fig.\@ \ref{\detokenize{Flocculation/Floc_Design:figure-g-velocity-profile}}}, which shows the velocity profile of flowing water.

\begin{figure}[htbp]
\centering
\capstart

\noindent\sphinxincludegraphics[width=0.500\linewidth]{{G_velocity_profile}.jpg}
\caption{The velocity profile of flowing fluid with uniform shear}\label{\detokenize{Flocculation/Floc_Design:id4}}\label{\detokenize{Flocculation/Floc_Design:figure-g-velocity-profile}}\end{figure}

\(G\) measures the magnitude of shear by using the velocity gradient of a fluid in space, \(\frac{\Delta \bar v}{\Delta h}\). This is essentially the same as the \(\frac{\delta u}{\delta y}\) term in fluid mechanics, which is found in the ubiquitous \sphinxhref{http://polymerdatabase.com/polymer\%20physics/images/Visc.png}{fluid-shear problem} as sourced from \sphinxhref{http://polymerdatabase.com/polymer\%20physics/Viscosity.html}{here.}

\(\bar G\) represents the average \(\frac{\Delta \bar v}{\Delta h}\) for the entire water volume under consideration, and is the parameter we will be using from now on. Unfortunately, it is unrealistic to measure \(\frac{\Delta \bar v}{\Delta h}\) for every parcel of the water in our flocculator and take an average. We need to approximate \(\bar G\) using measureable parameters.

The parameter that serves as the basis for obtaining \(\bar G\) is \(\varepsilon\), which represents the \sphinxstylestrong{energy dissipation} rate of a fluid \sphinxstyleemphasis{normalized by its mass}. The units of \(\varepsilon\) are Watts per kilogram:
\begin{equation}\label{equation:Flocculation/Floc_Design:Flocculation/Floc_Design:1}
\begin{split}\varepsilon = \left[ \frac{W}{Kg} \right] = \left[ \frac{J}{s \cdot Kg} \right] = \left[ \frac{N \cdot m}{s \cdot Kg} \right] = \left[ \frac{kg \cdot m \cdot m}{s^2 \cdot s \cdot Kg} \right] = \left[ \frac{m^2}{s^3} \right] = \left[ \frac{[L]^2}{[T]^3} \right]\end{split}
\end{equation}
There are at least two ways to think about \(\varepsilon\). One is through \(G\). Imagine that a fluid has \sphinxstyleemphasis{no viscosity} ; there is no internal friction caused by fluid flow. No matter how high \(G\) becomes, no energy is dissipated. Now image a honey, which has a very high viscosity. Making honey flow fast requires a lot of energy over a short period of time, which means a high energy dissipation rate. This explanation allows us to understand the equation for \(\varepsilon\) in terms of:math:\sphinxtitleref{G} and \(\nu\). \sphinxhref{https://app.knovel.com/web/view/khtml/show.v/rcid:kpMWHWTPD1/cid:kt00AD4KW1/viewerType:khtml/root\_slug:mwh-s-water-treatment/url\_slug:principles-reactor-analysis?\&b-toc-cid=kpMWHWTPD1\&b-toc-url-slug=coagulation-flocculation\&b-toc-title=MWH\%E2\%80\%99s\%20Water\%20Treatment\%20-\%20Principles\%20and\%20Design\%20(3rd\%20Edition)\&page=80\&view=collapsed\&zoom=1)}{See this textbook} for the derivation of the following equation:
\begin{equation}\label{equation:Flocculation/Floc_Design:Flocculation/Floc_Design:2}
\begin{split}\varepsilon = \nu G^2\end{split}
\end{equation}
Which means we can solve for \(G\):
\begin{equation}\label{equation:Flocculation/Floc_Design:Flocculation/Floc_Design:3}
\begin{split}G = \sqrt{\frac{\varepsilon}{\nu}}\end{split}
\end{equation}
Energy dissipation rate is, fortunately, easier to determine than collision potential. This is due to the second way to think about \(\varepsilon\), which is using head loss. In any reactor, a flocculator in this case, the total energy dissipated is simply the head loss, \(h_L\). The amount of time required to dissipate that energy is the residence time of the water in the reactor, \(\theta\). Accounting for the fact that ‘head’ energy is due to gravity \(g\), we have all the parameters needed to determine another equation for energy dissipation rate:
\begin{equation}\label{equation:Flocculation/Floc_Design:Flocculation/Floc_Design:4}
\begin{split}\bar \varepsilon = \frac{g h_L}{\theta}\end{split}
\end{equation}
Note that the equation above is for \(\bar \varepsilon\), not \(\varepsilon\). Since the head loss term we are using, \(h_L\), occurs over the entire reactor, it can only be used to find an average energy dissipation rate for the entire reactor. Combining the equations above, \(G = \sqrt{\frac{\varepsilon}{\nu}}\) and \(\bar \varepsilon = \frac{g h_L}{\theta}\), we can get an equation for \(\bar G\) in terms of easily measureable parameters:
\begin{equation}\label{equation:Flocculation/Floc_Design:Flocculation/Floc_Design:5}
\begin{split}\bar G = \sqrt{\frac{g h_L}{\nu \theta}}\end{split}
\end{equation}
We can use this to obtain a final equation for collision potential of a reactor:
\begin{equation}\label{equation:Flocculation/Floc_Design:Flocculation/Floc_Design:6}
\begin{split}\bar G \theta = \sqrt{\frac{g h_L \theta}{\nu}}\end{split}
\end{equation}
\sphinxstylestrong{Note:} When we say \(G \theta\) we are almost always referring to \(\bar G \theta\).


\subsection{Generating Head Loss with Baffles}
\label{\detokenize{Flocculation/Floc_Design:generating-head-loss-with-baffles}}

\subsubsection{\sphinxstylestrong{What are Baffles?}}
\label{\detokenize{Flocculation/Floc_Design:what-are-baffles}}
Now that we know how to measure collision potential with head loss, we need a way to actually generate head loss. While both major or minor losses can be the design basis, it generally makes more sense to use major losses only for very low-flow flocculation (lab-scale) and minor losses for higher flows, as flocculation with minor losses tends to be more space-efficient. Since this book focuses on town and village-scale water treatment (5 L/S to 120 L/S), we will use minor losses as our design basis.

To generate minor losses, we need to create flow expansions. AguaClara does this with \sphinxstylestrong{baffles}, which are obstructions in the channel of a flocculator to force the flow to switch directions by 180°. Baffles in AguaClara plants are plastic sheets, and all of the baffles in one flocculator channel are connected to form a \sphinxstylestrong{baffle module.} \hyperref[\detokenize{Flocculation/Floc_Design:figure-ac-flocculator}]{Fig.\@ \ref{\detokenize{Flocculation/Floc_Design:figure-ac-flocculator}}} shows an AguaClara flocculator and \hyperref[\detokenize{Flocculation/Floc_Design:figure-baffle-module}]{Fig.\@ \ref{\detokenize{Flocculation/Floc_Design:figure-baffle-module}}} shows the assembly of a baffle module.

\begin{figure}[htbp]
\centering
\capstart

\noindent\sphinxincludegraphics[width=1.000\linewidth]{{AC_flocculator}.jpg}
\caption{Clockwise from the top left the images show: the outline of the entire flocculator, some top and bottom baffles in the channels, the 4 flocculator channels in this flocculator, and the flow path of water through the flocculator}\label{\detokenize{Flocculation/Floc_Design:id5}}\label{\detokenize{Flocculation/Floc_Design:figure-ac-flocculator}}\end{figure}

\begin{figure}[htbp]
\centering
\capstart

\noindent\sphinxincludegraphics[width=0.500\linewidth]{{Baffle_module}.jpg}
\caption{Before being inserted into the floccualtor channel, the baffle module is constructed as a unit as shown here.}\label{\detokenize{Flocculation/Floc_Design:id6}}\label{\detokenize{Flocculation/Floc_Design:figure-baffle-module}}\end{figure}

AguaClara flocculators, like the one pictured above, are called \sphinxstylestrong{vertical hydraulic flocculators} because the baffles force the flow vertically up and down. If the baffles were instead arranged to force the flow side-to-side, the flocculator would be called a \sphinxstylestrong{horizontal hydraulic flocculator}. AguaClara uses vertical flocculators because they are more efficient when considering plant area. They are deeper than horizontal flocculators, which allows them to have a smaller \sphinxhref{https://simple.wikipedia.org/wiki/Plan\_view}{plan-view area} and thus to be cheaper.


\subsubsection{\sphinxstylestrong{Finding the Minor Loss of a Baffle}}
\label{\detokenize{Flocculation/Floc_Design:finding-the-minor-loss-of-a-baffle}}
Before beginning this section, it is important to understand how water flows through a baffled flocculator. This flow path is shown in \hyperref[\detokenize{Flocculation/Floc_Design:figure-flocculator-flow}]{Fig.\@ \ref{\detokenize{Flocculation/Floc_Design:figure-flocculator-flow}}}. Take note of the thin red arrows; they indicate the compression of the flow around a baffle.

\begin{figure}[htbp]
\centering
\capstart

\noindent\sphinxincludegraphics[width=600\sphinxpxdimen]{{Flocculator_flow}.jpg}
\caption{Flow path through a vertical flow hydraulic flocculator}\label{\detokenize{Flocculation/Floc_Design:id7}}\label{\detokenize{Flocculation/Floc_Design:figure-flocculator-flow}}\end{figure}

Since baffles are the source of head loss via minor losses, we need to find the minor loss coefficient of one baffle if we want to be able to quantify its head loss. To do this, we apply fluid mechanics intuition and check it against a computational fluid dynamics (CFD) simulation. Flow around a 90° bend has a vena contracta value of around \(\Pi_{vc} = 0.62\). Flow around a 180° bend therefore has a value of \(\color{red}{\Pi_{vc, \, baffle} = \Pi_{vc}^2 = 0.384}\). This number is roughly confirmed with CFD, as shown in the image below.

\begin{figure}[htbp]
\centering
\capstart

\noindent\sphinxincludegraphics[width=100\sphinxpxdimen]{{CFD_vc_baffle}.jpg}
\caption{The 180° bend at the end of a baffle results in a dramatic flow contraction with all of the flow passing through less than 40\% of the space between the baffles.}\label{\detokenize{Flocculation/Floc_Design:id8}}\label{\detokenize{Flocculation/Floc_Design:figure-cfd-vc-baffle}}\end{figure}

We can therefore state with reasonable accuracy that, when most contracted, the flow around a baffle goes through 38.4\% of the area it does when expanded, or \(A_{contracted} = \Pi_{vc, \, baffle} A_{expanded}\). Through the \sphinxtitleref{:ref:{}`third form of the minor loss equation \textless{}heading\_minor\_losses\textgreater{}}, \(h_e = K \frac{\bar v_{out}^2}{2g}\) and its definition of the minor loss coefficient, \(K = \left( \frac{A_{out}}{A_{in}} -1 \right)^2\), we can determine a \(k\) for flow around a single baffle:
\begin{align}\label{equation:Flocculation/Floc_Design:Flocculation/Floc_Design:7}\!\begin{aligned}
K_{baffle} = \left( \frac{A_{expanded}}{A_{contracted}} -1 \right)^2\\
K_{baffle} = \left( \frac{\rlap{\Big/} A_{expanded}}{\Pi_{vc, \, baffle} \rlap{\Big/} A_{expanded}} -1 \right)^2\\
K_{baffle} = \left( \frac{1}{0.384} -1 \right)^2\\
\color{red}{K_{baffle} = 2.56}\\
\end{aligned}\end{align}
This \(K_{baffle}\) has been used to design many flocculators in AguaClara plants. However, its value has not yet been rigorously tested for AguaClara plants the field. Therefore it might actually deviate from \(2.56\). Research and testing the \(K\) of a baffle in an AguaClara plant is ongoing, but for now the designs made under the assumption that \(\color{red}{K_{baffle} = 2.56}\) are functioning very well in AguaClara plants. Although research has been done by many academics on the minor loss coefficient, including \sphinxhref{https://iwaponline.com/aqua/article/47/3/142/31711/Design-of-around-the-end-hydraulic-flocculators}{this paper by Haarhoff in 1998}  (DOI: 10.2166/aqua.1998.20), the \(K_{baffle}\) values found are context dependent and empirically based. For AguaClara flocculator parameters, literature suggest a \(K_{baffle}\) value between \(2.5\) and \(4\).


\subsection{Flocculator Efficiency}
\label{\detokenize{Flocculation/Floc_Design:flocculator-efficiency}}
When designing an effective and efficient flocculator, there are two main problems that we seek to avoid:
\begin{enumerate}
\item {} 
Having certain sections in the flocculator with such high local \(G\) values that our big, fluffy flocs are sheared apart into smaller flocs.

\item {} 
Having dead space. Dead space means volume within the flocculator that is not being used to facilitate collisions. Dead space occurs after the flow has fully expanded from flowing around a baffle and before it reaches the next baffle.

\end{enumerate}

Fortunately for us, both problems can be quantified with a single ratio:
\begin{equation}\label{equation:Flocculation/Floc_Design:Flocculation/Floc_Design:8}
\begin{split}\Pi_{\bar G}^{G_{Max}} = \frac{G_{Max}}{\bar G}\end{split}
\end{equation}
High values of \(\Pi_{\bar G}^{G_{Max}}\) occur when one or both of the previous problems is present. If certain sections in the flocculator have very high local \(G\) values, then \(G_{Max}\) becomes large. If the flocculator has a lot of dead space, then \(\bar G\) becomes small. Either way, \(\Pi_{\bar G}^{G_{Max}}\) becomes larger.

\sphinxstylestrong{Note:} Recall the relationship between \(G\) and \(\varepsilon\) : \(G = \sqrt{ \frac{\varepsilon}{\nu} }\). From this relationship, we can see that \(G \propto \sqrt{\varepsilon}\). Thus, by defining  \(\Pi_{\bar G}^{G_{Max}}\), we can also define a ratio for Max to average energy dissipation rate:
\begin{equation}\label{equation:Flocculation/Floc_Design:Flocculation/Floc_Design:9}
\begin{split}\Pi_{\bar \varepsilon}^{\varepsilon_{Max}} = \left( \Pi_{\bar G}^{G_{Max}} \right)^2\end{split}
\end{equation}
Therefore, by making our \(\Pi_{\bar G}^{G_{Max}}\) as small as possible, we can be sure that our flocculator is efficient, and we no longer have to account for the previously mentioned problems. \sphinxhref{https://iwaponline.com/aqua/article/50/3/149/30498/Towards-optimal-design-parameters-for-around-the}{A paper by Haarhoff and van der Walt in 2001} (DOI: 10.2166/aqua.2001.0014) uses CFD to show that the minimum \(\Pi_{\bar G}^{G_{Max}}\) attainable in a hydraulic flocculator is \(\Pi_{\bar G}^{G_{Max}} = \sqrt{2} \approx 1.4\), which means that \(\Pi_{\bar \varepsilon}^{\varepsilon_{Max}} = \left( \Pi_{\bar G}^{G_{Max}} \right)^2 \approx 2\). So how do we optimize an AguaClara flocculator to make sure \(\Pi_{\bar G}^{G_{Max}} = \sqrt{2}\)?

We define and optimize a performance metric:
\begin{equation}\label{equation:Flocculation/Floc_Design:Flocculation/Floc_Design:10}
\begin{split}\frac{H_e}{S} = \Pi_{H_eS}\end{split}
\end{equation}
Where \(H_e\) is the distance between flow expansions in the flocculator and \(S\) is the spacing between baffles. For now, \(H_e\) is approximated as the height of water in the flocculator.

Since \(G_{Max}\) is determined by the fluid mechanics of flow around a baffle, our main concern is eliminating dead space in the flocculator. We do this by placing an upper limit on \(\frac{H_e}{S}\). To determine this upper limit, we need to find the distance it takes for the flow to fully expand after it has contracted around a baffle. We base this on the rule of thumb for flow expansion, \_**\textless{}font color=”red”\textgreater{}RESEARCHED BY GERHART JIRKA FIND A REFERENCE THAT’S BETTER THAN ONE OF MONROE’S POWERPOINTS**\_: a jet doubles its initial diameter/length once it travels 10 times the distance of its original diameter/length\textless{}/font\textgreater{}. If this is confusing, refer to the equation and image below:
\begin{equation}\label{equation:Flocculation/Floc_Design:Flocculation/Floc_Design:11}
\begin{split}\frac{x}{10} = D - D_0\end{split}
\end{equation}
\begin{figure}[htbp]
\centering
\capstart

\noindent\sphinxincludegraphics[width=400\sphinxpxdimen]{{Jet_expansion_flocculator}.jpg}
\caption{A turbulent jet expands in width by one unit for every 10 units downstream.}\label{\detokenize{Flocculation/Floc_Design:id9}}\label{\detokenize{Flocculation/Floc_Design:figure-jet-expansion-flocculator}}\end{figure}

Using the equation and image above, we can find the distance required for the flow to fully expand around a baffle as a function of baffle spacing \(S\). We do this by substituting  \(D_0 = (0.384 S)\) along with \(D = S\) to approximate how much distance, \(x = H_e\), the contracted flow has to cover.
\begin{equation}\label{equation:Flocculation/Floc_Design:Flocculation/Floc_Design:12}
\begin{split}\frac{H_e}{10} = S - (0.384 S)
\frac{H_e}{10} = 0.616 S
H_e = 6.16S
\frac{H_e}{S} = 6.16
\Pi_{H_eS_{Max}} = \frac{H_e}{S} = 6.16 \approx 6\end{split}
\end{equation}
This is the highest allowable \(\Pi_{H_eS}\) that we can design while ensuring that there is no dead space in the flocculator.

\begin{figure}[htbp]
\centering
\capstart

\noindent\sphinxincludegraphics[width=0.500\linewidth]{{CFD_baffle_image}.jpg}
\caption{High \(\frac{H_e}{S}\) ratios result in flocculator zones with low velocity gradients that don’t contribute effectively.}\label{\detokenize{Flocculation/Floc_Design:id10}}\label{\detokenize{Flocculation/Floc_Design:figure-cfd-baffle-image}}\end{figure}

\begin{figure}[htbp]
\centering
\capstart

\noindent\sphinxincludegraphics[width=700\sphinxpxdimen]{{CFD_full_channel}.jpg}
\caption{Each bend creates a flow contraction and when the flow expands it converts kinetic energy into turbulent eddies and fluid deformation. The fluid deformation is what ultimately creates collisions between particles.}\label{\detokenize{Flocculation/Floc_Design:id11}}\label{\detokenize{Flocculation/Floc_Design:figure-cfd-full-channel}}\end{figure}

In order to have a robust design process for a baffle module, we need to have some flexibility in the \(\Pi_{H_eS} = \frac{H_e}{S}\) ratio. Since we found \(\Pi_{H_eS_{Max}}\) previously, we must now find the lowest functional \(\frac{H_e}{S}\) ratio, \(\Pi_{H_eS_{Min}}\).

AguaClara uses a fairly straightforward way of setting \(\Pi_{H_eS_{Min}}\). It is based on the distance between the water level and the bottom baffle (which is the same distance between the flocculator floor and a top baffle). This distance is referred to as the slot width (\sphinxhref{http://aqua.iwaponline.com/content/47/3/142}{Haarhoff 1998})  DOI: 10.2166/aqua.1998.20”) and is defined by the slot width ratio, which describes the slot width as a function of baffle spacing \(S\). Slot width is shown in the following image:

\begin{figure}[htbp]
\centering
\capstart

\noindent\sphinxincludegraphics[width=600\sphinxpxdimen]{{Slot_width_description}.jpg}
\caption{The space between the bottom of the upper baffle and the floor of the flocculator is defined as the slot width.}\label{\detokenize{Flocculation/Floc_Design:id12}}\label{\detokenize{Flocculation/Floc_Design:figure-slot-width-description}}\end{figure}

AguaClara uses a slot width ratio of 1 for its flocculators. This number has been the topic of much hydraulic flocculation research, and values between 1 and 1.5 are generally accepted for hydraulic flocculators. See the following paper and book respectively for more data on slot width ratios and other hydraulic flocculator parameters: \phantomsection\label{\detokenize{Flocculation/Floc_Design:id1}}{\hyperref[\detokenize{Flocculation/Floc_Design:floc-haarhoff-design-1998}]{\sphinxcrossref{{[}Haa98{]}}}}, \phantomsection\label{\detokenize{Flocculation/Floc_Design:id2}}{\hyperref[\detokenize{Flocculation/Floc_Design:floc-schulz-surface-1992}]{\sphinxcrossref{{[}SODA92{]}}}}. We base our slot width ratio of 1 on research done by \phantomsection\label{\detokenize{Flocculation/Floc_Design:id3}}{\hyperref[\detokenize{Flocculation/Floc_Design:floc-haarhoff-towards-2001}]{\sphinxcrossref{{[}HWJ01{]}}}} on optimizing hydraulic flocculator parameters to maximize flocculator efficiency.

The minimum \(\Pi_{H_eS}\) allowable depends on the slot with ratio. If \(\Pi_{H_eS}\) is less than twice the slot width ratio, the water would flow straight through the flocculator without having to bend around the baffles. This means that the flocculator would not be generating almost any head loss, and the top and bottom of the flocculator will largely be dead space. See the following image for an example:

\begin{figure}[htbp]
\centering
\capstart

\noindent\sphinxincludegraphics[width=600\sphinxpxdimen]{{HeS_ratio_min}.jpg}
\caption{The minimum \(\frac{H_e}{S}\) ratio is set by the need to prevent short circuiting through the flocculator.}\label{\detokenize{Flocculation/Floc_Design:id13}}\label{\detokenize{Flocculation/Floc_Design:figure-hes-ratio-min}}\end{figure}

Thus, \(\Pi_{H_eS_{Min}}\) should be at least twice the slot width ratio, \(\Pi_{H_eS_{Min}} = 2\). Historically, AguaClara plants have been designed using \(\Pi_{H_eS_{Min}} = 3\). This adds a safety factor of sorts, ensuring that the flow does not short-circuit through the flocculator and also allowing more space for the flow to expand after each contraction.
\begin{equation}\label{equation:Flocculation/Floc_Design:Flocculation/Floc_Design:13}
\begin{split}\Pi_{H_eS_{Min}} = \frac{H_e}{S} = 3\end{split}
\end{equation}
Finally, we describe a range of \(\Pi_{H_eS}\) that we can use to design an AguaClara flocculator:
\begin{equation}\label{equation:Flocculation/Floc_Design:Flocculation/Floc_Design:14}
\begin{split}3 < \Pi_{H_eS} < 6\end{split}
\end{equation}

\subsubsection{Obstacles}
\label{\detokenize{Flocculation/Floc_Design:obstacles}}
Knowing that efficient flocculators require an \(\frac{H_e}{S}\) ratio that lies between 3 and 6, we need to understand how that impacts the flocculator design. Keeping \(\frac{H_e}{S}\) between two specific values limits the options for baffle spacing and quantity, due to the flocculator having certain size constraints before beginning the design of the baffles. This limitation places an upper limit on the amount of head loss that a baffled flocculator can generate, since the number of baffles is limited by space and baffles are what cause head loss. This is unfortunate, it means that baffled flocculators under certain size specifications can’t be designed to generate certain values of \(\bar \varepsilon\) and \(\bar G\) \sphinxstyleemphasis{while remaining efficient and maintaining} \(3 < \Pi_{H_eS} < 6\). This problem only arises for low flow plants, usually below \(Q_{Plant} = 20 {\rm \frac{L}{s}}\).

To get around this problem, AguaClara included ‘obstacles,’ or half-pipes to contract the flow after the flow expands around one baffle and before it reaches the next baffle. The purpose of these obstacles is to provide extra head loss in between baffles. They also generate head loss via minor losses, \sphinxstyleemphasis{and one obstacle is designed to have the same :math:{}`K{}` as one baffle}. Introducing obstacles slightly alters how we think about \(H_e\). In a flocculator where there are just baffles and no obstacles, then \(H_e = H\), since the height of water in the flocculator is equal to the distance between expansions. When obstacles are added, however, then \(H_e = \frac{H}{1 + n_{obstacles}}\), where \(n_{obstacles}\) is the number of obstacles between two baffles.

\sphinxstylestrong{Baffle space} is the term we use for the space between two baffles. The number of flow expansions per baffle space is \(n_{expansions} = 1 + n_{obstacles}\). The \(1\) is because the baffle itself causes a flow expansion.

These obstacles serve as ‘pseudo-baffles’. They allow for \(\frac{H}{S}`\) to exceed 6, while maintaining maximum flocculator efficiency since, \(\frac{H_e}{S}\) can still be between 3 and 6. Obstacles make it possible to design smaller flocculators without compromising flocculation efficiency. \hyperref[\detokenize{Flocculation/Floc_Design:figure-floc-module-with-obstacles}]{Fig.\@ \ref{\detokenize{Flocculation/Floc_Design:figure-floc-module-with-obstacles}}} and \hyperref[\detokenize{Flocculation/Floc_Design:figure-floc-flow-with-obstacles}]{Fig.\@ \ref{\detokenize{Flocculation/Floc_Design:figure-floc-flow-with-obstacles}}} show these obstacles and how they affect the flow in a flocculator.

\begin{figure}[htbp]
\centering
\capstart

\noindent\sphinxincludegraphics[width=800\sphinxpxdimen]{{Floc_module_with_obstacles}.jpg}
\caption{Obstacles are added so that the flow continually contracts and expands. Additional obstacles are needed for low flow plants where the spacing between baffles is small realtive to the flocculator depth.}\label{\detokenize{Flocculation/Floc_Design:id14}}\label{\detokenize{Flocculation/Floc_Design:figure-floc-module-with-obstacles}}\end{figure}

\begin{figure}[htbp]
\centering
\capstart

\noindent\sphinxincludegraphics[width=900\sphinxpxdimen]{{Floc_flow_with_obstacles}.jpg}
\caption{Obstacles ensure that there aren’t any zones with low velocity gradients.}\label{\detokenize{Flocculation/Floc_Design:id15}}\label{\detokenize{Flocculation/Floc_Design:figure-floc-flow-with-obstacles}}\end{figure}


\section{AguaClara Design of Hydraulic, Vertical Flow Flocculators}
\label{\detokenize{Flocculation/Floc_Design:aguaclara-design-of-hydraulic-vertical-flow-flocculators}}
AguaClara’s approach to flocculator design is the same as it is for any other unit process. First, critical design criteria, called inputs, are established. These criteria represent the priorities that the rest of the design will be based around. Once these parameters are established, then the other parameters of the design, which are dependent on the inputs, are calculated based on certain constraints.

Take the CDC as an example of this design process in {\hyperref[\detokenize{Flow_Control_and_Measurement/FCM_Design:title-flow-control-design}]{\sphinxcrossref{\DUrole{std,std-ref}{Flow Control and Measurement Design}}}}; its inputs are \(h_{L_{Max}}\), \(\sum K\), \(\Pi_{Error}\), and the discrete dosing tube diameters  \(D\) that are available at hardware stores or pipe suppliers. Its dependent variables include the number and length of the dosing tubes and the flow through the CDC system.

The flocculator is more complex to design than the CDC, as it has more details and parameters and the equations for those details and parameters are very interdependent. Therefore, there are many ways to design an AguaClara flocculator, and many different sets of critical design criteria to begin with. Enumerated below is the current AguaClara approach.
\begin{enumerate}
\item {} \begin{description}
\item[{Input parameters}] \leavevmode\begin{itemize}
\item {} 
Specify:
- \(h_{L_{floc}}\), head loss
- \(\bar G \theta\), collision potential
- \(Q\), plant flow rate
- \(H\), height of water \sphinxstyleemphasis{at the end of the flocculator}
- \(L_{Max, \, sed}\), max length of a flocculator channel based on sedimentation tank length
- \(W_{Min, \, human}\) minimum width of a single channel based on the width of the average human hip (someone’s got to go down there…)

\item {} 
Find:
- \(\bar G\), average velocity gradient
- \(\theta\), hydraulic retention time
- \(\rlap{-}V_{floc}\), flocculator volume

\end{itemize}

\end{description}

\item {} \begin{description}
\item[{Physical dimensions}] \leavevmode\begin{itemize}
\item {} 
Calculate:
- \(L_{channel}\), actual channel length
- \(n_{channels}\), amount of channels
- \(W_{channel}\), actual channel width

\end{itemize}

\end{description}

\item {} \begin{description}
\item[{Hydraulic parameters}] \leavevmode\begin{itemize}
\item {} 
Calculate:
- \(H_e\), distance between baffle/obstacle induced flow expansions
- \(n_{obstacles}\), amount of obstacles per baffle space
- \(S\), baffle spacing, distance between baffles

\end{itemize}

\end{description}

\end{enumerate}

\begin{figure}[htbp]
\centering
\capstart

\noindent\sphinxincludegraphics[width=600\sphinxpxdimen]{{Flocculator_physical_parameters}.jpg}
\caption{Flocculator geometry definition including the effect of baffle thickness. Accounting for baffle thickness would be particularly important if \sphinxhref{https://en.wikipedia.org/wiki/Ferrocement}{ferrocement} or wood were used for baffles.}\label{\detokenize{Flocculation/Floc_Design:id16}}\label{\detokenize{Flocculation/Floc_Design:figure-floculator-physical-parameters}}\end{figure}


\subsection{Input Parameters}
\label{\detokenize{Flocculation/Floc_Design:input-parameters}}

\subsubsection{Specify}
\label{\detokenize{Flocculation/Floc_Design:specify}}
We start by making sure that our flocculator will be able to flocculate effectively by defining \(h_{L_{floc}}\) and \(\bar G \theta\). Fixing these two parameters initially allows us to easily find all other parameters which determine flocculator performance. Here are the current standards in AguaClara flocculators:
- \(h_{L_{floc}} = 40 \, {\rm cm}\)
- \(\bar G \theta = 37,000\)

The plant flow rate \(Q\) is defined by the needs of the community that the plant is being desiged for. Additionally, the height of water \sphinxstyleemphasis{at the end} of the flocculator, \(H\), the \sphinxstyleemphasis{maximum} length of the flocculator based on the length of the sedimentation tank length, \(L_{Max, \, sed}\), and the \sphinxstyleemphasis{minimum} width of a flocculator channel required for a human to fit inside, \(W_{Min, \, human}\), are also defined initially. Ordinarilly in AguaClara plants, the flocculator occupies the same length dimension as the sedimentation tanks, which is why the length constraint exists. See \hyperref[\detokenize{Flocculation/Floc_Design:figure-physical-design-criteria-floc}]{Fig.\@ \ref{\detokenize{Flocculation/Floc_Design:figure-physical-design-criteria-floc}}} for a representation of how the flocculator and sedimentation tanks are placed in a plant.
\begin{itemize}
\item {} 
\(H = 2 \, {\rm m}\)

\item {} 
\(L_{Max, \, sed} = 6 \, {\rm m}\)

\item {} 
\(W_{Min, \, human} = 45 \, {\rm cm}\)

\end{itemize}

\begin{figure}[htbp]
\centering
\capstart

\noindent\sphinxincludegraphics[width=600\sphinxpxdimen]{{Physical_design_criteria}.jpg}
\caption{Layout of flocculator and sedimentation tanks that was adopted starting with the 2nd AguaClara plant in Tamara, Honduras in 2008.}\label{\detokenize{Flocculation/Floc_Design:id17}}\label{\detokenize{Flocculation/Floc_Design:figure-physical-design-criteria-floc}}\end{figure}


\subsubsection{Find}
\label{\detokenize{Flocculation/Floc_Design:find}}
We can rearrange the equation for \(\bar G\) from the section on collision potential, \(\bar G = \sqrt{\frac{g h_L}{\nu \theta}}\), to solve for \(\bar G\) in terms of \(\bar G \theta\):
\begin{equation}\label{equation:Flocculation/Floc_Design:Flocculation/Floc_Design:15}
\begin{split}\bar G = \frac{g h_{L_{floc}}}{\nu (\bar G \theta)}\end{split}
\end{equation}
Now that we have \(\bar G\), we can very easily find \(theta\):
\begin{equation}\label{equation:Flocculation/Floc_Design:Flocculation/Floc_Design:16}
\begin{split}\theta = \frac{\bar G \theta}{\bar G}\end{split}
\end{equation}
Finally, we take retention time \(\theta\) over plant flow rate \(Q\) to get the required volume of the flocculator:
\begin{equation}\label{equation:Flocculation/Floc_Design:Flocculation/Floc_Design:17}
\begin{split}\rlap{-} V_{floc} = \frac{\theta}{Q}\end{split}
\end{equation}
Now that we have the basic parameters defined, we can start to design the details of the flocculator, starting from the physical dimensions.


\subsection{Physical Dimensions}
\label{\detokenize{Flocculation/Floc_Design:physical-dimensions}}
Deriving the equations required to find the physical dimensions now and the hydraulic parameters (baffle/obstacle design) in the next section requires many steps. To simplify this design explanation the equation derivations are developed in {\hyperref[\detokenize{Review/Review_Fluid_Mechanics_Derivations:title-review-fluid-mechanics-derivations}]{\sphinxcrossref{\DUrole{std,std-ref}{Review: Fluid Mechanics Derivations}}}}. All complex equations which seemingly came out of nowhere will be derived in the derivation sheet.


\subsubsection{Length}
\label{\detokenize{Flocculation/Floc_Design:length}}
Flocculator length, \(L_{channel}`\) must meet two constraints: it must be less than or equal to the length of the sedimentation tanks, as the flocculator is adjacent to the sed tanks. This constraint is \(L_{Max, \, sed}\). Next, the flocculator must be short enough to make sure the target volume of the flocculator is met, while still allowing for a human to fit inside \(L_{Max, \, \rlap{-} V}\). \sphinxstylestrong{The constraint that wins out is the one that results in the *smaller* length value}.
\begin{equation}\label{equation:Flocculation/Floc_Design:Flocculation/Floc_Design:18}
\begin{split}L_{Max, \, sed} = 6 \, {\rm m}
L_{Max, \, \rlap{-}V} = \frac{\rlap{-} V}{n_{Min, \, channels} W_{Min, \, human} H}\end{split}
\end{equation}
\begin{DUlineblock}{0em}
\item[] Such that:
\item[] \(n_{Min, \, channels} = 2\)
\end{DUlineblock}

The reason why \(W_{Min, \, human}\) is used is because it represents the absolute minimum of flocculator channel width. If the width ends up being larger, the length will decrease. \(n_{Min, \, channels} = 2\)  to make sure that the flow ends up on the correct side of the sedimentation tank, as the image below shows. Note that there can only be an even number of flocculator channels, as explained in the image’s caption.

The equation for \sphinxstyleemphasis{actual} flocculator length is therefore:
\begin{equation}\label{equation:Flocculation/Floc_Design:Flocculation/Floc_Design:19}
\begin{split}L_{channel} = {\rm min}(L_{Max, \, sed}, \, L_{Max, \, \rlap{-} V})\end{split}
\end{equation}
\begin{figure}[htbp]
\centering
\capstart

\noindent\sphinxincludegraphics[width=600\sphinxpxdimen]{{Floc_channels}.jpg}
\caption{There are an even amount of flocculator channels to keep the AguaClara plant layout consistent for flows greater than 12 L/s. This ensures that the entrance tank, filter box, and filters can be kept in the same places across plants.}\label{\detokenize{Flocculation/Floc_Design:id18}}\label{\detokenize{Flocculation/Floc_Design:figure-floc-channels}}\end{figure}


\subsubsection{Width and Number of Channels}
\label{\detokenize{Flocculation/Floc_Design:width-and-number-of-channels}}
The width of a single flocculator channel must meet the following conditions:
- Maintain \(\bar G\) at the value found in the inputs section
- Allow for \(3 < \frac{H_e}{S} < 6\). Recall that \(\frac{H_e}{S} =  \Pi_{H_eS}\)
- Allow for a human to be able to fit into a flocculator channel

The first two conditions are wrapped up into the following equation, {\hyperref[\detokenize{Flocculation/Floc_Derivations:title-flocculation-derivations}]{\sphinxcrossref{\DUrole{std,std-ref}{which is derived here}}}}
\begin{equation}\label{equation:Flocculation/Floc_Design:Flocculation/Floc_Design:20}
\begin{split}W_{Min, \, \Pi_{H_eS}} = \frac{\Pi_{H_eS}Q}{H_e}\left( \frac{K}{2 H_e \nu \bar G^2} \right)^\frac{1}{3}\end{split}
\end{equation}
This equation represents the absolute smallest width of a flocculator channel if we consider the lowest value of \(\Pi_{H_eS}\) and the highest possible value of \(H_e\):

\(H_e = H_{e_{Max}} = H = 2 \, {\rm m}\), this implies that there are no obstacles between baffles
\(\Pi_{H_eS} = \Pi_{ {HS}_{Min} } = 3\)

Recall our other width constraint, \(W_{Min, \, human} = 45 \, {\rm cm}\), which is based on our desire to have a human be able to fit into the channels. The governing constraint is the \sphinxstyleemphasis{larger} value of \(W_{Min}\):
\begin{equation}\label{equation:Flocculation/Floc_Design:Flocculation/Floc_Design:21}
\begin{split}W_{Min} = {\rm max}(W_{Min, \, \Pi_{H_eS}}, \, W_{Min, \, human})\end{split}
\end{equation}
We can find the number of channels, \(n_{channels}\) and their actual width in one last step, by finding the \sphinxstyleemphasis{total flocculator width} if there were no channels and dividing that by the minimum flocculator width, \(W_{Min}\), found above. The equation for total flocculator width is based on our target volume:
\begin{equation}\label{equation:Flocculation/Floc_Design:Flocculation/Floc_Design:22}
\begin{split}W_{total} = \frac{\rlap{-} V}{H L_{channel}}\end{split}
\end{equation}
Finally:
\begin{equation}\label{equation:Flocculation/Floc_Design:Flocculation/Floc_Design:23}
\begin{split}n_{channels} = \frac{W_{total}}{W_{Min}}\end{split}
\end{equation}
\begin{DUlineblock}{0em}
\item[] Such that:
\item[] \(n_{channels}\) is an even number and is not 0. Usually, \(n_{channels}\) is either 2 or 4.
\end{DUlineblock}

Now that we know \(n_{channels}\), we can find the actual width of a channel, \(W_{channel}\).
\begin{equation}\label{equation:Flocculation/Floc_Design:Flocculation/Floc_Design:24}
\begin{split}W_{channel} = \frac{W_{total}}{n_{channels}}\end{split}
\end{equation}

\subsection{Hydraulic Parameters}
\label{\detokenize{Flocculation/Floc_Design:hydraulic-parameters}}
Now that the physical dimensions of the flocculator have been defined, the baffle module needs to be designed. The parameter on which most others are based is the distance between flow expansions, \(H_e\). Recall that \(H_e = H\) when there are no obstacles in between baffles.


\subsubsection{Height Between Expansions \protect\(H_e\protect\) and Number of Obstacles per Baffle Space \protect\(n_{obstacles}\protect\)}
\label{\detokenize{Flocculation/Floc_Design:height-between-expansions-and-number-of-obstacles-per-baffle-space}}
We have a range of possible \(H_e\) values based on our window of \(3 < \frac{H_e}{S} < 6\). However, we have a limitation and a preference which shape how we design \(H_e\). Our limitation is that there can only be an integer number of obstacles. Our preference is to have as few obstacles as possible to make the baffle module as easy to fabricate as possible. Therefore, we want \(H_e\) to be closer to \(6\) than it is to \(3\); we are looking for \(H_{e_{Max}}\).

We calculate \(H_{e_{Max}}\) based on the physical flocculator dimensions. The equation for \(H_e\) is obtained by rearranging one of the equations for minimum channel width found above, \(W_{Min, \, \Pi_{H_eS}} = \frac{\Pi_{H_eS}Q}{H_e}\left( \frac{K}{2 H_e \nu \bar G^2} \right)^\frac{1}{3}\). Because we have already design the channel width, we substitute \({W_{channel}\) for \(W_{Min, \, \Pi_{H_eS}}\). Since we are looking for \(H_{e_{Max}}\), we also substitute \(\Pi_{{HS}_{Max}}\) for \(\Pi_{H_eS}\). The result is:
\begin{equation}\label{equation:Flocculation/Floc_Design:Flocculation/Floc_Design:25}
\begin{split}H_{e_{Max}} = \left[ \frac{K}{2 \nu \bar G^2} \left( \frac{Q \Pi_{{HS}_{Max}}}{W_{channel}} \right)^3 \right]^\frac{1}{4}\end{split}
\end{equation}
Note that this is the \sphinxstyleemphasis{maximum} distance between flow expansions, and does not account for the limitation that there must be an integer number of obstacles per baffle space. Thus, we need to find the \sphinxstyleemphasis{actual} distance between flow expansions. To do this, we determine and round up the number of expansions per baffle space using the ceiling function:
\begin{equation}\label{equation:Flocculation/Floc_Design:Flocculation/Floc_Design:26}
\begin{split}n_{expansions} = {\rm ceil}\left( \frac{H}{H_{e_{Max}}} \right)\end{split}
\end{equation}
If we had used the floor() function instead, we would find that \(H_e\) would be larger than our upper bound, \(H_{e_{Max}}\). From here, we can easily get to the actual number of flow expansions per baffle spacing:
\begin{equation}\label{equation:Flocculation/Floc_Design:Flocculation/Floc_Design:27}
\begin{split}H_e = \frac{H}{n_{expansions}}\end{split}
\end{equation}
Finally, we can obtain the number of obstacles per baffle space. The \(- 1\) in the equation is because the baffles themselves provide one flow expansion per baffle space.
\begin{equation}\label{equation:Flocculation/Floc_Design:Flocculation/Floc_Design:28}
\begin{split}n_{obstacles} = \frac{H}{H_e} - 1\end{split}
\end{equation}

\subsubsection{\sphinxstylestrong{Baffle Spacing :math:{}`S{}`}}
\label{\detokenize{Flocculation/Floc_Design:baffle-spacing-math-s}}
Finally, we can find the space between baffles, \(S\). The equation for \(S\) is taken from an intermediate step in the \(W_{Min, \, \Pi_{H_eS}}\) derivation where we obtained, \(W = \frac{Q}{S}\left( \frac{K}{2 H_e \nu \bar G^2} \right)^\frac{1}{3}\). Rearranging for \(S\), we get:
\begin{equation}\label{equation:Flocculation/Floc_Design:Flocculation/Floc_Design:29}
\begin{split}S = \left( \frac{K}{2 H_e \bar G^2 \nu } \right)^\frac{1}{3} \frac{Q}{W_{channel}}\end{split}
\end{equation}
Fortunately, we either know or have already design for all the parameters in this equation


\section{Checking the Flocculator Design}
\label{\detokenize{Flocculation/Floc_Design:checking-the-flocculator-design}}
We then compare \(n_{spaces, \, required}\) to \(n_{spaces, \, actual}\) to make sure that they are equal.


\subsection{Average Velocity in the Flocculator Check}
\label{\detokenize{Flocculation/Floc_Design:average-velocity-in-the-flocculator-check}}
As water flows through the flocculators, the flocs will get larger and larger. As a result, their terminal sedimentation velocity will increase. This is what we want. However, we need to make sure that the flocs don’t settle in the flocculator; that they instead all settle in the sedimentation tank. To make sure of this, we need to make sure that the velocity of water in the flocculator is high enough to scour any flocs that fall to the bottom of the flocculator. The velocity required to scour flocs from the bottom and avoid floc accumulation is around \(v_{scour} = 15 \, {\rm \frac{cm}{s}}\). We need to check our average velocity \(\bar v\) against this value.
\begin{equation}\label{equation:Flocculation/Floc_Design:Flocculation/Floc_Design:30}
\begin{split}\bar v = \frac{Q}{W_{channel} S}\end{split}
\end{equation}

\subsection{Residence Time of Water in the Flocculator Check}
\label{\detokenize{Flocculation/Floc_Design:residence-time-of-water-in-the-flocculator-check}}
It is now time to make our final check. We need to make sure that our actual residence time is \sphinxstyleemphasis{at least} as much as we designed for. Fortunately, in our design we did not account for the change in water level throughout the flocculator due to head loss. Therefore, the actual volume of water in the flocculator is actually greater than \(\rlap{-} V_{floc}\). See \hyperref[\detokenize{Flocculation/Floc_Design:figure-flocculator-head-loss}]{Fig.\@ \ref{\detokenize{Flocculation/Floc_Design:figure-flocculator-head-loss}}} for clarification.

\begin{figure}[htbp]
\centering
\capstart

\noindent\sphinxincludegraphics[width=400\sphinxpxdimen]{{Flocculator_head_loss}.jpg}
\caption{The water level in the flocculator decreases due to head loss. Flocculators may occupy multiple channels, but this extra triangle of water exists in any case.}\label{\detokenize{Flocculation/Floc_Design:id19}}\label{\detokenize{Flocculation/Floc_Design:figure-flocculator-head-loss}}\end{figure}

Thus, the actual average water level in the flocculator is \(H + \frac{h_{L_{floc}}}{2}\). Thus, the actual residence time is:
\begin{equation}\label{equation:Flocculation/Floc_Design:Flocculation/Floc_Design:31}
\begin{split}\theta_{actual} = \frac{n_{channels} L_{channel} W_{channel} \left( H + \frac{h_{L_{floc}}}{2} \right)} {Q}\end{split}
\end{equation}
Check to see if \(\theta_{actual}\) is greater than \(\theta\).




\chapter{Flocculation Derivations}
\label{\detokenize{Flocculation/Floc_Derivations:flocculation-derivations}}\label{\detokenize{Flocculation/Floc_Derivations:title-flocculation-derivations}}\label{\detokenize{Flocculation/Floc_Derivations::doc}}

\section{Design Equations for the Flocculator}
\label{\detokenize{Flocculation/Floc_Derivations:design-equations-for-the-flocculator}}\label{\detokenize{Flocculation/Floc_Derivations:heading-design-equations-for-the-flocculator}}
This document contains the derivation for the minimum allowable width of
a flocculator channel based on the requirements that
\(3 < \Pi_{H_eS} < 6\) and that we maintain the \(\bar G\) that
serves as a basis for design. The final parameter derived is
\(W_{Min, \, \Pi_{H_eS}}\).


\subsection{Width}
\label{\detokenize{Flocculation/Floc_Derivations:width}}
Our two restrictions are: - Ensuring that we maintain the \(\bar G\)
we get based on our input parameters - Ensuring that
\(3 < \frac{H_e}{S} < 6\)

First, we begin by setting the two equations for energy dissipation
rate, \(\bar \varepsilon = \nu \bar G^2\) and
\(\bar \varepsilon = \frac{g h_{L_{floc}}}{\theta}\) equal to each
other to bring \(\bar G\) into the equation.
\begin{equation}\label{equation:Flocculation/Floc_Derivations:Flocculation/Floc_Derivations:0}
\begin{split}\nu \bar G^2 = \frac{g h_{L_{floc}}}{\theta}\end{split}
\end{equation}
\sphinxstylestrong{Very Important Note:}

For the following steps, we will consider the flow through \sphinxstylestrong{*a single
flow expansion :math:{}`H\_e{}`, not through the entire flocculator*}. This
could be from baffle to obstacle, obstacle to baffle, obstacle to
obstacle, or baffle to baffle depending on how many obstacles are in the
design. This means that we are briefly redefining \(\theta\) to be
the time it takes for the flow to fully expand after a flow contraction.
\(\theta\) no longer represents the time it takes for the flow to go
through the entire flocculator.

From here we make three subsequent substitutions: first
\(h_{L_{floc}} = K \frac{\bar v^2}{2g}\), then
\(\theta = \frac{H_e}{\bar v}\), and finally
\(\bar v = \frac{Q}{WS}\)
\begin{equation}\label{equation:Flocculation/Floc_Derivations:Flocculation/Floc_Derivations:1}
\begin{split}\nu \bar G^2 = K \frac{\bar v^2}{2 \theta}\end{split}
\end{equation}\begin{equation}\label{equation:Flocculation/Floc_Derivations:Flocculation/Floc_Derivations:2}
\begin{split}\nu \bar G^2 = K \frac{\bar v^3}{2 H_e}\end{split}
\end{equation}\begin{equation}\label{equation:Flocculation/Floc_Derivations:Flocculation/Floc_Derivations:3}
\begin{split}\nu \bar G^2 = \frac{K}{2 H_e} \left( \frac{Q}{WS} \right)^3\end{split}
\end{equation}
Now we can solve this equation for channel width, \(W\).
\begin{equation}\label{equation:Flocculation/Floc_Derivations:Flocculation/Floc_Derivations:4}
\begin{split}W = \frac{Q}{S}\left( \frac{K}{2 H_e \nu \bar G^2} \right)^\frac{1}{3}\end{split}
\end{equation}
From here, we can define \(\Pi_{H_eS} = \frac{H_e}{S}\) and
substitute \(S = \frac{H_e}{\Pi_{H_eS}}\) into the previous equation
for \(W\) to get \(W_{Min, \, \Pi_{H_eS}}\):
\begin{equation}\label{equation:Flocculation/Floc_Derivations:Flocculation/Floc_Derivations:5}
\begin{split}\color{purple}{
W_{Min, \, \Pi_{H_eS}} = \frac{\Pi_{H_eS}Q}{H_e}\left( \frac{K}{2 H_e \nu \bar G^2} \right)^\frac{1}{3}
}\end{split}
\end{equation}
\begin{DUlineblock}{0em}
\item[] This equation represents the absolute smallest width of a flocculator
channel if we consider the lowest value of \(\Pi_{H_eS}\) and the
highest possible value of \(H_e\):
\item[] \(H_e = H\), this implies that there are no obstacles between
baffles
\item[] \(\Pi_{H_eS} = 3\)
\end{DUlineblock}


\chapter{Sedimentation Introduction}
\label{\detokenize{Sedimentation/Sed_Intro:sedimentation-introduction}}\label{\detokenize{Sedimentation/Sed_Intro:sedimentation-intro}}\label{\detokenize{Sedimentation/Sed_Intro::doc}}
The improved performance is due to 3 factors. First, the inlet manifold has a diffuser system that straightens the fluid jets that are exiting the manifold so that they have no horizontal velocity component. This is critical because even a small horizontal velocity causes a large scale circulation that transports flocs directly to the top of the sedimentation tank. Inlet manifolds without flow straightening diffusers are commonly used in vertical flow sedimentation tanks including designs by leading competitors.

Second, the diffusers create a line jet that spans the entire length of the sedimentation tank. The line jet enters a jet reverser and the vertical upward jet momentum is used to resuspend flocs that have settled to the bottom of the sedimentation tank. The resuspended flocs form a fluidized bed (floc blanket) with a suspended solids concentrations of approximately 1-5 g/L. The high concentration of particles leads to an increase in collisions and particle aggregation. The floc blanket reduces settled water turbidity by a factor of 10 (Garland et al., 2017) and provides two additional benefits. The floc blanket creates a uniform vertical velocity of water entering the plate settlers and the floc blanket transports excess flocs to a floc hopper for final removal by opening a small drain valve.
Third, the bottom geometry is shaped so that all flocs that settle are transported to the jet reverser. Thus there is no accumulation of settled flocs in the main sedimentation basin. Sludge that is allowed to accumulate in the bottom of sedimentation tanks in tropical and temperate decomposes anaerobically and generates methane. The methane forms gas bubbles that carry suspended solids to the top of the sedimentation tank and cause a reduction in particle removal efficiency.  The AguaClara sedimentation tank bottom geometry prevents sludge accumulation.
The hydraulic self cleaning sedimentation tank with a high performing floc blanket, zero sludge accumulation, and with no moving parts outperforms conventional sedimentation tanks on capital cost, performance, and maintenance costs. Mechanical sludge removal systems are well known to be costly to install and a challenge to maintain.


\section{Floc blankets}
\label{\detokenize{Sedimentation/Sed_Intro:floc-blankets}}\label{\detokenize{Sedimentation/Sed_Intro:id1}}
See the Pan American Health Organization, (PAHO) manual on theory of rapid sand filtration plants (page 289) for reasons why floc blankets should not be used! According to PAHO floc blankets are not recommended for small communities who lack highly trained personal to operate the plant and floc blanket should only be used where plant flow rates and water quality are constant. Each of these constraints was due to the inadequate design of previous floc blanket reactors that made operation difficult.


\subsection{Floc blanket hypotheses}
\label{\detokenize{Sedimentation/Sed_Intro:floc-blanket-hypotheses}}\label{\detokenize{Sedimentation/Sed_Intro:id2}}
The floc blanket mechanism responsible for reduced settled water turbidity has been elusive.
- not flocculation between particles delivered from the flocculator because \$Gtheta\$ generated by the shear of the settling flocs and the hydraulic residence time of the floc blanket is insufficient to cause significant
- The floc blanket consists of settling flocs that are maintained in suspension by the upwardly flowing water.


\section{Floc Hopper}
\label{\detokenize{Sedimentation/Sed_Intro:floc-hopper}}\label{\detokenize{Sedimentation/Sed_Intro:id3}}

\section{Plate Settlers}
\label{\detokenize{Sedimentation/Sed_Intro:plate-settlers}}\label{\detokenize{Sedimentation/Sed_Intro:id4}}

\section{Manifold Hydraulics}
\label{\detokenize{Sedimentation/Sed_Intro:manifold-hydraulics}}\label{\detokenize{Sedimentation/Sed_Intro:id5}}

\chapter{Sedimentation Examples}
\label{\detokenize{Sedimentation/Sed_Examples:sedimentation-examples}}\label{\detokenize{Sedimentation/Sed_Examples:id1}}\label{\detokenize{Sedimentation/Sed_Examples::doc}}
Design a tube settler for a laboratory scale sedimentation tank. The vertical section of the sedimentation tank has a net upflow velocity of 3 mm/s. This velocity is maintained in the tube settler, \(V_\alpha\). The target capture velocity is 0.2 mm/s. The tube settler diameter is 2.54 cm.
\begin{equation}\label{equation:Sedimentation/Sed_Examples:Sedimentation/Sed_Examples:0}
\begin{split}\frac{\bar v_{\uparrow}}{v_c} = \frac{L}{D} \cos \alpha \sin \alpha + \sin ^2 \alpha\end{split}
\end{equation}\begin{equation}\label{equation:Sedimentation/Sed_Examples:Sedimentation/Sed_Examples:1}
\begin{split}\bar v_\uparrow = \bar v_\alpha\sin \alpha\end{split}
\end{equation}
Solve for the length of the tube settler.
\begin{equation}\label{equation:Sedimentation/Sed_Examples:Sedimentation/Sed_Examples:2}
\begin{split}L = \frac{D}{\cos \alpha}\left(\frac{\bar v_\alpha}{\bar v_c} - \sin \alpha\right)\end{split}
\end{equation}
\fvset{hllines={, ,}}%
\begin{sphinxVerbatim}[commandchars=\\\{\}]
\PYG{k+kn}{from} \PYG{n+nn}{aide\PYGZus{}design}\PYG{n+nn}{.}\PYG{n+nn}{play} \PYG{k}{import}\PYG{o}{*}
\PYG{n}{v\PYGZus{}alpha} \PYG{o}{=} \PYG{l+m+mi}{3} \PYG{o}{*} \PYG{n}{u}\PYG{o}{.}\PYG{n}{mm}\PYG{o}{/}\PYG{n}{u}\PYG{o}{.}\PYG{n}{s}
\PYG{n}{v\PYGZus{}c} \PYG{o}{=} \PYG{l+m+mi}{1} \PYG{o}{*} \PYG{n}{u}\PYG{o}{.}\PYG{n}{mm}\PYG{o}{/}\PYG{n}{u}\PYG{o}{.}\PYG{n}{s}
\PYG{n}{D} \PYG{o}{=} \PYG{l+m+mf}{2.54} \PYG{o}{*} \PYG{n}{u}\PYG{o}{.}\PYG{n}{cm}
\PYG{n}{alpha} \PYG{o}{=} \PYG{l+m+mi}{60} \PYG{o}{*} \PYG{n}{u}\PYG{o}{.}\PYG{n}{deg}

\PYG{k}{def} \PYG{n+nf}{L\PYGZus{}settler}\PYG{p}{(}\PYG{n}{D}\PYG{p}{,}\PYG{n}{alpha}\PYG{p}{,}\PYG{n}{v\PYGZus{}alpha}\PYG{p}{,}\PYG{n}{v\PYGZus{}c}\PYG{p}{)}\PYG{p}{:}
  \PYG{k}{return} \PYG{n}{D}\PYG{o}{/}\PYG{n}{np}\PYG{o}{.}\PYG{n}{cos}\PYG{p}{(}\PYG{n}{alpha}\PYG{p}{)}\PYG{o}{*}\PYG{p}{(}\PYG{n}{v\PYGZus{}alpha}\PYG{o}{/}\PYG{n}{v\PYGZus{}c} \PYG{o}{\PYGZhy{}} \PYG{n}{np}\PYG{o}{.}\PYG{n}{sin}\PYG{p}{(}\PYG{n}{alpha}\PYG{p}{)}\PYG{p}{)}

\PYG{n+nb}{print}\PYG{p}{(}\PYG{n}{L\PYGZus{}settler}\PYG{p}{(}\PYG{n}{D}\PYG{p}{,}\PYG{n}{alpha}\PYG{p}{,}\PYG{n}{v\PYGZus{}alpha}\PYG{p}{,}\PYG{l+m+mi}{1}\PYG{o}{*}\PYG{n}{u}\PYG{o}{.}\PYG{n}{mm}\PYG{o}{/}\PYG{n}{u}\PYG{o}{.}\PYG{n}{s}\PYG{p}{)}\PYG{p}{)}
\PYG{n+nb}{print}\PYG{p}{(}\PYG{n}{L\PYGZus{}settler}\PYG{p}{(}\PYG{n}{D}\PYG{p}{,}\PYG{n}{alpha}\PYG{p}{,}\PYG{n}{v\PYGZus{}alpha}\PYG{p}{,}\PYG{l+m+mf}{0.2}\PYG{o}{*}\PYG{n}{u}\PYG{o}{.}\PYG{n}{mm}\PYG{o}{/}\PYG{n}{u}\PYG{o}{.}\PYG{n}{s}\PYG{p}{)}\PYG{p}{)}
\end{sphinxVerbatim}

The tube settler above the floc hopper needs to be 72 cm long. The tube settler should provide a capture velocity of at least 1 mm/s prior to the floc hopper. Thus there should be 11 cm below the floc hopper.


\chapter{Sedimentation Theory and Future Work}
\label{\detokenize{Sedimentation/Sed_Theory_and_Future_Work:sedimentation-theory-and-future-work}}\label{\detokenize{Sedimentation/Sed_Theory_and_Future_Work:id1}}\label{\detokenize{Sedimentation/Sed_Theory_and_Future_Work::doc}}

\section{Floc recycle}
\label{\detokenize{Sedimentation/Sed_Theory_and_Future_Work:floc-recycle}}\label{\detokenize{Sedimentation/Sed_Theory_and_Future_Work:id2}}
We hypothesize that the flocs in floc blankets serve as collectors that primary particles attach to. We suspect that collisions between primary particles and large flocs are possible in the sedimentation tank because the rotational velocity of the flocs is small relative to the sedimentation velocity of the flocs. If the rotational velocity of the flocs is small, then a stagnation point will exist on the floc and a finite flow of fluid will come within a primary particle radius of the floc. Thus we expect primary particle removal in floc blankets to be proportional to the number of collectors that a primary particle passes while in the floc blanket.

The number of collectors that a primary particle passes is proportional to the solids concentration (a surrogate for the number concentration of flocs), the primary particle residence time in the floc blanket, and the sedimentation velocity of the flocs. The sedimentation velocity of the flocs is important because that is what causes a relative velocity between the primary particles and the flocs.

As we have explored increasing the upflow velocity in sedimentation tanks the performance has dropped markedly. This is undoubtedly due in part to the combined effective of a very dilute floc blanket at high upflow velocities AND a low residence time for the primary particles.

Would it be possible to increase the concentration of the floc blanket and thus increase the collision rate? At 3 mm/s upflow velocity there are very few flocs that can stay in the floc blanket. We need a mechanism to transport flocs to the bottom of the floc blanket and return them again after they are carried to the top of the floc blanket.

We propose to test this by installing a settled floc recycle line. The recycle line will connect to the bottom surface of the tube settler below the location of the floc weir. From there is will carry concentrated sludge to the very bottom of the sedimentation tank where it will pass through the wall of the sedimentation tank. Increasing the amount of recycle flow will both increase the solids concentration in the floc blanket and decrease the primary particle residence time in the floc blanket.

There must be an optimal amount of recycled flocs for a floc blanket. Of course, one possiblity is that the optimal recycle is zero. Recycled flocs increase the floc blanket concentration and thus increase the rate of collisions between primary particles and flocs. The recycled flocs also decrease the residence time in the floc blanket and thus decrease the total number of collisions between primary particles and flocs. It may be more complicated than this because the hindered sedimentation velocity of the flocs in the floc blanket is also a function of their concentration.

Our goal is to find the optimal recycle ratio. Optimal is defined as the maximum collision potential. Collision potential for the floc blanket is proportional to to the collision rate times the hydraulic residence time. The collision rate is proportional to the solids concentration and the hindered sedimentation velocity of those flocs. The collision potential is thus proportional to the total number of flocs that a primary particle passes on its way through the floc blanket.
\begin{equation}\label{equation:Sedimentation/Sed_Theory_and_Future_Work:Sedimentation/Sed_Theory_and_Future_Work:0}
\begin{split}CP_{fb} \propto C_{fb} \theta_{fb} v_{hindered}\end{split}
\end{equation}
The residence time in the floc blanket is given by
\begin{equation}\label{equation:Sedimentation/Sed_Theory_and_Future_Work:Sedimentation/Sed_Theory_and_Future_Work:1}
\begin{split}\theta_{fb} = \frac{H_{fb}}{v_{fb}}\end{split}
\end{equation}\begin{equation}\label{equation:Sedimentation/Sed_Theory_and_Future_Work:Sedimentation/Sed_Theory_and_Future_Work:2}
\begin{split}v_{fb} = \frac{Q_{plant} + Q_{recycle}}{A_{fb}}\end{split}
\end{equation}\begin{equation}\label{equation:Sedimentation/Sed_Theory_and_Future_Work:Sedimentation/Sed_Theory_and_Future_Work:3}
\begin{split}Q_{recycle} = \Pi_{recycle}Q_{plant}\end{split}
\end{equation}
The velocity up through the floc blanket without recycle is defined as
\begin{equation}\label{equation:Sedimentation/Sed_Theory_and_Future_Work:Sedimentation/Sed_Theory_and_Future_Work:4}
\begin{split}v_{up} = \frac{Q_{plant}}{A_{fb}}\end{split}
\end{equation}\begin{equation}\label{equation:Sedimentation/Sed_Theory_and_Future_Work:Sedimentation/Sed_Theory_and_Future_Work:5}
\begin{split}v_{fb} = v_{up}\left( 1 + \Pi_{recycle} \right)\end{split}
\end{equation}
Now we need equations for the concentration in the floc blanket. This is based on mass conservation such that the mass in the floc blanket is constant. There is a hindered sedimentation velocity of the flocs that results in a reduction of the mass flux out of the top of the control volume.
\begin{equation}\label{equation:Sedimentation/Sed_Theory_and_Future_Work:Sedimentation/Sed_Theory_and_Future_Work:6}
\begin{split}C_{fb}\left(\frac{ Q_{plant}+Q_{recycle} }{A_{fb}}-v_{hindered}\right) A_{fb}= C_{plant}Q_{plant} + C_{recycle}Q_{recycle}\end{split}
\end{equation}\begin{equation}\label{equation:Sedimentation/Sed_Theory_and_Future_Work:Sedimentation/Sed_Theory_and_Future_Work:7}
\begin{split}C_{fb}\left(\frac{ Q_{plant}+\Pi_{recycle}Q_{plant} }{A_{fb}}-v_{hindered}\frac{Q_{plant}}{Q_{plant}}\right) A_{fb}= C_{plant}Q_{plant} + C_{recycle}\Pi_{recycle}Q_{plant}\end{split}
\end{equation}\begin{equation}\label{equation:Sedimentation/Sed_Theory_and_Future_Work:Sedimentation/Sed_Theory_and_Future_Work:8}
\begin{split}C_{fb}\left( 1+\Pi_{recycle} -\frac{v_{hindered}}{v_{up}}\right) = C_{plant} + C_{recycle}\Pi_{recycle}\end{split}
\end{equation}\begin{equation}\label{equation:Sedimentation/Sed_Theory_and_Future_Work:Sedimentation/Sed_Theory_and_Future_Work:9}
\begin{split}C_{fb} = \frac{C_{plant} + C_{recycle}\Pi_{recycle}}{\left(1+\Pi_{recycle}-\frac{v_{hindered}}{v_{up}}\right)}\end{split}
\end{equation}
Now we can substitute to get the collision potential as a function of the flow rates.
\begin{equation}\label{equation:Sedimentation/Sed_Theory_and_Future_Work:Sedimentation/Sed_Theory_and_Future_Work:10}
\begin{split}CP_{fb} \propto \frac{C_{plant} + C_{recycle}\Pi_{recycle}}{\left(1+\Pi_{recycle}-\frac{v_{hindered}}{v_{up}}\right)\left( 1 + \Pi_{recycle} \right)}  \frac{H_{fb}v_{hindered}} {v_{up}}\end{split}
\end{equation}
We estimate the hindered sedimentation velocity to be 1 mm/s since that is what occurs in a 1 mm/s upflow velocity floc blanket. Ideally we would have a hindered sedimentation velocity as a function of the concentration of flocs in the floc blanket. The concentration of recycled flocs is assumed to be approximately 20 g/L based on Casey Garland measurements of the solids concentration in the floc hopper sludge.

\fvset{hllines={, ,}}%
\begin{sphinxVerbatim}[commandchars=\\\{\}]
\PYG{k+kn}{from} \PYG{n+nn}{aide\PYGZus{}design}\PYG{n+nn}{.}\PYG{n+nn}{play} \PYG{k}{import}\PYG{o}{*}
\PYG{n}{D\PYGZus{}fb}\PYG{o}{=}\PYG{l+m+mf}{2.5}\PYG{o}{*}\PYG{n}{u}\PYG{o}{.}\PYG{n}{cm}
\PYG{n}{A\PYGZus{}fb} \PYG{o}{=} \PYG{n}{pc}\PYG{o}{.}\PYG{n}{area\PYGZus{}circle}\PYG{p}{(}\PYG{n}{D\PYGZus{}fb}\PYG{p}{)}
\PYG{n}{H\PYGZus{}fb} \PYG{o}{=} \PYG{l+m+mi}{1} \PYG{o}{*} \PYG{n}{u}\PYG{o}{.}\PYG{n}{m}
\PYG{n}{v\PYGZus{}hindered} \PYG{o}{=} \PYG{l+m+mi}{1} \PYG{o}{*} \PYG{n}{u}\PYG{o}{.}\PYG{n}{mm}\PYG{o}{/}\PYG{n}{u}\PYG{o}{.}\PYG{n}{s}
\PYG{n}{C\PYGZus{}fb\PYGZus{}conventional} \PYG{o}{=} \PYG{l+m+mi}{3} \PYG{o}{*} \PYG{n}{u}\PYG{o}{.}\PYG{n}{g}\PYG{o}{/}\PYG{n}{u}\PYG{o}{.}\PYG{n}{L}
\PYG{n}{C\PYGZus{}recycle} \PYG{o}{=} \PYG{l+m+mi}{20} \PYG{o}{*} \PYG{n}{u}\PYG{o}{.}\PYG{n}{g}\PYG{o}{/}\PYG{n}{u}\PYG{o}{.}\PYG{n}{L}
\PYG{n}{C\PYGZus{}plant} \PYG{o}{=} \PYG{l+m+mi}{100} \PYG{o}{*} \PYG{n}{u}\PYG{o}{.}\PYG{n}{NTU}
\PYG{n}{v\PYGZus{}up} \PYG{o}{=} \PYG{l+m+mi}{3} \PYG{o}{*} \PYG{n}{u}\PYG{o}{.}\PYG{n}{mm}\PYG{o}{/}\PYG{n}{u}\PYG{o}{.}\PYG{n}{s}


\PYG{k}{def} \PYG{n+nf}{CP}\PYG{p}{(}\PYG{n}{H\PYGZus{}fb}\PYG{p}{,}\PYG{n}{v\PYGZus{}up}\PYG{p}{,}\PYG{n}{v\PYGZus{}hindered}\PYG{p}{,}\PYG{n}{Pi\PYGZus{}recycle}\PYG{p}{,}\PYG{n}{C\PYGZus{}plant}\PYG{p}{,}\PYG{n}{C\PYGZus{}recycle}\PYG{p}{)}\PYG{p}{:}
  \PYG{k}{return} \PYG{p}{(}\PYG{n}{H\PYGZus{}fb}\PYG{o}{*}\PYG{n}{v\PYGZus{}hindered}\PYG{o}{/}\PYG{n}{v\PYGZus{}up}\PYG{o}{*}\PYG{p}{(}\PYG{n}{C\PYGZus{}plant}\PYG{o}{+}\PYG{n}{C\PYGZus{}recycle}\PYG{o}{*}\PYG{n}{Pi\PYGZus{}recycle}\PYG{p}{)}\PYG{o}{/}\PYG{p}{(}\PYG{p}{(}\PYG{l+m+mi}{1}\PYG{o}{+}\PYG{n}{Pi\PYGZus{}recycle}\PYG{p}{)}\PYG{o}{*}\PYG{p}{(}\PYG{l+m+mi}{1}\PYG{o}{+}\PYG{n}{Pi\PYGZus{}recycle}\PYG{o}{\PYGZhy{}}\PYG{n}{v\PYGZus{}hindered}\PYG{o}{/}\PYG{n}{v\PYGZus{}up}\PYG{p}{)}\PYG{p}{)}\PYG{p}{)}\PYG{o}{.}\PYG{n}{to\PYGZus{}base\PYGZus{}units}\PYG{p}{(}\PYG{p}{)}
\PYG{n}{Pi\PYGZus{}recycle\PYGZus{}max} \PYG{o}{=} \PYG{l+m+mi}{2}
\PYG{n}{Pi\PYGZus{}recycle} \PYG{o}{=} \PYG{n}{np}\PYG{o}{.}\PYG{n}{arange}\PYG{p}{(}\PYG{l+m+mi}{0}\PYG{p}{,}\PYG{n}{Pi\PYGZus{}recycle\PYGZus{}max}\PYG{p}{,}\PYG{l+m+mf}{0.1}\PYG{p}{)}
\PYG{n}{fig}\PYG{p}{,} \PYG{n}{ax} \PYG{o}{=} \PYG{n}{plt}\PYG{o}{.}\PYG{n}{subplots}\PYG{p}{(}\PYG{p}{)}
\PYG{n}{x}\PYG{o}{=}\PYG{n}{np}\PYG{o}{.}\PYG{n}{array}\PYG{p}{(}\PYG{p}{[}\PYG{l+m+mi}{0}\PYG{p}{,}\PYG{n}{Pi\PYGZus{}recycle\PYGZus{}max}\PYG{p}{]}\PYG{p}{)}
\PYG{n}{yscale} \PYG{o}{=} \PYG{p}{(}\PYG{n}{C\PYGZus{}fb\PYGZus{}conventional}\PYG{o}{*}\PYG{n}{H\PYGZus{}fb}\PYG{o}{*}\PYG{n}{v\PYGZus{}hindered}\PYG{o}{/}\PYG{p}{(}\PYG{l+m+mi}{1}\PYG{o}{*}\PYG{n}{u}\PYG{o}{.}\PYG{n}{mm}\PYG{o}{/}\PYG{n}{u}\PYG{o}{.}\PYG{n}{s}\PYG{p}{)}\PYG{p}{)}\PYG{o}{.}\PYG{n}{to\PYGZus{}base\PYGZus{}units}\PYG{p}{(}\PYG{p}{)}
\PYG{n}{yscale}
\PYG{n}{y}\PYG{o}{=}\PYG{n}{np}\PYG{o}{.}\PYG{n}{array}\PYG{p}{(}\PYG{p}{[}\PYG{l+m+mi}{1}\PYG{p}{,}\PYG{l+m+mi}{1}\PYG{p}{]}\PYG{p}{)}\PYG{o}{*}\PYG{n}{yscale}
\PYG{n}{ax}\PYG{o}{.}\PYG{n}{plot}\PYG{p}{(}\PYG{n}{x}\PYG{p}{,}\PYG{n}{y}\PYG{p}{)}
\PYG{n}{ax}\PYG{o}{.}\PYG{n}{plot}\PYG{p}{(}\PYG{n}{Pi\PYGZus{}recycle}\PYG{p}{,}\PYG{n}{CP}\PYG{p}{(}\PYG{n}{H\PYGZus{}fb}\PYG{p}{,}\PYG{n}{v\PYGZus{}up}\PYG{p}{,}\PYG{n}{v\PYGZus{}hindered}\PYG{p}{,}\PYG{n}{Pi\PYGZus{}recycle}\PYG{p}{,}\PYG{n}{C\PYGZus{}plant}\PYG{p}{,}\PYG{n}{C\PYGZus{}recycle}\PYG{p}{)}\PYG{p}{)}
\PYG{n}{imagepath} \PYG{o}{=} \PYG{l+s+s1}{\PYGZsq{}}\PYG{l+s+s1}{Sedimentation/Images/}\PYG{l+s+s1}{\PYGZsq{}}
\PYG{n}{ax}\PYG{o}{.}\PYG{n}{set}\PYG{p}{(}\PYG{n}{xlabel}\PYG{o}{=}\PYG{l+s+s1}{\PYGZsq{}}\PYG{l+s+s1}{recycle ratio}\PYG{l+s+s1}{\PYGZsq{}}\PYG{p}{,} \PYG{n}{ylabel}\PYG{o}{=}\PYG{l+s+s1}{\PYGZsq{}}\PYG{l+s+s1}{Collision Potential (kg/m\PYGZca{}2)}\PYG{l+s+s1}{\PYGZsq{}}\PYG{p}{)}
\PYG{n}{ax}\PYG{o}{.}\PYG{n}{legend}\PYG{p}{(}\PYG{p}{[}\PYG{l+s+s2}{\PYGZdq{}}\PYG{l+s+s2}{no recycle at 1 mm/s}\PYG{l+s+s2}{\PYGZdq{}}\PYG{p}{,}\PYG{l+s+s2}{\PYGZdq{}}\PYG{l+s+s2}{with recycle at 3 mm/s}\PYG{l+s+s2}{\PYGZdq{}}\PYG{p}{]}\PYG{p}{)}
\PYG{n}{fig}\PYG{o}{.}\PYG{n}{savefig}\PYG{p}{(}\PYG{n}{imagepath}\PYG{o}{+}\PYG{l+s+s1}{\PYGZsq{}}\PYG{l+s+s1}{fb\PYGZus{}recycle\PYGZus{}ratio}\PYG{l+s+s1}{\PYGZsq{}}\PYG{p}{)}
\PYG{n}{plt}\PYG{o}{.}\PYG{n}{show}\PYG{p}{(}\PYG{p}{)}
\end{sphinxVerbatim}

Here are the results.
\begin{quote}
\end{quote}

\begin{figure}[htbp]
\centering
\capstart

\noindent\sphinxincludegraphics[width=700\sphinxpxdimen]{{fb_recycle_ratio}.png}
\caption{Collision potential comparison in a 1 m deep floc blanket.}\label{\detokenize{Sedimentation/Sed_Theory_and_Future_Work:id4}}\label{\detokenize{Sedimentation/Sed_Theory_and_Future_Work:collision-potential-with-sludge-recycle}}\end{figure}

This analysis suggest that a recycle flow rate that is between 0.5 and 1.5 at a net upflow velocity of 3 mm/s could produce collision potential that is 2/3 of the collision potential with a 1 mm/s upflow velocity. Thus a 3 mm/s sed tank with 1.5 m of floc blanket and recycle might be able to perform at the same level as a 1 mm/s sed tank with a 1 m floc blanket.

The next step is to design the recycle tube. The recycle tube could be inclined to promote additional consolidation to reduce the amount of water that is recycled. The slope would need to be about 60 degrees. We could experiment with the design of the recycle line if it were made of flexible tubing.

It is expected that the consolidated sludge will flow by gravity because of its higher density. The big unknown is what diameter recycle line is needed for a lab scale test with a 2.5 cm diameter sedimentation tank.

The recycle sludge has a density given by
\begin{equation}\label{equation:Sedimentation/Sed_Theory_and_Future_Work:Sedimentation/Sed_Theory_and_Future_Work:11}
\begin{split}\rho_{sludge} = \left( 1 - \frac{\rho_{H_2O}}{\rho_{Clay}} \right) C_{sludge} + \rho_{H_2O}\end{split}
\end{equation}
The piezometric head (measured in equivalent change in height of the recycle line liquid) that is causing the flow through the recycle line is equal to the difference in density between the recycled sludge and the floc blanket times the height of the floc blanket normalized by the recycle line density.
\begin{equation}\label{equation:Sedimentation/Sed_Theory_and_Future_Work:Sedimentation/Sed_Theory_and_Future_Work:12}
\begin{split}H_l = H_{fb}\frac{\rho_{sludge} - \rho_{fb}}{\rho_{sludge}}\end{split}
\end{equation}
Substitute to replace the sludge and floc blanket densities.
\begin{equation}\label{equation:Sedimentation/Sed_Theory_and_Future_Work:Sedimentation/Sed_Theory_and_Future_Work:13}
\begin{split}H_l = H_{fb}\frac{\left( 1 - \frac{\rho_{H_2O}}{\rho_{Clay}} \right) C_{sludge} + \rho_{H_2O} -\left[  \left( 1 - \frac{\rho_{H_2O}}{\rho_{Clay}} \right) C_{fb} + \rho_{H_2O} \right]} {\left( 1 - \frac{\rho_{H_2O}}{\rho_{Clay}} \right) C_{sludge} + \rho_{H_2O}}\end{split}
\end{equation}
Simplify the equation for the head loss in the recycle tube.
\begin{equation}\label{equation:Sedimentation/Sed_Theory_and_Future_Work:Sedimentation/Sed_Theory_and_Future_Work:14}
\begin{split}H_l = H_{fb}\frac{ C_{sludge} -C_{fb}} { C_{sludge} + \frac{\rho_{H_2O}\rho_{Clay}}{  \rho_{Clay} -\rho_{H_2O} }}\end{split}
\end{equation}
The recycle tube is assumed to be sloped at 60 degrees from the horizontal to enable further consolidation. The length of the recycle tube is
\begin{equation}\label{equation:Sedimentation/Sed_Theory_and_Future_Work:Sedimentation/Sed_Theory_and_Future_Work:15}
\begin{split}L_{tube} = H_{fb}/sin(60)\end{split}
\end{equation}
We will assume that the dynamic viscosity of the sludge is the same as the dynamic viscosity of water. We will calculate the kinematic viscosity of the sludge by dividing the dynamic viscosity of water by the density of the recycle.

Now we can solve for the required tube diameter

\fvset{hllines={, ,}}%
\begin{sphinxVerbatim}[commandchars=\\\{\}]
\PYG{k+kn}{from} \PYG{n+nn}{aide\PYGZus{}design}\PYG{n+nn}{.}\PYG{n+nn}{play} \PYG{k}{import}\PYG{o}{*}
\PYG{n}{Temperature}\PYG{o}{=} \PYG{l+m+mi}{20}\PYG{o}{*}\PYG{n}{u}\PYG{o}{.}\PYG{n}{degC}
\PYG{n}{D\PYGZus{}fb}\PYG{o}{=}\PYG{l+m+mf}{2.5}\PYG{o}{*}\PYG{n}{u}\PYG{o}{.}\PYG{n}{cm}
\PYG{n}{A\PYGZus{}fb} \PYG{o}{=} \PYG{n}{pc}\PYG{o}{.}\PYG{n}{area\PYGZus{}circle}\PYG{p}{(}\PYG{n}{D\PYGZus{}fb}\PYG{p}{)}
\PYG{n}{H\PYGZus{}fb} \PYG{o}{=} \PYG{l+m+mf}{1.5} \PYG{o}{*} \PYG{n}{u}\PYG{o}{.}\PYG{n}{m}
\PYG{n}{Angle\PYGZus{}tube} \PYG{o}{=} \PYG{l+m+mi}{60}\PYG{o}{*}\PYG{n}{u}\PYG{o}{.}\PYG{n}{deg}
\PYG{n}{L\PYGZus{}tube} \PYG{o}{=} \PYG{n}{H\PYGZus{}fb}\PYG{o}{/}\PYG{n}{np}\PYG{o}{.}\PYG{n}{sin}\PYG{p}{(}\PYG{n}{Angle\PYGZus{}tube}\PYG{p}{)}
\PYG{n}{density\PYGZus{}clay}\PYG{o}{=}\PYG{l+m+mi}{2650}\PYG{o}{*}\PYG{n}{u}\PYG{o}{.}\PYG{n}{kg}\PYG{o}{/}\PYG{n}{u}\PYG{o}{.}\PYG{n}{m}\PYG{o}{*}\PYG{o}{*}\PYG{l+m+mi}{3}

\PYG{n}{H\PYGZus{}l} \PYG{o}{=} \PYG{n}{H\PYGZus{}fb}\PYG{o}{*}\PYG{p}{(}\PYG{n}{C\PYGZus{}recycle}\PYG{o}{\PYGZhy{}}\PYG{n}{C\PYGZus{}fb}\PYG{p}{)}\PYG{o}{/}\PYG{p}{(}\PYG{n}{C\PYGZus{}recycle}\PYG{o}{+}\PYG{p}{(}\PYG{p}{(}\PYG{n}{pc}\PYG{o}{.}\PYG{n}{density\PYGZus{}water}\PYG{p}{(}\PYG{n}{Temperature}\PYG{p}{)}\PYG{o}{*}\PYG{n}{density\PYGZus{}clay}\PYG{p}{)}\PYG{o}{/}\PYG{p}{(}\PYG{n}{density\PYGZus{}clay}\PYG{o}{\PYGZhy{}}\PYG{n}{pc}\PYG{o}{.}\PYG{n}{density\PYGZus{}water}\PYG{p}{(}\PYG{n}{Temperature}\PYG{p}{)}\PYG{p}{)}\PYG{p}{)}\PYG{p}{)}
\PYG{n}{H\PYGZus{}l}
\PYG{n}{Q\PYGZus{}plant}\PYG{o}{=}\PYG{n}{v\PYGZus{}up}\PYG{o}{*}\PYG{n}{A\PYGZus{}fb}
\PYG{n}{Pi\PYGZus{}recycle}\PYG{o}{=}\PYG{l+m+mf}{0.5}
\PYG{n}{density\PYGZus{}recycle} \PYG{o}{=} \PYG{p}{(}\PYG{l+m+mi}{1} \PYG{o}{\PYGZhy{}} \PYG{n}{pc}\PYG{o}{.}\PYG{n}{density\PYGZus{}water}\PYG{p}{(}\PYG{n}{Temperature}\PYG{p}{)}\PYG{o}{/}\PYG{n}{density\PYGZus{}clay}\PYG{p}{)}\PYG{o}{*}\PYG{n}{C\PYGZus{}recycle} \PYG{o}{+} \PYG{n}{pc}\PYG{o}{.}\PYG{n}{density\PYGZus{}water}\PYG{p}{(}\PYG{n}{Temperature}\PYG{p}{)}
\PYG{n}{nu\PYGZus{}recycle} \PYG{o}{=} \PYG{n}{pc}\PYG{o}{.}\PYG{n}{viscosity\PYGZus{}dynamic}\PYG{p}{(}\PYG{n}{Temperature}\PYG{p}{)}\PYG{o}{/}\PYG{n}{density\PYGZus{}recycle}
\PYG{n}{D\PYGZus{}recycle} \PYG{o}{=} \PYG{n}{pc}\PYG{o}{.}\PYG{n}{diam\PYGZus{}pipe}\PYG{p}{(}\PYG{n}{Q\PYGZus{}plant}\PYG{o}{*}\PYG{n}{Pi\PYGZus{}recycle}\PYG{p}{,}\PYG{n}{H\PYGZus{}l}\PYG{p}{,}\PYG{n}{L\PYGZus{}tube}\PYG{p}{,}\PYG{n}{nu\PYGZus{}recycle}\PYG{p}{,}\PYG{l+m+mf}{0.01}\PYG{o}{*}\PYG{n}{u}\PYG{o}{.}\PYG{n}{mm}\PYG{p}{,}\PYG{l+m+mi}{2}\PYG{p}{)}
\PYG{n}{D\PYGZus{}recycle}\PYG{o}{.}\PYG{n}{to}\PYG{p}{(}\PYG{n}{u}\PYG{o}{.}\PYG{n}{mm}\PYG{p}{)}
\PYG{n}{D\PYGZus{}recycle}\PYG{o}{.}\PYG{n}{to}\PYG{p}{(}\PYG{n}{u}\PYG{o}{.}\PYG{n}{inch}\PYG{p}{)}
\end{sphinxVerbatim}

The head loss in the recycle tube is approximately 1.6 cm in a 1.5 m deep floc blanket.

The recycle line will be installed between the bottom of the tube settler and the inlet to the sedimentation tank. The recycle line will connect  directly to the side of the sedimentation tank to minimize minor losses. We will use a 0.25” ID, 3/8”OD clear flexible tube for the recycle line. We will use PVC glue to attach the flexible tube to the rigid clear PVC tubing.

It is possible that it will be necessary to prevent flow in the recycle line initially so that it doesn’t flow upward. Once the tube begins filling with solids it should be possible for it to start flowing downwards.


\section{Floc Volcanoes}
\label{\detokenize{Sedimentation/Sed_Theory_and_Future_Work:floc-volcanoes}}\label{\detokenize{Sedimentation/Sed_Theory_and_Future_Work:id3}}
Floc volcanoes are caused by differences in temperature between the water that is in a sedimentation tank and the incoming water. If the incoming water is warmer than the water that is already in the sedimentation tank, then the incoming water will be buoyant and will rise quickly to the top of the sedimentation tank and carry flocs to the effluent launder.

Temperature fluctuations can be especially pronounced with small scale water supplies where small streams and small diameter transmission lines can be exposed to the sun and can warm up dramatically oduring a few hours of sunshine. Given that temperature changes and density changes can not easily be engineered, the only solution that we have is to reduce the time that water spends in the sedimentation tank so that the influent water is closer to the average temperature of the water in the sedimentation tank. Solar heating causing the raw water temperature to go from a minimum at 6 am to a maximum at 1 pm. AguaClara sedimentation tanks currently have a residence time of approximately 2 m / (1 mm/s) or 2000 s. We anticipate that by increasing the upflow velocity and by introducing floc recycle that the effects of temperature induced floc volcanoes will be reduced.


\chapter{Filtration Design}
\label{\detokenize{Filtration/Filtration_Design:filtration-design}}\label{\detokenize{Filtration/Filtration_Design:title-filtration}}\label{\detokenize{Filtration/Filtration_Design::doc}}
This section deals with search for a self-backwashing filter which is functional over a wide range of flows. While in basic concept,running water through a sand bed, filters are the simplest of the unit processes, the are probably the most complex in design within an AguaClara plant because they are not inherently self-cleaning. Additionally. The search to overcome this problem has led to the development of a Stacked Rapid Sand Filter (StaRS Filter) a novel filter design which provides a hydraulic backwash system and works over a large range of flows with some adpatation for small flows.


\section{Important Terms and Equations}
\label{\detokenize{Filtration/Filtration_Design:important-terms-and-equations}}\label{\detokenize{Filtration/Filtration_Design:heading-filtration-terms}}
Terms:
\begin{enumerate}
\item {} 
Porosity

\item {} 
StaRS Filter

\item {} 
Backwash

\end{enumerate}

Equations:
\begin{enumerate}
\item {} 
\end{enumerate}


\section{AguaClara Flow Control and Measurement Technologies}
\label{\detokenize{Filtration/Filtration_Design:aguaclara-flow-control-and-measurement-technologies}}\label{\detokenize{Filtration/Filtration_Design:heading-aguaclara-filtration-technologies}}
What it is
What it does and why
How it works
Notes


\subsection{Filter Types}
\label{\detokenize{Filtration/Filtration_Design:filter-types}}\label{\detokenize{Filtration/Filtration_Design:heading-filter-types}}
The principal difference in various filters is the difference in velocity of water through them. This in turn determines the plan area of the filter for a particular flow.

For a multistage filter system, the filter areas for the Dynamic, Roughing, and Slow Sand filters are included in a total area. For a given flow, and the velocity for each filter type the total area is:
\begin{equation}\label{equation:Filtration/Filtration_Design:area_stuff}
\begin{split}  A_{Total} = \frac{Q}{v_{Dynamic}} + \frac{Q}{v_{Rough}} + \frac{Q}{v_{Slow}}\end{split}
\end{equation}
Using this formula it becomes easy to see the relative sizes of different filter systems. \sphinxstylestrong{Note: Clare, consider adding the table from slide 23 here.}

Understanding the amount of area requires for this component makes it easy to see why certain systems would be preferable to others, but also that overall filtration is only a polishing step and cannot treat as well as other unit processes, predominantly the flocculation-sedimentation combination.


\section{Porosity}
\label{\detokenize{Filtration/Filtration_Design:porosity}}\label{\detokenize{Filtration/Filtration_Design:heading-porosity}}
In understading how sand filtration works, porosity is one of the most important concepts to be familiar with. Porosity refers to the ratio of the void volume to the total volume of a control volume.
\begin{equation}\label{equation:Filtration/Filtration_Design:porosity}
\begin{split}  \phi_{FiSand} = \frac{\rlap{-} V_{voids}}{\rlap{-} V_{total}}\end{split}
\end{equation}
Porosity is determined by the geometry of the material in the control volume, but also by the size of the particles involved. If you have three different sized spheres (such as .0um clay, .2mm sand, and 1 cm gravel) in three different buckets, each bucket will have the same porosity. To minimize the porosity, the three materials could be mixed because the smaller materials would be filling the pore space of the larger material.

\begin{figure}[htbp]
\centering
\capstart

\noindent\sphinxincludegraphics{{figure_porosity}.png}
\caption{Within each box, the spheres are different sizes. However the total porosity is the same. To minimize the pore space, the smaller particles could be used to fill the spore space between the larger particles, though in a filter this is not necessarily ideal.}\label{\detokenize{Filtration/Filtration_Design:id2}}\label{\detokenize{Filtration/Filtration_Design:figure-porosity}}\end{figure}

One way that the relative size of particles is characterized is by describing the size of the smallest 10\% of grains, and the smallest 60\% of grains. That is:

\(D_{10}\) = the sieve size that passes 10\% by mass of sand through

\(D_{60}\) = the sieve sixe that passes 60\% by mass of sand through

\(D_{10}\) is used for particle removal models, and \(D_{60}\) is used for hydrualic modeling.

The relationship of the two is the uniformity coeffecient:
\begin{equation}\label{equation:Filtration/Filtration_Design:uniformity_coefficient}
\begin{split}  UC = \frac{D_{60}}{D_{10}}\end{split}
\end{equation}
The uniformity coefficient describes the uniformity of the sand. A \(UC = 1\) indicates that every grain of sand is the same size, which is the ideal case. A large \(UC\) is indicative of a wide range of grain sizes which can cause trouble during filter operation and backwash, as stratification occurs and the porosity changes with respect to depth in the filter.

During backwash sand is fluidized and the sand bed expands. This expansion causes a change in porosity of the sand bed (as the volume of water occupied by the sand is increased). The porosity and height of the sand bed are directly related through the following equation:
\begin{equation}\label{equation:Filtration/Filtration_Design:backwash_porosity}
\begin{split}  \phi_{FiSandBw} = \frac{\phi_{FiSand} H_{FiSand} A_{Fi} + \left( H_{FiSandBw} - H_{FiSand} \right) A_{Fi}}{H_{FiSandBw} A_{Fi}}\end{split}
\end{equation}
\begin{DUlineblock}{0em}
\item[] Such that:
\item[] \(phi_{FiSandBw}\) = sand porosity during backwash
\item[] \(phi_{FiSand}\) = settled sand porosity
\item[] \(H_{FiSand}\) = height of sand in the filter
\item[] \(H_{FiSandBw}\) = height of sand during backwash
\item[] \(A_{Fi}\) = filter area
\end{DUlineblock}

From this it becomes possible to directly relate porosity (as above) to the filter expansion ratio, which is simply the ratio of the heights of the expanded sand bed and the settled sand bed:
\begin{equation}\label{equation:Filtration/Filtration_Design:filter_expansion_ratio}
\begin{split}\Pi_{FiBw} = \frac{H_{FiSandBw}}{H_{FiSand}}\end{split}
\end{equation}
\begin{DUlineblock}{0em}
\item[] Such that:
\item[] \(Pi_{FiBw}\) = the expansion ratio value
\item[] \(H_{FiSand}\) = height of sand in the filter
\item[] \(H_{FiSandBw}\) = height of sand during backwash
\end{DUlineblock}


\section{Headloss Requirements}
\label{\detokenize{Filtration/Filtration_Design:headloss-requirements}}\label{\detokenize{Filtration/Filtration_Design:heading-headloss-requirements}}
One of the key parameters in design of a filter is the headloss through the system because it determines the required fluid velocity for backwash. The Karmen Kozeny Equation, an adaptation of the Hagen Pouseille equation (ref from elsewhere) describes the headloss through a clean bed during filtration.
\begin{equation}\label{equation:Filtration/Filtration_Design:karmen_kozeny_clean_bed}
\begin{split}  \frac{h_l}{H_{FiSand}} = 36 k \frac{\left( 1 - \phi_{FiSand} \right)^2}{\phi_{FiSand}^3} \frac{\nu V_{Fi}}{g D_{60}^2}\end{split}
\end{equation}
\begin{DUlineblock}{0em}
\item[] Such that:
\item[] \(h_l\) = headloss in sand bed
\item[] \(H_{FiSand}\) = the sand bed depth/length of flow paths
\item[] \(phi_{FiSand}\) = porosity of sand
\item[] \(nu\) = kinematic viscosity
\item[] \(V_{Fi}\) = the water velocity in the filter
\item[] \(D_{60}\) = the size of the sand
\item[] \(g\) = gravity
\item[] \(k\) = Kozeny constant (5 for most filtration cases)
\end{DUlineblock}

This equation is valid for Reynolds numbers less than 6. Where:
\({\rm Re}  = \frac{D_{60} V_{Fi}}{\nu}\)

The headloss during backwash is taken as the design parameter, so other values are constructed around it.

The following equation describes the headloss through the fluidized bed:

\begin{DUlineblock}{0em}
\item[] Such that:
\item[] \(h_{l_{FiBw}\) = the headloss in the fluidized bed
\item[] \(H_{FiSand}\) =  the depth of the settled sand bed
\item[] \(phi_{FiSand}\) = the settled sand porosity
\item[] \(rho_{Sand}\)  = the sand density
\item[] \(rho_{Water}\) = the water density
\end{DUlineblock}

Using these two equations the minimum velocity for snad fluidization can be found!

From this equation it can easily be seen that if the diameter of the sand at the top is half the diameter of the sand at the bottom, it will fluidize at one quarter the velocity. This result indicates that fluidization occurring at the top of the filter is \sphinxstylestrong{not} indicative of fluidization at the bottom.


\section{Backwash}
\label{\detokenize{Filtration/Filtration_Design:backwash}}\label{\detokenize{Filtration/Filtration_Design:heading-backwash}}
When considering backwash design, there are two main factors that constitute a dilemma. The first, backwash velocity must be must greater than filtration velocity (to expand the sand bed), and second, the backwash water must be clean water (cleaning with dirty water introduces more particles into the filter). This limits the paths water can take during the backwash process. The conventional options include pumping it back from the storage tank, using a set of parallel fiters to backwash one filter at a time, or storing the filtered water at an adequate elevation. Due to energy limitations and space constraints, the conventional solutions are simply not feasible for this system. Examples that illustrate why they cannot work can be found in the derivations sections(?)(or the examples?)

\sphinxstylestrong{brief example here?}

To avoid electricity, pumps can be immeidately ruled out.

Parallel filters would require too much area and wouldn’t work well under low flow conditions:

Given:
\begin{equation}\label{equation:Filtration/Filtration_Design:filter_base_conditions}
\begin{split}  Q_{Plant} = 6 \, \frac{L}{s} \,\,\,\,\, V_{Fi} = 1.8 \, \frac{mm}{s} \,\,\,\,\, V_{Bw} = 9 \, \frac{mm}{s}\end{split}
\end{equation}
As the ratio of the backwash velocity to the filter velocity is 5, 5 filters will be needs to provide enough flow to backash one: Therefore the number of parallel filters is 6:

\(N_{Fi} = \frac{V_{Bw}}{V_{Fi}} + 1 = 6\)

In this system, the water exiting five of the filters would be diverted to backwash one of the other filters. In addtion to requiring the plan view area of 6 filters, each filter would need to be backwashed independently, meaning it would take 6x longer and use 6x the water as compared to just having one filter. Another detriment to this system is that in low flows (such as drought conditions) not enough water would be passing through the system to backwash at points since all the water is diverted to backwash.

The third option, elevating the filtered water to provide enough head to cause backwash, is also unfeasible.

\sphinxstylestrong{add the third one at some later point if it’s useful}

How can we find a solution?

If the velocities could be more similar the filter could work!

This could be accomplished in several ways: such as decreasing the media density thus lowering velocity to fluidize it, decrease the media diameter thus lowering the fluidization velocity, or make a more compact filter which filters in parallel and backwashes in series.

As changing the material characteristics of the sand is challenging, a more compact filter is the chosen design. As it happens this innovation results in a more concpetually difficult filter. In the design, six layers of sand are stacked, there are four inlets, and three outlets which are all in use during filtration. During backwash only one inlet is used and the backwash water is discarded through a separate manifold. Throughout this section, figures and images will be the best methods to understand the design flow through the system, and will be supplemented by the text.

This overall design can be seen in Figure XXXXX.

(figure of the full system)

Tasks for clare for Thursday + Friday morning: insert images! none of them are in yet. Save as pngs. streamline the way you want this to work as well. like overall structure

In is most basic schema, the filter is a series of pipes leading into a deep box with 1.2 meters of sand (for most filters)

As a parcel of water traveling in the filter the first part of the filter is the inlet box. The inlet box is a shallow box with four holes in the bottom. The holes lead into four pipes which lead into different levels of the sand filter. At the outlet of each of these pipes into the sand filter is a structure designed to spread the flow over the entire footprint area of the filter. These structures have slots which allow water out of the inlet pipes into the sand bed. Across a layer of sand from the inlet is an outlet pipe in the same shapes as the slotted pipe inlets. Water passes into the pipes and up to the fitler outlet box where it only needs to be chlorinated before being distributed.

Steps of designing a filter.
1.


\section{Siphon}
\label{\detokenize{Filtration/Filtration_Design:siphon}}\label{\detokenize{Filtration/Filtration_Design:id1}}

\chapter{Filtration Derivations}
\label{\detokenize{Filtration/Filtration_Derivations:filtration-derivations}}\label{\detokenize{Filtration/Filtration_Derivations:title-filtration-derivations}}\label{\detokenize{Filtration/Filtration_Derivations::doc}}\phantomsection\label{\detokenize{Filtration/Filtration_Derivations:derivation-backwash-headloss-force-balance}}
To determine the head loss during backwash a force balance should be performed between the water and the sand per unit of filter area (thus pressure values will be yielded). A schematic for this system is shown below:

\begin{figure}[htbp]
\centering

\noindent\sphinxincludegraphics[width=0.500\linewidth]{{figure_force_balance}.png}
\label{\detokenize{Filtration/Filtration_Derivations:figure-force-balance}}\end{figure}

The pressure from the water:
\begin{equation}\label{equation:Filtration/Filtration_Derivations:Filtration/Filtration_Derivations:0}
\begin{split}P_{Manometer} = \rho_{Water} g \left( H_{W_1} + H_{W_2} + \phi_{FiSand} H_{FiSand} \right) + \rho_{Sand} g \left( 1 - \phi_{FiSand} \right) H_{FiSand}\end{split}
\end{equation}
\begin{DUlineblock}{0em}
\item[] Such that:
\item[] \(P_{Manometer} =\) total height from the bottom of the filter to the inlet box
\item[] \(\rho_{Water} =\) density of water
\item[] \(H_{W_1} =\) the distnace from the top of the settled sand bed to the water surface in the filter
\item[] \(H_{W_2} =\) the height of the water below the sand bed but within the filter
\item[] \(\phi_{FiSand} =\) porosity of sand
\item[] \(H_{FiSand} =\) height of the filter bed
\item[] \(\rho_{Sand} =\) density of sand
\end{DUlineblock}

The pressure from the sand:
\begin{equation}\label{equation:Filtration/Filtration_Derivations:Filtration/Filtration_Derivations:1}
\begin{split}P_{Manometer} = \rho_{Water} g \left( H_{W_1} + H_{W_2} + H_{FiSand} + h_{l_{FiBw}} \right)\end{split}
\end{equation}
\begin{DUlineblock}{0em}
\item[] Such that:
\item[] \(h_{l_{FiBw}} =\) the difference in height of the inlet and water surface height during backwash; the backwash head loss
\end{DUlineblock}

Setting them equal for a force balance:
\begin{equation}\label{equation:Filtration/Filtration_Derivations:Filtration/Filtration_Derivations:2}
\begin{split}\rho_{Water} g \left( H_{W_1} + H_{W_2} + \phi_{FiSand} H_{FiSand} \right) + \rho_{Sand} g \left( 1 - \phi_{FiSand} \right) H_{FiSand} = \rho_{Water} g \left( H_{W_1} + H_{W_2} + H_{FiSand} + h_{l_{FiBw}} \right)\end{split}
\end{equation}
Which simplifies to:
\begin{equation}\label{equation:Filtration/Filtration_Derivations:Filtration/Filtration_Derivations:3}
\begin{split}h_{l_{FiBw}} = \frac{\rho_{Sand} - \rho_{Water}}{\rho_{Water}} \left( 1 - \phi_{FiSand} \right) H_{FiSand}
or
h_{l_{FiBw}} = H_{FiSand} \left( 1 - \phi_{FiSand} \right)  \left( \frac{\rho_{Sand}}{\rho_{Water}} - 1 \right)\end{split}
\end{equation}
This result gives a ratio of the head loss during backwash to the height difference during forward operation. With \(\phi_{FiSand} = 0.4\) and \(\rho_{Sand} = 2650 kg/m^3\) the value of this ratio is:
\begin{equation}\label{equation:Filtration/Filtration_Derivations:Filtration/Filtration_Derivations:4}
\begin{split}\left( 1- \Phi_{FiSand} \right) \left( \frac{\rho_{FiSand}}{\rho_{Water}} - 1 \right) = 0.99\end{split}
\end{equation}
Thus:
\begin{equation}\label{equation:Filtration/Filtration_Derivations:Filtration/Filtration_Derivations:5}
\begin{split}h_{l_{FiBw}} = H_{FiSand} * 0.99\end{split}
\end{equation}

\chapter{Dissolved Gas Introduction}
\label{\detokenize{Dissolved_Gas/DG_Intro:dissolved-gas-introduction}}\label{\detokenize{Dissolved_Gas/DG_Intro:title-dissolved-gas-introduction}}\label{\detokenize{Dissolved_Gas/DG_Intro::doc}}
Dissolved gas supersaturation and subsequent bubble formation can cause significant performance deterioration in drinking water treatment plants. Supersaturation means that the dissolved gas concentration is greater than the equilibrium concentration. The liquid phase equilibrium concentration of a gas is directly proportional to the absolute partial pressure of the gas in the gas phase. This relationship is described by Henry’s law.


\section{Henry’s Law}
\label{\detokenize{Dissolved_Gas/DG_Intro:henry-s-law}}\label{\detokenize{Dissolved_Gas/DG_Intro:heading-henrys-law}}

\section{Effects of Gas Bubbles in Water Treatment Plants}
\label{\detokenize{Dissolved_Gas/DG_Intro:effects-of-gas-bubbles-in-water-treatment-plants}}\label{\detokenize{Dissolved_Gas/DG_Intro:heading-effects-of-gas-bubbles-in-water-treatment-plants}}
The American Water Works Association recognized the detrimental effects of gas bubble evolution in drinking water treatment plants and sponsored research at Virginia Polytechnic Institute and State University. \sphinxhref{../\_static/references/PaoloScardinaDissertation2004.pdf}{Paolo Scardina’s Dissertation} provides an excellent review of the problems that are caused by gas bubbles.

Dissolved gas removal is also a goal after anaerobic wastewater treatment for enhanced removal of dissolved methane.


\subsection{Flocculation}
\label{\detokenize{Dissolved_Gas/DG_Intro:flocculation}}\label{\detokenize{Dissolved_Gas/DG_Intro:heading-dg-flocculation}}
Minimal adverse effects. Foam produced can dry and be blown around the water treatment plant. Nuisance and aesthetically unpleasing.


\subsection{Floc Blankets}
\label{\detokenize{Dissolved_Gas/DG_Intro:floc-blankets}}\label{\detokenize{Dissolved_Gas/DG_Intro:heading-dg-floc-blankets}}
Disrupts floc blanket, breaks flocs, carries flocs into plate settlers.


\subsection{Sedimentation}
\label{\detokenize{Dissolved_Gas/DG_Intro:sedimentation}}\label{\detokenize{Dissolved_Gas/DG_Intro:heading-dg-sedimentation}}
Bubbles carry flocs rapidly to the top of the sedimentation tank where they form a surface mat of flocs. The surface mat of flocs isn’t necessarily harmful to plant performance, but it is a nuisance and requires that the operators scrape the flocs to remove them from the sedimentation tank. Some of the rising flocs are swept into the exit manifold and thus the settled water turbidity can be significantly increased by these rising flocs.


\subsection{Filtration}
\label{\detokenize{Dissolved_Gas/DG_Intro:filtration}}\label{\detokenize{Dissolved_Gas/DG_Intro:heading-dg-filtration}}
Filtration performance can be severely harmed by gas bubbles. The bubbles form in the pores of the filter and are unable to leave the pore due to surface tension. Filter media can fill rapidly with gas bubbles and this results in increased head loss or decreased flow through the filter. Bubble formation can significantly reduce the filter run time. The amount of bubble formation is strongly dependent on the pressure in the filter. Filters that operate with a deep column of water on top of the filter bed can be unaffected by dissolved gasses if the high pressure within the filter raises the equilibrium dissolved gas concentration above the actual dissolved gas concentration. Enclosed Stacked Rapid Sand Filters are more vulnerable to bubble formation that Open Stacked Rapid Sand Filters because of the difference in pressure in the sand bed. This pressure analysis also reveals that gas bubble formation will tend to occur at the lowest pressure zone in the filter bed. This low pressure zone can occur in a down flow filter right below the partially clogged section of the bed because of the decrease in pressure due to head loss.

At the El PODA Nicaragua AguaClara plant there was so much dissolved air in the influent water that the filters could only operate one hour before requiring backwash.


\section{Dissolved Gas Sources}
\label{\detokenize{Dissolved_Gas/DG_Intro:dissolved-gas-sources}}\label{\detokenize{Dissolved_Gas/DG_Intro:heading-dissolved-gas-sources}}\begin{itemize}
\item {} 
high pressure regions in transmission lines where the pipeline flow capacity exceeds the need AND where the flow control is restricted by upstream supply rather than by downstream head loss control (meaning by partially closing a valve!).

\item {} 
Temperature increase that reduces the equilibrium concentration. This gas source is observed in the AguaClara lab facilities where hot and cold incoming tap water are mixed to produce “room temperature” water. The increase in water temperature results in bubble formation.

\end{itemize}


\section{Dissolved Gas Concentration Reduction}
\label{\detokenize{Dissolved_Gas/DG_Intro:dissolved-gas-concentration-reduction}}\label{\detokenize{Dissolved_Gas/DG_Intro:heading-dissolved-gas-concentration-reduction}}

\subsection{Methods to provide nucleation sites}
\label{\detokenize{Dissolved_Gas/DG_Intro:methods-to-provide-nucleation-sites}}\begin{itemize}
\item {} 
aeration
- air can be pumped through diffusers
- air can be sucked into the diffusers if the reactor operated at less than atmospheric pressure

\item {} 
fluidized bed containing particles that have nucleation sites

\end{itemize}

The fluidized nucleation site particles must be dense enough and large enough so that attached gas bubbles can’t carry them out of the degassing reactor.


\subsection{Methods to reduce degassing reactor pressure}
\label{\detokenize{Dissolved_Gas/DG_Intro:methods-to-reduce-degassing-reactor-pressure}}
The top of the degassing reactor can be located several meters above the entrance tank of the water treatment plant. The higher the tank (up to a max of 10 m) the more pronounced the pressure reduction will be and the more effective the reactor will be at degassing. The degassing reactor will produce a water and bubble effluent that must be

Can we make some sort of a diagnostic guide based on symptoms? Make the following into a giant table.


\chapter{Problems}
\label{\detokenize{Troubleshooting/Troubleshooting:problems}}\label{\detokenize{Troubleshooting/Troubleshooting::doc}}
Temperature fluctuations

Calcium from calcium hypochlorite combines with carbonate in the water to form low solubility calcium carbonate.


\chapter{Solutions}
\label{\detokenize{Troubleshooting/Troubleshooting:solutions}}
Drip chlorine into water rather than injecting it to eliminate formation of precipitate at the injection point

Use valves at the end of the pipeline to control transmission line flow rate rather than upstream control. Review the transmission line to ensure that all sections of the line have flow controlled by limiting available driving head. Air entrainment occurs when the available head exceeds the head required to transmit the target flow.


\begin{savenotes}\sphinxattablestart
\raggedright
\sphinxcapstartof{table}
\sphinxcaption{Table of symptoms, problems, and solutions for AguaClara plant operation.}\label{\detokenize{Troubleshooting/Troubleshooting:id1}}\label{\detokenize{Troubleshooting/Troubleshooting:table-troubleshooting}}
\sphinxaftercaption
\begin{tabulary}{\linewidth}[t]{|T|T|T|}
\hline
\sphinxstyletheadfamily 
Observation
&\sphinxstyletheadfamily 
Problem
&\sphinxstyletheadfamily 
Solution
\\
\hline
Air bubbles
&
Air entrainment in the transmission line and transport to a high pressure zone in the pipeline where the air is dissolved in the water.
&\\
\hline
Bubbles in sedimentation tanks
&&\\
\hline
Bubbles in EStaRS
&
water entering the plant is supersaturated with air and EStaRS filters operate at very low pressure (compared with OStaRS)
&
eliminate air entrainment in transmission line
\\
\hline&&
add a unit process (TBD) that removes excess dissolved air
\\
\hline
Rising flocs
&&\\
\hline
Short filter runtimes
&
poor performance of floc/sed system
&\\
\hline
Gradual increase in post backwash head loss in filters
&&\\
\hline
Scale deposition in the distribution system
&&\\
\hline
Clogging of chlorination system tubes and formation of precipitate at the injection point
&&\\
\hline
\end{tabulary}
\par
\sphinxattableend\end{savenotes}

\sphinxhref{https://github.com/AguaClara/Textbook/releases/latest}{PDF and LaTeX versions} %
\begin{footnote}[1]\sphinxAtStartFootnote
PDF and LaTeX versions may contain visual oddities because it is generated automatically. The website is the recommended way to read this textbook. \sphinxhref{https://github.com/AguaClara/Textbook}{Please visit our GitHub site} to submit an issue, contribute, or comment.
%
\end{footnote}.
\paragraph{\sphinxstylestrong{Notes}}

\begin{sphinxthebibliography}{SODA92}
\bibitem[Haa98]{\detokenize{Haa98}}{\phantomsection\label{\detokenize{Flocculation/Floc_Design:floc-haarhoff-design-1998}} 
J. Haarhoff. Design of around-the-end hydraulic flocculators. \sphinxstyleemphasis{Journal of Water Supply: Research and Technology-Aqua}, 47(3):142\textendash{}152, May 1998. URL: \sphinxurl{https://iwaponline.com/aqua/article/47/3/142/31711/Design-of-around-the-end-hydraulic-flocculators}, \sphinxhref{https://doi.org/10.2166/aqua.1998.20}{doi:10.2166/aqua.1998.20}.
}
\bibitem[HWJ01]{\detokenize{HWJ01}}{\phantomsection\label{\detokenize{Flocculation/Floc_Design:floc-haarhoff-towards-2001}} 
Johannes Haarhoff, Van Der Walt, and Jeremia J. Towards optimal design parameters for around-the-end hydraulic flocculators. \sphinxstyleemphasis{Journal of Water Supply: Research and Technology-Aqua}, 50(3):149\textendash{}160, May 2001. URL: \sphinxurl{https://iwaponline.com/aqua/article/50/3/149/30498/Towards-optimal-design-parameters-for-around-the}, \sphinxhref{https://doi.org/10.2166/aqua.2001.0014}{doi:10.2166/aqua.2001.0014}.
}
\bibitem[SODA92]{\detokenize{SODA92}}{\phantomsection\label{\detokenize{Flocculation/Floc_Design:floc-schulz-surface-1992}} 
Christopher R. Schulz, Daniel A. Okun, David Donaldson, and John Austin. \sphinxstyleemphasis{Surface water treatment for communities in developing countries}. John Wiley \& Sons, 1992. URL: \sphinxurl{http://www.google.com}.
}
\end{sphinxthebibliography}



\renewcommand{\indexname}{Index}
\printindex
\end{document}