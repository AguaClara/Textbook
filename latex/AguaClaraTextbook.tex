%% Generated by Sphinx.
\def\sphinxdocclass{report}
\documentclass[letterpaper,10pt,english]{sphinxmanual}
\ifdefined\pdfpxdimen
   \let\sphinxpxdimen\pdfpxdimen\else\newdimen\sphinxpxdimen
\fi \sphinxpxdimen=.75bp\relax

\PassOptionsToPackage{warn}{textcomp}
\usepackage[utf8]{inputenc}
\ifdefined\DeclareUnicodeCharacter
 \ifdefined\DeclareUnicodeCharacterAsOptional
  \DeclareUnicodeCharacter{"00A0}{\nobreakspace}
  \DeclareUnicodeCharacter{"2500}{\sphinxunichar{2500}}
  \DeclareUnicodeCharacter{"2502}{\sphinxunichar{2502}}
  \DeclareUnicodeCharacter{"2514}{\sphinxunichar{2514}}
  \DeclareUnicodeCharacter{"251C}{\sphinxunichar{251C}}
  \DeclareUnicodeCharacter{"2572}{\textbackslash}
 \else
  \DeclareUnicodeCharacter{00A0}{\nobreakspace}
  \DeclareUnicodeCharacter{2500}{\sphinxunichar{2500}}
  \DeclareUnicodeCharacter{2502}{\sphinxunichar{2502}}
  \DeclareUnicodeCharacter{2514}{\sphinxunichar{2514}}
  \DeclareUnicodeCharacter{251C}{\sphinxunichar{251C}}
  \DeclareUnicodeCharacter{2572}{\textbackslash}
 \fi
\fi
\usepackage{cmap}
\usepackage[T1]{fontenc}
\usepackage{amsmath,amssymb,amstext}
\usepackage{babel}
\usepackage{times}
\usepackage[Bjarne]{fncychap}
\usepackage[,numfigreset=1,mathnumfig]{sphinx}

\usepackage{geometry}

% Include hyperref last.
\usepackage{hyperref}
% Fix anchor placement for figures with captions.
\usepackage{hypcap}% it must be loaded after hyperref.
% Set up styles of URL: it should be placed after hyperref.
\urlstyle{same}
\addto\captionsenglish{\renewcommand{\contentsname}{Contributor's guide}}

\addto\captionsenglish{\renewcommand{\figurename}{Fig.}}
\addto\captionsenglish{\renewcommand{\tablename}{Table}}
\addto\captionsenglish{\renewcommand{\literalblockname}{Listing}}

\addto\captionsenglish{\renewcommand{\literalblockcontinuedname}{continued from previous page}}
\addto\captionsenglish{\renewcommand{\literalblockcontinuesname}{continues on next page}}

\addto\extrasenglish{\def\pageautorefname{page}}

\setcounter{tocdepth}{0}


        \usepackage{cancel}
    

\title{AguaClara Textbook Documentation}
\date{Jul 16, 2018}
\release{}
\author{AguaClara Cornell}
\newcommand{\sphinxlogo}{\vbox{}}
\renewcommand{\releasename}{}
\makeindex

\begin{document}

\maketitle
\sphinxtableofcontents
\phantomsection\label{\detokenize{index::doc}}


This textbook is written and maintained in \sphinxhref{https://github.com/AguaClara/Textbook}{Github} via \sphinxhref{http://www.sphinx-doc.org/en/master/}{Sphinx}. It uses and refers to AguaClara code and functions in \sphinxhref{https://github.com/AguaClara/aide\_design}{aide\_design}. Listed below are the versions of the programs we use:


\begin{savenotes}\sphinxattablestart
\centering
\sphinxcapstartof{table}
\sphinxcaption{These are the software versions used to compile this textbook}\label{\detokenize{index:id2}}\label{\detokenize{index:software-versions}}
\sphinxaftercaption
\begin{tabular}[t]{|\X{10}{20}|\X{10}{20}|}
\hline
\sphinxstyletheadfamily 
Software
&\sphinxstyletheadfamily 
version
\\
\hline
Sphinx
&
1.7.5
\\
\hline
aide\_design
&
0.0.12
\\
\hline
Anaconda
&
4.5.4
\\
\hline
Python
&
3.6.5
\\
\hline
\end{tabular}
\par
\sphinxattableend\end{savenotes}


\chapter{Introduction to RST and Sphinx for Textbook Contributors}
\label{\detokenize{Textbook_Creation_Help/rst_intro:introduction-to-rst-and-sphinx-for-textbook-contributors}}\label{\detokenize{Textbook_Creation_Help/rst_intro:title-rst-intro}}\label{\detokenize{Textbook_Creation_Help/rst_intro::doc}}

\section{What is RST?}
\label{\detokenize{Textbook_Creation_Help/rst_intro:what-is-rst}}\label{\detokenize{Textbook_Creation_Help/rst_intro:heading-what-is-rst}}
RST stands for ReStructured Text. It is the standard markup language used for documenting python packages. \sphinxhref{http://www.sphinx-doc.org/en/master/}{Sphinx} is the Python package that generates an html website from RST files, and it is what we are using to generate this site. To read more about why we chose RST over markdown or Latex, read the following section, {\hyperref[\detokenize{Textbook_Creation_Help/rst_intro:heading-why-rst}]{\sphinxcrossref{\DUrole{std,std-ref}{Why RST?}}}}.


\subsection{Why RST?}
\label{\detokenize{Textbook_Creation_Help/rst_intro:why-rst}}\label{\detokenize{Textbook_Creation_Help/rst_intro:heading-why-rst}}
In the beginning, we used markdown. As we tried to add different features to markdown (\DUrole{red}{colored words}, image sizes, citations), we were forced to use raw html and various pre-processors. With these various band-aid solutions came added complexity. Adding sections became cumbersome and awkward as it required ill-defined html. Additionally, providing site-wide style updates was prohibitively time-consuming and complex. Essentially, we were trying to pack too much functionality into markdown. In the search for an alternative, restructured text provided several advantages. Out of the box, RST supports globally-defined styles, figure numbering and referencing, Latex function rendering, image display customization and more. Furthermore, restructured text was already the language of choice for the AIDE ecosystem’s documentation.


\section{Setting up RST for Development}
\label{\detokenize{Textbook_Creation_Help/rst_intro:setting-up-rst-for-development}}\label{\detokenize{Textbook_Creation_Help/rst_intro:heading-setting-up-rst}}
There are two ways to \sphinxstyleemphasis{quickly} view an RST file. The first is using an \sphinxhref{https://ide.atom.io/}{Atom} plugin that renders the view alongside the source code. This is a good initial test to make sure the RST is proper RST and looks \sphinxstyleemphasis{mostly} correct. However, some functionality, such as any extensions provided by \sphinxhref{http://www.sphinx-doc.org/en/master/}{Sphinx} won’t run in the preview. In order to see the final html that will display on the website, you’ll need to use the second method, running sphinx locally to fully generate the html code. Once you are satisfied with your work and want to push it to the textbook, you’ll need to incorporate it to the master branch. To do so, refer to {\hyperref[\detokenize{Textbook_Creation_Help/rst_intro:publishing-online}]{\sphinxcrossref{Publishing online}}}.


\subsection{Installing the Atom Plugins}
\label{\detokenize{Textbook_Creation_Help/rst_intro:installing-the-atom-plugins}}\label{\detokenize{Textbook_Creation_Help/rst_intro:heading-installing-atom}}
If you are using the Atom IDE to write RST, you can use the \sphinxhref{https://atom.io/packages/rst-preview-pandoc}{rst-preview-pandoc} plugin to auto-generate a live RST preview within atom (much like the markdown-preview-plus preview page.) To get rst-preview working, you’ll need to install \sphinxhref{https://atom.io/packages/language-restructuredtext}{language-restructuredtext} via atom and \sphinxhref{https://pandoc.org/installing.html}{Pandoc} via your command line (\sphinxcode{\sphinxupquote{pip install pandoc}}). If everything worked, you can use \sphinxcode{\sphinxupquote{ctrl + shift + e}} to toggle a display window for the live-updated RST preview.


\subsection{Building RST Locally with Sphinx}
\label{\detokenize{Textbook_Creation_Help/rst_intro:building-rst-locally-with-sphinx}}\label{\detokenize{Textbook_Creation_Help/rst_intro:heading-building-rst-locally}}
We use \sphinxhref{http://www.sphinx-doc.org/en/master/}{Sphinx} to build RST locally and remotely. Follow these steps to get \sphinxhref{http://www.sphinx-doc.org/en/master/}{Sphinx} and run it locally:
\begin{enumerate}
\item {} 
Install \sphinxhref{http://www.sphinx-doc.org/en/master/}{Sphinx}, disqus, and a sphinx visual theme using pip: \sphinxcode{\sphinxupquote{pip install sphinx -{-}user -U}} and \sphinxcode{\sphinxupquote{pip install git+https://github.com/rmk135/sphinxcontrib-disqus}}.

\item {} 
Generate all the html by navigating in the command line to the source directory /Textbook and creating the build in that directory with the command line \sphinxcode{\sphinxupquote{make html}}.

\item {} 
View the html generated in the /Textbook/\_build directory by copying the full file path of /Textbook/\_build/html/index.html and pasting it into your browser.

\end{enumerate}

\begin{sphinxadmonition}{note}{Note:}
Regarding \sphinxstylestrong{1.} the master branch for the package implementing disqus in sphinx \sphinxhref{https://github.com/Robpol86/sphinxcontrib-disqus/pull/7}{is broken}, which is why we use a non-standard pip/online installation. If you already have the incorrect sphinx-disqus version installed, uninstall it with \sphinxcode{\sphinxupquote{pip uninstall sphinxcontrib-disqus}} before installing the functioning version.
\end{sphinxadmonition}


\subsection{Publishing Online}
\label{\detokenize{Textbook_Creation_Help/rst_intro:publishing-online}}\label{\detokenize{Textbook_Creation_Help/rst_intro:heading-publishing-online}}
We use \sphinxhref{https://travis-ci.org/}{Travis} to ensure this site will always contain functional builds. To publish online, you need to:
\begin{enumerate}
\item {} 
Always test your build by first :ref:{}` building RST locally \textless{}heading\_building\_rst\_locally\textgreater{}{}`, and then following the {\hyperref[\detokenize{Textbook_Creation_Help/rst_intro:heading-testing-online}]{\sphinxcrossref{\DUrole{std,std-ref}{testing online}}}} instructions. Once you like how your build looks, follow the steps below to introduce it into the master branch.

\item {} 
Submit a \sphinxhref{https://github.com/AguaClara/Textbook/pulls}{pull request to master}. You’ll need to ask for someone else to review your work at this stage- “request reviewers”. Every pull request \sphinxstylestrong{must} be reviewed by at least one other person.

\item {} 
\sphinxhref{https://travis-ci.org/}{Travis} will build the site using \sphinxhref{http://www.sphinx-doc.org/en/master/}{Sphinx}, and if there aren’t any errors, Travis will report success to GitHub on the “checks” part of the pull request.

\item {} 
All your requested reviewers must now approve and comment on  your commit before the merge is allowed.

\item {} 
Once the PR passes Travis and is approved by another author, feel free to “merge to master.”

\item {} \begin{description}
\item[{To release the master branch, (build the html, pdf, and latex, and upload the pdf to Pages) you’ll need to publish a \sphinxhref{https://github.com/AguaClara/Textbook/releases/new}{GitHub release}. Include a \sphinxhref{https://semver.org/}{semver} version number as the tag (under “Tag: Choose or create”), and a brief description of the updates under “Release Title”. Finally, for the description, detail the changes as much as you see fit and when ready, hit “Publish release”. Example:}] \leavevmode\begin{itemize}
\item {} 
Tag name: 0.1.5

\item {} 
Release title: Filtration section maintenance

\item {} 
Description: Added filter code from aide\_design 0.2.6. Also updated all broken external links.

\end{itemize}

\end{description}

\item {} 
Travis will rebuild the site and push the html to Pages, and the PDF and LaTeX to GitHub Releases under the tag name.

\end{enumerate}

\begin{sphinxadmonition}{important}{Important:}
If your changes to the master branch aren’t pushing to gh-pages, then check the status of the \sphinxhref{https://travis-ci.org/AguaClara/Textbook}{Travis build here}.
\end{sphinxadmonition}


\subsection{Testing Online}
\label{\detokenize{Textbook_Creation_Help/rst_intro:testing-online}}\label{\detokenize{Textbook_Creation_Help/rst_intro:heading-testing-online}}
To test exactly what will be published, we have a test branch. The test branch is built by Travis and contains all the processed html that Travis produces in \_build/html. This branch is populated when ANY COMMIT IS PUSHED. Meaning, the last commit to be pushed, if it passes the Travis tests, will be built and the output will be placed in the test branch. Also, if the PDF=True environment variable is triggered for a Travis build, the PDF will also be generated and placed in the test branch. To do this, use the “Trigger Build” option in Travis and put:

\fvset{hllines={, ,}}%
\begin{sphinxVerbatim}[commandchars=\\\{\}]
\PYG{n}{script}\PYG{p}{:}
    \PYG{o}{\PYGZhy{}} \PYG{n}{PDF}\PYG{o}{=}\PYG{k+kc}{True}
\end{sphinxVerbatim}

\sphinxhref{https://rawgit.com/AguaClara/Textbook/test/html/index.html}{The website output is viewable here}.


\subsection{Sharing Test Output}
\label{\detokenize{Textbook_Creation_Help/rst_intro:sharing-test-output}}
if you want to share what your latest branch developments look like without having whoever is viewing it actually have to build it, you can push a commit, and find the \sphinxhref{https://rawgit.com/}{rawgit URL with this site} by entering the URL of the git file within the test branch that you’d like to share. Furthermore, if you want to point to the commit so that even if someone else pushes, the URL will still point to the code you intend it to, make sure to include the commit SHA within the rawgit URL like so: \sphinxurl{https://rawgit.com/AguaClara/Textbook/e5693e0485702b95e11d4d6bdf76d07c42fdbf99/html/index.html}. That link will never change where it is pointing. To share the PDF output, follow the {\hyperref[\detokenize{Textbook_Creation_Help/rst_intro:heading-testing-online}]{\sphinxcrossref{\DUrole{std,std-ref}{testing online}}}} instructions to build the PDF, and point to the commit with the PDF. Happy testing!


\section{Brief Best Practices}
\label{\detokenize{Textbook_Creation_Help/rst_intro:brief-best-practices}}\label{\detokenize{Textbook_Creation_Help/rst_intro:heading-brief-best-practices}}
When writing RST, there are often many ways to write the same thing. Almost always, the way with the fewest number of characters is the best way. Ideally, never copy and paste.


\subsection{How do I write RST?}
\label{\detokenize{Textbook_Creation_Help/rst_intro:how-do-i-write-rst}}\label{\detokenize{Textbook_Creation_Help/rst_intro:heading-how-do-i-write-rst}}
RST is friendly to learn and powerful. There are many useful cheatsheets, like \sphinxhref{https://thomas-cokelaer.info/tutorials/sphinx/rest\_syntax.html\#inserting-code-and-literal-blocks}{this one} and the next page on this site: \DUrole{xref,std,std-ref}{Functionality in RST and AguaClara Convention}, which is specifically for AguaClara and this textbook project. When you start writing RST, look at the cheat sheets all the time. Use \sphinxcode{\sphinxupquote{ctrl-f}} all the time to find whatever you need.

\sphinxstylestrong{Things not covered in most cheat sheets which are of critical importance:}
\begin{itemize}
\item {} 
A document is referred to by its title, as defined between the \sphinxcode{\sphinxupquote{*****}} signs at the top of the page, \sphinxstylestrong{NOT} the filename. So it is critical to have a title.

\item {} 
Anything else you’d like to add for the future…

\end{itemize}


\subsection{Example to Start From}
\label{\detokenize{Textbook_Creation_Help/rst_intro:example-to-start-from}}\label{\detokenize{Textbook_Creation_Help/rst_intro:heading-example-to-start-from}}
This file is written in RST. You can start there! Just click on “View page source” at the top of the page.

Also, the next page contains the convention, and is where we specify all AguaClara RST best practices: \DUrole{xref,std,std-ref}{Functionality in RST and AguaClara Convention}. I recommend looking at the raw RST and the rendered html side by side.


\section{Converting Markdown to RST}
\label{\detokenize{Textbook_Creation_Help/rst_intro:converting-markdown-to-rst}}\label{\detokenize{Textbook_Creation_Help/rst_intro:heading-converting-md-to-rst}}
Ideally, use pandoc to do the conversion in the command line: \sphinxcode{\sphinxupquote{pandoc -{-}from=markdown -{-}to=rst -{-}output=my\_file.rst my\_file.md}}.
Raw html will not be converted (because it is not actually markdown), and tables are converted poorly.
You’ll need to carefully review any page converted with pandoc.


\chapter{Parameter Convention List}
\label{\detokenize{Textbook_Creation_Help/parameter_convention_list:parameter-convention-list}}\label{\detokenize{Textbook_Creation_Help/parameter_convention_list:title-parameter-convention-list}}\label{\detokenize{Textbook_Creation_Help/parameter_convention_list::doc}}

\begin{savenotes}\sphinxattablestart
\centering
\sphinxcapstartof{table}
\sphinxcaption{Relevant Dimensions}\label{\detokenize{Textbook_Creation_Help/parameter_convention_list:id1}}\label{\detokenize{Textbook_Creation_Help/parameter_convention_list:table-dimension-table}}
\sphinxaftercaption
\begin{tabular}[t]{|\X{30}{90}|\X{30}{90}|\X{30}{90}|}
\hline
\sphinxstyletheadfamily 
Dimension
&\sphinxstyletheadfamily 
Abbreviation
&\sphinxstyletheadfamily 
Base Unit
\\
\hline
Length
&
\([L]\)
&
meter
\\
\hline
Mass
&
\([M]\)
&
kilogram
\\
\hline
Time
&
\([T]\)
&
second
\\
\hline
\end{tabular}
\par
\sphinxattableend\end{savenotes}

If you would like to be able to \sphinxcode{\sphinxupquote{ctrl+f}} some variables, click on ‘View page source’ on the top right of this window. If you want to know what a greek variable is but don’t know what it’s called, you can view the source text on the file where you found the variable. nu, mu, eta, who actually remembers what these all look like? The letter ‘v’ should sue ‘nu’ for copyright infringement. Or is it the other way around?


\begin{savenotes}\sphinxatlongtablestart\begin{longtable}{|\X{10}{70}|\X{15}{70}|\X{45}{70}|}
\caption{Parameter Guide\strut}\label{\detokenize{Textbook_Creation_Help/parameter_convention_list:id2}}\label{\detokenize{Textbook_Creation_Help/parameter_convention_list:table-parameter-table}}\\*[\sphinxlongtablecapskipadjust]
\hline
\sphinxstyletheadfamily 
Parameter
&\sphinxstyletheadfamily 
Description
&\sphinxstyletheadfamily 
Units
\\
\hline
\endfirsthead

\multicolumn{3}{c}%
{\makebox[0pt]{\sphinxtablecontinued{\tablename\ \thetable{} -- continued from previous page}}}\\
\hline
\sphinxstyletheadfamily 
Parameter
&\sphinxstyletheadfamily 
Description
&\sphinxstyletheadfamily 
Units
\\
\hline
\endhead

\hline
\multicolumn{3}{r}{\makebox[0pt][r]{\sphinxtablecontinued{Continued on next page}}}\\
\endfoot

\endlastfoot

\(m\)
&
Mass
&
\([M]\)
\\
\hline
\(z\)
&
Elevation
&
\([L]\)
\\
\hline
\(L\)
&
Length
&
\([L]\)
\\
\hline
\(W\)
&
Width
&
\([L]\)
\\
\hline
\(H\)
&
Height
&
\([L]\)
\\
\hline
\(D\)
&
Diameter
&
\([L]\)
\\
\hline
\(r\)
&
Radius
&
\([L]\)
\\
\hline
\(A\)
&
Area
&
\([L]^2\)
\\
\hline
\(\rlap{-} V\)
&
Volume
&
\([L]^3\)
\\
\hline
\(v\)
&
Velocity
&
\(\frac{[L]}{[T]}\)
\\
\hline
\(Q\)
&
Flow rate
&
\(\frac{[L]^3}{[T]}\)
\\
\hline
\(n\)
&
Number, Amount
&
Dimensionless
\\
\hline
\(C\)
&
Concentration
&
\(\frac{[M]}{[L]^3}\)
\\
\hline
\(p\)
&
Pressure
&
\(\frac{[M]}{[L][T]^2}\)
\\
\hline
\(g\)
&
Acceleration due to Gravity
&
\(\frac{[L]}{[T]^2}\)
\\
\hline
\(\rho\)
&
Density
&
\(\frac{[M]}{[L]^3}\)
\\
\hline
\(\mu\)
&
Dynamic viscosity
&
\(\frac{[M]}{[T][L]}\)
\\
\hline
\(\nu\)
&
Kinematic viscosity
&
\(\frac{[L]^2}{[T]}\)
\\
\hline
\(h\)
&
Head, Elevation
&
\([L]\)
\\
\hline
\(h_L\)
&
Headloss
&
\([L]\)
\\
\hline
\(h_{\rm f}\)
&
Major Loss (friction)
&
\([L]\)
\\
\hline
\(\epsilon\)
&
Surface roughness
&
\([L]\)
\\
\hline
\(\rm{f}\)
&
Darcy-Weisbach friction factor
&
Dimensionless
\\
\hline
\({\rm Re}\)
&
Reynolds Number
&
Dimensionless
\\
\hline
\(h_e\)
&
Minor Loss (expansion)
&
\([L]\)
\\
\hline
\(K\)
&
Minor Loss coefficient
&
Dimensionless
\\
\hline
\(\Pi\)
&
Dimensionless Proportionality Ratio
&
Dimensionless
\\
\hline
\(\Pi_{vc}\)
&
Vena Contracta Area Ratio
&
Dimensionless
\\
\hline
\(\Pi_{Error}\)
&
Linearity Error Ratio
&
Dimensionless
\\
\hline
\(M\)
&
Fluid Momentum
&
\(\frac{[M][L]}{[T]^2}\)
\\
\hline
\(F\)
&
Force
&
\(\frac{[M][L]}{[T]^2}\)
\\
\hline
\(t\)
&
Time
&
\([T]\)
\\
\hline
\(\theta\)
&
Residence Time
&
\([T]\)
\\
\hline
\(G\)
&
Velocity Gradient/Fluid Deformation
&
\(\frac{1}{[T]}\)
\\
\hline
\(\varepsilon\)
&
Energy Dissipation Rate
&
\(\frac{[L]^2}{[T]^3}\)
\\
\hline
\(\Pi_{\bar G}^{G_{Max}}\)
&
\(\frac{G_{Max}}{\bar G}\) Ratio in a Reactor
&
Dimensionless
\\
\hline
\(\Pi_{\bar \varepsilon}^{\varepsilon_{Max}}\)
&
\(\frac{\varepsilon_{Max}}{\bar \varepsilon}\) Ratio in a Reactor
&
Dimensionless
\\
\hline
\(\Pi_{HS}\)
&
Height to Baffle Spacing in a Flocculator
&
Dimensionless
\\
\hline
\(H_e\)
&
Height Between Flow Expansions in a Flocculator
&
\([L]\)
\\
\hline
\(S\)
&
Spacing Between Two Objects
&
\([L]\)
\\
\hline
\(B\)
&
Center-to-Center Spacing Between Two Objects
&
\([L]\)
\\
\hline
\(T\)
&
Object Thickness
&
\([L]\)
\\
\hline
\(P\)
&
Power
&
\(\frac{[M][L]^2}{[T]^3}\)
\\
\hline
\(\eta_K\)
&
Kolmogorov Length Scale
&
\([L]\)
\\
\hline
\(\lambda_\nu\)
&
Inner Viscous Length Scale
&
\([L]\)
\\
\hline
\(\Pi_{K\nu}\)
&
Ratio of Inner Viscous Length Scale to Kolmogorov Length Scale
&
Dimensionless
\\
\hline
\(\Lambda\)
&
Distance Between Particles
&
\([L]\)
\\
\hline
\end{longtable}\sphinxatlongtableend\end{savenotes}

This is a place holder for ‘Chapter 1: Introduction’ files


\chapter{Review: Fluid Mechanics}
\label{\detokenize{Review/Review_Fluid_Mechanics:review-fluid-mechanics}}\label{\detokenize{Review/Review_Fluid_Mechanics:title-review-fluid-mechanics}}\label{\detokenize{Review/Review_Fluid_Mechanics::doc}}
This document is meant to be a refresher on fluid mechanics. It will only cover the topics of fluids mechanics that will be used heavily in the course.

If you wish to review fluid mechanics in (much) more detail, please refer to \sphinxhref{https://github.com/AguaClara/CEE4540\_Master/wiki/Fluids-Review-Guide}{this guide} Note that to view this link, you will need a Github accounts. If you wish to review from a legitimate textbook, you can find a pdf of good book by Frank White \sphinxhref{https://hellcareers.files.wordpress.com/2016/01/fluid-mechanics-seventh-edition-by-frank-m-white.pdf}{here}.


\section{Important Terms and Equations}
\label{\detokenize{Review/Review_Fluid_Mechanics:important-terms-and-equations}}\label{\detokenize{Review/Review_Fluid_Mechanics:heading-fluids-terms-eqs}}
\sphinxstylestrong{Terms:}
\begin{enumerate}
\item {} 
{\hyperref[\detokenize{Review/Review_Fluid_Mechanics:heading-laminar-and-turbulent-flow}]{\sphinxcrossref{\DUrole{std,std-ref}{Laminar}}}}

\item {} 
{\hyperref[\detokenize{Review/Review_Fluid_Mechanics:heading-laminar-and-turbulent-flow}]{\sphinxcrossref{\DUrole{std,std-ref}{Turbulent}}}}

\item {} 
{\hyperref[\detokenize{Review/Review_Fluid_Mechanics:heading-laminar-and-turbulent-flow}]{\sphinxcrossref{\DUrole{std,std-ref}{Viscosity}}}}

\item {} 
{\hyperref[\detokenize{Review/Review_Fluid_Mechanics:heading-streamlines-and-control-volumes}]{\sphinxcrossref{\DUrole{std,std-ref}{Streamline}}}}

\item {} 
{\hyperref[\detokenize{Review/Review_Fluid_Mechanics:heading-streamlines-and-control-volumes}]{\sphinxcrossref{\DUrole{std,std-ref}{Control Volume}}}}

\item {} 
{\hyperref[\detokenize{Review/Review_Fluid_Mechanics:heading-bernoulli-equation}]{\sphinxcrossref{\DUrole{std,std-ref}{Head}}}}

\item {} 
{\hyperref[\detokenize{Review/Review_Fluid_Mechanics:heading-head-loss}]{\sphinxcrossref{\DUrole{std,std-ref}{Head loss}}}}

\item {} 
{\hyperref[\detokenize{Review/Review_Fluid_Mechanics:heading-head-loss-elevation-difference-trick}]{\sphinxcrossref{\DUrole{std,std-ref}{Driving head}}}}

\item {} 
{\hyperref[\detokenize{Review/Review_Fluid_Mechanics:heading-what-is-a-vena-contracta}]{\sphinxcrossref{\DUrole{std,std-ref}{Vena Contracta/Coefficient of Contraction}}}}

\end{enumerate}

\sphinxstylestrong{Equations:}
\begin{enumerate}
\item {} 
Continuity equation: \eqref{equation:Review/Review_Fluid_Mechanics:continuity_equation}

\item {} 
Reynolds number \eqref{equation:Review/Review_Fluid_Mechanics:reynolds_number_equation}

\item {} 
Bernoulli equation \eqref{equation:Review/Review_Fluid_Mechanics:bernoulli_equation}

\item {} 
Energy equation \eqref{equation:Review/Review_Fluid_Mechanics:energy_equation}

\item {} 
Darcy-Weisbach equation \eqref{equation:Review/Review_Fluid_Mechanics:darcy_weisbach}

\item {} 
Swamee-Jain equation \eqref{equation:Review/Review_Fluid_Mechanics:swamee_jain}

\item {} 
Hagen-Poiseuille equation \eqref{equation:Review/Review_Fluid_Mechanics:hagen_poiseuille}

\item {} 
Orifice equation \eqref{equation:Review/Review_Fluid_Mechanics:orifice_equation}

\end{enumerate}


\section{Introductory Concepts}
\label{\detokenize{Review/Review_Fluid_Mechanics:introductory-concepts}}\label{\detokenize{Review/Review_Fluid_Mechanics:heading-introductory-concepts}}
Before diving in to the rest of this document, there are a few important concepts to focus on which will be the foundation for building your understanding of fluid mechanics. One must walk before they can run, and similarly, the basics of fluid mechanics must be understood before moving on to the more fun (and exciting!) sections of this document.


\subsection{Continuity Equation}
\label{\detokenize{Review/Review_Fluid_Mechanics:continuity-equation}}\label{\detokenize{Review/Review_Fluid_Mechanics:heading-continuity-equation}}
Continuity is simply an application of mass balance to fluid mechanics. It states that the cross sectional area \(A\) that a fluid flows through multiplied by the fluid’s average flow velocity \(\bar v\) must equal the fluid’s flow rate \(Q\):
\begin{equation}\label{equation:Review/Review_Fluid_Mechanics:continuity_equation}
\begin{split}  Q = \bar v A\end{split}
\end{equation}
\begin{sphinxadmonition}{note}{Note:}
The line above the \(v\) is called a ‘bar,’ and represents an average. Any variable can have a bar. In this case, we are adding the bar to velocity \(v\), turning it into average velocity \(\bar v\). This variable is pronounced ‘v bar.’
\end{sphinxadmonition}

In this course, we deal primarily with flow through pipes. For a circular pipe, \(A = \pi r^2\). Substituting diameter in for radius, \(r = \frac{D}{2}\), we get \(A = \frac{\pi D^2}{4}\). You will often see this form of the continuity equation being used to relate the a pipe’s flow rate to its diameter and the velocity of the fluid flowing through it:
\begin{equation}\label{equation:Review/Review_Fluid_Mechanics:Review/Review_Fluid_Mechanics:0}
\begin{split}Q = \bar v \frac{\pi D^2}{4}\end{split}
\end{equation}
The continuity equation is also useful when flow is going from one geometry to another. In this case, the flow in one geometry must be the same as the flow in the other, \(Q_1 = Q_2\), which yields the following equations:
\begin{equation}\label{equation:Review/Review_Fluid_Mechanics:Review/Review_Fluid_Mechanics:1}
\begin{split}\bar v_1 A_1 = \bar v_2 A_2\end{split}
\end{equation}\begin{equation}\label{equation:Review/Review_Fluid_Mechanics:Review/Review_Fluid_Mechanics:2}
\begin{split}\bar v_1 \frac{\pi D_1^2}{4} = \bar v_2 \frac{\pi D_2^2}{4}\end{split}
\end{equation}
\begin{DUlineblock}{0em}
\item[] Such that:
\item[] \(Q =\) fluid flow rate
\item[] \(\bar v =\) fluid average velocity
\item[] \(A =\) pipe area
\item[] \(r =\) pipe radius
\item[] \(D =\) pipe diameter
\end{DUlineblock}

An example of changing flow geometries is when a change in pipe size occurs in a circular piping system, as is demonstrated below. The flow through \({\rm pipe} \, 1\) must be the same as the flow through \({\rm pipe} \, 2\).

\begin{figure}[htbp]
\centering
\capstart

\noindent\sphinxincludegraphics[width=700\sphinxpxdimen]{{continuity_pipes}.png}
\caption{Flow going from a small diameter pipe to a large one. The continuity principle states that the flow through each pipe must be the same.}\label{\detokenize{Review/Review_Fluid_Mechanics:id1}}\label{\detokenize{Review/Review_Fluid_Mechanics:figure-continuity-pipes}}\end{figure}


\subsection{Laminar and Turbulent Flow}
\label{\detokenize{Review/Review_Fluid_Mechanics:laminar-and-turbulent-flow}}\label{\detokenize{Review/Review_Fluid_Mechanics:heading-laminar-and-turbulent-flow}}
Considering that this class deals with the flow of water through a water treatment plant, understanding the characteristics of the flow is very important. Thus, it is necessary to understand the most common characteristic of fluid flow: whether it is \sphinxstylestrong{laminar} or \sphinxstylestrong{turbulent}. \sphinxhref{https://en.wikipedia.org/wiki/Laminar\_flow}{Laminar} flow is very smooth and highly ordered. \sphinxhref{https://en.wikipedia.org/wiki/Turbulence}{Turbulent} flow is chaotic, messy, and disordered. The best way to understand each flow and what it looks like is visually, like in the wikipedia figure below \sphinxhref{https://youtu.be/qtvVN2qt968?t=131}{or in this video}. Please ignore the part of the video after the image of the tap.

\begin{figure}[htbp]
\centering
\capstart

\noindent\sphinxincludegraphics[width=400\sphinxpxdimen]{{Wikipedia_laminar_turbulent}.png}
\caption{This is a beautiful example of the difference between ordered and smooth laminar flow and chaotic turbulent flow.}\label{\detokenize{Review/Review_Fluid_Mechanics:id2}}\label{\detokenize{Review/Review_Fluid_Mechanics:figure-wikipedia-laminar-turbulent}}\end{figure}

A numeric way to determine whether flow is laminar or turbulent is by finding the \sphinxhref{https://en.wikipedia.org/wiki/Reynolds\_number}{Reynolds number}, \({\rm Re}\). The Reynolds number is a dimensionless parameter that compares inertia, represented by the average flow velocity \(\bar v\) times a length scale \(D\) to \sphinxhref{https://en.wikipedia.org/wiki/Viscosity}{viscosity}, represented by the kinematic viscosity \(\nu\). \sphinxhref{https://www.youtube.com/watch?v=DVQw0svRHZA}{Click here} for a brief video explanation of viscosity. If the Reynolds number is less than 2,100 the flow is considered laminar. If it is more than 2,100, it is considered turbulent.
\begin{equation}\label{equation:Review/Review_Fluid_Mechanics:Review/Review_Fluid_Mechanics:3}
\begin{split}{\rm Re = \frac{inertia}{viscosity}} = \frac{\bar vD}{\nu}\end{split}
\end{equation}
\sphinxhref{https://en.wikipedia.org/wiki/Laminar\%E2\%80\%93turbulent\_transition}{The transition between laminar and turbulent flow is not yet well understood}, which is why the concept of transitional flow is often simplified and neglected to make it possible to code for laminar or turbulent flow, which are better understood. We will assume that the transition occurs at \(\rm{Re} = 2100\). In aide\_design, this parameter shows us as \sphinxcode{\sphinxupquote{pc.RE\_TRANSITION\_PIPE}}.

Fluid can flow through very many different geometries, like a pipe, a rectangular channel, or any other shape. To account for this, the characteristic length scale for the Reynolds number, which was written in the equation above as \(D\), is quantified as the \sphinxhref{https://www.engineeringtoolbox.com/hydraulic-equivalent-diameter-d\_458.html}{hydraulic diameter}, \(D_h\) when considering a general cross-sectional area. For circular pipes, which are the most common geometry you’ll encounter in this class, the hydraulic diameter is simply the pipe’s diameter, \(D_h = D\).

Here are other commonly used forms of the Reynolds number equation \sphinxstyleemphasis{for circular pipes}. They are the same as the one above, just with the substitutions \(Q = \bar v \frac{\pi D^2}{4}\) and \(\nu = \frac{\mu}{\rho}\)
\begin{equation}\label{equation:Review/Review_Fluid_Mechanics:reynolds_number_equation}
\begin{split}  {\rm Re} = \frac{\bar vD}{\nu} = \frac{4Q}{\pi D\nu} = \frac{\rho \bar vD}{\mu}\end{split}
\end{equation}
\begin{DUlineblock}{0em}
\item[] Such that:
\item[] \(Q\) = fluid flow rate in pipe
\item[] \(D\) = pipe diameter
\item[] \(\bar v\) = fluid velocity
\item[] \(\nu\) = fluid kinematic viscosity
\item[] \(\mu\) = fluid dynamic viscosity
\end{DUlineblock}


\sphinxstrong{See also:}


\sphinxstylestrong{Function in aide\_design:} \sphinxcode{\sphinxupquote{pc.re\_pipe(FlowRate, Diam, Nu)}} Returns the Reynolds number \sphinxstyleemphasis{in a circular pipe}. Functions for finding the Reynolds number through other flow conduits and geometries can also be found in \sphinxhref{https://github.com/AguaClara/aide\_design/blob/master/aide\_design/physchem.py}{physchem.py} within aide\_design.



\begin{sphinxadmonition}{note}{Note:}
\sphinxstylestrong{Definition of Flow Regimes:} Laminar and turbulent flow are described as two different \sphinxstylestrong{flow regimes}. When there is a characteristic of flow and different categories of the characteristic, each category is referred to as a flow regime. For example, the Reynolds number describes a flow characteristic, and its categories, referred to as flow regimes, are laminar or turbulent.
\end{sphinxadmonition}


\subsection{Streamlines and Control Volumes}
\label{\detokenize{Review/Review_Fluid_Mechanics:streamlines-and-control-volumes}}\label{\detokenize{Review/Review_Fluid_Mechanics:heading-streamlines-and-control-volumes}}
Both \sphinxhref{https://en.wikipedia.org/wiki/Streamlines,\_streaklines,\_and\_pathlines}{streamlines} and \sphinxhref{https://www.engineersedge.com/fluid\_flow/control\_volume.htm}{control volumes} are tools to compare different parts of a system. For this class, this system will always be hydraulic.

Imagine water flowing through a pipe. A streamline is the path that a particle would take if it could be placed in the fluid without changing the original flow of the fluid. A more technical definition is “a line which is everywhere parallel to the local velocity vector.” Computational tools, \sphinxhref{https://www.nuclear-power.net/wp-content/uploads/2016/05/Flow-Regime.png?4b884b}{dyes (in water)}, or \sphinxhref{https://www.youtube.com/watch?v=E9ZSAX56m0E\&t=59s}{smoke (in air)} can be used to visualize streamlines.

A \sphinxstylestrong{control volume} is just an imaginary 3-dimensional shape in space. Its boundaries can be placed anywhere by the person applying the control volume, and once set the boundaries remain fixed in space over time. These boundaries are usually chosen to compare two relevant surfaces to each other. These surfaces are called \sphinxstyleemphasis{Control Surfaces}. The entirety of a control volume is usually not shown, as it is often unnecessary. This is demonstrated in the following image:

\begin{figure}[htbp]
\centering
\capstart

\noindent\sphinxincludegraphics[width=650\sphinxpxdimen]{{control_volume_simplification}.png}
\caption{While the image on the left indicates a complete control volume, control volumes are usually shortened to only include the relevant control surfaces, in which the control volume intersects the fluid. This is shown in the image on the right.}\label{\detokenize{Review/Review_Fluid_Mechanics:id3}}\label{\detokenize{Review/Review_Fluid_Mechanics:figure-control-volume-simplification}}\end{figure}

\begin{sphinxadmonition}{important}{Important:}
Many images will be used over the course of this class to show hydraulic systems. A standardized system of lines will be used throughout them all to distinguish reference elevations from control volumes from streamlines. This system is described in the image below.
\end{sphinxadmonition}

\begin{figure}[htbp]
\centering
\capstart

\noindent\sphinxincludegraphics[width=650\sphinxpxdimen]{{image_control_volumes}.png}
\caption{On the right, a control volume is applied to a hydraulic system. On the left, a streamline is applied to a hydraulic system. A figure-convention for control volumes and streamlines will be very helpful throughout this course as there will be very, very many figures.}\label{\detokenize{Review/Review_Fluid_Mechanics:id4}}\label{\detokenize{Review/Review_Fluid_Mechanics:figure-image-control-volumes}}\end{figure}


\section{The Bernoulli and Energy Equations}
\label{\detokenize{Review/Review_Fluid_Mechanics:the-bernoulli-and-energy-equations}}\label{\detokenize{Review/Review_Fluid_Mechanics:heading-bernoulli-and-energy-equations}}
As explained in almost every fluid mechanics class, the Bernoulli and energy equations are incredibly useful in understanding the transfer of the fluid’s energy throughout a streamline or through a control volume. The Bernoulli equation applies to two different points along one streamline, whereas the energy equation applies to fluid entering and exiting a control volume. The energy of a fluid has three forms: pressure, potential (deriving from elevation), and kinetic (deriving from velocity).


\subsection{The Bernoulli Equation}
\label{\detokenize{Review/Review_Fluid_Mechanics:the-bernoulli-equation}}\label{\detokenize{Review/Review_Fluid_Mechanics:heading-bernoulli-equation}}
These three forms of energy expressed above make up the Bernoulli equation:
\begin{equation}\label{equation:Review/Review_Fluid_Mechanics:bernoulli_equation}
\begin{split}  \frac{p_1}{\rho g} + {z_1} + \frac{v_1^2}{2g} = \frac{p_2}{\rho g} + {z_2} + \frac{v_2^2}{2g}\end{split}
\end{equation}
\begin{DUlineblock}{0em}
\item[] Such that:
\item[] \(p\) = pressure
\item[] \(\rho\) = fluid density
\item[] \(g\) = acceleration due to gravity, in aide\_design as \sphinxcode{\sphinxupquote{con.GRAVITY}}
\item[] \(z\) = elevation relative to a reference
\item[] \(v\) = fluid velocity
\end{DUlineblock}

Notice that each term in this form of the Bernoulli equation has units of \([L]\), even though the terms represent the energy of the fluid, which has units of \(\frac{[M] \cdot [L]^2}{[T]^2}\). When energy of the fluid is described in units of length, the term used is called \sphinxstylestrong{head} and referred to as \(h\).

There are two important distinctions to keep in mind when using head to talk about a fluid’s energy. First is that head is dependent on the density of the fluid under consideration. Take mercury, for example, which is around 13.6 times more dense than water. 1 meter of mercury head is therefore equivalent to around 13.6 meters of water head. Second is that head is independent of the amount of fluid being considered, \sphinxstyleemphasis{as long as all the fluid is the same density}. Thus, raising 1 liter of water up by one meter and raising 100 liters of water up by one meter are both equivalent to giving the water 1 meter of water head, even though it requires 100 times more energy to raise the hundred liters than to raise the single liter. Since we are concerned mainly with water in this class, we will refer to ‘water head’ simply as ‘head’.

Going back to the Bernoulli equation, the \(\frac{p}{\rho g}\) term is called the pressure head, \(z\) is called the elevation head, and \(\frac{v^2}{2g}\) is the velocity head. The following diagram shows these various forms of head via a 1 meter deep bucket (left) and a jet of water shooting out of the ground (right).

\begin{figure}[htbp]
\centering
\capstart

\noindent\sphinxincludegraphics[width=650\sphinxpxdimen]{{different_forms_of_head}.png}
\caption{The three forms of hydraulic head.}\label{\detokenize{Review/Review_Fluid_Mechanics:id5}}\label{\detokenize{Review/Review_Fluid_Mechanics:figure-different-forms-of-head}}\end{figure}


\subsubsection{Assumption in using the Bernoulli equation}
\label{\detokenize{Review/Review_Fluid_Mechanics:assumption-in-using-the-bernoulli-equation}}
Though there are \sphinxhref{https://en.wikipedia.org/wiki/Bernoulli\%27s\_principle\#Incompressible\_flow\_equation}{many assumptions needed to confirm that the Bernoulli equation can be used}, the main one for the purpose of this class is that energy is not gained or lost throughout the streamline being considered. If we consider more precise fluid mechanics terminology, then “friction by viscous forces must be negligible.” What this means is that the fluid along the streamline being considered is not losing energy to viscosity. As a result, using the Bernoulli equation implies that energy can’t be gained or lost. It can only be transferred between its three forms.


\subsubsection{Example problems}
\label{\detokenize{Review/Review_Fluid_Mechanics:example-problems}}
\sphinxhref{https://www.teachengineering.org/content/cub\_/lessons/cub\_bernoulli/cub\_bernoulli\_lesson01\_bepworksheetas\_draft4\_tedl\_dwc.pdf}{Here is a simple worksheet with very straightforward example problems using the Bernoulli equation.} Note that the solutions use the pressure-form of the Bernoulli equation. This just means that every term in the equation is multiplied by \(\rho g\), so the pressure term is just \(P\). The form of the equation does not affect the solution to the problem it helps solved.


\subsection{The Energy Equation}
\label{\detokenize{Review/Review_Fluid_Mechanics:the-energy-equation}}\label{\detokenize{Review/Review_Fluid_Mechanics:heading-energy-equation}}
The assumption necessary to use the Bernoulli equation, which is stated above, represents the key difference between the Bernoulli equation and the energy equation for the purpose of this class. The energy equation accounts for the potential addition or loss of fluid energy within the control volume. (L)oss of energy is usually due to viscous friction resisting fluid flow, \(h_L\), or the charging of a (T)urbine, \(h_T\). The most common input of fluid energy into a system is usually caused by a (P)ump within the control volume, \(h_P\).
\begin{equation}\label{equation:Review/Review_Fluid_Mechanics:Review/Review_Fluid_Mechanics:4}
\begin{split}\frac{p_{1}}{\rho g} + z_{1} + \alpha_{1} \frac{\bar v_{1}^2}{2g} + h_P = \frac{p_{2}}{\rho g} + z_{2} + {\alpha_{2}} \frac{\bar v_{2}^2}{2g} + h_T + h_L\end{split}
\end{equation}
You’ll also notice the \(\alpha\) term attached to the velocity head. This is a correction factor for kinetic energy, and will be neglected in this class; we assume that its value is 1. In the Bernoulli equation, the velocity of a streamline of the fluid is considered, \(v\). The energy equation, however compares control surfaces instead of streamlines, and the velocities across a control surface many not all be the same. Hence, \(\bar v\) is used to represent the average velocity. Since AguaClara does not use pumps nor turbines, \(h_P = h_T = 0\). With these simplifications, the energy equation can be written as follows:
\begin{equation}\label{equation:Review/Review_Fluid_Mechanics:energy_equation}
\begin{split}  \frac{p_{1}}{\rho g} + z_{1} + \frac{\bar v_{1}^2}{2g} = \frac{p_{2}}{\rho g} + z_{2} + \frac{\bar v_{2}^2}{2g} + h_L\end{split}
\end{equation}
\sphinxstylestrong{This is the form of the energy equation that you will see over and over again in this book.} To summarize, the main difference between the Bernoulli equation and the energy equation for the purposes of this class is energy loss. The energy equation accounts for the fluid’s loss of energy over time while the Bernoulli equation does not. So how can the fluid lose energy?


\section{Headloss}
\label{\detokenize{Review/Review_Fluid_Mechanics:headloss}}\label{\detokenize{Review/Review_Fluid_Mechanics:heading-head-loss}}
\sphinxstylestrong{Head(L)oss}, \(h_L\) is a term that is ubiquitous in both this class and fluid mechanics in general. Its definition is exactly as it sounds: it refers to the loss of energy of a fluid as it flows through space. There are two components to head loss: major losses caused by (f)riction between the fluid the surface it’s flowing over, \(h_{\rm{f}}\), and minor losses caused by fluid-fluid internal friction resulting from flow (e)xpansions, \(h_e\). These two components combine such that \(h_L = h_{\rm{f}} + h_e\).


\subsection{Major Losses}
\label{\detokenize{Review/Review_Fluid_Mechanics:major-losses}}\label{\detokenize{Review/Review_Fluid_Mechanics:heading-major-losses}}
These losses are the result of friction between the fluid and the surface over which the fluid is flowing. A force acting parallel to a surface is referred to as \sphinxhref{https://en.wikipedia.org/wiki/Shear\_force}{shear}. It can therefore be said that major losses are the result of shear between the fluid and the surface it’s flowing over. To help in understanding major losses, consider the following example: imagine, as you have so often in physics class, pushing a large box across the ground. Friction is what resists your efforts to push the box. The farther you push the box, the more energy you expend pushing against friction. The same is true for water moving through a pipe, where water is analogous to the box you want to move, the pipe is similar to the floor that provides the friction, and the major losses of the water through the pipe is analogous to the energy \sphinxstylestrong{you} expend by pushing the box.

In this class, we will be dealing primarily with major losses in circular pipes, as opposed to channels or pipes with other geometries. Fortunately for us, Henry Darcy and Julius Weisbach came up with a handy equation to determine the major losses in a circular pipe \sphinxstyleemphasis{under both laminar and turbulent flow conditions}. Their equation is logically and unoriginally named the \sphinxhref{https://en.wikipedia.org/wiki/Darcy\%E2\%80\%93Weisbach\_equation}{Darcy-Weisbach equation}. It is shown below:
\begin{equation}\label{equation:Review/Review_Fluid_Mechanics:darcy_weisbach}
\begin{split}  h_{\rm{f}} \, = \, {\rm{f}} \frac{L}{D} \frac{\bar v^2}{2g}\end{split}
\end{equation}
Substituting the continuity equation \(Q = \bar vA\) in the form of \(\bar v^2 = \frac{16Q^2}{\pi^2 D^4}\) gives another, equivalent form of Darcy-Weisbach which uses flow, \(Q\), instead of velocity, \(\bar v\):
\begin{equation}\label{equation:Review/Review_Fluid_Mechanics:Review/Review_Fluid_Mechanics:5}
\begin{split}h_{\rm{f}} \, = \,{\rm{f}} \frac{8}{g \pi^2} \frac{LQ^2}{D^5}\end{split}
\end{equation}
\begin{DUlineblock}{0em}
\item[] Such that:
\item[] \(h_{\rm{f}}\) = major loss
\item[] \(\rm{f}\) = Darcy friction factor
\item[] \(L\) = pipe length
\item[] \(Q\) = pipe flow rate
\item[] \(D\) = pipe diameter
\end{DUlineblock}


\sphinxstrong{See also:}


\sphinxstylestrong{Function in aide\_design:} \sphinxcode{\sphinxupquote{pc.headloss\_fric(FlowRate, Diam, Length, Nu, PipeRough)}} Returns only major losses. Works for both laminar and turbulent flow. PipeRough describes the pipe roughness \(\epsilon\) described shortly below.



Darcy-Weisbach is wonderful because it applies to both laminar and turbulent flow regimes and contains relatively easy to measure variables. The one exception is the Darcy friction factor, \(\rm{f}\). This parameter is an approximation for the magnitude of friction between the pipe walls and the fluid, and its value changes depending on the whether or not the flow is laminar or turbulent, and varies with the Reynolds number in both flow regimes.

For laminar flow, the friction factor can be determined from the following equation:
\begin{equation}\label{equation:Review/Review_Fluid_Mechanics:Review/Review_Fluid_Mechanics:6}
\begin{split}{\rm{f}} = \frac{64}{\rm{Re}}\end{split}
\end{equation}
For turbulent flow, the friction factor is more difficult to determine. In this class, we will use the \sphinxhref{https://en.wikipedia.org/wiki/Darcy\_friction\_factor\_formulae\#Swamee\%E2\%80\%93Jain\_equation}{Swamee-Jain equation}:
\begin{equation}\label{equation:Review/Review_Fluid_Mechanics:swamee_jain}
\begin{split}  {\rm{f}} = \frac{0.25} {\left[ \log \left( \frac{\epsilon }{3.7D} + \frac{5.74}{{\rm Re}^{0.9}} \right) \right]^2}\end{split}
\end{equation}
\begin{DUlineblock}{0em}
\item[] Such that:
\item[] \(\epsilon\) = pipe roughness, \([L]\)
\item[] \(D\) = pipe diameter, \([L]\)
\end{DUlineblock}


\sphinxstrong{See also:}


\sphinxstylestrong{Function in aide\_design:} \sphinxcode{\sphinxupquote{pc.fric(FlowRate, Diam, Nu, PipeRough)}} Returns \(\rm{f}\) for laminar \sphinxstyleemphasis{or} turbulent flow. For laminar flow, use zero for the \sphinxcode{\sphinxupquote{PipeRough}} input.



The simplicity of the equation for \(\rm{f}\) during laminar flow allows for substitutions to create a very useful, simplified equation for major losses during laminar flow. This simplification combines the Darcy-Weisbach equation, the equation for the Darcy friction factor during laminar flow, and the Reynold’s number formula:
\begin{equation}\label{equation:Review/Review_Fluid_Mechanics:Review/Review_Fluid_Mechanics:7}
\begin{split}h_{\rm{f}} \, = \,{\rm{f}} \frac{8}{g \pi^2} \frac{LQ^2}{D^5}\end{split}
\end{equation}\begin{equation}\label{equation:Review/Review_Fluid_Mechanics:Review/Review_Fluid_Mechanics:8}
\begin{split}{\rm{f}} = \frac{64}{\rm{Re}}\end{split}
\end{equation}\begin{equation}\label{equation:Review/Review_Fluid_Mechanics:Review/Review_Fluid_Mechanics:9}
\begin{split}{\rm{Re}}=\frac{4Q}{\pi D\nu}\end{split}
\end{equation}
To form the \sphinxhref{https://en.wikipedia.org/wiki/Hagen\%E2\%80\%93Poiseuille\_equation}{Hagen-Poiseuille equation} for major losses during laminar flow, and \sphinxstyleemphasis{only} during laminar flow:
\begin{equation}\label{equation:Review/Review_Fluid_Mechanics:hagen_poiseuille}
\begin{split}  h_{\rm{f}} = \frac{128\mu L Q}{\rho g\pi D^4}\end{split}
\end{equation}\begin{equation}\label{equation:Review/Review_Fluid_Mechanics:Review/Review_Fluid_Mechanics:10}
\begin{split}h_{\rm{f}} = \frac{32\nu L\bar v}{ g D^2}\end{split}
\end{equation}
The significance of this equation lies in its relationship between \(h_{\rm{f}}\) and \(Q\). Hagen-Poiseuille shows that the terms are directly proportional (\(h_{\rm{f}} \propto Q\)) during laminar flow, while Darcy-Weisbach shows that \(h_{\rm{f}}\) grows with the square of \(Q\) during turbulent flow (\(h_{\rm{f}} \propto Q^2\)). As you will soon see, minor losses, \(h_e\), will grow with the square of \(Q\) in both laminar and turbulent flow. This has implications that will be discussed in a future chapter: {\hyperref[\detokenize{Flow_Control_and_Measurement/FCM_Design:title-flow-control-design}]{\sphinxcrossref{\DUrole{std,std-ref}{Flow Control and Measurement Design}}}}.

In 1944, Lewis Ferry Moody plotted a ridiculous amount of experimental data, gathered by many people, on the Darcy-Weisbach friction factor to create what we now call the \sphinxhref{https://en.wikipedia.org/wiki/Moody\_chart}{Moody diagram}. This diagram has makes it easy to find the friction factor \(f\). \(\rm{f}\) is plotted on the left-hand y-axis, relative pipe roughness \(\frac{\epsilon}{D}\) is on the right-hand y-axis, and Reynolds number \(\rm{Re}\) is on the x-axis. The Moody diagram is an alternative to computational methods for finding \(\rm{f}\).

\begin{figure}[htbp]
\centering
\capstart

\noindent\sphinxincludegraphics[width=650\sphinxpxdimen]{{Moody}.jpg}
\caption{This is the famous and famously useful Moody diagram.}\label{\detokenize{Review/Review_Fluid_Mechanics:id6}}\label{\detokenize{Review/Review_Fluid_Mechanics:figure-moody}}\end{figure}


\subsection{Minor Losses}
\label{\detokenize{Review/Review_Fluid_Mechanics:minor-losses}}\label{\detokenize{Review/Review_Fluid_Mechanics:heading-minor-losses}}
Unfortunately, there is no simple ‘pushing a box across the ground’ example to explain minor losses. So instead, consider a \sphinxhref{https://www.youtube.com/watch?v=5spXXZX55C8}{hydraulic jump}. In the video, you can see lots of turbulence and eddies in the transition region between the fast, shallow flow and the slow, deep flow. The high amount of mixing of the water in the transition region of the hydraulic jump results in significant friction \sphinxstyleemphasis{between water and water}. This turbulent, eddy-induced, fluid-fluid friction results in  minor losses, much like fluid-pipe friction results in major losses.

As occurs in a hydraulic jump, a flow expansion (from shallow flow to deep flow) creates the turbulent eddies that result in minor losses. This will be a recurring theme in throughout the course: \sphinxstylestrong{minor losses are caused by flow expansions}. Imagine a pipe fitting that connects a small diameter pipe to a large diameter one, as shown in \hyperref[\detokenize{Review/Review_Fluid_Mechanics:figure-minor-loss-pipe-frd}]{Fig.\@ \ref{\detokenize{Review/Review_Fluid_Mechanics:figure-minor-loss-pipe-frd}}} below. The flow must expand to fill up the entire large diameter pipe. This expansion creates turbulent eddies near the union between the small and large pipes, and these eddies result in minor losses. You may already know the equation for minor losses, but understanding where it comes from is very important for effective AguaClara plant design. For this reason, you are strongly recommended to read through its full derivation: {\hyperref[\detokenize{Review/Review_Fluid_Mechanics_Derivations:title-review-fluid-mechanics-derivations}]{\sphinxcrossref{\DUrole{std,std-ref}{Review: Fluid Mechanics Derivations}}}}.

There are three forms of the minor loss equation that you will see in this class:
\begin{equation}\label{equation:Review/Review_Fluid_Mechanics:Review/Review_Fluid_Mechanics:11}
\begin{split}{\rm{ \mathbf{First \, form:} }} \quad h_e = \frac{\left( \bar v_{in}  - \bar v_{out} \right)^2}{2g}\end{split}
\end{equation}\begin{equation}\label{equation:Review/Review_Fluid_Mechanics:Review/Review_Fluid_Mechanics:12}
\begin{split}{\rm{ \mathbf{Second \, form:} }} \quad h_e = \left( 1 - \frac{A_{in}}{A_{out}} \right)^2 \, \frac{\bar v_{in}^2}{2g} \, \, = \, \, K_e^{'} \frac{\bar v_{in}^2}{2g}, \quad {\rm where} \quad K_e^{'} = \left( 1 - \frac{A_{in}}{A_{out}} \right)^2\end{split}
\end{equation}\begin{equation}\label{equation:Review/Review_Fluid_Mechanics:Review/Review_Fluid_Mechanics:13}
\begin{split}\color{purple}{
{\rm{ \mathbf{Third \, form:} }} \quad h_e = \left( \frac{A_{out}}{A_{in}} -1 \right)^2 \, \frac{\bar  v_{out}^2}{2g} \, \, = \, \, K_e \frac{\bar v_{out}^2}{2g}, \quad {\rm where} \quad K_e = \left( \frac{A_{out}}{A_{in}} - 1 \right)^2
}\end{split}
\end{equation}
\begin{DUlineblock}{0em}
\item[] Such that:
\item[] \(K_e^{'}, \,\, K_e\) = minor loss coefficients, dimensionless
\end{DUlineblock}

\begin{sphinxadmonition}{note}{Note:}
You will most often see \(K_e^{'}\) and \(K_e\) used without the \(e\) subscript,  as \(K^{'}\) and \(K\).
\end{sphinxadmonition}


\sphinxstrong{See also:}


\sphinxstylestrong{Function in aide\_design:} \sphinxcode{\sphinxupquote{pc.headloss\_exp\_general(Vel, KMinor)}} Returns \(h_e\). Can be either the second or third form due to user input of both velocity and minor loss coefficient. It is up to the user to use consistent \(\bar v\) and \(K_e\).




\sphinxstrong{See also:}


\sphinxstylestrong{Function in aide\_design:} \sphinxcode{\sphinxupquote{pc.headloss\_exp(FlowRate, Diam, KMinor)}} Returns \(h_e\). Uses third form, \(K_e\).



\begin{figure}[htbp]
\centering
\capstart

\noindent\sphinxincludegraphics[width=650\sphinxpxdimen]{{minor_loss_pipe}.png}
\caption{The \(in\) and \(out\) subscripts in each of the three forms of the minor loss equation refer to this diagram that was used for the derivation.}\label{\detokenize{Review/Review_Fluid_Mechanics:id7}}\label{\detokenize{Review/Review_Fluid_Mechanics:figure-minor-loss-pipe-frd}}\end{figure}

The second and third forms are the ones which you are probably most familiar with. The distinction between them, however, is critical. First, consider the magnitudes of \(A_{in}\) and \(A_{out}\). \(A_{in}\) can never be larger than \(A_{out}\), because the flow is expanding. When flow expands, the cross-sectional area it flows through must increase. As a result, both \(\frac{A_{out}}{A_{in}} > 1\) and \(\frac{A_{in}}{A_{out}} < 1\) must always be true. This means that \(K^{'}\) can never be greater than 1, while \(K\) technically has no upper limit.

If you have taken CEE 3310, you have seen tables of minor loss coefficients \sphinxhref{https://www.engineeringtoolbox.com/minor-loss-coefficients-pipes-d\_626.html}{like this
one}, and they almost all have coefficients greater than 1. This implies that these tables use the third form of the minor loss equation as we have defined it, where the velocity is \(\bar v_{out}\). There is a good reason for using the third form over the second one: \(\bar v_{out}\) is far easier to determine than \(\bar v_{in}\). Consider flow through a pipe elbow, as shown in the image below.

\begin{figure}[htbp]
\centering
\capstart

\noindent\sphinxincludegraphics[width=650\sphinxpxdimen]{{minor_loss_elbow}.png}
\caption{Flow around a pipe elbow results in a minor loss. ‘Control surface 1’ can be abbreviated as ‘CS 1’}\label{\detokenize{Review/Review_Fluid_Mechanics:id8}}\label{\detokenize{Review/Review_Fluid_Mechanics:figure-minor-loss-elbow}}\end{figure}

In order to find \(\bar v_{out}\), we first need to know what (or where) is \(out\) and what is \(in\). A simple way to distinguish the two surfaces is that \(in\) occurs when the flow is most contracted, and \(out\) occurs when the flow has fully expanded after that maximal contraction. Going on these guidelines, Control surface ‘2’ (CS 2) in the figure above above would be \(in\), since it represents the most contracted flow in the elbow-pipe system. Therefore, CS 3 would be \(out\), as it represents the flow having fully expanded after its compression at CS 2.

\(\bar v_{out}\) is easy to determine because it is the velocity of the fluid as it flows through the entire area of the pipe. Thus, \(\bar v_{out}\) can be found with the continuity equation, since the flow through the pipe and its diameter are easy to measure, \(\bar v_{out} = \frac{4 Q}{\pi D^2}\). On the other hand, \(\bar v_{in}\) is difficult to find, as the area of the contracted flow is dependent on the exact geometry of the elbow. This is why the third form of the minor loss equation, as we have defined it, is the most common:
\begin{equation}\label{equation:Review/Review_Fluid_Mechanics:Review/Review_Fluid_Mechanics:14}
\begin{split}h_e = K \frac{\bar v_{out}^2}{2g} = \,\,\,\, \left( \frac{A_{out}}{A_{in}} -1 \right)^2 \frac{\bar v_{out}^2}{2g}\end{split}
\end{equation}
\begin{sphinxadmonition}{note}{Note:}
When considering a hydraulic system within a control volume, there can be many sources of minor losses. Instead of saying \(h_e = K_1 \frac{\bar v_{out}^2}{2g} + K_2 \frac{\bar v_{out}^2}{2g} + ...\) we can simply lump all of the minor loss coefficients into one: \(\sum K = K_1 + K_2 + ...\). Thus, it is also common to see this form of the minor loss equation when finding the minor loss across control volumes: \(\sum K \frac{v_{out}^2}{2g}\).
\end{sphinxadmonition}


\subsection{Head Loss = Elevation Difference Trick}
\label{\detokenize{Review/Review_Fluid_Mechanics:head-loss-elevation-difference-trick}}\label{\detokenize{Review/Review_Fluid_Mechanics:heading-head-loss-elevation-difference-trick}}
This trick, also called the ‘control volume trick,’ or more colloquially, the ‘head loss trick,’ is incredibly useful for simplifying hydraulic systems and is used all the time in this class.

Consider the following figure:

\begin{figure}[htbp]
\centering
\capstart

\noindent\sphinxincludegraphics[width=650\sphinxpxdimen]{{head_loss_trick}.png}
\caption{A typical hydraulic system can be used to understand the head loss trick.}\label{\detokenize{Review/Review_Fluid_Mechanics:id9}}\label{\detokenize{Review/Review_Fluid_Mechanics:figure-head-loss-trick}}\end{figure}

In systems like this, where an elevation difference is causing water to flow, the elevation difference is called the \sphinxstylestrong{driving head}. In the system above, the driving head is the elevation difference between the water level and the end of the tubing. Usually, driving head is written as \(\Delta z\) or \(\Delta h\), though above it is labelled as \(h_L\). Doesn’t \(h_L\) refer to head loss though? Yes it does! Referring to \(\Delta h\) or \(\Delta z\) \sphinxstyleemphasis{IS} the head loss trick, and how it works is explained in the following paragraphs and equations.

The figure is technically violating the energy equation by saying that the elevation difference between the water in the tank and the end of the tube is \(h_L\). It implies that all of the driving head, \(\Delta z\), is lost to head loss. Since all of the energy is gone, there should not be water flowing out of the tubing. But there is. Let’s apply the energy equation across the control surfaces shown in the figure. Pressures at both ends are atmospheric and the velocity of water at the top of tank is negligible.
\begin{equation}\label{equation:Review/Review_Fluid_Mechanics:Review/Review_Fluid_Mechanics:15}
\begin{split}\cancel{ \frac{p_{1}}{\rho g} } + z_{1} + \cancel{ \frac{\bar v_{1}^2}{2g} } = \cancel{ \frac{p_{2}}{\rho g} } + z_{2} + \frac{\bar v_{2}^2}{2g} + h_L\end{split}
\end{equation}
We now get:
\begin{equation}\label{equation:Review/Review_Fluid_Mechanics:Review/Review_Fluid_Mechanics:16}
\begin{split}\Delta z = \frac{\bar v_2^2}{2g} + h_L\end{split}
\end{equation}
This equation contradicts the figure above, which says that \(\Delta z = h_L\) and neglects \(\frac{\bar v_2^2}{2g}\). The figure above is correct, however, if you apply the head loss trick. The trick incorporates the \(\frac{\bar v_2^2}{2g}\) term \sphinxstyleemphasis{into} the \(h_L\) term as a minor loss. See the math below:
\begin{equation}\label{equation:Review/Review_Fluid_Mechanics:Review/Review_Fluid_Mechanics:17}
\begin{split}\Delta z = \frac{\bar v_2^2}{2g} + h_e + h_f\end{split}
\end{equation}\begin{equation}\label{equation:Review/Review_Fluid_Mechanics:Review/Review_Fluid_Mechanics:18}
\begin{split}\Delta z = \frac{\bar v_2^2}{2g} + \left( \sum K \right) \frac{\bar v_2^2}{2g} + h_f\end{split}
\end{equation}\begin{equation}\label{equation:Review/Review_Fluid_Mechanics:Review/Review_Fluid_Mechanics:19}
\begin{split}\Delta z = \left( 1 + \sum K \right) \frac{\bar v_2^2}{2g} + h_f\end{split}
\end{equation}\begin{equation}\label{equation:Review/Review_Fluid_Mechanics:Review/Review_Fluid_Mechanics:20}
\begin{split}\Delta z = \left( \sum K \right) \frac{\bar v_2^2}{2g} + h_f\end{split}
\end{equation}
This last step incorporated the kinetic energy term of the energy equation, \(\frac{\bar v_2^2}{2g}\), into the minor loss equation by saying that its \(K\) is 1 and incorporating that 1 into \(\sum K\). From here, we reverse our steps to get \(\Delta z = h_L\), starting with \(h_e = \left( \sum K \right) \frac{\bar v_2^2}{2g}\)
\begin{equation}\label{equation:Review/Review_Fluid_Mechanics:Review/Review_Fluid_Mechanics:21}
\begin{split}\Delta z = h_e + h_f\end{split}
\end{equation}\begin{equation}\label{equation:Review/Review_Fluid_Mechanics:Review/Review_Fluid_Mechanics:22}
\begin{split}\Delta z = h_L\end{split}
\end{equation}
By applying the head loss trick, you are considering the entire flow of the fluid out of a control volume as energy lost via minor losses. This is just an algebraic trick, the only thing to remember when applying this trick is that \(\sum K\) will always be at least 1, even if there are no ‘real’ minor losses in the system.


\section{The Orifice Equation}
\label{\detokenize{Review/Review_Fluid_Mechanics:the-orifice-equation}}\label{\detokenize{Review/Review_Fluid_Mechanics:heading-the-orifice-equation}}
This equation is one that you’ll see and use again and again throughout this class. Understanding it now will be invaluable, as future concepts will use and build on this equation.


\subsection{What is a Vena Contracta?}
\label{\detokenize{Review/Review_Fluid_Mechanics:what-is-a-vena-contracta}}\label{\detokenize{Review/Review_Fluid_Mechanics:heading-what-is-a-vena-contracta}}
Before describing the equation, we must first understand the concept of a \sphinxhref{https://en.wikipedia.org/wiki/Vena\_contracta}{vena contracta}. Refer to the figure below.

\begin{figure}[htbp]
\centering
\capstart

\noindent\sphinxincludegraphics[width=650\sphinxpxdimen]{{sluice_gate_vena_contracta}.png}
\caption{This figure shows flow around a sluice gate. Since streamlines can’t make sharp turns, the flow is forced to gradually curve and contract to an area smaller than the area of the gate.}\label{\detokenize{Review/Review_Fluid_Mechanics:id10}}\label{\detokenize{Review/Review_Fluid_Mechanics:figure-sluice-gate-vena-contracta}}\end{figure}

The flow contracts as the fluid moves past the gate. This happens because the fluid can’t make a sharp turn as it tries to go around the gate, as indicated by the streamline in the figure. Instead, the most extreme streamline makes a gradual change in direction. As a result of this gradual turn, the flow contracts and the cross-sectional area the fluid is flowing decreases.

The term ‘vena contracta’ describes the phenomenon of contracting flow due to streamlines being unable to make sharp turns. \(\Pi_{vc}\) is a dimensionless ratio comparing the flow area at the point of maximal contraction, \(A_{downstream}\), and the flow area \sphinxstyleemphasis{before} the contraction, \(A_{gate}\). In the figure above, the equation for the vena contracta coefficient would be:
\begin{equation}\label{equation:Review/Review_Fluid_Mechanics:Review/Review_Fluid_Mechanics:23}
\begin{split}\Pi_{vc} = \frac{A_{downstream}}{A_{gate}}\end{split}
\end{equation}
When the most extreme turn a streamline must make is 90°, the value of the vena contracta coefficient is close to 0.62. This parameter value, 0.62, is in aide\_design as \sphinxcode{\sphinxupquote{pc.RATIO\_VC\_ORIFICE}}. The vena contracta coefficient value is a function of the flow geometry. Since the ratio always puts the most contracted area over the least contracted area, \(\Pi_{vc}\) is always less than 1.

\begin{sphinxadmonition}{important}{Important:}
\sphinxstylestrong{A vena contracta coefficient is not a minor loss coefficient.} Though the equations for the two both involve contracted and non-contracted areas, these coefficients are not the same. Minor losses coefficients imply energy loss, and vena contractas do not. Minor losses coefficients deal with flow expansions, and vena contracas deal with flow contractions. Confusing the two coefficients is common mistake that this paragraph will hopefully help you to avoid.
\end{sphinxadmonition}

\begin{sphinxadmonition}{note}{Note:}
Note that what this class calls \(\Pi_{vc}\) is often referred to as a ‘Coefficient of Contraction,’ \(C_c\), in other engineering courses and settings.
\end{sphinxadmonition}


\subsection{Origin of the Orifice Equation}
\label{\detokenize{Review/Review_Fluid_Mechanics:origin-of-the-orifice-equation}}
The orifice equation is derived from the Bernoulli equation as applied to the purple points in the following image:

\begin{figure}[htbp]
\centering
\capstart

\noindent\sphinxincludegraphics[width=650\sphinxpxdimen]{{hole_in_a_bucket}.png}
\caption{Flow through a hole in the bottom of a bucket is a great example of the orifice equation.}\label{\detokenize{Review/Review_Fluid_Mechanics:id11}}\label{\detokenize{Review/Review_Fluid_Mechanics:figure-hole-in-a-bucket}}\end{figure}

At point 1, the pressure is atmospheric and the instantaneous velocity is negligible as the water level in the bucket drops slowly. At point 2, the pressure is also atmospheric. We define the difference in elevations between the two points, \(z_1 - z_2\), to be \(\Delta h\). With these simplifications \((p_1 = \bar v_1 = p_2 = 0)\) and assumptions \((z_A - z_B = \Delta h)\), the Bernoulli equation becomes:
\begin{equation}\label{equation:Review/Review_Fluid_Mechanics:Review/Review_Fluid_Mechanics:24}
\begin{split}\Delta h = \frac{\bar v_2^2}{2g}\end{split}
\end{equation}
Substituting the continuity equation \(Q = \bar v A\) in the form of \(\bar v_2^2 = \frac{Q^2}{A_{vc}^2}\), the vena contracta coefficient in the form of \(A_{vc} = \Pi_{vc} A_{or}\) yields:
\begin{equation}\label{equation:Review/Review_Fluid_Mechanics:Review/Review_Fluid_Mechanics:25}
\begin{split}\Delta h = \frac{Q^2}{2g \Pi_{vc}^2 A_{or}^2}\end{split}
\end{equation}
Which, rearranged to solve for \(Q\) gives \sphinxstylestrong{The Orifice Equation:}
\begin{equation}\label{equation:Review/Review_Fluid_Mechanics:orifice_equation}
\begin{split}  Q = \Pi_{vc} A_{or} \sqrt{2g\Delta h}\end{split}
\end{equation}
\begin{DUlineblock}{0em}
\item[] Such that:
\item[] \(\Pi_{vc}\) = 0.62 = vena contracta coefficient, in aide\_design as \sphinxcode{\sphinxupquote{pc.RATIO\_VC\_ORIFICE}}
\item[] \(A_{or}\) = orifice area- NOT contracted flow area
\item[] \(\Delta h\) = elevation difference between orifice and water level
\end{DUlineblock}


\sphinxstrong{See also:}


\sphinxstylestrong{Equation in aide\_design:} \sphinxcode{\sphinxupquote{pc.flow\_orifice(Diam, Height, RatioVCOrifice)}} Returns flow through a horizontal orifice.




\sphinxstrong{See also:}


\sphinxstylestrong{Equation in aide\_design:} \sphinxcode{\sphinxupquote{pc.flow\_orifice\_vert(Diam, Height, RatioVCOrifice)}} Returns flow through a vertical orifice. The height parameter refers to height above the center of the orifice.



There are two configurations for an orifice in the tank holding a fluid: horizontal and vertical. These are both displayed in the figure below. The orifice equation written is for a horizontal orifice; the equation for flow through vertical orifice equation requires integration or the orifice equation across its height to return the correct flow. This is explored in the Flow Control and Measurement Examples section.

\begin{figure}[htbp]
\centering
\capstart

\noindent\sphinxincludegraphics[width=650\sphinxpxdimen]{{vertical_and_horizontal_orifices}.png}
\caption{The descriptions ‘vertical’ and ‘horizontal’ \sphinxstylestrong{apply to the orientation of the orifices,} not to the orientation of the fluid coming out of the orifices.}\label{\detokenize{Review/Review_Fluid_Mechanics:id12}}\label{\detokenize{Review/Review_Fluid_Mechanics:figure-vertical-and-horizontal-orifices}}\end{figure}


\section{Section Summary}
\label{\detokenize{Review/Review_Fluid_Mechanics:section-summary}}\label{\detokenize{Review/Review_Fluid_Mechanics:heading-fr-section-summary}}\begin{enumerate}
\item {} 
\sphinxstylestrong{Introductory Concepts:}
\begin{quote}
\begin{itemize}
\item {} 
\sphinxstylestrong{Continuity} means that the mass of a fluid is conserved as it flows, and implies a constant density. The continuity equation has two purposes:
\begin{quote}
\begin{enumerate}
\item {} 
Relating the average velocity of a fluid, \(\bar v\), to its flow rate, \(Q\), via the cross-sectional area, \(A\), that it flows through. When the fluid is flowing in a pipe, we can simply this even further to relate the flow rate and velocity to the pipe’s diameter, \(D\). The final equation below is only used for circular pipes, as it includes a pipe diameter.

\end{enumerate}
\begin{equation}\label{equation:Review/Review_Fluid_Mechanics:Review/Review_Fluid_Mechanics:26}
\begin{split}Q = \bar v A = \bar v \frac{\pi D^2}{4}\end{split}
\end{equation}\begin{enumerate}
\item {} 
Finding the average velocity or flow when the geometry of a fluid’s flow changes, as the mass of the fluid must be conserved when it transitions through flow geometries.

\end{enumerate}
\begin{equation}\label{equation:Review/Review_Fluid_Mechanics:Review/Review_Fluid_Mechanics:27}
\begin{split}Q_1 = Q_2\end{split}
\end{equation}\begin{equation}\label{equation:Review/Review_Fluid_Mechanics:Review/Review_Fluid_Mechanics:28}
\begin{split}\bar v_1 A_1 = \bar v_2 A_2\end{split}
\end{equation}\begin{equation}\label{equation:Review/Review_Fluid_Mechanics:Review/Review_Fluid_Mechanics:29}
\begin{split}\bar v_1 \frac{\pi D_1^2}{4} = \bar v_2 \frac{\pi D_2^2}{4}\end{split}
\end{equation}\end{quote}

\item {} 
\sphinxstylestrong{Laminar and Turbulent flow} describe the disorder and chaos of fluid flow. The \sphinxstylestrong{Reynolds number,} \({\rm Re}\) is used to distinguish laminar from turbulent flow. For \({\rm Re} < 2100\), flow is considered laminar. For \({\rm Re} > 2100\), flow is considered turbulent. The equations for the Reynolds number are below:

\end{itemize}
\begin{equation}\label{equation:Review/Review_Fluid_Mechanics:Review/Review_Fluid_Mechanics:30}
\begin{split}{\rm Re} = \frac{\bar vD}{\nu} = \frac{4Q}{\pi D\nu} = \frac{\rho \bar vD}{\mu}\end{split}
\end{equation}\begin{itemize}
\item {} 
\sphinxstylestrong{Control volumes vs Streamlines.} This section is quite short, a summary would simply repeat what the sections says. The section is its own summary; read it here: {\hyperref[\detokenize{Review/Review_Fluid_Mechanics:streamlines-and-control-volumes}]{\sphinxcrossref{Streamlines and Control Volumes}}}

\end{itemize}
\end{quote}

\item {} 
\sphinxstylestrong{Bernoulli vs Energy equations:} The Bernoulli equation assumes that energy is conserved throughout a streamline or control volume. The Energy equation assumes that there is energy loss, or head loss \(h_L\). This head loss is composed of major losses, \(h_{\rm{f}}\), and minor losses, \(h_e\).

\end{enumerate}
\begin{quote}

Bernoulli equation:
\begin{equation}\label{equation:Review/Review_Fluid_Mechanics:Review/Review_Fluid_Mechanics:31}
\begin{split}\frac{p_1}{\rho g} + {z_1} + \frac{\bar v_1^2}{2g} = \frac{p_2}{\rho g} + {z_2} + \frac{\bar v_2^2}{2g}\end{split}
\end{equation}
Energy equation, simplified to remove pumps, turbines, and \(\alpha\) factors:
\begin{equation}\label{equation:Review/Review_Fluid_Mechanics:Review/Review_Fluid_Mechanics:32}
\begin{split}\frac{p_{1}}{\rho g} + z_{1} + \frac{\bar v_{1}^2}{2g} = \frac{p_{2}}{\rho g} + z_{2} + \frac{\bar v_{2}^2}{2g} + h_L\end{split}
\end{equation}\end{quote}
\begin{enumerate}
\setcounter{enumi}{2}
\item {} 
\sphinxstylestrong{Major losses:} Defined as the energy loss due to shear between the walls of the pipe/flow conduit and the fluid. The Darcy-Weisbach equation is used to find major losses in both laminar and turbulent flow regimes. The equation for finding the Darcy friction factor, \(\rm{f}\), changes depending on whether the flow is laminar or turbulent. The Moody diagram is a common graphical method for finding \(\rm{f}\). During laminar flow, the Hagen-Poiseuille equation, which is just a combination of Darcy-Weisbach, Reynolds number, and \({\rm{f}} = \frac{64}{\rm{Re}}\), can be used

\end{enumerate}
\begin{quote}

Darcy-Weisbach equation:
\begin{equation}\label{equation:Review/Review_Fluid_Mechanics:Review/Review_Fluid_Mechanics:33}
\begin{split}h_{\rm{f}} = {\rm{f}} \frac{L}{D} \frac{\bar v^2}{2g}\end{split}
\end{equation}
For water treatment plant design we tend to use plant flow rate, \(Q\), as our master variable and thus we have.
\begin{equation}\label{equation:Review/Review_Fluid_Mechanics:Review/Review_Fluid_Mechanics:34}
\begin{split}h_{\rm{f}} = {\rm{f}} \frac{8}{g \pi^2} \frac{LQ^2}{D^5}\end{split}
\end{equation}
\(\rm{f}\) for laminar flow:
\begin{equation}\label{equation:Review/Review_Fluid_Mechanics:Review/Review_Fluid_Mechanics:35}
\begin{split}{\rm{f}} = \frac{64}{\rm{Re}} = \frac{16 \pi D \nu}{Q} = \frac{64 \nu}{\bar v D}\end{split}
\end{equation}
\(\rm{f}\) for turbulent flow:
\begin{equation}\label{equation:Review/Review_Fluid_Mechanics:Review/Review_Fluid_Mechanics:36}
\begin{split}{\rm{f}} = \frac{0.25} {\left[ \log \left( \frac{\epsilon }{3.7D} + \frac{5.74}{{\rm Re}^{0.9}} \right) \right]^2}\end{split}
\end{equation}
Hagen-Poiseuille equation for laminar flow:
\begin{equation}\label{equation:Review/Review_Fluid_Mechanics:Review/Review_Fluid_Mechanics:37}
\begin{split}h_{\rm{f}} = \frac{32\mu L \bar v}{\rho gD^2} = \frac{128\mu Q}{\rho g\pi D^4}\end{split}
\end{equation}\end{quote}
\begin{enumerate}
\setcounter{enumi}{3}
\item {} 
\sphinxstylestrong{Minor losses:} Defined as the energy loss due to the generation of turbulent eddies when flow expands. Once more: minor losses are caused by flow expansions. There are three forms of the minor loss equation, two of which look the same but use different coefficients (\(K^{'}\) vs \(K\)) and velocities (\(\bar v_{in}\) vs \(\bar v_{out}\)). \sphinxstyleemphasis{Make sure the coefficient you select is consistent with the velocity you use}. The third form, written in purple, is the most commonly used form of the minor loss equation.

\end{enumerate}
\begin{equation}\label{equation:Review/Review_Fluid_Mechanics:Review/Review_Fluid_Mechanics:38}
\begin{split}{\rm{ \mathbf{First \, form:} }} \quad h_e = \frac{\left( \bar v_{in}  - \bar v_{out} \right)^2}{2g}\end{split}
\end{equation}\begin{equation}\label{equation:Review/Review_Fluid_Mechanics:Review/Review_Fluid_Mechanics:39}
\begin{split}{\rm{ \mathbf{Second \, form:} }} \quad h_e = \left( 1 - \frac{A_{in}}{A_{out}} \right)^2 \, \frac{\bar v_{in}^2}{2g} \, \, = \, \, K_e^{'} \frac{\bar v_{in}^2}{2g}, \quad {\rm where} \quad K_e^{'} = \left( 1 - \frac{A_{in}}{A_{out}} \right)^2\end{split}
\end{equation}\begin{equation}\label{equation:Review/Review_Fluid_Mechanics:Review/Review_Fluid_Mechanics:40}
\begin{split}\color{purple}{
{\rm{ \mathbf{Third \, form:} }} \quad h_e = \left( \frac{A_{out}}{A_{in}} -1 \right)^2 \, \frac{\bar  v_{out}^2}{2g} \, \, = \, \, K_e \frac{\bar v_{out}^2}{2g}, \quad {\rm where} \quad K_e = \left( \frac{A_{out}}{A_{in}} - 1 \right)^2
}\end{split}
\end{equation}\begin{enumerate}
\setcounter{enumi}{4}
\item {} 
\sphinxstylestrong{Major and minor losses vary with flow:} While it is generally important to know how increasing or decreasing flow will affect head loss, it is even more important for this class to understand exactly how flow will affect head loss. As the table below shows, head loss will always be proportional to flow squared during turbulent flow. During laminar flow, however, the exponent on \(Q\) will be between 1 and 2 depending on the proportion of major to minor losses.

\end{enumerate}


\begin{savenotes}\sphinxattablestart
\centering
\sphinxcapstartof{table}
\sphinxcaption{Proportionality between head loss \(h_L\) and flow rate \(Q\) for different flow regimes and types of head loss.}\label{\detokenize{Review/Review_Fluid_Mechanics:id13}}\label{\detokenize{Review/Review_Fluid_Mechanics:table-h-q-proportionality}}
\sphinxaftercaption
\begin{tabular}[t]{|\X{10}{30}|\X{10}{30}|\X{10}{30}|}
\hline

\(h_L propto Q^?\)
&\sphinxstyletheadfamily 
Major Losses
&\sphinxstyletheadfamily 
Minor Losses
\\
\hline
Laminar
&
\(Q\)
&
\(Q^2\)
\\
\hline
Turbulent
&
\(Q^2\)
&
\(Q^2\)
\\
\hline
\end{tabular}
\par
\sphinxattableend\end{savenotes}
\begin{enumerate}
\setcounter{enumi}{5}
\item {} 
The \sphinxstylestrong{head loss trick}, also called the control volume trick, can be used to incorporate the ‘kinetic energy out’ term of the energy equation, \(\frac{\bar v_2^2}{2g}\), into head loss as a minor loss with \(K = 1\), so the minor loss equation becomes \(\left( 1 + \sum K \right) \frac{\bar v^2}{2g}\). This is used to be able to say that \(\Delta z = h_L\) and makes many equation simplifications possible in the future.

\item {} 
\sphinxstylestrong{Orifice equation and vena contractas:} The orifice equation is used to determine the flow out of an orifice given the elevation of water above the orifice. This equation introduces the concept of vena contracta, which describes flow contraction due to the inability of streamlines to make sharp turns. The equation shows that the flow out of an orifice is proportional to the square root of the driving head, \(Q \propto \sqrt{\Delta h}\). Depending on the orientation of the orifice, vertical (like a hole in the side of a bucket) or horizontal (like a hole in the bottom of a bucket), a different equation in aide\_design should be used.

\end{enumerate}
\begin{quote}

The Orifice Equation:
\begin{equation}\label{equation:Review/Review_Fluid_Mechanics:Review/Review_Fluid_Mechanics:41}
\begin{split}Q = \Pi_{vc} A_{or} \sqrt{2g\Delta h}\end{split}
\end{equation}\end{quote}


\chapter{Review: Fluid Mechanics Derivations}
\label{\detokenize{Review/Review_Fluid_Mechanics_Derivations:review-fluid-mechanics-derivations}}\label{\detokenize{Review/Review_Fluid_Mechanics_Derivations:title-review-fluid-mechanics-derivations}}\label{\detokenize{Review/Review_Fluid_Mechanics_Derivations::doc}}

\section{Minor Loss Equation}
\label{\detokenize{Review/Review_Fluid_Mechanics_Derivations:minor-loss-equation}}\label{\detokenize{Review/Review_Fluid_Mechanics_Derivations:heading-minor-loss-equation-derivation}}
This section contains the derivation of the minor loss equation using the following figure as a reference. The derivation begins with a slightly simplified energy equation across the control volume shown. Our energy equation begins with \(h_P\) and \(h_T\) having been
eliminated.

\begin{figure}[htbp]
\centering
\capstart

\noindent\sphinxincludegraphics[width=700\sphinxpxdimen]{{minor_loss_pipe}.png}
\caption{This is the system we will use to derive the minor loss equation.}\label{\detokenize{Review/Review_Fluid_Mechanics_Derivations:id1}}\label{\detokenize{Review/Review_Fluid_Mechanics_Derivations:figure-minor-loss-pipe}}\end{figure}
\begin{equation}\label{equation:Review/Review_Fluid_Mechanics_Derivations:Review/Review_Fluid_Mechanics_Derivations:0}
\begin{split}\frac{p_{in}}{\rho g} + {z_{in}} + \frac{\bar v_{in}^2}{2g} = \frac{p_{out}}{\rho g} + z_{out} + \frac{\bar v_{out}^2}{2g} + h_L\end{split}
\end{equation}
Since the elevations at the center of the \(in\) and \(out\) control surfaces are the same, we can eliminate \(z_{in}\) and \(z_{out}\). As we are considering such a small length of pipe, we will neglect the major loss component of head loss. Thus, \(h_L = h_e + \cancel{h_f}\). The following three equations are all the same, simply rearranged to solve for \(h_e\).
\begin{equation}\label{equation:Review/Review_Fluid_Mechanics_Derivations:Review/Review_Fluid_Mechanics_Derivations:1}
\begin{split}\frac{p_{in}}{\rho g} + \frac{\bar v_{in}^2}{2g} = \frac{p_{out}}{\rho g} + \frac{\bar v_{out}^2}{2g} + h_e\end{split}
\end{equation}\begin{equation}\label{equation:Review/Review_Fluid_Mechanics_Derivations:Review/Review_Fluid_Mechanics_Derivations:2}
\begin{split}\frac{p_{in} - p_{out}}{\rho g} = \frac{\bar v_{out}^2 - \bar v_{in}^2}{2g} + h_e\end{split}
\end{equation}\begin{equation}\label{equation:Review/Review_Fluid_Mechanics_Derivations:minor_loss_energy_eq}
\begin{split}  h_e = \frac{p_{in} - p_{out}}{\rho g} + \frac{\bar v_{in}^2 - \bar v_{out}^2}{2g}\end{split}
\end{equation}
This last equation has \(h_e\) as a function of four variables \((p_{in}, \, p_{out}, \, v_{in}\), and \(v_{out})\); we would like it to be a function of only one. Thus, we will invoke conservation of momentum in the horizontal direction across our control volume to remove variables. The difference in momentum from the \(in\) point to the \(out\) point is driven by the pressure difference between each end of the control volume. We will be considering the pressure at the centroid of our control surfaces, and we will neglect shear along the pipe walls. After these assumptions, our momentum equation becomes the following:
\begin{equation}\label{equation:Review/Review_Fluid_Mechanics_Derivations:Review/Review_Fluid_Mechanics_Derivations:3}
\begin{split}M_{in, \, x} + M_{out, \, x} = F_{p_{in, \, x}} + F_{p_{out, \, x}}\end{split}
\end{equation}
\begin{DUlineblock}{0em}
\item[] Such that:
\item[] \(M_{x}\) = momentum flowing through the control volume in the x-direction
\item[] \(F_{p_x}\) = force due to pressure acting on the boundaries of the control volume in the x-direction
\end{DUlineblock}

Recall that momentum is mass times velocity for solids, \(m v\), with units of \(\frac{[M][L]}{[T]}\). Since we consider water flowing through a pipe, there is not one singular mass or one singular velocity. Instead, there is a mass flow rate, or a mass per time indicated by \(\dot m = \rho Q\), which has units of \(\frac{[M]}{[T]}\). Therefore, the momentum for a fluid is \(\rho Q \bar v\). Applying the continuity equation \(Q = \bar v A\), we get to the following equation for the momentum of a fluid flowing through a pipe which we will use in this derivation, \(M = \rho \bar v^2 A\). The pressure force is simply the pressure at the centroid of the flow multiplied by the area the pressure is acting upon, \(p A\).

To ensure correct sign convention, we will make each side of the equation negative for reasons discussed shortly. Since \(\bar v_{in} > \bar v_{out}\), the left hand side will be \(M_{out} - M_{in}\) in order to be negative. The reduction in velocity from \(in\) to \(out\) causes an increase in pressure, therefore \(p_{in} - p_{out}\) is negative. With these substitutions, the conservation of momentum equation becomes as follows:
\begin{equation}\label{equation:Review/Review_Fluid_Mechanics_Derivations:Review/Review_Fluid_Mechanics_Derivations:4}
\begin{split}M_{out} - M_{in} = p_{in} - p_{out}\end{split}
\end{equation}\begin{equation}\label{equation:Review/Review_Fluid_Mechanics_Derivations:Review/Review_Fluid_Mechanics_Derivations:5}
\begin{split}\rho \bar v_{out}^2 A_{out} - \rho \bar v_{in}^2 A_{in} = p_{in} A_{out} - p_{out} A_{out}\end{split}
\end{equation}
Note that the area term attached to \(p_{in}\) is actually \(A_{out}\) instead of \(A_{in}\), as one might think. This is because \(A_{out} = A_{in}\). We chose our control volume to start a few millimeters into the larger pipe, which means that the cross-sectional area does not change over the course of the control volume.

Dividing both sides of the equation by \(A_{out} \rho g\), we obtain the following equation, which contains the very same pressure term as our adjusted energy equation above, equation \eqref{equation:Review/Review_Fluid_Mechanics_Derivations:minor_loss_energy_eq}. This is why we chose a negative sign convention.
\begin{equation}\label{equation:Review/Review_Fluid_Mechanics_Derivations:Review/Review_Fluid_Mechanics_Derivations:6}
\begin{split}\frac{p_{in} - p_{out}}{\rho g} = \frac{\bar v_{out}^2 - \bar v_{in}^2 \frac{A_{in}}{A_{out}}}{g}\end{split}
\end{equation}
Now, we combine the momentum, continuity, and adjusted energy equations:
\begin{equation}\label{equation:Review/Review_Fluid_Mechanics_Derivations:Review/Review_Fluid_Mechanics_Derivations:7}
\begin{split}{\rm{Energy \, equation:}} \,\,\,  h_e = \frac{p_{in} - p_{out}}{\rho g} + \frac{\bar v_{in}^2 - \bar v_{out}^2}{2g}\end{split}
\end{equation}\begin{equation}\label{equation:Review/Review_Fluid_Mechanics_Derivations:Review/Review_Fluid_Mechanics_Derivations:8}
\begin{split}{\rm{Momentum \, equation:}} \,\,\, \frac{p_{in} - p_{out}}{\rho g} = \frac{\bar v_{out}^2 - \bar v_{in}^2 \frac{A_{in}}{A_{out}}}{g}\end{split}
\end{equation}\begin{equation}\label{equation:Review/Review_Fluid_Mechanics_Derivations:Review/Review_Fluid_Mechanics_Derivations:9}
\begin{split}{\rm{Continuity \, equation:}} \,\,\, \frac{A_{in}}{A_{out}} = \frac{\bar v_{out}}{\bar v_{in}}\end{split}
\end{equation}
To obtain an equation for minor losses with just two variables, \(\bar v_{in}\) and \(\bar v_{out}\).
\begin{equation}\label{equation:Review/Review_Fluid_Mechanics_Derivations:Review/Review_Fluid_Mechanics_Derivations:10}
\begin{split}h_e = \frac{\bar v_{out}^2 - \bar v_{in}^2\frac{\bar v_{out}}{\bar v_{in}}}{g} + \frac{\bar v_{in}^2 - \bar v_{out}^2}{2g}\end{split}
\end{equation}
Now we will combine the two terms. The numerator and denominator of the first term, \(\frac{\bar v_{out}^2 - \bar v_{in}^2\frac{\bar v_{out}}{\bar v_{in}}}{g}\) will be multiplied by \(2\) to become \(\frac{2 \bar v_{out}^2 - 2 \bar v_{in}^2\frac{\bar v_{out}}{\bar v_{in}}}{2 g}\). The equation then looks like:
\begin{equation}\label{equation:Review/Review_Fluid_Mechanics_Derivations:Review/Review_Fluid_Mechanics_Derivations:11}
\begin{split}h_e = \frac{\bar v_{out}^2 - 2 \bar v_{in} \bar v_{out} + \bar v_{in}^2}{2g}\end{split}
\end{equation}

\subsection{Final Forms of the Minor Loss Equation}
\label{\detokenize{Review/Review_Fluid_Mechanics_Derivations:final-forms-of-the-minor-loss-equation}}\label{\detokenize{Review/Review_Fluid_Mechanics_Derivations:heading-final-minor-loss-equations}}
Factoring the numerator yields to the first ‘final’ form of the minor loss equation:
\begin{equation}\label{equation:Review/Review_Fluid_Mechanics_Derivations:Review/Review_Fluid_Mechanics_Derivations:12}
\begin{split}{\rm{ \mathbf{First \, form:} }} \quad h_e = \frac{\left( \bar v_{in}  - \bar v_{out} \right)^2}{2g}\end{split}
\end{equation}
From here, the two other forms of the minor loss equation can be derived by solving for either \(\bar v_{in}\) or \(\bar v_{out}\) using the ubiquitous continuity equation \(\bar v_{in} A_{in} = \bar v_{out} A_{out}\):
\begin{equation}\label{equation:Review/Review_Fluid_Mechanics_Derivations:Review/Review_Fluid_Mechanics_Derivations:13}
\begin{split}{\rm{ \mathbf{Second \, form:} }} \quad h_e = \left( 1 - \frac{A_{in}}{A_{out}} \right)^2 \, \frac{\bar v_{in}^2}{2g} \, \, = \, \, K_e^{'} \frac{\bar v_{in}^2}{2g}, \quad {\rm where} \quad K_e^{'} = \left( 1 - \frac{A_{in}}{A_{out}} \right)^2\end{split}
\end{equation}\begin{equation}\label{equation:Review/Review_Fluid_Mechanics_Derivations:minor_loss_equation}
\begin{split}   \color{purple}{
   {\rm{ \mathbf{Third \, form:} }} \quad h_e = \left( \frac{A_{out}}{A_{in}} -1 \right)^2 \, \frac{\bar  v_{out}^2}{2g} \, \, = \, \, K_e \frac{\bar v_{out}^2}{2g}, \quad {\rm where} \quad K_e = \left( \frac{A_{out}}{A_{in}} - 1 \right)^2
   }\end{split}
\end{equation}
\begin{sphinxadmonition}{note}{Note:}
You will often see \(K_e^{'}\) and \(K_e\) used without the \(e\) subscript, they will appear as \(K^{'}\) and \(K\).
\end{sphinxadmonition}

Being familiar with these three forms and how they are used will be of great help throughout the class. The third form is the one that is most commonly used.


\chapter{Flow Control and Measurement Introduction}
\label{\detokenize{Flow_Control_and_Measurement/FCM_Intro:flow-control-and-measurement-introduction}}\label{\detokenize{Flow_Control_and_Measurement/FCM_Intro:title-flow-control-intro}}\label{\detokenize{Flow_Control_and_Measurement/FCM_Intro::doc}}

\section{Tank with a Valve}
\label{\detokenize{Flow_Control_and_Measurement/FCM_Intro:tank-with-a-valve}}\label{\detokenize{Flow_Control_and_Measurement/FCM_Intro:heading-tank-with-a-valve}}

\subsection{Flow \protect\(Q\protect\) and Water Level \protect\(h\protect\) as a Function of Time}
\label{\detokenize{Flow_Control_and_Measurement/FCM_Intro:flow-and-water-level-as-a-function-of-time}}\label{\detokenize{Flow_Control_and_Measurement/FCM_Intro:heading-qh-as-a-function-of-t}}
Our first step is to see if we can get constant head out of a simple system. The most simple flow control system is a bucket or tank with a hole in it. This system is too coarse to provide constant head. One step above that is a bucket or tank with a valve. This is where we begin our search for constant head.

Using the setup of in the image below, we derive the following equation for flow \(Q\) through the valve as a function of time \(t\). The derivation is found here: {\hyperref[\detokenize{Flow_Control_and_Measurement/FCM_Derivations:heading-flow-for-a-tank-with-a-valve}]{\sphinxcrossref{\DUrole{std,std-ref}{ for a Tank with a Valve}}}}. You are advised to read through it if you are at all confused about this equation.
\begin{equation}\label{equation:Flow_Control_and_Measurement/FCM_Intro:Q_tank_with_valve}
\begin{split}  \frac{Q}{Q_0} = 1 - \frac{1}{2} \frac{t}{t_{Design}} \frac{h_{Tank}}{h_0}\end{split}
\end{equation}
\begin{DUlineblock}{0em}
\item[] Such that:
\item[] \(Q\) = \(Q(t)\) = flow of hypochlorite through valve at time \(t\)
\item[] \(Q_0\) = flow of hypochlorite through valve at time \(t = 0\)
\item[] \(t\) = elapsed time
\item[] \(t_{Design}\) = time it \sphinxstyleemphasis{would} take for tank to empty if flow stayed constant at \(Q_0\), which it does not
\item[] \(h_{Tank}\) = elevation of water level with reference to tank bottom at time \(t\) = 0
\item[] \(h_0\) = elevation of water level with reference to the valve at time \(t = 0\)
\end{DUlineblock}

\begin{figure}[htbp]
\centering
\capstart

\noindent\sphinxincludegraphics[width=600\sphinxpxdimen]{{hypochlorinator_variable_explanation}.png}
\caption{This figure shows the variables that are defined in the equation above.}\label{\detokenize{Flow_Control_and_Measurement/FCM_Intro:id1}}\label{\detokenize{Flow_Control_and_Measurement/FCM_Intro:figure-hypochlorinator-variable-explanation-design}}\end{figure}

This equation has historically give students some trouble, and while its nuances are explained in the derivation, they will be quickly summarized here:
\begin{itemize}
\item {} 
\(t_{Design}\) is \sphinxstylestrong{NOT} the time it takes to drain the tank. It is the time that it \sphinxstyleemphasis{would} take to drain the tank \sphinxstyleemphasis{if} the flow rate at time \(t = 0\), \(Q_0\), were the flow rate forever, which it is not. \(t_{Design}\) was used in the derivation to simplify the equation, which is why this potentially-confusing parameter exists. The actual time it takes to drain the tank lies somewhere between \(t_{Design}\) and \(2 \, t_{Design}\) and depends on the ratio \(\frac{h_{Tank}}{h_0}\).

\item {} 
\(h_{Tank}\) is not the same as \(h_{0}\). \(h_{Tank}\) is the height of water level in the tank with reference to the tank bottom. \(h_{0}\) is the water level in the tank with reference to the valve. Neither change with time, they both refer to the water level at one instance in time, \(t = 0\). Therefore, \(h_{0} \geq h_{Tank}\) is always true. If the tank is elevated far above the valve, then the \(h_{0} > > h_{Tank}\). If the valve is at the same elevation as the bottom of the tank, then \(h_{0} = h_{Tank}\). Please refer to the figure above to clarify \(h_{0}\) and \(h_{Tank}\).

\end{itemize}

We can use the proportionality \(Q \propto \sqrt{h}\), which applies to both minor losses and orifices to form a relationship between water level in the tank \(h\) and time \(t\). This proportionality comes from rearranging the minor loss equation \(h = K \frac{Q^2}{2 g A^2}\) for \(Q\) instead of \(h\). A table of proportionality between \(Q\) and math:\sphinxtitleref{h} can be found in \hyperref[\detokenize{Review/Review_Fluid_Mechanics:table-h-q-proportionality}]{Table \ref{\detokenize{Review/Review_Fluid_Mechanics:table-h-q-proportionality}}}

Using equation \eqref{equation:Flow_Control_and_Measurement/FCM_Intro:Q_tank_with_valve} and this proportionality relationship, we make the following plots. On the left, the valve is at the same elevation as the bottom of the tank, or \(h_{Tank} = h_0\). Our attempt to get a continuous flow rate out of this system is to make \(\frac{h_{Tank}}{h_0}\) very small by elevating the tank far above the valve. On the right, \(\frac{h_{Tank}}{h_0} = \frac{1}{50}\). While the plot looks great and provides essentially constant head, elevating the tank by 50 times its height is not realistic. The ‘tank with a valve’ is not a solution to the constant head problem.

\begin{figure}[htbp]
\centering
\capstart

\noindent\sphinxincludegraphics[width=600\sphinxpxdimen]{{tank_valve_play}.png}
\caption{These graphs show how manipulation of the variables in the \(Q(t)\) expression can result in effectively constant head.}\label{\detokenize{Flow_Control_and_Measurement/FCM_Intro:id2}}\label{\detokenize{Flow_Control_and_Measurement/FCM_Intro:figure-tank-valve-play}}\end{figure}


\subsection{Drain System for a Tank}
\label{\detokenize{Flow_Control_and_Measurement/FCM_Intro:drain-system-for-a-tank}}\label{\detokenize{Flow_Control_and_Measurement/FCM_Intro:heading-drain-system-for-a-tank}}
While the ‘tank with a valve’ scenario is not a good constant head solution, we can use our understanding of the system to properly design drain systems for AguaClara reactors like flocculators and sedimentation tanks, since they are just tanks with valves. The derivation for the following equation is here, along with more details on AguaClara’s pipe stub method for draining tanks: {\hyperref[\detokenize{Flow_Control_and_Measurement/FCM_Derivations:heading-diameter-and-time-tank-drain-equation}]{\sphinxcrossref{\DUrole{std,std-ref}{ and  for Tank Drain Equation}}}}. The derived ‘Tank Drain’ equation is as follows:
\begin{equation}\label{equation:Flow_Control_and_Measurement/FCM_Intro:Flow_Control_and_Measurement/FCM_Intro:0}
\begin{split}D_{Pipe} = \sqrt{ \frac{8 L_{Tank} W_{Tank}}{\pi t_{Drain}}} {\left( \frac{H_{Tank} \sum K }{2g} \right)^{\frac{1}{4}}}\end{split}
\end{equation}
The equation can also be rearranged to solve for the time it would take to drain a tank given its dimensions and a certain drain pipe size:
\begin{equation}\label{equation:Flow_Control_and_Measurement/FCM_Intro:Flow_Control_and_Measurement/FCM_Intro:1}
\begin{split}t_{Drain} =  \frac{8 L_{Tank} W_{Tank}}{\pi D_{Pipe}^2} {\left( \frac{H_{Tank} \sum K }{2g} \right)^{\frac{1}{2}}}\end{split}
\end{equation}
\begin{DUlineblock}{0em}
\item[] Such that:
\item[] \(D_{Pipe}\) = Diameter of the drain piping
\item[] \(L_{Tank}, W_{Tank}, H_{Tank}\) = Tank dimensions
\item[] \(t_{Drain}\) = Time it takes to drain the tank
\item[] \(\sum K\) = Sum of all the minor loss coefficients in the system
\end{DUlineblock}

\begin{figure}[htbp]
\centering
\capstart

\noindent\sphinxincludegraphics[width=600\sphinxpxdimen]{{pipe_stub_drainage_variables}.png}
\caption{Variables for draining a tank}\label{\detokenize{Flow_Control_and_Measurement/FCM_Intro:id3}}\label{\detokenize{Flow_Control_and_Measurement/FCM_Intro:figure-pipe-stub-drainage-variables-in-derivation}}\end{figure}


\chapter{Flow Control and Measurement Design}
\label{\detokenize{Flow_Control_and_Measurement/FCM_Design:flow-control-and-measurement-design}}\label{\detokenize{Flow_Control_and_Measurement/FCM_Design:title-flow-control-design}}\label{\detokenize{Flow_Control_and_Measurement/FCM_Design::doc}}
This section explores AguaClara’s search for constant head in checmial dosing. The term \sphinxstylestrong{constant head} means that the driving head of a system, \(\Delta z\) or \(\Delta h\), does not change over time, even as water flows through or out of the system. Constant head implies constant flow, since the energy driving the flow does not change.

The challenge of constant head in chemical dosing for water treatment plants is not \sphinxstyleemphasis{just} providing one continuous flow of chemicals; it is also varying that flow of chemicals as the flow rate through the plant changes, so that the concentration of chemicals in the raw water stays the same.


\section{Important Terms and Equations}
\label{\detokenize{Flow_Control_and_Measurement/FCM_Design:important-terms-and-equations}}\label{\detokenize{Flow_Control_and_Measurement/FCM_Design:heading-fcm-terms-eqs}}
\sphinxstylestrong{Terms:}
\begin{enumerate}
\item {} 
Dose

\item {} 
Coagulant

\item {} 
Chlorination

\item {} 
Turbidity

\item {} 
Organic Matter

\item {} 
Constant Head Tank

\item {} 
Sutro weir

\end{enumerate}

\sphinxstylestrong{Equations:}
\begin{enumerate}
\item {} 
Hagen-Poiseuille equation

\end{enumerate}


\section{AguaClara Flow Control and Measurement Technologies}
\label{\detokenize{Flow_Control_and_Measurement/FCM_Design:aguaclara-flow-control-and-measurement-technologies}}\label{\detokenize{Flow_Control_and_Measurement/FCM_Design:heading-aguaclara-flow-control-and-measurement-technologies}}
Each technology or component for this section will have five subsections:
\begin{itemize}
\item {} 
\sphinxstylestrong{What it is}

\item {} 
\sphinxstylestrong{What it does and why}

\item {} 
\sphinxstylestrong{How it works}

\item {} 
\sphinxstylestrong{Notes}

\end{itemize}

Before diving into the technologies, recall the purpose of the chemicals that we are seeking to constantly \sphinxstylestrong{dose}, and why it is important to keep a constant, specific dose. Also recall that ‘dose’ means ‘concentration of chemical’ \sphinxstyleemphasis{in the water we are trying to treat}, not in the stock tanks of the chemicals. \sphinxhref{https://en.wikipedia.org/wiki/Coagulation\_(water\_treatment)}{Coagulant} like alum, PAC, and some iron-based chemicals are used to turn small particles into bigger particles, allowing them to be captured more easily. Waters with high \sphinxhref{https://en.wikipedia.org/wiki/Turbidity}{turbidity}, indicative of a lot of particles like clay and bacteria, require more coagulant to treat effectively. Additionally, waters with a lot of \sphinxhref{https://en.wikipedia.org/wiki/Organic\_matter}{organic matter} require significantly more coagulant to treat. \sphinxhref{https://en.wikipedia.org/wiki/Water\_chlorination}{Chlorine} is used to disinfect water that has already been fully treated. A proper and consistent chlorine dose is required, as too low of a dose creates a risk of reintroduction of pathogens in the distribution system and too high of a dose increases the risk of carcinogenic \sphinxhref{https://en.wikipedia.org/wiki/Disinfection\_by-product}{disinfection byproduct} formation.

\begin{sphinxadmonition}{important}{Important:}
This section will often refer to the proportionality between flow \(Q\) and head \(\Delta h\) (recall that \(\Delta h = h_L\) after applying the head loss trick) by using the ‘proportional to’ symbol, \(\propto\). It is important to remember that it doesn’t necessarily matter whether \(Q\) or \(h_L\) goes first, \(Q \propto \sqrt{h_L}\) is equivalent to saying that \(h_L \propto Q^2\).
\end{sphinxadmonition}


\subsection{“Almost Linear” Flow Controller}
\label{\detokenize{Flow_Control_and_Measurement/FCM_Design:almost-linear-flow-controller}}\label{\detokenize{Flow_Control_and_Measurement/FCM_Design:heading-almost-linear-flow-controller}}

\subsubsection{What it is}
\label{\detokenize{Flow_Control_and_Measurement/FCM_Design:what-it-is}}
This device consists of a bottle of chemical solution, called the \sphinxstylestrong{Constant Head Tank} (CHT), a float valve to keep a solution in the CHT at a constant water level, a flexible tube starting at the bottom of the CHT, and many precisely placed and equally spaced holes in a pipe, as the image below shows. The holes in the pipe hold the other end of the tube that starts at the CHT.

Chemical solution, either coagulant or chlorine, is stored in a stock tank somewhere above the CHT. A different tube connects the stock tank to the float valve within the CHT.


\subsubsection{What it does and why}
\label{\detokenize{Flow_Control_and_Measurement/FCM_Design:what-it-does-and-why}}
This flow controller provides a constant flow of chemical solution to the water in the plant. When the end of the flexible tube is placed in a hole, the elevation difference between the water level in the bottle and the hole is set and does not change unless the tube is then placed in another hole. Thus, a constant flow is provided while the end of the tube is not moved.

As has been mentioned previously, the amount of chlorine and coagulant that must be added to the raw water changes depending on the flow rate of the plant; the change is necessary to keep the dose constant. More water flowing through the plant means more chlorine is necessary to maintain the dose of chlorine in the treated water. For coagulant, there are also other factors aside from plant flow rate that impact the required dose, including the turbidity and amount of organic matter in the water. The operator must be able to change the dose of both coagulant and chlorine quickly and easily, and they must be able to know the value of the new dose they set. The “Almost Linear” Flow Controller accomplishes this by having a large number of holes in the flow control pipe next to the CHT. This large number of holes gives the operator many options for adjusting the dose, and let them quickly change the flow of chemicals into the raw water by moving the end of the flexible tube from one hole to another.


\subsubsection{How it works}
\label{\detokenize{Flow_Control_and_Measurement/FCM_Design:how-it-works}}
The idea behind this flow controller is to have a linear relationship between \(Q\) and \(h_L\) (remember that \(h_L\) is equal to \(\Delta h\) when we apply the head loss trick), which can be written as \(Q \propto h_L\). Here, \(Q\) is the flow of chemicals out of the flexible tube, and \(h_L\) is the elevation difference between the water level in the CHT and the end of the flexible tube.

As you remember from section 1.5, the summary of Fluids Review, \(Q \propto \Delta h\), or \(\Delta h \propto Q\) as it was written in the section summary, is only true for the combination of major losses and laminar flow, which makes applicable the Hagen-Poiseuille equation. Therefore, the flow must always be laminar in the flexible tube that goes between the CHT and the holes, and major losses must far exceed minor losses.

It is easy to design for laminar flow, but the “Almost Linear” Flow Controller was unable to make major losses far exceed minor losses. The bending in the flexible tube caused a lot of minor losses which changed in magnitude depending on exactly how the tube was bent. This made the flow controller “almost linear,” but that wasn’t good enough.


\subsubsection{Notes}
\label{\detokenize{Flow_Control_and_Measurement/FCM_Design:notes}}\begin{itemize}
\item {} 
This flow controller is \sphinxstylestrong{no longer used by AguaClara.}

\item {} 
The tube connecting the CHT to the outlet of chemicals must really belong and, more importantly, \sphinxstylestrong{straight} to form a linear relationship between driving head and flow. This was not true for the “Almost Linear” Flow Controller. When you read about the Linear Chemical Flow Controller (CDC), you will be learning about the replacement to the “Almost Linear” Flow Controller’s replacement.

\end{itemize}


\subsection{Linear Flow Orifice Meter (LFOM)}
\label{\detokenize{Flow_Control_and_Measurement/FCM_Design:linear-flow-orifice-meter-lfom}}\label{\detokenize{Flow_Control_and_Measurement/FCM_Design:heading-lfom}}

\subsubsection{What it is}
\label{\detokenize{Flow_Control_and_Measurement/FCM_Design:id1}}
The LFOM is a weir shape cut into a pipe. It was meant to imitate \sphinxhref{http://www.nptel.ac.in/courses/105106114/pdfs/Unit14/14\_3b.pdf}{the Sutro Weir} while being far easier to build. The LFOM is a pipe with rows of holes, or orifices, drilled into it. There are progressively fewer holes per row as you move up the LFOM, as the shape is meant to resemble half a parabola on each side. The size of all holes is the same, and the amount of holes per row are precisely calculated. Water in the entrance tank flows into and down the LFOM, towards the rapid mix and flocculator.

\begin{figure}[htbp]
\centering
\capstart

\noindent\sphinxincludegraphics[width=600\sphinxpxdimen]{{sutro_v_lfom}.png}
\caption{On the left is a sutro weir. On the right is AguaClara’s approximation of the sutro weir’s geometery. This elegant innovation is called a linear flow orifice meter, or LFOM for short.}\label{\detokenize{Flow_Control_and_Measurement/FCM_Design:id9}}\label{\detokenize{Flow_Control_and_Measurement/FCM_Design:figure-sutro-v-lfom}}\end{figure}


\subsubsection{What it does and why}
\label{\detokenize{Flow_Control_and_Measurement/FCM_Design:id2}}
The LFOM does one thing and serves two purposes.

What it does:

\sphinxstylestrong{The LFOM creates a linear relationship between water level in the entrance tank and the flow out of the entrance tank.} \sphinxstyleemphasis{It does not control the flow through the plant}. If the LFOM were replaced with a hole in the bottom of the entrance tank, the same flow rate would go through the plant, the only difference being that the water level in the entrance tank would scale with flow squared \(h \propto Q^2\) instead of \(h \propto Q\). For example, if an LFOM has 10 rows of holes and has been designed for a plant whose maximum flow rate is 10 L/s, then the operator knows that the number of rows submerged in water is equal to the flow rate of the plant in L/s. So if the water were up to the third row of holes, there would be 3 L/s of water flowing through the plant.

Why it is useful:
\begin{enumerate}
\item {} 
Allows the operator to measure the flow through the plant quickly and easily, explained above.

\item {} 
Allows for the Linear Chemical Dose Controller, which will be explained next, to automatically adjust the flow of coagulant/chlorine into the plant as the plant flow rate changes. This means the operator would only need to adjust the flow of coagulant when there is a change in turbidity or organic matter.

\end{enumerate}


\subsubsection{How it works}
\label{\detokenize{Flow_Control_and_Measurement/FCM_Design:id3}}
This is best understood with examples. By shaping a weir differently, different relationships between \(Q\) and \(h\) are formed:
* In the case of a \sphinxhref{https://swmm5.files.wordpress.com/2016/09/image00124.jpg}{rectangular weir}, \(Q \propto h^{\frac{3}{2}}\)
* In the case of a \sphinxhref{https://swmm5.files.wordpress.com/2016/09/image0096.jpg}{v-notch weir}, \(Q \propto h^{\frac{5}{2}}\)
* In the case of a \sphinxhref{http://www.engineeringexcelspreadsheets.com/wp-content/uploads/2012/11/Sutro-Weir-Diagram1.jpg}{Sutro weir} and thus LFOM, \(Q \propto h\).


\subsubsection{Notes}
\label{\detokenize{Flow_Control_and_Measurement/FCM_Design:id4}}\begin{itemize}
\item {} 
The LFOM is not perfect. Before the water level reaches the second row of holes, the LFOM is simulating a rectangular weir, and thus \(h \not\propto Q\). The Sutro weir also experiences this problem.

\item {} 
If the water level exceeds the topmost row of the LFOM’s orifices, the linearity also breaks down. The entire LFOM begins to act like an orifice, the exponent of \(Q\) in \(h \propto Q\) becomes greater than 1. This is because the LFOM approaches orifice behavior, and for orifices, \(h \propto Q^2\).

\end{itemize}


\subsection{Linear Chemical Dose Controller (CDC)}
\label{\detokenize{Flow_Control_and_Measurement/FCM_Design:linear-chemical-dose-controller-cdc}}\label{\detokenize{Flow_Control_and_Measurement/FCM_Design:heading-linear-cdc}}
Since the Linear Chemical Dose Controller has become the standard in AguaClara, it is often simply called the Chemical Dose Controller, \sphinxstylestrong{or CDC for short}. It can be confusing to describe with words, so be sure to flip through the slides in the ‘Flow Control and Measurement’ powerpoint, as they contain very, very, helpful diagrams of the CDC.


\subsubsection{What it is}
\label{\detokenize{Flow_Control_and_Measurement/FCM_Design:id5}}
The CDC brings together the LFOM and many improvements to the “Almost Linear” Flow Controller. Let’s break it down, with the image below as a guide.
\begin{enumerate}
\item {} 
Start at the Constant Head Tank (CHT). This is the same set up as the “Almost Linear” Flow Controller. The stock tank feeds into the CHT, and the float valve makes sure that the water level in the constant head tank is always the same.

\end{enumerate}

2. Now the tubes. These fix the linearity problems that were the main problem in the “Almost Linear” Flow Controller.
* The tube connected to the bottom of the CHT is large diameter to minimize any head loss through it.
\begin{itemize}
\item {} 
The three thin, straight tubes are designed to generate a lot of major losses and to minimize any minor losses. This is to make sure that major losses far exceed any minor losses, which will ensure that the Hagen-Poiseuille equation is applicable and that flow will be directly proportional to the head, \(Q \propto \Delta h\). Why are there 3 tubes?
\begin{enumerate}
\item {} 
\sphinxstylestrong{3 short instead of 1 short} Removing 2 of the 3 tubes would mean 3 times the flow through the remaining tube. This means the velocity in the tube would be 3 times as fast. Since minor losses scale with \(v^2\) and major losses only scale with \(v\), this would increase the ratio of \(\rm{\frac{minor \, losses}{major \, losses}}\), which would break the linearity we’re trying to achieve. It would also increase the total head loss through the system, resulting in a lower maximum flow rate than before.

\item {} 
\sphinxstylestrong{1 long instead of 3 short} One tube whose length is equal to the three combined would be inconveniently long, and would suffer from the same problems as above. There would be even more head loss through the tube, since its length would be longer.

\end{enumerate}

\item {} 
The large-diameter tube on the right of the three thin, straight tubes is where the chemicals flow out. The end of the tube is connected to both a slider and a ‘drop tube.’ The drop tube allows for supercritical flow of the chemical leaving the dosing tubes; once the chemical enters the drop tube it falls freely and no longer affects the CDC system.

\end{itemize}
\begin{enumerate}
\setcounter{enumi}{2}
\item {} 
The slider rests on a lever. This lever is the critical part of the CDC, it connects the water level in the entrance tank, which is adjusted by the LFOM, to the difference in head between the CHT and the end of the dosing tube. This allows the flow of chemicals to automatically adjust to a change in the plant flow rate, maintaining a constant dose in the plant water. One end of the lever tracks the water level in the entrance tank by using a float. The counterweight on the other side of the lever is to make sure the float ‘floats,’ since this float is usually made of PVC, which is more dense than water.

\item {} 
The slider itself controls the dose of chemicals. For any given plant flow rate, the slider can be adjusted to increase or decrease the amount of chemical flowing through the plant.

\end{enumerate}

\begin{figure}[htbp]
\centering
\capstart

\noindent\sphinxincludegraphics[width=600\sphinxpxdimen]{{cdc_labelled}.png}
\caption{This is the setup of the chemical dose controller.}\label{\detokenize{Flow_Control_and_Measurement/FCM_Design:id10}}\label{\detokenize{Flow_Control_and_Measurement/FCM_Design:figure-cdc-labelled}}\end{figure}


\subsubsection{What it does and why}
\label{\detokenize{Flow_Control_and_Measurement/FCM_Design:id6}}
The CDC makes it easy and accurate to dose chemicals. The flow of chemicals automatically adjusts to changes in the plant flow rate to keep a constant dose, set by the operator. When a turbidity event occurs, the operator can change the dose of coagulant by moving the coagulant slider \sphinxstyleemphasis{lower} on the lever to increase the dose. The slider has labelled marks so the operator can record the dose accurately.


\subsubsection{How it works}
\label{\detokenize{Flow_Control_and_Measurement/FCM_Design:id7}}
A lot of design has gone into the CDC. The design equations and their derivations that the following steps are based on can be found here: {\hyperref[\detokenize{Flow_Control_and_Measurement/FCM_Derivations:heading-design-equations-for-the-cdc}]{\sphinxcrossref{\DUrole{std,std-ref}{Design Equations for the Linear Chemical Dose Controller (CDC)}}}}, and you are very, very strongly encouraged to read them.

The CDC can be designed manually using the equations from the derivation linked above or via aide\_design, using the equations found in \sphinxhref{https://github.com/AguaClara/aide\_design/blob/master/aide\_design/cdc\_functions.py}{cdc\_functions.py}. Either way, the design algorithm is roughly the same:
\begin{enumerate}
\item {} 
Calculate the maximum flow rate, \(Q_{Max, \, Tube}\), through each available dosing tube diameter \(D\) that keeps error due to minor losses below 10\% of total head loss. Recall that tubing diameter is an array, as there are many diameters available at hardware stores and suppliers. This means that for each step, there will be as many solutions as there are reasonable diameters available.

\end{enumerate}
\begin{equation}\label{equation:Flow_Control_and_Measurement/FCM_Design:Flow_Control_and_Measurement/FCM_Design:0}
\begin{split}Q_{Max, \, Tube} = \frac{\pi D^2}{4} \sqrt{\frac{2 h_L g \Pi_{Error}}{\sum{K} }}\end{split}
\end{equation}\begin{enumerate}
\setcounter{enumi}{1}
\item {} 
Calculate how much flow of chemical needs to pass through the CDC at maximum plant flow and maximum chemical dose. This depends on the concentration of chemicals in the stock tank.

\end{enumerate}
\begin{equation}\label{equation:Flow_Control_and_Measurement/FCM_Design:Flow_Control_and_Measurement/FCM_Design:1}
\begin{split}Q_{Max, \, CDC} = \frac{Q_{Plant} \cdot C_{Dose, \, Max}}{C_{StockTank}}\end{split}
\end{equation}\begin{enumerate}
\setcounter{enumi}{2}
\item {} 
Calculate the number of dosing tubes required if the tubes flow at  maximum capacity (round up)

\end{enumerate}
\begin{equation}\label{equation:Flow_Control_and_Measurement/FCM_Design:Flow_Control_and_Measurement/FCM_Design:2}
\begin{split}n_{Tubes} = {\rm ceil} \left( \frac{Q_{Max, \, CDC}}{Q_{Max, \, Tube}} \right)\end{split}
\end{equation}\begin{enumerate}
\setcounter{enumi}{3}
\item {} 
Calculate the length of dosing tube(s) that correspond to each available tube diameter.

\end{enumerate}
\begin{equation}\label{equation:Flow_Control_and_Measurement/FCM_Design:Flow_Control_and_Measurement/FCM_Design:3}
\begin{split}L_{Min} = \left( \frac{g h_L \pi D^4}{128 \nu Q_{Max}} - \frac{Q_{Max}}{16 \pi \nu} \sum{K} \right)\end{split}
\end{equation}\begin{enumerate}
\setcounter{enumi}{4}
\item {} 
Select a tube length from your array of solutions. Pick the longest dosing tube that you can, keeping in mind that the tube(s) must be able to fit in the plant and can’t be longer than the length of the plant wall it will be placed along.

\item {} 
Finally, select the dosing tube diameter and flow rate corresponding to the selected tube length.

\end{enumerate}


\subsubsection{Notes}
\label{\detokenize{Flow_Control_and_Measurement/FCM_Design:id8}}
Nothing in life is perfect, and the CDC is no exception. It has a few causes of inaccuracy which go beyond non-zero minor losses:
* Float valves are not perfect. There will still be minor fluctuations of the fluid level in the CHT which will result in imperfect dosing.
* Surface tension may resist the flow of chemicals from the dosing tube into the drop tube during low flows. Since the CDC design does not consider surface tension, this is a potential source of error.
* The lever and everything attached to it are not weightless. Changing the dose of coagulant or chlorine means moving the slider along the lever. Since the slider and tubes attached to it (drop tube, dosing tube) have mass, moving the slider means that the torque of the lever is altered. This means that the depth that the float is submerged is changed, which affects \(\Delta h\) of the system. This can be remedied by making the float’s diameter as large as possible, which makes these fluctuations small. This problem can not be avoided entirely.


\section{Section Summary}
\label{\detokenize{Flow_Control_and_Measurement/FCM_Design:section-summary}}\label{\detokenize{Flow_Control_and_Measurement/FCM_Design:heading-fcm-section-summary}}
1. \sphinxstylestrong{Tank with a valve:}
.. math:

\fvset{hllines={, ,}}%
\begin{sphinxVerbatim}[commandchars=\\\{\}]
\PYGZbs{}\PYG{n}{frac}\PYG{p}{\PYGZob{}}\PYG{n}{Q}\PYG{p}{\PYGZcb{}}\PYG{p}{\PYGZob{}}\PYG{n}{Q\PYGZus{}0}\PYG{p}{\PYGZcb{}} \PYG{o}{=} \PYG{l+m+mi}{1} \PYG{o}{\PYGZhy{}} \PYGZbs{}\PYG{n}{frac}\PYG{p}{\PYGZob{}}\PYG{l+m+mi}{1}\PYG{p}{\PYGZcb{}}\PYG{p}{\PYGZob{}}\PYG{l+m+mi}{2}\PYG{p}{\PYGZcb{}} \PYGZbs{}\PYG{n}{frac}\PYG{p}{\PYGZob{}}\PYG{n}{t}\PYG{p}{\PYGZcb{}}\PYG{p}{\PYGZob{}}\PYG{n}{t\PYGZus{}}\PYG{p}{\PYGZob{}}\PYG{n}{Design}\PYG{p}{\PYGZcb{}}\PYG{p}{\PYGZcb{}} \PYGZbs{}\PYG{n}{frac}\PYG{p}{\PYGZob{}}\PYG{n}{h\PYGZus{}}\PYG{p}{\PYGZob{}}\PYG{n}{Tank}\PYG{p}{\PYGZcb{}}\PYG{p}{\PYGZcb{}}\PYG{p}{\PYGZob{}}\PYG{n}{h\PYGZus{}0}\PYG{p}{\PYGZcb{}}
\end{sphinxVerbatim}

This equation describes flow \(Q\) as a function of time \(t\) of a fluid leaving a tank through a valve. Attempting to get this ‘tank with a valve’ system to yield constant head means raising the tank far, far above the valve that controls the flow. This is unreasonable when designing a flow control system for constant dosing, but can be used to design systems to drain a tank. See the section above for a description of the variables in the equation.
\begin{enumerate}
\setcounter{enumi}{1}
\item {} 
\sphinxstylestrong{LFOM:} The LFOM makes the water level in the entrance tank linear with respect to the flow out of the entrance tank. This is useful in measuring the flow and is a critical component in AguaClara’s chemical dosing system. The LFOM \sphinxstyleemphasis{measures} the flow through the plant, it does not \sphinxstyleemphasis{control} the flow through the plant.

\item {} 
\sphinxstylestrong{The Linear Chemical Dose Controller (CDC)} combines the:
* linear relationship between water level and flow in the entrance tank caused by the LFOM,
* linear relationship between elevation difference and flow caused by the Hagen-Poiseuille equation, which is only valid for major losses under laminar flow, and
* a lever to link the two linear relationships

\end{enumerate}

To keep the chemical dose constant by automatically adjusting the addition of coagulant and chlorine as the plant flow rate varies. Two sliders on the lever allows the operator to change the dose of coagulant and chlorine independently of the plant flow rate.


\chapter{Flow Control and Measurement Derivations}
\label{\detokenize{Flow_Control_and_Measurement/FCM_Derivations:flow-control-and-measurement-derivations}}\label{\detokenize{Flow_Control_and_Measurement/FCM_Derivations:title-flow-control-derivations}}\label{\detokenize{Flow_Control_and_Measurement/FCM_Derivations::doc}}

\section{\protect\(Q(t)\protect\) for a Tank with a Valve}
\label{\detokenize{Flow_Control_and_Measurement/FCM_Derivations:for-a-tank-with-a-valve}}\label{\detokenize{Flow_Control_and_Measurement/FCM_Derivations:heading-flow-for-a-tank-with-a-valve}}
This document contains the derivation of the flow through a tank-with-a-valve over time, \(Q(t)\). Our reference will be a simple hypochlorinator, shown in the following image. In the image, a hypochlorite solution is slowly dripping and mixing with piped source water, thereby disinfecting it. The valve is almost closed to make sure that the hypochlorite solution drips instead of flows. At the end of this document is an image which shows the variables in the final equation.

\begin{figure}[htbp]
\centering
\capstart

\noindent\sphinxincludegraphics[width=600\sphinxpxdimen]{{drip_hypochlorinator}.png}
\caption{This is a common setup for chlorinating water before distributing it to a nearby community.}\label{\detokenize{Flow_Control_and_Measurement/FCM_Derivations:id1}}\label{\detokenize{Flow_Control_and_Measurement/FCM_Derivations:figure-drip-hypochlorinator}}\end{figure}

This derivation begins by finding two equations for flow, \(Q\), through the hypochlorinator and setting them equal to each other. First, the rate of change of the volume of hypochlorite solution in the tank is equivalent to the flow out of the hypochlorinator. Since the volume of hypochlorite solution in the tank is equal to the tank’s cross-sectional area times it height, we get the following equation:
\begin{equation}\label{equation:Flow_Control_and_Measurement/FCM_Derivations:Flow_Control_and_Measurement/FCM_Derivations:0}
\begin{split}Q =  - \frac{d\rlap{-}V}{dt} = - \frac{{A_{Tank}}dh}{dt}\end{split}
\end{equation}
\begin{DUlineblock}{0em}
\item[] Such that:
\item[] \(\frac{d\rlap{-}V}{dt}\) = rate of change in volume of solution in the tank
\item[] \(\frac{dh}{dt}\) = rate of change in height of water (hypochlorite solution) level with time
\end{DUlineblock}

Our other equation for flow is the head loss equation. Since major losses are negligible for a short pipe-low flow rate system, we only need to consider minor losses. The only real minor loss in this system occurs in the almost-closed valve that is dripping the hypochlorite solution. However, we will also use the head loss trick. Therefore, the total driving head of the system \(h\) is equal to the minor losses:
\begin{equation}\label{equation:Flow_Control_and_Measurement/FCM_Derivations:Flow_Control_and_Measurement/FCM_Derivations:1}
\begin{split}h = h_e = \left( \sum K \right) \frac{Q^2}{2gA_{Valve}^2}\end{split}
\end{equation}
Bear in mind that this is the second form of the minor loss equation as described in {\hyperref[\detokenize{Review/Review_Fluid_Mechanics_Derivations:heading-final-minor-loss-equations}]{\sphinxcrossref{\DUrole{std,std-ref}{this derivation}}}}. Rearranging the minor loss equation to solve for \(Q\), it looks like this:
\begin{equation}\label{equation:Flow_Control_and_Measurement/FCM_Derivations:Flow_Control_and_Measurement/FCM_Derivations:2}
\begin{split}Q = A_{Valve} \sqrt{\frac{2 h_e g}{\sum K}}\end{split}
\end{equation}
Now we can set both equations for \(Q\) equal to each other and move them both to one side:
\begin{equation}\label{equation:Flow_Control_and_Measurement/FCM_Derivations:Flow_Control_and_Measurement/FCM_Derivations:3}
\begin{split}A_{Tank} \frac{dh}{dt} + A_{Valve} \sqrt{\frac{2gh}{\sum K}} = 0\end{split}
\end{equation}
From here, calculus and equation substitution dominate the derivation. Separating the variables of the equation immediately above, we get the following integral:
\begin{equation}\label{equation:Flow_Control_and_Measurement/FCM_Derivations:Flow_Control_and_Measurement/FCM_Derivations:4}
\begin{split}\frac{ -A_{Tank}}{{A_{Valve}} \sqrt{\frac{2g}{\sum K}} }   \int \limits_{h_0}^h \frac{dh}{\sqrt h} = \int \limits_0^t {dt}\end{split}
\end{equation}
Which, when integrated, yields:
\begin{equation}\label{equation:Flow_Control_and_Measurement/FCM_Derivations:Flow_Control_and_Measurement/FCM_Derivations:5}
\begin{split}\frac{ -A_{Tank}}{A_{Valve} \sqrt{ \frac{2g}{\sum K}} } \cdot 2 \left( \sqrt{h} - \sqrt{h_0} \right) = t\end{split}
\end{equation}
And solved for \(\sqrt{h}\) returns:
\begin{equation}\label{equation:Flow_Control_and_Measurement/FCM_Derivations:Flow_Control_and_Measurement/FCM_Derivations:6}
\begin{split}\sqrt h  = \sqrt{h_0} - t \frac{A_{Valve}}{2 A_{tank}} \sqrt {\frac{2g}{\sum K}}\end{split}
\end{equation}
At this point, the steps and equation substitutions may begin to seem unintuitive. Do not worry if you do not understand why \sphinxstyleemphasis{exactly} a particular substitution is occurring. Since we determined above that \(h_e = h\), our equation above for \(\sqrt{h}\) is also an equation for \(\sqrt{h_e}\). As such, we will plug the equation above back into the minor loss equation solved for \(Q\) from above, \(Q = A_{Valve} \sqrt{\frac{2 h_e g}{\sum K}}\), to produce:
\begin{equation}\label{equation:Flow_Control_and_Measurement/FCM_Derivations:Flow_Control_and_Measurement/FCM_Derivations:7}
\begin{split}Q = A_{Valve} \sqrt{\frac{2g}{\sum K}} \left( \sqrt{h_0}  - t \frac{A_{Valve}}{2 A_{tank}} \sqrt{\frac{2g}{\sum K}} \right)\end{split}
\end{equation}
Now we can focus on getting rid of the variables \(A_{Valve}\), \(\sum K\), and \(A_{tank}\). By using the minor loss equation once more, we can remove both \(A_{Valve}\) and \(\sum K\). Consider the initial state of the system, when the hypochlorinator is set up and starts dropping its first few drops of hypochlorite solution into the water. The initial flow rate, \(Q_0\), and elevation difference between the water level and the valve, \(h_0\), can be input into the minor loss equation, which can then be solved for \(A_{Valve}\):
\begin{equation}\label{equation:Flow_Control_and_Measurement/FCM_Derivations:Flow_Control_and_Measurement/FCM_Derivations:8}
\begin{split}A_{Valve} = \frac{Q_{0}}{ \sqrt{ \frac{2 h_0 g}{\sum K}} }\end{split}
\end{equation}
Plugging this equation for \(A_{Valve}\) into the equation for \(Q\) just above, we get the following two equations, in which the second equation is a simplified version of the first:
\begin{equation}\label{equation:Flow_Control_and_Measurement/FCM_Derivations:Flow_Control_and_Measurement/FCM_Derivations:9}
\begin{split}Q = Q_0 \frac{1}{\sqrt{h_0}} \left( \sqrt{h_0} - \frac{Q_0 t}{2 A_{Tank} \sqrt{h_0}} \right)\end{split}
\end{equation}\begin{equation}\label{equation:Flow_Control_and_Measurement/FCM_Derivations:Flow_Control_and_Measurement/FCM_Derivations:10}
\begin{split}\frac{Q}{Q_0} = 1 - \frac{t Q_0}{2 A_{Tank} h_0}\end{split}
\end{equation}
This next step will eliminate \(A_{Tank}\). However, it requires some clever manipulation that has a tendency to cause some confusion. We will define a new parameter, \(t_{Design}\), which represents the time it would take to empty the tank \sphinxstylestrong{if the initial flow rate through the valve, :math:{}`Q\_0{}`, stays constant in time}. Of course, the flow \(Q\) through the valve does not stay constant in time, which is why this derivation document exists. But imagining this hypothetical \(t_{Design}\) parameter allows us to form the following equation:
\begin{equation}\label{equation:Flow_Control_and_Measurement/FCM_Derivations:Flow_Control_and_Measurement/FCM_Derivations:11}
\begin{split}Q_0 t_{Design} = A_{Tank} h_{Tank}\end{split}
\end{equation}
This equation describes draining all the hypochlorite solution from the tank. The volume of the solution, \(A_{Tank} h_{Tank}\), is drained in \(t_{Design}\). Rearranged, the equation becomes:
\begin{equation}\label{equation:Flow_Control_and_Measurement/FCM_Derivations:Flow_Control_and_Measurement/FCM_Derivations:12}
\begin{split}\frac{Q_0}{A_{Tank}} = \frac{h_{Tank}}{t_{Design}}\end{split}
\end{equation}
\begin{DUlineblock}{0em}
\item[] Such that:
\item[] \(h_{Tank}\) = elevation of water level in the tank with reference to tank bottom at the initial state, \(t = 0\)
\end{DUlineblock}

Here lies another common source of confusion. \(h_{Tank}\) is not the same as \(h_{0}\). \(h_{Tank}\) is the height of water level in the tank with reference to the tank bottom. \(h_{0}\) is the water level in the tank with reference to the valve. Therefore, \(h_{0} \geq h_{Tank}\) is true if the valve is located at or below the bottom of the tank. If the tank is elevated far above the valve, then the \(h_{0} > > h_{Tank}\). If the valve is at the same elevation as the bottom of the tank, then \(h_{0} = h_{Tank}\). Please refer to the following image to clarify \(h_{0}\) and \(h_{Tank}\). Also note that both \(h_{Tank}\) and \(h_{0}\) are not variables, they are constants which are defined by the initial state of the hypochlorinator, when the solution just begins to flow.

\begin{figure}[htbp]
\centering
\capstart

\noindent\sphinxincludegraphics[width=600\sphinxpxdimen]{{hypochlorinator_variable_explanation}.png}
\caption{\(Q_0 =\) initial flow rate of hypochlorite solution at time \(t = 0\), \(t_{Design} =\) time it would take to drain the tank if flow was held constant at \(Q_0\)}\label{\detokenize{Flow_Control_and_Measurement/FCM_Derivations:id2}}\label{\detokenize{Flow_Control_and_Measurement/FCM_Derivations:figure-hypochlorinator-variable-explanation}}\end{figure}

Finally, our fabricated equivalence, \(\frac{Q_0}{A_{Tank}} = \frac{h_{Tank}}{t_{Design}}\) can be plugged into \(\frac{Q}{Q_0} = 1 - \frac{t Q_0}{2 A_{Tank} h_0}\) to create the highly useful equation for flow rate as a function of time for a drip hypochlorinator:
\begin{equation}\label{equation:Flow_Control_and_Measurement/FCM_Derivations:Flow_Control_and_Measurement/FCM_Derivations:13}
\begin{split}\color{purple}{
\frac{Q}{Q_0} = 1 - \frac{1}{2} \frac{t}{t_{Design}} \frac{h_{Tank}}{h_0}
}\end{split}
\end{equation}
Which can be slightly rearranged to yield:
\begin{equation}\label{equation:Flow_Control_and_Measurement/FCM_Derivations:Flow_Control_and_Measurement/FCM_Derivations:14}
\begin{split}\color{purple}{
Q(t) = Q_0 \left( 1 - \frac{1}{2} \frac{t}{t_{Design}} \frac{h_{Tank}}{h_0} \right)
}\end{split}
\end{equation}
\begin{DUlineblock}{0em}
\item[] Such that:
\item[] \(Q = Q(t)\) = flow of hypochlorite through valve at time \(t\)
\item[] \(t\) = elapsed time
\item[] \(t_{Design}\) = time it would take for tank to empty \sphinxstyleemphasis{if} flow stayed constant at \(Q_0\), which it does not
\item[] \(h_{Tank}\) = elevation of water level with reference to tank bottom
\item[] \(h_0\) = elevation of water level with reference to the valve
\end{DUlineblock}

“How does this ‘tank with a valve’ scenario differ from the ‘hole in a bucket’ scenario?” some might ask. If you are interested, you may go through the derivation on your own using the orifice equation instead of the minor loss equation for the first step. If you do so you’ll find that the equation remains almost the same, the only difference being that the \(\frac{h_{Tank}}{h_0}\) term drops out for an orifice, as \(h_{Tank} = h_0\). The big difference in the systems lies with the flexibility of having a valve. It can be tightened or loosened to change the flow rate, whereas changing the size of an orifice multiple times in a row is not recommended and is usually irreversible.


\section{\protect\(D(t)\protect\) and \protect\(t(D)\protect\) for Tank Drain Equation}
\label{\detokenize{Flow_Control_and_Measurement/FCM_Derivations:and-for-tank-drain-equation}}\label{\detokenize{Flow_Control_and_Measurement/FCM_Derivations:heading-diameter-and-time-tank-drain-equation}}
This document contains the derivation of \(D_{Pipe}\), which is the pipe diameter necessary to install in a drain system to entirely drain a tank in time \(t_{Drain}\).

First, it is necessary to understand how AguaClara tank drains work and what they look like. Many tanks, including the flocculator and entrance tank, have a hole in their bottoms which are fitted with \sphinxhref{https://www.mrpoolman.com.au/assets/thumbL/16057.jpg}{pipe couplings}. During normal operation, these couplings have pipe stubs in them, and the pipe stubs are tall enough to go above the water level in the tank and not allow water to flow into the drain. When the pipe stub is removed, the water begins to flow out of the drain, as the image below indicates. The drain pipe consists of pipe and one elbow, shown in the image.

\begin{figure}[htbp]
\centering
\capstart

\noindent\sphinxincludegraphics[width=600\sphinxpxdimen]{{pipe_stub_drainage}.png}
\caption{This is AguaClara’s alternatives to having valves.}\label{\detokenize{Flow_Control_and_Measurement/FCM_Derivations:id3}}\label{\detokenize{Flow_Control_and_Measurement/FCM_Derivations:figure-pipe-stub-drainage}}\end{figure}

While AguaClara sedimentation tanks use valves instead of pipe to begin the process of draining, the actual drain piping system is the same, pipe and an elbow. The equation that will soon be derived applies to both pipe stub and valve drains.

We will start the derivation from the following equation, which is found in an intermediate step from the ‘\(Q(t)\) {\hyperref[\detokenize{Flow_Control_and_Measurement/FCM_Derivations:heading-flow-for-a-tank-with-a-valve}]{\sphinxcrossref{\DUrole{std,std-ref}{ for a Tank with a Valve}}}}.’ While this system does not have a valve, it has other sources of minor loss and therefore the equation is still valid.
\begin{equation}\label{equation:Flow_Control_and_Measurement/FCM_Derivations:Flow_Control_and_Measurement/FCM_Derivations:15}
\begin{split}\sqrt h  = \sqrt{h_0} - t \frac{A_{Valve}}{2 A_{Tank}} \sqrt {\frac{2g}{K}}\end{split}
\end{equation}
We need to make some adjustments to this equation before proceeding, to make it applicable for this new drain-system scenario. First, we want to assume that the tank has fully drained. Thus, \(t = t_{Drain}\) and \(h = 0\). Next, we recall that the tank drain is not actually a valve, but just pipe and an elbow, so \(A_{Valve} = A_{Pipe}\). Additionally, there can be multiple points of minor loss in the drain system: the entrance from the tank into the drain pipe, the elbow, and potentially the exit of the water out of the drain pipe. When considering a sedimentation tank, the open valve required to begin drainage also has a minor loss associated with it. Therefore, it is necessary to substitute \(\sum K\) for \(K\) With these substitutions, the equation becomes:
\begin{equation}\label{equation:Flow_Control_and_Measurement/FCM_Derivations:Flow_Control_and_Measurement/FCM_Derivations:16}
\begin{split}\sqrt{h_0}  = t_{Drain} \frac{A_{Pipe}}{2 A_{Tank}} \sqrt {\frac{2g}{\sum K}}\end{split}
\end{equation}
Now, with the knowledge that \(A_{Pipe} = \frac{\pi D_{Pipe}^2}{4}\) and rearranging to solve for \(D_{Pipe}\), we obtain the following equation:
\begin{equation}\label{equation:Flow_Control_and_Measurement/FCM_Derivations:Flow_Control_and_Measurement/FCM_Derivations:17}
\begin{split}D_{Pipe} = \sqrt{ \frac{8 A_{Tank}}{\pi t_{Drain}} \sqrt{ \frac{h_0 \sum K}{2g} } }\end{split}
\end{equation}
To get the equation in terms of easily measureable tank parameters, we substitute \(L_{Tank} W_{Tank}\) for \(A_{Tank}\). To maintain consistency in variable names, we substitute \(H_{Tank}\) for \(h_0\).

\begin{sphinxadmonition}{note}{Note:}
By saying that \(h_0 = H_{Tank}\), we are making the assumption that the pipe drain is at the same elevation as the bottom of the tank. The pipe drain is actually a little lower than the bottom of the tank, but that would make the tank drain faster than \(t_{Drain}\), which is preferred. Therefore, we are designing a slight safety factor when we say that \(h_0 = H_{Tank}\).
\end{sphinxadmonition}

Finally, we arrive at the equation for drain pipe sizing:
\begin{equation}\label{equation:Flow_Control_and_Measurement/FCM_Derivations:Flow_Control_and_Measurement/FCM_Derivations:18}
\begin{split}\color{purple}{
D_{Pipe} = \sqrt{ \frac{8 L_{Tank} W_{Tank}}{\pi t_{Drain}}} \left( \frac{H_{Tank} \sum K}{2g} \right)^{\frac{1}{4}}
}\end{split}
\end{equation}
We can also easily rearrange to find the time required to drain a tank given a drain diameter:
\begin{equation}\label{equation:Flow_Control_and_Measurement/FCM_Derivations:Flow_Control_and_Measurement/FCM_Derivations:19}
\begin{split}\color{purple}{
t_{Drain} = \frac{8 L_{Tank} W_{Tank}}{\pi D_{Pipe}^2} \sqrt{ \frac{H_{Tank} \sum K}{2g} }
}\end{split}
\end{equation}
Such that the variables are as the appear in the image below.

\begin{figure}[htbp]
\centering
\capstart

\noindent\sphinxincludegraphics[width=600\sphinxpxdimen]{{pipe_stub_drainage_variables}.png}
\caption{\(L_{Tank}\) is the length of the tank which goes the page. \(K\) is the aggregate minor loss coefficient of the drain system.}\label{\detokenize{Flow_Control_and_Measurement/FCM_Derivations:id4}}\label{\detokenize{Flow_Control_and_Measurement/FCM_Derivations:figure-pipe-stub-drainage-variables}}\end{figure}


\section{Design Equations for the Linear Chemical Dose Controller (CDC)}
\label{\detokenize{Flow_Control_and_Measurement/FCM_Derivations:design-equations-for-the-linear-chemical-dose-controller-cdc}}\label{\detokenize{Flow_Control_and_Measurement/FCM_Derivations:heading-design-equations-for-the-cdc}}
This document will include the equation derivations required to design a CDC system. The most important restriction in this design process is maintaining linearity between head \(h\) and flow \(Q\), which is the entire purpose of the CDC. Recall that major losses under laminar flow scale with \(Q\) and minor losses scale with \(Q^2\) Since it is impossible to remove minor losses from the system entirely, we will simply try to make minor losses very small compared to major losses. The CDC does this by including ‘dosing tube(s),’ which are long, straight tubes designed to generate a lot of major losses. There can be one tube or multiple, depending on the design conditions.

We will use the ‘head loss trick’ that was introduced in the Fluids Review section. Therefore, the elevation difference between the water level in the constant head tank (CHT) and the end of the tube connected to the slider, \(\Delta h\), is equal to the head loss between the two points, \(h_L\). Thus, \(\Delta h = h_L = h_e + h_f\).

\begin{sphinxadmonition}{note}{Note:}
There are a lot of equations in this section, and they may quickly get confusing. They are color coded in an attempt to make them easier to follow. There are two final design equations: \(\color{purple}{\bar v_{Max}}\) and math:\sphinxtitleref{color\{purple\}\{L\_\{Min\}\}}, and they will be written in \(\color{purple}{\rm{purple \, text \, coloring}}\) to make them noticeable.
\end{sphinxadmonition}

\begin{figure}[htbp]
\centering
\capstart

\noindent\sphinxincludegraphics[width=600\sphinxpxdimen]{{CDC_derivation}.png}
\caption{Visual representation of CDC.}\label{\detokenize{Flow_Control_and_Measurement/FCM_Derivations:id5}}\label{\detokenize{Flow_Control_and_Measurement/FCM_Derivations:figure-cdc-derivation}}\end{figure}


\subsection{CDC Design Equation Derivation}
\label{\detokenize{Flow_Control_and_Measurement/FCM_Derivations:cdc-design-equation-derivation}}\label{\detokenize{Flow_Control_and_Measurement/FCM_Derivations:heading-cdc-design-equation-derivations}}
\begin{sphinxadmonition}{important}{Important:}
\sphinxstylestrong{When designing the CDC, there are a few parameters which are picked and set initially, before applying any equations. These parameters are:}
\end{sphinxadmonition}
\begin{enumerate}
\item {} 
\(D\) = tube diameter. only certain tubing diameters are manufactured (like \(\frac{x}{16}\) inch), so an array of available tube diameters is set initially.

\item {} 
\(\sum K\) = sum of minor loss coefficients for the whole system. This is also set initially, it is usually 2.

\item {} 
\(h_{L_{Max}}\) = maximum elevation difference between CHT water level and outlet of solution. This parameter is usually 20 cm.

\end{enumerate}

We begin by defining the head loss through the system \(h_L\), which is equivalent to defining the driving head \(\Delta h\). Major losses will be coded as red.
\begin{equation}\label{equation:Flow_Control_and_Measurement/FCM_Derivations:Flow_Control_and_Measurement/FCM_Derivations:20}
\begin{split}\color{red}{
  h_{\rm{f}} = \frac{128\nu LQ}{g\pi D^4}
  }\end{split}
\end{equation}
\begin{DUlineblock}{0em}
\item[] Such that:
\item[] \(\nu\) = kinematic viscosity \sphinxstyleemphasis{of the solution going through the dosing tube(s)}. This is either coagulant or chlorine
\item[] \(Q\) = flow rate through the dosing tube(s)
\item[] \(L\) = length of the dosing tube(s)
\end{DUlineblock}

\begin{sphinxadmonition}{note}{Note:}
‘Tube(s)’ is used because there may be 1 or more dosing tubes depending on the particular design.
\end{sphinxadmonition}

Minor losses are equal to:
\begin{equation}\label{equation:Flow_Control_and_Measurement/FCM_Derivations:Flow_Control_and_Measurement/FCM_Derivations:21}
\begin{split}h_e = \frac{8 Q^2}{g \pi^2 D^4} \sum{K}\end{split}
\end{equation}
Therefore, the total head loss is a function of flow, and is shown in the following equation.
\begin{equation}\label{equation:Flow_Control_and_Measurement/FCM_Derivations:Flow_Control_and_Measurement/FCM_Derivations:22}
\begin{split}h_L(Q) =
{\color{red}{
  \frac{128\nu L Q}{g \pi D^4}}} +
  \frac{8Q^2}{g \pi^2 D^4} \sum K\end{split}
\end{equation}
Blue will be used to reference \sphinxstyleemphasis{actual} head loss from now on. This is the same equation as above.
\begin{equation}\label{equation:Flow_Control_and_Measurement/FCM_Derivations:Flow_Control_and_Measurement/FCM_Derivations:23}
\begin{split}\color{blue}{
  h_L(Q) = \left( \frac{128\nu L}{g \pi D^4} + \frac{8Q}{g \pi ^2 D^4} \sum{K} \right) Q
  }\end{split}
\end{equation}
This equation is not linear with respect to flow. We can make it linear by turning the variable \(Q\) in the \(\frac{8Q}{g \pi ^2 D^4} \sum{K}\) term into a constant. To do this, we pick a maximum flow rate of coagulant/chlorine through the dose controller, \(Q_{Max}\), and put that into the term in place of \(Q\). The term becomes \(\frac{8Q_{Max}}{g \pi ^2 D^4} \sum{K}\), and our linearized model of head loss, coded as green, becomes:
\begin{equation}\label{equation:Flow_Control_and_Measurement/FCM_Derivations:Flow_Control_and_Measurement/FCM_Derivations:24}
\begin{split}\color{green}{
  h_{L_{linear}}(Q) = \left( \frac{128\nu L}{g \pi D^4} + \frac{8Q_{Max}}{g \pi ^2 D^4} \sum{K} \right) Q
  }\end{split}
\end{equation}
Here is a plot of the three colored equations above. Our goal is to minimize the minor losses in the system; to bring the red and blue curves as close as possible to the green one.

\begin{figure}[htbp]
\centering
\capstart

\noindent\sphinxincludegraphics[width=600\sphinxpxdimen]{{CDC_linearity_model}.png}
\caption{MathCAD generated graph for linearity error analysis. TODO: make this in python}\label{\detokenize{Flow_Control_and_Measurement/FCM_Derivations:id6}}\label{\detokenize{Flow_Control_and_Measurement/FCM_Derivations:figure-cdc-linearity-model}}\end{figure}


\subsubsection{Designing for the error constraint, \protect\(\Pi_{Error}\protect\)}
\label{\detokenize{Flow_Control_and_Measurement/FCM_Derivations:designing-for-the-error-constraint}}
\begin{sphinxadmonition}{important}{Important:}
The first step in the design is to make sure that major losses far exceed minor losses. This will result in an equation for the maximum velocity that can go through the dosing tube(s), \(\color{purple}{\bar v_{Max} }\).
\end{sphinxadmonition}

Minor losses will never be 0, so how much error in our linearity are we willing to accept? Let’s define a new parameter, \(\Pi_{Error}\), as the maximum amount of error we are willing to accept. We are ok with 10\% error or less, so \(\Pi_{Error} = 0.1\).
\begin{equation}\label{equation:Flow_Control_and_Measurement/FCM_Derivations:Flow_Control_and_Measurement/FCM_Derivations:25}
\begin{split}\Pi_{Error} = \frac{\color{green}{ h_{L_{linear}} } - \color{blue}{ h_L }}{\color{green}{ h_{L_{linear}} }} = 1 - \frac{\color{blue}{ h_L }}{\color{green}{ h_{L_{linear}} }}\end{split}
\end{equation}\begin{equation}\label{equation:Flow_Control_and_Measurement/FCM_Derivations:Flow_Control_and_Measurement/FCM_Derivations:26}
\begin{split}1 - \Pi_{Error} = \frac{\color{blue}{ h_L }}{\color{green}{ h_{L_{linear}} }}\end{split}
\end{equation}
Now we plug \(\color{blue}{ h_L(Q) }\) and \(\color{green}{ h_{L_{linear}} }\) back into the equation for \(1 - \Pi_{Error}\) and take the limit as \(Q \rightarrow 0\), as that is when the relative difference between actual head loss and our linear model for head loss is the greatest.
\begin{equation}\label{equation:Flow_Control_and_Measurement/FCM_Derivations:Flow_Control_and_Measurement/FCM_Derivations:27}
\begin{split}1 - \Pi_{Error} =
  \frac{ \color{blue}{
  \left( \frac{128 \nu L}{g \pi D^4} +
  \cancel{\frac{8Q}{g \pi^2 D^4} \sum{K}}
  \right) Q
  }}
  {\color{green}{
  \left( \frac{128 \nu L}{g \pi D^4} + \frac{8 Q_{Max}}{g \pi^2 D^4} \sum{K} \right) Q
  }}
  =     \frac{\left( \frac{128 \nu L}{g \pi D^4} \right)}{\left( \frac{128 \nu L}{g \pi D^4} + \frac{8 Q_{Max}}{g \pi^2 D^4} \sum{K} \right)}\end{split}
\end{equation}
The next steps are algebraic rearrangements to solve for \(L\). This \(L\) describes the \sphinxstyleemphasis{minimum} length of dosing tube necessary to meet our error constraint at \sphinxstyleemphasis{maximum} flow. Thus, we will refer to it as \(L_{Min, \, \Pi_{Error}}\).
\begin{equation}\label{equation:Flow_Control_and_Measurement/FCM_Derivations:Flow_Control_and_Measurement/FCM_Derivations:28}
\begin{split}\left( 1 - \Pi_{Error} \right)  \frac{128 \nu L}{g \pi D^4} + \left( 1 - \Pi_{Error} \right) \frac{8 Q_{Max}}{g \pi ^2 D^4} \sum{K}  =  \frac{128 \nu L}{g \pi D^4}\end{split}
\end{equation}\begin{equation}\label{equation:Flow_Control_and_Measurement/FCM_Derivations:Flow_Control_and_Measurement/FCM_Derivations:29}
\begin{split}- \Pi_{Error} \frac{128 \nu L}{g \pi D^4} + \left( 1 - \Pi_{Error} \right) \frac{8 Q_{Max}}{g \pi^2 D^4} \sum{K}  = 0\end{split}
\end{equation}\begin{equation}\label{equation:Flow_Control_and_Measurement/FCM_Derivations:Flow_Control_and_Measurement/FCM_Derivations:30}
\begin{split}L = \left( \frac{1 - \Pi_{Error}}{\Pi_{Error}} \right) \frac{Q_{Max}}{16 \nu \pi} \sum{K}\end{split}
\end{equation}\begin{equation}\label{equation:Flow_Control_and_Measurement/FCM_Derivations:Flow_Control_and_Measurement/FCM_Derivations:31}
\begin{split}L_{Min, \, \Pi_{Error}} = L = \left( \frac{1 - \Pi_{Error}}{\Pi_{Error}} \right) \frac{Q_{Max}}{16 \nu \pi} \sum{K}\end{split}
\end{equation}
\begin{DUlineblock}{0em}
\item[] Note that this equation is independent of head loss.
\end{DUlineblock}

Unfortunately, both \(L_{Min, \, \Pi_{Error}}\) and \(Q_{Max}\) are unknowns. We can plug this equation for \(L_{Min, \, \Pi_{Error}}\) back into the head loss equation at maximum flow, which is \(h_{L_{Max}} = \left( \frac{128\nu L Q_{Max}}{g \pi D^4} + \frac{8Q_{Max}^2}{g \pi ^2 D^4} \sum{K} \right)\) and rearrange for \(Q_{Max}\) to get:
\begin{equation}\label{equation:Flow_Control_and_Measurement/FCM_Derivations:Flow_Control_and_Measurement/FCM_Derivations:32}
\begin{split}Q_{Max} = \frac{\pi D^2}{4} \sqrt{\frac{2 h_{L_{Max}} g \Pi_{Error}}{\sum K }}\end{split}
\end{equation}

\sphinxstrong{See also:}


\sphinxstylestrong{Function in aide\_design} \sphinxcode{\sphinxupquote{cdc.max\_linear\_flow(Diam, HeadlossCDC, Ratio\_Error, KMinor)}} Returns the maximum flow \(Q_{Max}\) that can go through a dosing tube will making sure that linearity between head loss and flow is conserved.



From this equation for \(Q_{Max}\), we can get to our first design equation, \(\color{purple}{\bar v_{Max}}\) by using the continuity equation \(\bar v_{Max} = \frac{Q_{Max}}{\frac{\pi D^2}{4}}\)
\begin{equation}\label{equation:Flow_Control_and_Measurement/FCM_Derivations:Flow_Control_and_Measurement/FCM_Derivations:33}
\begin{split}\color{purple}{
  \bar v_{Max} = \sqrt{ \frac{2 h_L g \Pi_{Error}}{\sum{K} }}
  }\end{split}
\end{equation}

\subsubsection{Designing for the proper amount of head loss, \protect\(h_{L_{Max}}\protect\)}
\label{\detokenize{Flow_Control_and_Measurement/FCM_Derivations:designing-for-the-proper-amount-of-head-loss}}
\begin{sphinxadmonition}{important}{Important:}
The second step in the design is to make sure that the maximum head loss corresponds to the maximum flow of chemicals. This will result in an equation for the length of the dosing tube(s), \(\color{purple}{L_{Min} }\).
\end{sphinxadmonition}

We previously derived an equation for the minimum length of the dosing tube(s), \(L_{Min, \, \Pi_{Error}}\), which was the minimum length needed to ensure that our linearity constraint was met. This equation is shown again below, in red:
\begin{equation}\label{equation:Flow_Control_and_Measurement/FCM_Derivations:Flow_Control_and_Measurement/FCM_Derivations:34}
\begin{split}\color{red}{
  L_{Min, \, \Pi_{Error}} = \left( \frac{1 - \Pi_{Error}}{\Pi_{Error}} \right) \frac{Q_{Max}}{16 \nu \pi} \sum{K}
  }\end{split}
\end{equation}
This equation does not, however, account for getting to the proper amount of head loss. If we were to use this equation to design the dosing tubes, we might not end up with the proper amount of flow \(Q_{Max}\) at the maximum head loss \(h_{L{Max}}\). So we need to double check to make sure that we get our desired head loss.

First, consider the head loss at maximum flow that was used to get the equation for \(Q_{Max}\):
\begin{equation}\label{equation:Flow_Control_and_Measurement/FCM_Derivations:Flow_Control_and_Measurement/FCM_Derivations:35}
\begin{split}h_{L_{Max}} = \left( \frac{128 \nu L{Q_{Max}}}{g \pi D^4} + \frac{8 Q_{Max}^2}{g \pi^2 D^4} \sum{K} \right)\end{split}
\end{equation}
Now that we know all of the parameters in this equation except for \(L\), we can solve the equation for \(L\). This the \sphinxstyleemphasis{shortest} tube that generates our required head loss, \(h_{L_{Max}}\).
\begin{equation}\label{equation:Flow_Control_and_Measurement/FCM_Derivations:Flow_Control_and_Measurement/FCM_Derivations:36}
\begin{split}\color{green}{
   L_{Min, \, head loss} = L = \left( \frac{g h_{L_{Max}} \pi D^4}{128 \nu Q_{Max}} - \frac{Q_{Max}}{16 \pi \nu} \sum{K} \right)
   }\end{split}
\end{equation}

\sphinxstrong{See also:}


\sphinxstylestrong{Function in aide\_design:} \sphinxcode{\sphinxupquote{cdc.\_length\_cdc\_tube\_array(FlowPlant, ConcDoseMax, ConcStock, DiamTubeAvail, HeadlossCDC, temp, en\_chem, KMinor)}} Returns \(\color{purple}{L_{Min}}\), takes in the flow rate input of \sphinxstyleemphasis{plant design flow rate}.




\sphinxstrong{See also:}


\sphinxstylestrong{Function in aide\_design:} \sphinxcode{\sphinxupquote{cdc.\_len\_tube(Flow, Diam, HeadLoss, conc\_chem, temp, en\_chem, KMinor)}} Returns \(\color{purple}{L_{Min}}\), takes in the flow rate input of \sphinxstyleemphasis{max flow rate through the dosing tube(s)}.



If you decrease the max flow \(Q_{Max}\) and hold \(h_{L_{Max}}\) constant, \(\color{green}{L_{Min, \, head loss}}\) becomes larger. This means that a CDC system for a plant of 40 \(\frac{L}{s}\) must be different than one for a plant of 20 \(\frac{L}{s}\). If we want to maintain the same head loss at maximum flow in both plants, then the dosing tube(s) will need to be a lot longer for the 20 \(\frac{L}{s}\) plant.

To visualize the distinction between \(\color{red}{  L_{Min, \, \Pi_{Error}}}\) and math:\sphinxtitleref{color\{green\}\{ L\_\{Min, , head loss\}\}}, see the following plot. \(\color{green}{ L_{Min, \, head loss}}\) is discontinuous because it takes in the smallest allowable tube diameter as an input. As the chemical flow rate through the dosing tube(s) decreases, the dosing tube diameter does as well. Whenever you see a jump in the green points, that means the tubing diameter has changed.

\begin{figure}[htbp]
\centering
\capstart

\noindent\sphinxincludegraphics[width=600\sphinxpxdimen]{{CDC_length_model}.png}
\caption{CDC length modeling in MathCAD.}\label{\detokenize{Flow_Control_and_Measurement/FCM_Derivations:id7}}\label{\detokenize{Flow_Control_and_Measurement/FCM_Derivations:figure-cdc-length-model}}\end{figure}

As you can see, the head loss constraint is more limiting than the linearity constraint when designing for tube length. Therefore, the design equation for tube length is the one which accounts for head loss. This is the second and final design equation for designing the CDC:
\begin{equation}\label{equation:Flow_Control_and_Measurement/FCM_Derivations:Flow_Control_and_Measurement/FCM_Derivations:37}
\begin{split}\color{purple}{
L_{Min} = L_{Min, \, head loss} = \left( \frac{g h_{L_{Max}} \pi D^4}{128 \nu Q_{Max}} - \frac{Q_{Max}}{16 \pi \nu} \sum{K} \right)
}\end{split}
\end{equation}
The equations for \(\color{purple}{\bar v_{Max}}\) and \(\color{purple}{L_{Min}}\) are the only ones you \sphinxstylestrong{need} to manually design a CDC.


\subsubsection{CDC Dosing Tube(s) Diameter \protect\(D_{Min}\protect\) Plots}
\label{\detokenize{Flow_Control_and_Measurement/FCM_Derivations:cdc-dosing-tube-s-diameter-plots}}
Below are equations which also govern the CDC and greatly aid in understanding the physics behind it, but are not strictly necessary in design.

By rearranging \(Q_{Max} = \frac{\pi D^2}{4} \sqrt{\frac{2 h_L g \Pi_{Error}}{\sum K }}\), we can solve for \(D\) to get the \sphinxstyleemphasis{minimum} diameter we can use assuming the shortest tube possible that meets the error constraint, \(\color{red}{L_{Min, \, \Pi_{Error}}}\). If we use a diameter smaller than \(D_{Min, \, \Pi_{Error}}\), we will not be able to simultaneously reach \(Q_{Max}\) and meet the error constraint \(\Pi_{Error}\).
\begin{equation}\label{equation:Flow_Control_and_Measurement/FCM_Derivations:Flow_Control_and_Measurement/FCM_Derivations:38}
\begin{split}\color{blue}{
D_{Min, \, \Pi_{Error}} = \left[ \frac{8 Q_{Max}^2 \sum K}{\Pi_{Error} h_l g \pi^2} \right]^{\frac{1}{4}}
}\end{split}
\end{equation}
We can also find the minimum diameter needed to guarantee laminar flow, which is another critical condition in the CDC design. We can do this by combining the equation for Reynolds number at the maximum \(\rm{Re}\) for laminar flow, \({\rm{Re}}_{Max} = 2100\) with the continuity equation at maximum flow:
\begin{equation}\label{equation:Flow_Control_and_Measurement/FCM_Derivations:Flow_Control_and_Measurement/FCM_Derivations:39}
\begin{split}{\rm Re}_{Max}  = \frac{\bar v_{Max} D}{\nu}\end{split}
\end{equation}\begin{equation}\label{equation:Flow_Control_and_Measurement/FCM_Derivations:Flow_Control_and_Measurement/FCM_Derivations:40}
\begin{split}\bar v_{Max} = \frac{4 Q_{Max}}{\pi D^2}\end{split}
\end{equation}
To get:
\begin{equation}\label{equation:Flow_Control_and_Measurement/FCM_Derivations:Flow_Control_and_Measurement/FCM_Derivations:41}
\begin{split}\color{red}{
D_{Min, \, Laminar} = \frac{4 Q_{Max}}{\pi \nu {\rm{Re}}_{Max}}
}\end{split}
\end{equation}
Combined with the discrete amount of tubing sizes (shown in dark green), we can create a graph of the three diameter constraints:

\begin{figure}[htbp]
\centering
\capstart

\noindent\sphinxincludegraphics[width=600\sphinxpxdimen]{{CDC_diameter_model}.png}
\caption{CDC diameter modeling in MathCAD.}\label{\detokenize{Flow_Control_and_Measurement/FCM_Derivations:id8}}\label{\detokenize{Flow_Control_and_Measurement/FCM_Derivations:figure-cdc-diameter-model}}\end{figure}

\sphinxhref{https://github.com/AguaClara/Textbook/releases/latest}{PDF and LaTeX versions} %
\begin{footnote}[1]\sphinxAtStartFootnote
PDF and LaTeX versions may contain visual oddities because it is generated automatically. The website is the recommended way to read this textbook. \sphinxhref{https://github.com/AguaClara/Textbook}{Please visit our GitHub site} to submit an issue, contribute, or comment.
%
\end{footnote}.
\paragraph{\sphinxstylestrong{Notes}}



\renewcommand{\indexname}{Index}
\printindex
\end{document}